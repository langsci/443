\addchap{\lsAbbreviationsTitle}
% \addchap{Abbreviations and symbols}

\begin{tabularx} {.45\textwidth}{>{\scshape}lQ}
1       & first person \\
2       & second person \\
3       & third person \\
a       & ambitransitive (in morpheme glosses) \\
A       & most “agent-like” argument of a transitive clause (primitive grammatical relation).  \\
abs       & absolutive case \\
act       & action (nominalizer -- \textsc{nr.act})\\
adj       & adjective \\
adjr       & adjectivizer \\
adv       & adverb \\
apl       & applicative \\
att       & speaker’s attitude \\
atn       & attention getter \\
ben       & beneficiary \\
caus       & causative \\
CL      & clause \\
cm       & completive particle -- “already” \\
cont       & continuous \\
ctr       & contrast \\
d1       & deictic near speaker \\
d2       & deictic near addressee \\
d3       & deictic near both speaker and addressee \\
d4       & deictic far away, out of sight \\
def       & definite demonstrative \\
deon       & deontic \\
det       & determiner \\
dist       & dist   ributive  \\
\end{tabularx}
\begin{tabularx} {.45\textwidth}{>{\scshape}lQ}
emph       & emphatic (prefix for demonstrative pronoun) \\
erg       & ergative case \\
excl       & exclusive \\
ext       & existential \\
f       & far distance in demonstrative determiners \\
fem       & feminine  \\
g       & given (existential -- \textsc{ext.g}) \\
gen       & genitive case \\
hap       & happenstantial \\
hsy       & hearsay \\
i       & intransitive (verbal inflection)\\
in       & indefinite (existential -- \textsc{ext.in})\\
inc       & incompletive aspect \\
incl        & inclusive \\
inj       & interjection \\
inst        & instrument \\
int       & intensifier \\
ir       & irrealis modality \\
irn       & irony \\
lk       & linker \\
loc       & locative \\
m       & medial distance for the demonstrative determiners \\
masc       & masculine \\
mp       & modifier phrase \\
n       & near distance for the demonstrative determiners \\
\end{tabularx}

\begin{tabularx} {.45\textwidth}{>{\scshape}lQ}
nabs       & non-absolutive case \\
neg       & negative \\
nr       & nominalizer \\
num       & numeral \\
O       & other argument of a transitive clause (primitive grammatical relation) \\
obl       & oblique \\
occ       & ocupation  \\
ord       & ordinal number \\
orig       & origin \\
opt       & optative \\
p       & personal name/pronoun, annotation for prenominal case markers \\
p       & plural in pronoun glosses (e.g., 3\textsc{p.abs}) = Third person plural absolutive.\\
pl       & plural marker, pluraction verbal affix \\
poss       & possessive \\
pp       & prepositional phrase \\
pr       & precise \\
\end{tabularx}
\begin{tabularx} {.45\textwidth}{>{\scshape}lQ}
purp       & purpose \\
r       & realis \\
RE      & referring expression \\
rec       & recipient  \\
red       & reduplication \\
rel       & people in a human relationsip \\
res       & resultative \\
rc       & relative clause \\
rp       & referring phrase \\
rq       & rhetorical question \\
s       & singular (in pronoun glosses) \\
S       & single argument (primitive grammatical relation) \\
sp       & specific \\
spat       & spatial marker \\
supl       & superlative \\
surp       & surprise \\
t       & transitive \\
v       & volitional \\
vr       & verbalizer  \\
\\
\end{tabularx}
\bigskip

\noindent
\begin{tabularx} {\textwidth}{lQ}
{}.       & two word gloss for one Kagayanen word in gloss line of examples \\
--       & separates morphemes in morphological analyses gloss line of examples \\
-       & glottal stop in the Kagayanen orthography line of examples. Also it is placed between two parts of a root that has reduplication as in \textit{sipit-sipit} ‘scorpion’. \\
...       & In analysis line represents conversationally dropped material, usually an inflectional affix. \\
\end{tabularx}
