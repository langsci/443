\chapter{Three Kagayanen Texts}\label{app:b}
\label{chap:threekagayanentexts}

The following are three texts, two oral and one written, that exemplify the texts in the large corpus that constitutes the main empirical basis for this grammar. The database as a whole is described in \chapref{chap:1} and in Appendix A. Each text is preceded by a brief introduction describing the speaker, the circumstances, and significance of the text. The text itself is presented first in the current standard Kagayanen orthography, and then in the grammatically analyzed and translated format consistent with the presentation of examples throughout this grammar.

\section{Oral expository -- \textit{Isturya ta kaan ta lao, kaanlao}: Lunar eclipse}

On January 21, 1988 Carol Pebley asked Mr. \name{Javier}{Carceler}, a retired elementary school teacher and principal, for a story to be recorded on tape. This is the story he told on the spot without any preparation. Later it was transcribed and checked many times by Mr. Carceler and other Kagayanen consultants. Mr. Carceler was a tremendous help in many aspects of the Kagayanen language project being a very skilled and prolific speaker and writer. We are greatly indebted to him for his friendship and help in learning the Kagayanen language.

% \vspace{12pt} % Please don't use extra vspaces! 

Kaanlao pagpati ta mga ittaw di ta Cagayancillo na daw bu\c{l}an an kaan ta lao, kaanlao, bakod kon an na bekkessan palam-ed din bu\c{l}an an. Palam-ed din bu\c{l}an an, ti bu\c{l}an an dili en magpawa. Ta mga ittaw an tak nakita danen na bu\c{l}an ya naduwad en tak palam-ed ta bekkessan na bakod, magpukpok danen an ta mga lata, mga drum o daw ano man na makaatag ta sikad sagbak aged na bekkessan an maadlek ilua din bu\c{l}an ya ig bekkessan an m\c{l}agan en ya. Paglua din ta bu\c{l}an ya miyag amba\c{l}en tan na bu\c{l}an i gapawa isab. Pagpawa isab ta bu\c{l}an an miyag amba\c{l}en tan na bekkessan ya gad\c{l}agan en tak naadlek ta sagbak ya na pabuat ta mga ittaw na papukpok danen mga lata an daw mga drum.

Piro yi daw intindien ta kagi an ta mga drum an daw lata na papukpok dili gani mamatian ta sikad madyo na nyaan duti basak i. Yan pa ayhan na mamatian ta bekkessan an na nyaan dya ta apaw ya na galam-ed ta bu\c{l}an?

Yon una an na pagpati ta mga ittaw. Piro anduni ta mga bag-ong tubo mga ittaw i u\c{l}a en gapati ta iling tan. Danen i nakaiskwila en kag naistudyuan en ta iskwilaan ta mga ittaw na kaan i lao bilang bu\c{l}an an naliperan ta a\c{l}o ta kalibutan. Tak naliperan ta a\c{l}o ta kalibutan an, bu\c{l}an an dili en magpawa ta miad. May kaanlao na ubos gid bu\c{l}an an naduwad, u\c{l}a gapawa. May kaanlao man na tiset nang gagwa an tak u\c{l}a kon nalam-ed ta miad ta bekkessan ya.

Dason eman ni papati ta mga ittaw di. Daw may kaan gani lao mga bai ya na may mga sawa na gabagnes dili kon mag-angad dya, dili kon magluag o mag-angad ta bu\c{l}an ya na kaanlao tak daw mag-angad danen dya ta bu\c{l}an ya na kaanlao o magluag danen an, bata kon an danen daw maggwa libat. Yon isya man na pagpati ta mga inay na gabagnes. Piro tama man na mga inay na galuag man ta kaanlao na daw matao gani bata an danen u\c{l}a man galibat.  

\ea
\gll Isturya  ta  kaan  ta  lao,  kaanlao \\
story  \textsc{nabs}  eat  \textsc{nabs}  sky.snake  lunar.eclipse \\
\glt `A story of the sky-snake's eating, lunar eclipse'
\z

\ea
\gll Kaanlao  pag--pati  ta  mga  ittaw  di  ta  Cagayancillo   na  daw  bu\c{l}an  an  kaan  ta  lao,  kaanlao,  bakod   kon  an  na  bekkessan  pa--lam-ed  din  bu\c{l}an  an. \\
lunar.eclipse  \textsc{nr.act}--believe  \textsc{nabs}  \textsc{pl}  person  \textsc{d1loc}  \textsc{nabs}  Cagayancillo
\textsc{lk}  if/when  moon  \textsc{def.m}  eat  \textsc{nabs}  sky.snake  lunar.eclipse  big
\textsc{hsy}  \textsc{def.m}  \textsc{lk}  snake  \textsc{t.r}--swallow  3\textsc{s.erg}  moon  \textsc{def.m} \\
\glt `Lunar eclipse is a belief of people on Cagayancillo that when the sky snake eats the moon, lunar eclipse, it is said, a big snake, s/he swallowed the moon.’
\z

\ea
\gll Pa--lam-ed  din  bu\c{l}an  an,  ti\footnotemark{}  bu\c{l}an  an  dili  en   mag--pawa. \\
\textsc{t.r}--swallow  3\textsc{s.erg}  moon  \textsc{def.m}  so  moon  \textsc{def.m}  \textsc{neg.ir}  \textsc{cm}
\textsc{i.ir}--bright \\
\footnotetext{The word \textit{ti} is a variant of the more common discourse particle \textit{ta} ‘so.’ (see \chapref{chap:pragmaticallymarkedstructures}, \sectref{sec:discourseparticles})}
\glt ‘S/he swallowed the moon so the moon will not become bright.’
\z

\ea
\gll Ta  mga  ittaw  an  tak  na--kita  danen  na  bu\c{l}an   ya  na--duwad en  tak  pa--lam-ed  ta  bekkessan   na  bakod,  mag--pukpok  danen  an  ta  mga  lata,  mga  drum   o  daw  ano\footnotemark{}  man  na  maka--atag  ta  sikad  sagbak   aged  na  bekkessan  an  ma--adlek,  i--lua  din  bu\c{l}an   ya  ig\footnotemark{}  bekkessan  an  m--d\c{l}agan  en  ya.\footnotemark{} \\
so  \textsc{pl}  person  \textsc{def.m}  because  \textsc{a.hap.r}--see  3\textsc{s.erg}  \textsc{lk}.  moon
\textsc{def.f}  \textsc{a.hap.r}--lose  \textsc{cm}  because  \textsc{t.r}--swallow  \textsc{nabs}  snake
\textsc{lk}  big  \textsc{i.ir}--beat  3\textsc{p.abs}  \textsc{def.m}  \textsc{nabs}  \textsc{pl}  can  \textsc{pl}  drum
or  if/when  what  also  \textsc{lk}  \textsc{i.hap.ir}--give  \textsc{nabs}  very  noise
so.that  \textsc{lk}  snake  \textsc{def.m}  \textsc{a.hap.ir}--afraid  \textsc{t.deon}--spit.out  3\textsc{s.erg}  moon
\textsc{def.f}  and  snake  \textsc{def.m}  \textsc{i.ir}--run  \textsc{cm}  \textsc{att} \\
\footnotetext{The word \textit{naduwad} ‘happened to be lost’ usually is pronounced as \textit{narwad} in realis modality and \textit{marwad} in irrealis modality.}
\footnotetext{The word \textit{ano} is a \isi{Tagalog} word meaning ‘what.’ Code switching to \isi{Tagalog} or \isi{Hiligaynon} is frequent in more formal speeches.}
\footnotetext{\textit{Ig} is a Visayan word, probably borrowed from \isi{Cuyunon}, meaning ‘and’. Like \textit{ano} and other borrowings, it is sometimes used in more formal types of oral speeches.}
\footnotetext{The attitude marker \textit{ya} is here to indicate an important exciting part of the story that the snake is gone. Since the snake is far away out of sight the marker \textit{ya} ‘far away’ is used instead of the \textit{i} ‘near to speaker’ or \textit{an} ‘somewhere in the area of addressees.’} 
\glt ‘So, as for the people, because they saw the moon happened to have disappeared because the big snake swallowed it, they will beat on cans, drums and whatever else that can give out very noisy (sound) so that when the snake is afraid, s/he will have to spit out the moon and the snake will run away.’
\z

\ea
\gll Pag--lua  din  ta  bu\c{l}an  ya,  miyag  amba\c{l}--en   tan  na  bu\c{l}an  i  ga--pawa  isab. \\
\textsc{nr.act}--spit.out  3\textsc{s.erg}  \textsc{nabs}  moon  \textsc{def.f}  want  say--\textsc{t.ir}
\textsc{d3nabs}   \textsc{lk}  moon  \textsc{def.n}  \textsc{i.r}--bright  again \\
\glt ‘When s/he spits out the moon that means to say that the moon is becoming bright again.’  
\z

\ea
\gll Pag--pawa  isab  ta  bu\c{l}an  an  miyag  amba\c{l}--en   tan  na  bekkessan  ya\footnotemark{}  ga--d\c{l}agan  en  tak   na--adlek  ta  sagbak  ya  na  pa--buat  ta  mga   ittaw  na  pa--pukpok  danen  mga  lata  an  daw  mga  drum. \\
\textsc{nr.act}--bright  again  \textsc{nabs}  moon  \textsc{def.m}  want  say--\textsc{t.ir}
\textsc{d3nabS}  \textsc{lk}  snake  \textsc{def.f}  \textsc{i.r}--run  \textsc{cm}  because
\textsc{a.hap.r--}afraid  \textsc{nabs}  noise  \textsc{def.f}  \textsc{lk}  \textsc{t.r}--make  \textsc{nabs}  \textsc{pl}
person  \textsc{lk}  \textsc{t.r}--beat  3\textsc{s.erg}  \textsc{pl}  can  \textsc{def.m}  and  \textsc{pl}  drum \\
\footnotetext{The form \textit{ya} here, in contrast to the \textit{ya} in the previous sentence, is the distal definite demonstrative. It is used here since the snake is far away, off stage and no longer a participant in the story.}
\glt ‘When the moon becomes bright again, that means to say that the snake already ran away because (s/he) is afraid of the noise that the people made when they beat on cans and drums.'
\z

\ea
\gll Piro yi daw intindi--en ta kagi an ta mga drum an daw lata na pa--pukpok dili gani ma--mati--an ta sikad madyo na nyaan duti\footnotemark{} basak i. \\
but  \textsc{d1abs}  if/when  understand--\textsc{t.ir}  1\textsc{p.incl.erg}  sound  \textsc{def.m}  \textsc{nabs}
\textsc{pl}  drum  \textsc{def.m}  and  can  \textsc{lk}  \textsc{t.r}--beat  \textsc{neg.ir}  truly  \textsc{a.hap.ir}--hear--\textsc{apl}
\textsc{nabs}  very  far  \textsc{lk}  \textsc{spat.def}  \textsc{d1loc}  land  \textsc{def.n} \\
\footnotetext{The word \textit{duti} `near, precise' locatonal adverb is a variant of the more comon \textit{unti} or \textit{ti}. It may be archaic.}
\glt ‘But if we understand this, the sound of the drums and cans that are beaten are truly not heard very far here on the land.’
\z

\ea
\gll Yan pa ayhan na ma--mati--an ta bekkessan an na nyaan dya ta apaw ya na ga--lam-ed ta bu\c{l}an? \\
\textsc{d2abs}  \textsc{inc}  perhaps  \textsc{lk}  \textsc{a.hap.ir}--hear--\textsc{apl}  \textsc{nabs}  snake  \textsc{def.m}
\textsc{lk}  \textsc{spat.def}  \textsc{d4loc}  \textsc{nabs}  above  \textsc{def.f}  \textsc{lk}  \textsc{i.r}--swallow
\textsc{nabs}  moon \\
\glt ‘Will that perhaps be heard by the snake that is above that swallowed the moon?’
\z

\ea
\gll Yon  una  an  na  pag--pati  ta  mga  ittaw. \\
\textsc{d3abs}  first  \textsc{def.m}  \textsc{lk}  \textsc{nr.act--}believe  \textsc{nabs}  \textsc{pl}  person \\
\glt ‘That is the long ago belief of people.’
\z

\ea
\gll  Piro  anduni  ta  mga  bag-ong-tubo,\footnotemark{}  mga  ittaw  i  u\c{l}a  en   ga--pati  ta  iling  tan. \\
but  now/today  \textsc{nabs}  \textsc{pl}  new.generation  \textsc{pl}  person  \textsc{def.n}  \textsc{neg.r}  \textsc{cm}
\textsc{i.r}--believe  \textsc{nabs}  like  \textsc{d3nabs} \\
\footnotetext{The words \textit{bag-ong tubo} form an idiom that is more usually pronounced \textit{bag-o tubo} ‘new generation’ literally ‘new growth.’ The word \textit{bag-ong} is a combination of the Kagayanen word \textit{bag-o} ‘new’ and the \isi{Tagalog} linker \textit{ng}.}
\glt ‘But today, the new generation, people no longer believe like that.’
\z

\ea
\gll Danen  i  naka--iskwila  en  kag\footnotemark{}  na--istudyo--an  en  ta   iskwila--an  ta  mga  ittaw  na  kaan  i  lao  bilang   bu\c{l}an  an  na--liped--an  ta  a\c{l}o  ta  kalibutan. \\
3\textsc{p.abs}  \textsc{def.n}  \textsc{i.hap.ir}--school  \textsc{cm}  and  \textsc{a.hap.r}--study--\textsc{apl}  \textsc{cm}  \textsc{nabs}
school--\textsc{nr}  \textsc{nabs}  \textsc{pl}  people  \textsc{lk}  eat  \textsc{def.n}  sky.snake  as moon  \textsc{def.m}  \textsc{a.hap.r}--block--\textsc{apl}  \textsc{nabs}  shadow  \textsc{nabs}  earth \\
\footnotetext{The word \textit{kag} is a \isi{Hiligaynon} word meaning ‘and’. It is used more often in formal oral speeches.}
\glt ‘They have gone to school and have studied in the school that a lunar eclipse is when the moon happens to be blocked by the shadow of the earth.’
\z

\ea
\gll Tak  na--liped--an  ta  a\c{l}o  ta  kalibutan  an,   bu\c{l}an  an  dili  en  mag--pawa  ta  miad. \\
because  \textsc{a.hap.r}--block--\textsc{apl}  \textsc{nabs}  shadow  \textsc{nabs}  earth  \textsc{def.m} moon  \textsc{def.m}  \textsc{neg.ir}  \textsc{cm}  \textsc{i.ir}--bright  \textsc{nabs}  well \\
\glt ‘Because it happens to be blocked with the shadow of the earth, the moon does not become very bright.’
\z

\ea
\gll May  kaanlao  na  ubos  gid  bu\c{l}an  an  na--duwad,   u\c{l}a  ga--pawa. \\
\textsc{ext.in}  lunar.eclipse  \textsc{lk}  completely  \textsc{int}  moon  \textsc{def.m}  \textsc{a.hap.r}--lose 
\textsc{neg.r}  \textsc{i.r--}bright \\
\glt ‘There are some lunar eclipses that the moon really completely disappears, it does not become bright.’
\z

\ea
\gll  May  kaanlao  man  na  tiset  nang  ga--gwa  an  tak  u\c{l}a   kon  na--lam-ed  ta  miad  ta  bekkessan  ya. \\
\textsc{ext.in}  lunar.eclipse  also  \textsc{lk}  little  only/just  \textsc{i.r}--out  \textsc{def.m}  because  \textsc{neg.r}
\textsc{hsy}  \textsc{a.hap.r}--swallow  \textsc{nabs}  well  \textsc{nabs}  snake  \textsc{def.f} \\
\glt ‘There are other lunar eclipses that just a little bit comes out because the snake did not happen to completely swallow it.’
\z

\ea
\gll  Dason  eman  ni  na  pa--pati  ta  mga  ittaw  di. \\
next  again.as.before  \textsc{d1abs}  \textsc{lk}  \textsc{t.r}--believe  \textsc{nabs}  \textsc{pl}  person  \textsc{d1loc} \\
\glt ‘This is another thing the people here believe.’
\z

\ea
\gll Daw  may  kaan  gani  lao,  mga  bai  ya  na  may   mga  sawa  na  ga--bagnes,  dili  kon  mag--angad  dya,  dili   kon  mag--luag  o  mag-angad  ta    bu\c{l}an  ya  na  kaanlao tak  daw  mag--angad  danen  dya  ta  bu\c{l}an  ya  na  kaanlao o  mag--luag  danen  an,  bata  kon  an  danen  daw mag--gwa  libat. \\
if/when  \textsc{ext.in}  eat  truly  sky.snake  \textsc{pl}  woman  \textsc{def.f}  \textsc{lk}  \textsc{ext.in}
\textsc{pl}  spouse  \textsc{lk}  \textsc{i.r}--pregnant  \textsc{neg.ir}  \textsc{hsy}  \textsc{i.ir}--look.up  \textsc{d4loc}  \textsc{neg.ir}
\textsc{hsy}  \textsc{i.ir}--watch  or  \textsc{i.ir}--look.up  \textsc{nabs}  moon  \textsc{def.f}  \textsc{lk}  lunar.eclipse 
because  if/when  \textsc{i.ir}--look.up  3\textsc{p.abs}  \textsc{d4loc}  \textsc{nabs}  moon  \textsc{def.f}  \textsc{lk}  lunar.eclipse
or  \textsc{i.ir}--watch  3\textsc{p.abs}  \textsc{def.m}  child  \textsc{hsy}  \textsc{def.m}  3\textsc{p.gen}  if/when
 \textsc{i.ir}--out  cross-eyed \\
\glt ‘If there is really a lunar eclipsse, married women who are pregnant do not look up there, do not watch or look up at the moon that is a lunar eclipse because if they look up there at the moon that is a lunar eclipse or they watch, it is said their child when it comes out will be cross-eyed.’
\z

\ea
\gll Yon  isya  man  na  pag--pati    ta  mga  inay  na  ga--bagnes. \\
\textsc{d3adj}  one  also  \textsc{lk}  \textsc{nr.act--}believe  \textsc{nabs}  \textsc{pl}  mother  \textsc{lk}  \textsc{i.r}--pregnant \\
\glt ‘That is also another belief of mothers who are pregnant.’
\z

\ea          %
%In the original text the speaker said here:
%Piro tama man na bata, a mga inay na galuag man ta kaanlao na daw matao gani bata an danen ula man galbat. 
%So there is a false start with the word bata ‘child’; Piro tama man na bata. The he stopped and change it to mga inay ‘mothers’.’ Should this be let in the text here or put in a footnote or just left out?
%Carol Pebley
%September 14, 2023, 2:50 PM
\gll Piro  tama  man  na  mga  inay  na  ga--luag  man  ta  kaanlao   na  daw  ma--tao  gani  bata  an  danen  u\c{l}a   man  ga--libat.\\
but  many  also  \textsc{lk}  \textsc{pl}  mother  \textsc{lk}  \textsc{i.r}--watch  \textsc{emph}  \textsc{nabs}  lunar.eclipse
\textsc{lk}  if/when  \textsc{a.hap.ir}--born  truly  child  \textsc{def.m}  3\textsc{p.gen}  \textsc{neg.r}
\textsc{emph}  \textsc{i.r}--cross.eyed \\
\glt ‘But, there are many mothers who watched the lunar eclipse who when truly their child is born (it) did not become cross-eyed.’
\z

\setcounter{equation}{0}
\section{Written narrative -- \textit{Isturya parti ta ganti}: Giant}

This is a narrative text written by Miss. \name{Jocelyn C.}{Bundal} on September 14, 1989, in response to a request from Carol Pebley to write some stories about anything she wanted to write about. She worked on the Kagayanen language project for many years, was involved in many aspects of the work, and contributed many oral and written stories. We are deeply grateful for all she has done.\bigskip



May isya na lugar na may tallo na mag-arey, bu\c{l}ag, piang daw buktot. Isya adlaw tallo i na mag-arey gapanaw na munta ta bukid aged mangita ta iran na pangabuian.

Ta iran na pagpanaw na naan ta da\c{l}an may nakita danen i na sikad baa\c{l} na ka\c{l}at. Patimod kon danen. Na naan eman kon ta unaan ya may nakita eman kon danen an na wasay. Patimod eman danen.

Lulugay na sigi kon danen an na panaw may nakita eman danen an na tambo\c{l}. Patimod eman danen.

Ta iran na pagpanaw na naan en danen ta sikad madyo na bukid may nakita buktot i na sikad kon bakod na ba\c{l}ay. Dayon kon amba\c{l} piang i na, ``Mos ilingan ta dya na ba\c{l}ay nan. Basi daw naan don ate na swirti." Dayon kon danen ilingan ba\c{l}ay ya.

Pag-abot danen ta ba\c{l}ay ya dayon kon amba\c{l} piang an na, ``Bakod man ti na ba\c{l}ay a!" Amba\c{l} man kon ya ta buktot a, ``Ganti taan gaistar ti na ba\c{l}ay tak sikad kon datas lamisaan an daw mga bangko." Na pasil-ing kon danen selled na isya nakita kon danen sikad tama na kwarta daw b\c{l}awan.

U\c{l}a nang kon lugay may namatian kon danen an na mga tikang. U\c{l}a kon danen i gasagbak. Namatian kon danen amba\c{l} ya ta ganti na may nasingngutan kon kanen an na bao. Listo kon amba\c{l} ganti na, ``Daw kino ka man na ittaw magwa ka ta ake na ba\c{l}ay." Dayon kon pilak ta piang ibitan din ya na ka\c{l}at. Nakulbaan ganti ya tak nakita din sikad baa\c{l} na ka\c{l}at. Salig ta ganti daw ka\c{l}at na nakita din daw buok ta ittaw na naan ta selled ta ba\c{l}ay din. Listo eman amba\c{l} ganti ya na, ``Daw mas ka pa bakod ki yaken magwa ka naan ta ba\c{l}ay ko an." Dayon kon pilak ta buktot wasay na pabitan din. Nakulbaan kon en ganti ya tak sikad kon bakod na ngipen ta ittaw an na naan ta selled ba\c{l}ay din. ``Piro bisan ino ka pa kabakod na ittaw ka dili a madlek ki kaon tak ake na ba\c{l}ay ni," amba\c{l} ta ganti ya. Dayon kon papakang bu\c{l}ag tambo\c{l} ya ta sikad tudo. Amba\c{l} kon ya ta ganti ya, ``Malin a di. Ittaw na gaistar ta ba\c{l}ay ko i mas pa bakod pa ki yaken." Dayon kon ganti ya \c{l}agan munta naan ta sikad madyo na bukid. Ta subla na adlek ta ganti ya, padilian din ba\c{l}ay din ya na bakod. Dayon kon tatawa tallo ya na mag-arey tak may ba\c{l}ay danen an en na istaran daw dili en danen mabellayan.

\ea
\gll Isturya parti ta ganti \\
story about \textsc{nabs} giant \\ 
\glt `A story about a giant'
\z

\ea
\gll May  isya  na  lugar  na  may  tallo  na  mag--{}-arey,  bu\c{l}ag,   piang  daw  buktot. \\
\textsc{ext.in}  one  \textsc{lk}  place  \textsc{lk}  \textsc{ext.in}  three  \textsc{lk}  \textsc{rel}--friend  blind lame  and  hunchback \\
\glt ‘There was a place where there were three friends, blind, lame and hunchback.’
\z

\ea
\gll Isya  adlaw  tallo  i  na  mag--{}-arey  ga--panaw   na  munta  ta  bukid  aged  ma--ngita   ta  iran  na  pangabui--an. \\
one  day  three  \textsc{def.n}  \textsc{lk}  \textsc{rel}--friend  \textsc{i.r}--walk/go
\textsc{lk}  toward  \textsc{nabs}  mountain  in.order.to  \textsc{a.hap.ir--search}
\textsc{nabs}  \textsc{3}\textsc{p.gen}  \textsc{lk}  living--\textsc{nr} \\
\glt ‘One day the three friends left going to the mountain to search for their means of making a living.’
\z

\ea
\gll  Ta  iran  na  pag--panaw  na  naan  ta  da\c{l}an   may  na--kita  danen  i  na  sikad  baa\c{l}  na  ka\c{l}at. \\
\textsc{nabs}  3\textsc{p.gen}  \textsc{lk}  \textsc{nr.act}--walk/go  \textsc{lk}  \textsc{spat.def}  \textsc{nabs}  road
\textsc{ext.in}  \textsc{a.hap.r}--see  3\textsc{p.abs}  \textsc{def.n}  \textsc{lk}  very  thick  \textsc{lk}  rope \\
\glt ‘During their walking on the road there was something they saw that was a very thick rope.’
\z

\ea
\gll  Pa--timod  kon  danen. \\
\textsc{t.r}--pick.up  \textsc{hsy}  3\textsc{p.erg} \\
\glt ‘They picked it up, it is said.’
\z

\ea
\gll Na  naan  eman  kon  ta  una--an  ya  may  na--kita   eman  kon  danen  an  na  wasay. \\
\textsc{lk}  \textsc{spat.def}  again.as.before  \textsc{hsy}  \textsc{nabs}  first--\textsc{nr}  \textsc{def.f}  \textsc{ext.in}  \textsc{a.hap.r}--see
again.as.before  \textsc{hsy}  3\textsc{p.abs}  \textsc{def.m}  \textsc{lk}  axe \\
\glt ‘When (being) as before in the place ahead, they saw something again as before that was an axe.’
\z

\ea
\gll  Pa--timod  eman  danen. \\
\textsc{t.r}--pick.up  again.as.before  3\textsc{p.erg} \\
\glt ‘They picked it up again as before.’
\z

\ea
\gll  Lu--lugay  na  sigi  kon  danen  an  na  panaw  may   na--kita  eman  danen  an  na  tambo\c{l}. \\
\textsc{red}--long.time  \textsc{lk}  continuously  \textsc{hsy}  3\textsc{p.abs}  \textsc{def.m}  \textsc{lk}  walk/go  \textsc{ext.in}
\textsc{a.hap.r}--see  again.as..before  3\textsc{p.abs}  \textsc{def.m}  \textsc{lk}  drum \\
\glt ‘After some time past when they continuously were walking there again as before was something they saw which was a drum.’
\z

\ea
\gll  Pa--timod  eman  danen. \\
\textsc{t.r}--pick.up  again.as.before  3\textsc{p.erg} \\
\glt ‘They picked it up again as before.’
\z

\ea
\gll  Ta  iran  na  pag--panaw  na  naan  en  danen  ta  sikad   madyo  na  bukid  may  na--kita  buktot  i  na  sikad   kon  bakod  na  ba\c{l}ay. \\
\textsc{nabs}  3\textsc{p.gen}  \textsc{lk}  \textsc{nr.act}--walk/go  \textsc{lk}  \textsc{spat.def}  \textsc{cm}  3\textsc{p.abs}  \textsc{nabs}  very
far  \textsc{lk}  mountain  \textsc{ext.in}  \textsc{a.hap.r}--see  hunchback  \textsc{def.n}  \textsc{lk}  very
\textsc{hsy}  big  \textsc{lk}  house \\
\glt ‘During their walking when they were on a very far away mountain the hunchback one saw something that was a very big house.’
\z

\ea
\gll Dayon\footnotemark{}  kon  amba\c{l}  piang  i  na,  ``Mos\footnotemark{}  0--iling--an   ta  dya  na  ba\c{l}ay\footnotemark{}  nan. \\
then  \textsc{hsy}  say  lame  \textsc{def.n}  \textsc{lk}  let’s.go  \textsc{t.ir}--go--\textsc{apl}
1\textsc{p.incl.erg}  \textsc{d4loc}  \textsc{lk}  house  \textsc{d3abs} \\
\footnotetext{When the adverb \textit{dayon} occurs sentence initially, as in this example, rather than its normal position following the verb, it indicates heighteed tension and exicitement in the story building up to the climax. This pattern is evident in the following sentence and others in this text.}
\footnotetext{The word \textit{mos} is a common expression in Kagayanen meaning ‘let’s go’. It clearly comes from Spanish \textit{vámonos} or \textit{vamos} `let's go'.}
\footnotetext{The words \textit{dya na ba\c{l}ay} sound like a calque from \isi{Hiligaynon}. In that language, the demonstrative \textit{dya} meaning ‘this’ may function as an adnominal adjective. In Kagayanen \textit{dya} is only a locative demonstrative pronoun and not a demonstrative adjective (see \chapref{chap:referringexpressions}, \sectref{sec:pronouns}).}
\glt ‘Then the lame one said, “Let’s go there to that house.’
\z

\ea
\gll  Basi  daw  naan  don  ate  na  swirti. \\
perhaps  if/when  \textsc{spat.def}  \textsc{d3loc}  1\textsc{p.incl.gen}  \textsc{lk}  luck \\
\glt ‘Perhaps our luck will be there.’
\z

\ea
\gll  Dayon  kon  danen  0--iling--an  ba\c{l}ay  ya. \\
right.away  \textsc{hsy}  3\textsc{p.erg}  \textsc{t.ir}--go--\textsc{apl}  house  \textsc{def.f} \\
\glt ‘Right away they went to the house.’
\z

\ea
\gll  Pag--{}-abot  danen  ta  ba\c{l}ay  ya  dayon  kon  amba\c{l}  piang  an   na,  ``Bakod  man  ti  na  ba\c{l}ay  a." \\
\textsc{nr.act}--arrive  3\textsc{p.gen}  \textsc{nabs}  house  \textsc{def.f}  right.away  \textsc{hsy}  say  lame  \textsc{def.m}
\textsc{lk}  big  \textsc{emph}  \textsc{d1nabs}  \textsc{lk}  house  \textsc{inj} \\
\glt ‘When they arrived at the house right away the lame one said, “What a big house truly!”’
\z

\ea
\gll Amba\c{l}  man  kon  ya  ta  buktot  a,  ``Ganti   taan  ga--istar  ti  na  ba\c{l}ay  tak  sikad  kon   datas  lamisa--an  an  daw  mga  bangko." \\
say  \textsc{emph}  \textsc{hsy}  \textsc{att}  \textsc{nabs}  hunchback  \textsc{emph}  giant
maybe  \textsc{i.r}--live  \textsc{d1nabs}  \textsc{lk}  house  because  very  \textsc{hsy}
high  table--\textsc{nr}  \textsc{def.m}  and  \textsc{pl}  chair \\
\glt ‘The hunchback also said, “A giant maybe lives in this house because the table and chairs are very high.”’
\z

\ea
\gll  Na  pa--sil-ing  kon  danen  selled  na  isya  na--kita  kon danen  sikad  tama  na  kwarta  daw  b\c{l}awan. \\
\textsc{lk}  \textsc{t.r}--peek.in  \textsc{hsy}  3\textsc{p.erg}  room  \textsc{lk}  one  \textsc{a.hap.r}--see  \textsc{hsy}
3\textsc{p.erg}  very  many/much  \textsc{lk}  money  and  gold \\
\glt ‘When they peaked inside one room they saw much money and gold.’
\z

\ea
\gll  U\c{l}a  nang  kon  lugay  may  na--mati--an  kon  danen  an   na  mga  tikang. \\
\textsc{neg.r}  only  \textsc{hsy}  long.time  \textsc{ext.in}  \textsc{a.hap.r}--hear--\textsc{apl}  \textsc{hsy}  3\textsc{p.erg}  \textsc{def.m}
\textsc{lk}  \textsc{pl}  footstep \\
\glt ‘Not a long time they heard something that was footsteps.’
\z

\ea
\gll  U\c{l}a  kon  danen  i  ga--sagbak. \\
\textsc{neg.r}  \textsc{hsy}  3\textsc{p.abs}  \textsc{def.n}  \textsc{i.r}--noisy \\
\glt ‘They were not noisy.’
\z

\ea
\gll  Na--mati--an  kon  danen  amba\c{l}  ya  ta  ganti  na   may  na--singngot--an  kon  kanen  an  na  bao. \\
\textsc{a.hap.r}--hear--\textsc{apl}  \textsc{hsy}  3\textsc{p.erg}  say  \textsc{def.f}  \textsc{nabs}  giant  \textsc{lk}
\textsc{ext.in}  \textsc{a.hap.r}--smell--\textsc{apl}  \textsc{hsy}  3\textsc{s.abs}  \textsc{def.m}  \textsc{lk}  odor \\
\glt ‘They heard what the giant said that he smells something that is an odor.’
\z

\ea
\gll Listo  kon  amba\c{l}  ganti  na,  “Daw  kino  ka  man  na  ittaw   ma--gwa  ka  ta  ake  na  ba\c{l}ay.” \\
quickly  \textsc{hsy}  say  giant  \textsc{lk}  if/when  who  2\textsc{s.abs}  \textsc{emph}  \textsc{lk}  person
\textsc{a.hap.ir}--out  2\textsc{s.abs}  \textsc{nabs}  1\textsc{s.gen}  \textsc{lk}  house \\
\glt ‘The giant quickly said, “Whoever person you are get out of my house.”’
\z

\ea
\gll  Dayon  kon  pilak  ta  piang  ...--ibit--an  din  ya  na  ka\c{l}at. \\
right.away  \textsc{hsy}  throw.out  \textsc{nabs}  lame  \textsc{t.r}--hold--\textsc{apl}  3\textsc{s.erg}  \textsc{def.f}  \textsc{lk}  rope \\
\glt ‘Right away the lame one threw out the rope that he was holding.’
\z

\ea
\gll  Na--kulba--an  ganti  ya  tak  na--kita  din  sikad  baa\c{l}  na  ka\c{l}at. \\
\textsc{a.hap.r--}nervou\textsc{s.--apl}  giant  \textsc{def.f}  because  \textsc{a.hap.r}--see  3\textsc{s.erg}  very  thick  \textsc{lk}  rope \\
\glt ‘The giant was nervous because he saw a thick rope.’
\z

\ea
\gll  Salig  ta  ganti  daw  ka\c{l}at  na  na--kita  din  daw  buok ta  ittaw   na  naan  ta  selled  ta  ba\c{l}ay  din. \\
thought.wrongly  \textsc{nabs}  giant  if/when  rope  \textsc{lk}  \textsc{a.hap.r}--see  3\textsc{s.erg}  if/when  hair 
\textsc{nabs}  person  \textsc{lk}  \textsc{spat.def}  \textsc{nabs}  inside  \textsc{nabs}  house  3\textsc{s.gen} \\
\glt  ‘The giant thought wrongly that the rope that he saw was the hair of the person who was in his house.’
\z

\ea
\gll Listo  eman  amba\c{l}  ganti  ya  na,  “Daw  mas  ka  pa   bakod  ki  yaken  ma--gwa  ka  naan  ta  ba\c{l}ay   ko  an.” \\
quickly  again.as.before  say  giant  \textsc{def.f}  \textsc{lk}  if/when  more  2\textsc{s.abs}  \textsc{inc} big  \textsc{obl.p}  1s  \textsc{a.hap.ir}--out  2\textsc{s.abs}  \textsc{spat.def}  \textsc{nabs}  house
1\textsc{s.gen}  \textsc{def.m} \\
\glt ‘Quickly the giant spoke again, “If you are bigger than me, you can get out of my house.”’
\z
\ea
\gll  Dayon  kon  pilak  ta  buktot  wasay  na  pa--ibit--an  din. \\
right.away  \textsc{hsy}  throw.out  \textsc{nabs}  hunchback  axe  \textsc{lk}  \textsc{t.r}--hold--\textsc{apl}  3\textsc{s.erg} \\
\glt ‘Right way the hunchback threw out the axe he was holding.’
\z

\ea
\gll  Na--kulba--an  kon  en  ganti  ya  tak  sikad  kon   bakod  na  ngipen  ta  ittaw  an  na  naan  ta  selled   ba\c{l}ay  din. \\
\textsc{a.hap.r--}nervou\textsc{s.--apl}  \textsc{hsy}  \textsc{cm}  giant  \textsc{def.f}  because  very  \textsc{hsy} big  \textsc{lk}  teeth  \textsc{nabs}  person  \textsc{def.m}  \textsc{lk}  \textsc{spat.def}  \textsc{nabs}  inside house  3\textsc{s.gen} \\
\glt ‘The giant was nervous because the tooth of the person who was inside his house were very big.’
\z

\ea
\gll  ``Piro  bisan  ino  ka  pa  ka--bakod  na  ittaw  ka   dili  a  ma--adlek  ki  kaon  tak  ake  na   ba\c{l}ay  ni,"  amba\c{l}  ta  ganti  ya. \\
but  even.though  what  2\textsc{s.abs}  \textsc{inc}  \textsc{nr}--big  \textsc{lk}  person  2\textsc{s.abs}
\textsc{neg.ir}  1\textsc{s.abs}  \textsc{a.hap.ir}--afraid  \textsc{obl.p}  2s  because  1\textsc{s.gen}  \textsc{lk}
house  \textsc{d1abs}  say  \textsc{nabs}  giant  \textsc{def.f} \\
\glt ‘“But no matter how big a person you are, I will not be afraid of you because this is my house,” the giant said.’
\z

\ea
\gll  Dayon  kon  pa--pakang  bu\c{l}ag  tambo\c{l}  ya  ta  sikad  tudo. \\
right.away  \textsc{hsy}  \textsc{t.r}--beat  blind  drum  \textsc{def.f}  \textsc{nabs}  very  intense. \\
\glt ‘Right away the blind one beat the drum very hard.’
\z

\ea
\gll  Amba\c{l}  kon  ya  ta  ganti  ya,  ``M--alin  a  di... \\
say  \textsc{hsy}  \textsc{att}  \textsc{nabs}  giant  \textsc{def.f}  \textsc{i.v.ir}--from  1\textsc{s.abs}  \textsc{d1loc} \\
\glt ‘The giant said, “I will leave here...’
\z

\ea
\gll  ... ittaw  na  ga--istar  ta  ba\c{l}ay  ko  i  mas  pa  bakod   pa   ki  yaken." \\
{} person  \textsc{lk}  \textsc{i.r}--live  \textsc{nabs}  house  1\textsc{s.gen}  \textsc{def.n}  more  \textsc{inc}  big  \textsc{emph}
\textsc{obl.p}  1s \\
\glt ‘...The person who lives in my house is even bigger than me.”’
\z

\ea
\gll  Dayon  kon  ganti  ya  d\c{l}agan  munta  naan  ta  sikad   madyo  na  bukid. \\
right.away  \textsc{hsy}  giant  \textsc{def.f}  run  toward  \textsc{spat.def}  \textsc{nabs}  very far  LK  mountain \\
\glt ‘Right away the giant ran away going to a very far away mountain.’
\z

\ea
\gll  Ta  sub\c{l}a  na  adlek  ta  ganti  ya,  pa--dili--an   din  ba\c{l}ay  din  ya  na  bakod. \\
\textsc{nabs}  too.much  \textsc{lk}  fear  \textsc{nabs}  giant  \textsc{def.f}  \textsc{t.r}--abandon--\textsc{apl}
3\textsc{s.erg}  house  3\textsc{s.gen}  \textsc{def.f}  \textsc{lk} big \\
\glt ‘From too much fear of the giant, he abandoned his big house.’
\z

\ea
\gll  Dayon  kon  ta--tawa  tallo  ya  na  mag-arey  tak   may  ba\c{l}ay  danen  an  en  na  istar--an  daw  dili  en   danen  ma--bellay--an. \\
right.away  \textsc{hsy}  \textsc{red}--laugh  three  \textsc{def.f}  \textsc{lk}  \textsc{rel}--friend  because
\textsc{ext.in}  house  3\textsc{p.abs}  \textsc{def.m}  \textsc{cm}  \textsc{lk}  live--\textsc{nr}  and  \textsc{neg.ir}  \textsc{cm}
3\textsc{p.abs}  \textsc{a.hap.ir}--tired--\textsc{apl} \\
\glt ‘Then the three friends were laughing because they have a house now to live in and they will not be harshipped anymore.’
\z

\setcounter{equation}{0}
\section{Oral Expository -- \textit{Isturya parti ta mangasawa}: Wedding}

Text \#3 is an oral expository text about wedding customs. During Carol Pebley’s time of language learning on Cagayancillo, \name{Darlie}{Bundal} was one of her main language teachers. In September, 1987 Carol asked Darlie to explain about wedding customs and Carol recorded what Darlie said on tape. Carol and several native speakers then transcribed and translated this story. The text was produced on the spot without any practice or preparation time. Carol is sincerely thankful for Darlie’s help, and for the help of several other Kagayanen consultants, \textit{terek ta tagipusuon} ‘straight from the heart.’

\vspace{12pt}

Bai daw mama na nubyuanay, daw gusto en na mangasawa, mapirsunal anay mama i ta ginikanan ta bai. Ansaan daw miyag. Daw miyag kani en ginikanan i ta mga bai en, yo en mangagon en mama i.

Tapos kagon mama i, ma\c{l}es en mga ginikanan i ta bai ta kagon ya ta mama ta ba\c{l}ay ta mama. 

Tapos, madisisyon en mga ginikanan i daw kan-o kasa\c{l} en mga kasa\c{l}en i. Daw dili pa gusto ta mga ginikanan, mama i masirbi anay ta ba\c{l}ay ta bai daw u\c{l}a pa daw kan-o kasa\c{l} danen an.

Paagi daw makasa\c{l}, bai i, daw makasa\c{l} appat abay din i. Appat na mama daw appat man na bai, daw flowergirl daw ringbearer. Isya na paris na abay a magda\c{l}a bilo. Isya an a kandila, singsing, daw kwarta. 

Daw miling kani ta altar en, mauna abay i punta ta tengnga ya ta unaan. Magsunod mga flowergirl daw ring bearers.

Tapos en, nya kasa\c{l}en ya na mama naan tagad ta tengnga ya ta simbaan. Tapos, bai i padu\c{l}{}-ong amay din naan tengnga daw paintriga din ta mama ya na sawaen din.

Daw tapos kani kasa\c{l} en, mga kasa\c{l}en i gabisa ta mga ginikanan danen daw gamusta ta mga arey danen. 

\ea
\gll Bai  daw  mama  na  nubyo--anay,  daw  gusto  en  na  ma--ngasawa, ma--pirsunal  anay  mama  i  ta  ginikanan  ta  bai.\\
woman  and  man  \textsc{lk}  boyfriend--\textsc{rec}  and  want  \textsc{cm}  \textsc{lk}  \textsc{a.hap.ir}--get.married
\textsc{a.hap.ir}--personal  first/for.awhile  man  \textsc{def.n}  \textsc{nabs}  parents  \textsc{nabs}  woman \\
\glt ‘A woman and a man who are boyfriend and girlfriend, if they want to get married, the man will first personally meet with the parents of the girl.’
\z

\ea
\gll \emptyset--Ansa--an  daw  miyag. \\
\textsc{t.ir}--ask--\textsc{apl}  if/when  agree \\
\glt  `(He) will ask (them) if they agree.’
\z

\ea
\gll  Daw  miyag  kani  en  ginikanan  i  ta  mga  bai  en   yo  en  ma--ng--kagon  en  mama  i. \\
if/when  agree  later  \textsc{cm}  parents  \textsc{def.n}  \textsc{nabs}  \textsc{pl}  woman  \textsc{cm}
\textsc{d}4\textsc{abs}  \textsc{cm}  \textsc{a.hap.ir}--\textsc{pl}--engage  \textsc{cm}  man  \textsc{def.n} \\
\glt `Later if the parents of the women agree (to the men marrying their daughters), that (is when) the men will be engaged (to the women).'
\z

\ea
\gll  Tapos  kagon  mama  i,  m--ba\c{l}es  en  mga ginikanan  i  ta  bai   ta  kagon  ya  ta  mama  ta  ba\c{l}ay  ta  mama. \\
then  engage  man  \textsc{def.n}  \textsc{i.v.ir}--respond  \textsc{cm}  \textsc{pl}  parents  \textsc{def.n}  \textsc{nabs}  woman
\textsc{nabs}  engage  \textsc{def.f}  \textsc{nabs}  man  \textsc{nabs}  house  \textsc{nabs}  man \\
\glt  `After the engaging of the man, the parents of the woman will respond to the engaging of the man at the house of the man.’
\z

\ea
\gll  Tapos  ma--disisyon  en  mga  ginikanan  i  daw  kan-o   kasa\c{l}  en  mga  kasa\c{l}--en  i. \\
then  \textsc{a.hap.ir}--decision  \textsc{cm}  \textsc{pl}  parents  \textsc{def.n}  and  when
wedding  \textsc{cm}  \textsc{pl}  wedding--\textsc{nr}  \textsc{def.n} \\
\glt  `Then the parents will decide when will be the wedding of the ones to be wedded.’
\z

\ea
\gll  Daw  dili  pa  gusto  ta  mga  ginikanan,  mama  i  ma--sirbi  anay  ta  ba\c{l}ay  ta  bai,   daw  u\c{l}a  pa  daw  kan-o  kasa\c{l}  danen  an. \\
if/when  \textsc{neg.ir}  \textsc{inc}  want  \textsc{nabs}  \textsc{pl}  parents
man  \textsc{def.n}  \textsc{a.hap.ir}--serve  first/for.awhile  \textsc{nabs}  house  \textsc{nabs}  woman
if/when  \textsc{neg.r}  \textsc{inc}  if/when  when  wedding  3\textsc{p.gen}  \textsc{def.n} \\
\glt  `If the parents do not yet want (the wedding to take place), the man will serve for awhile in the house of the woman, if (it is not yet known) when will be their wedding.’
\z

\largerpage
\ea
\gll  Paagi  daw  ma--kasa\c{l},  bai  i, daw ma--kasa\c{l}  appat  abay  din  i. \\
way/means  if/when  \textsc{a.hap.ir}--wedding  woman  \textsc{def.n} if/when \textsc{a.hap.ir}--wedding  four  wedding.attendant  3\textsc{s.gen}  \textsc{def.n} \\
\glt  `The way when having a wedding is the woman, when having a wedding, her wedding attendants are four.’
\z

\ea
Appat  na  mama,  daw  appat  man  na  bai,  daw  flowergirl   daw  ringbearer.  \\
four  \textsc{lk}  man  and  four  also  \textsc{lk}  woman  and  flowergirl
and  ringbearer. \\
\glt  `(There are) four men and four women and a flowergirl and a ring bearer.’
\z

\ea
\gll  Isya  na  paris  na  abay  a  mag--da\c{l}a  ta  bilo.\footnotemark{} \\
one  \textsc{lk}  pair  \textsc{lk}  wedding.attendant  \textsc{ctr}  \textsc{i.ir}--take/carry  \textsc{nabs}  white.string \\
\footnotetext{The white string is put around the bride and bridegroom symbolizing unity.}
\glt  `One pair of attendants will carry a white string.’
\z

\ea
\gll  Isya  an  a  kandila,  singsing,  daw  kwarta. \\
one  \textsc{def.m}  \textsc{ctr}  candle  ring  and  money \\
\glt  `One (another pair will carry) a candle, (another pair will carry) a ring and (another pair will carry) money.' `
\z

\ea
\gll  Daw  m--iling  kani  ta  altar  en,  ma--una  abay  i   punta  ta  tengnga  ya  ta  una--an. \\
if/when  \textsc{i.v.ir}--go  later  \textsc{nabs}  altar  \textsc{cm}  \textsc{a.hap.ir}--first  wedding.attendant  \textsc{def.n}
go  \textsc{nabs}  middle  \textsc{def.f}  \textsc{nabs}  first--\textsc{nr} \\
\glt  `When going later to the altar, the wedding attendants will go first to the middle of the front (of the church).’
\z

\ea
\gll  Mag--sunod  mga  flowergirl  daw  ringbearer. \\
\textsc{i.ir}--follow  \textsc{pl}  flowergirl  and  ringbearer. \\
\glt `The flowergirl and ringbearer will follow.’
\z

\ea
\gll  Tapos  en  nya  kasa\c{l}--en  ya  na  mama,  naan    tagad  ta  tengnga  ya  ta  simba--an. \\
then  \textsc{cm}  \textsc{d}4\textsc{abs}  wedding--\textsc{nr}  \textsc{def.f}  \textsc{lk}  man  \textsc{spat.def}
wait  \textsc{nabs}  middle  \textsc{def.f}  \textsc{nabs}  worship--\textsc{nr} \\
\glt  `Then that one, the man to be wedded, waits in the middle of the church.’
\z

\ea
\gll  Tapos  bai  i  pa--du\c{l}{}-ong  amay  din  naan   tengnga  daw  intriga  ta  mama  ya  na  sawa--en  din. \\
then  woman  \textsc{def.n}  \textsc{t.r}--accompany.somewhere  father  3\textsc{s.gen}  \textsc{spat.def}
middle  and  turn.over  \textsc{nabs}  man  \textsc{def.f}  \textsc{lk}  spouse--\textsc{nr}  3\textsc{s.erg} \\
\glt `Then the woman, her father accompanies (her) to the middle and hands (her) over to the man whom she will marry.’
\z

\ea
\gll  Daw  tapos  kani  kasa\c{l}  en,  mga  kasa\c{l}--en  i   ga--bisa  ta  mga  ginikanan  danen  daw    ma--ng--kamusta  ta  mga  arey  danen. \\
if/when  finished   later  wedding  \textsc{cm}  \textsc{pl}  wedding--\textsc{nr}  \textsc{def.n}
\textsc{i.r}--bless.older.person  \textsc{nabs}  \textsc{pl}  parents  3\textsc{p.gen}  and
\textsc{a.hap.ir--pl--}greet  \textsc{nabs}  \textsc{pl}  friend  3\textsc{p.gen} \\
\glt  `When later the wedding is finished, the ones being wedded bless their parents and greet their friends.’
\z
