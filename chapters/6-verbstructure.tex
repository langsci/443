\chapter{Verb structure and inflection}
\label{chap:verbstructure}
\section{Introduction}
\label{sec:introduction-6}

While \isi{Referring Phrases} mention people and things that are involved in a situation being communicated (\chapref{chap:referringexpressions}), the predicating word, or \textit{predicate,}\is{predicate} is the part of the clause that describes the situation itself. In Kagayanen discourse, the most frequent clause types begin with a \isi{predicating word}, usually an \textit{inflected verb}\is{inflected verbs}. In addition to functioning as predicates in basic clauses, verb-like elements also function in \isi{nominalization}s (\chapref{chap:referringexpressions}), and various types of dependent clause constructions\is{dependent clauses} (\chapref{chap:clausecombining}). In \chapref{chap:non-verbalclauses} we discussed predicates that are not based on verbs. In this chapter we will discuss the general structure of verbal predicates. We argue for a major distinction between \textit{inflectional morphology}\is{inflectional morphology} and \textit{stem-forming morphology}\is{stem-forming morphology}, and describe the \isi{inflectional paradigm}. In \chapref{chap:stemformingprocesses} we describe and exemplify eight stem-forming morphological processes. In \chapref{chap:verbclasses-1} we identify several morphosyntactically distinct \textit{verb classes}\is{verb classes}, and attempt to describe their usages in terms of the parameters of \textit{situation type}\is{situation type}, \textit{volitionality}\is{volitionality} and \textit{semantic transitivity}\is{semantic transitivity}\is{transitivity!semantic}. In \chapref{chap:verbclasses-2}, we take a semantics prior approach to several other verb classes, all of which exhibit specialized morphosyntactic behavior.

\section{Inflection}
\label{sec:verbstructureandinflection}
\is{inflection|(}

There are two classes of morphological processes in Kagayanen verbs which we describe as \textit{inflectional}\is{inflectional} and \textit{stem-forming} processes\is{stem-forming}.\footnote{We use the term “(morphological) process”\is{morphological processes} as a cover term for \isi{prefixation}, \isi{suffixation}, \isi{infixation}, and \isi{reduplication}. Reduplication does not play a role in inflection in Kagayanen, therefore we can refer to the set of inflections as “\isi{inflectional affixes}” (see \tabref{tab:inflectionalaffixes}). However, the stem-forming group includes two types of reduplication, so the group as a whole can only accurately be described as “processes”.}  The difference between these two classes is that inflectional processes are strictly \isi{paradigmatic}, whereas stem-forming processes are not. To clarify this important distinction, we need to introduce the difference between a \textit{paradigm}\is{paradigm} and a \textit{word family}\is{word family} (following a long tradition in linguistics, summarized nicely in \citealt[14]{haspelmath2002}). We will do this using the verb forms of English.

The inflectional paradigm of English verbs consists of exactly five forms:

\ea
\label{bkm:Ref59175832}
Paradigm for the English verb “do”. \\
Past tense:        \textit{did} \\
Present participle:        \textit{doing} \\
Past participle:      \textit{done} \\
“S-form” (3\textsuperscript{rd} person singular, present):  \textit{does} \\
Bare form:    \textit{do}
\z

Whenever the verb “do” is used in discourse, it must occur in one of these five forms. Notice that the bare form, sometimes called the “zero form”, is a member of this paradigm. This is because English speakers subconsciously know that any verb must embody one of these values, therefore when no overt marker (-\textit{d}, -\textit{ing}, -\textit{n}, or -\textit{s}) occurs, it leaves a “conspicuous absense”. This conspicuous absense is then available as the marker (or \textit{exponent}\is{exponent}) of a particular value in the list. A zero is as strong a “marker” of a member of a paradigm as any overt affix.

Contrast the paradigm in \REF{bkm:Ref59175832} with a family of words also based on the root \textit{do}:

\ea
\label{bkm:Ref59176157}Word family based on the root “do”: \\
Reiterative:    \textit{redo} \\
Reversative:    \textit{undo} \\
Abilitative:    \textit{doable} \\
Reiterative abilitative:  \textit{redoable} \\
Reversative abilitative:  \textit{undoable} \\
. . .
\z

The forms in \REF{bkm:Ref59176157} can be thought of as a partial \isi{word family} because they are all based on the same root. However, they are not a paradigm. There are several reasons this list of forms is not a \isi{paradigm}. First, they do not vary according to any coherent semantic dimension – their meanings are logically independent of one another: reversative, reiterative, ability, etc. Second, the meaning elements they express may combine with one another, for example \textit{undoable} combines the notions of reversative and ability; \textit{redoable} combines reiteration with ability. Forms such as \textit{un-redo}, \textit{re-undo}, and \textit{re-undoable}, though perhaps odd-sounding out of context, are also conceivable members of this family. Third, none of the affixes that define a word family are required for the root to be used in discourse. Fourth, these forms do not have analogies for every English verb. For example forms like *\textit{resee}, *\textit{unremember}, *\textit{bleedable} and many others are extremely difficult to find appropriate communicative contexts for. Finally, the forms \textit{undo} and \textit{redo} are verbs, just like the root \textit{do}. As such, they also occur in the five possible inflectional forms for English verbs: \textit{undid}, \textit{redoing}, \textit{undone}, \textit{redoes,} \textit{undo}, and so on. In other words, the affixes that define the word family may occur with each other, and with the inflectional forms in the paradigm.

In contrast, the inflections, as given in \REF{bkm:Ref59175832}, are mutually exclusive – a form cannot logically express present tense and past tense at the same time; 3\textsuperscript{rd} person singular and non-3\textsuperscript{rd} person singular at the same time, and so on. Thus forms with multiple inflections are impossible: *\textit{diding}, *\textit{doinged}, *\textit{done-ing}, *\textit{dids}, *\textit{dones}, etc. Note that these forms are strictly uninterpretable; they do not just “sound odd out of context” the way \textit{un-redo} and \textit{re-undo} might.

Dictionary makers usually consider members of a \isi{word family} to be different words: \textit{do}, \textit{redo}, and \textit{undo} probably have their own entries in a reasonably complete dictionary of English (with possible cross-references to one-another). The forms in a paradigm, on the other hand, are not considered different dictionary entries. They are versions of the same lexical item. You might say the members of a \isi{paradigm} are like one person wearing different sets of clothing, whereas the members of a \isi{word} family are like different people with a family resemblance to one another.

Paradigms and word families occur in many languages of the world; language communities everywhere seem to find groups similar to these useful for expressing meanings. Philippine languages, and Kagayanen in particular, have much richer and more powerful paradigms\is{paradigm} and \isi{word families} than English does. Because English has relatively simple morphology, it is easier to show the difference between \isi{paradigmatic} and \isi{non-paradigmatic} morphological values in English than it is in Kagayanen, though the distinction is still very relevant to Kagayanen. The principles we have just outlined apply equally to paradigms and word families in Kagayanen.

 inflectional affixes \is{inflectional affixes}in Kagayanen define a paradigm. There are exactly twelve possible inflectional forms for every verb (recall that English has only five). One and only one of these forms is required in order for a word to function as a verbal predicate in a main, independent clause. Furthermore, the inflectional affixes express points on two well-defined and intersecting dimensions: \isi{grammatical transitivity} and \isi{modality} (see \sectref{bkm:Ref58564378}). The meaning of each affix in the inflectional paradigm logically excludes the others: A verb cannot be both realis and irrealis in modality; grammatically transitive and simultaneously grammatically intransitive, and so on. Furthermore, a zero form is not simply the lack of inflection, but rather it expresses very specific inflectional values, namely \isi{transitive} and \isi{irrealis}. The thirteen possible inflectional forms of any verb may be compared to the different clothes that one person might wear (see \tabref{tab:inflectionalaffixes}).

Stem-forming processes\is{stem-forming processes}, on the other hand, are non-paradigmatic—they may combine with one another to form complex meaningful stems that are then available for inflection. They also express meaning elements that are logically independent of one another, such as \isi{causative}, \isi{applicative}, \isi{pluraction}, \isi{reciprocity} and \isi{aspect}. Roots with \isi{stem-forming} affixes can be compared to members of a family, all of whom resemble one another, but who are clearly different people. For example, consider a family of words based on the root \textit{atag} ‘to give.’

\ea
\label{bkm:Ref59345903}
“Family” of stems based on the root \textit{{}-atag}: \\
\begin{tabbing}
\hspace{4cm} \= \hspace{2.4cm} \= \kill    
bare root: \>   {}-\textit{atag}      \> ‘give’ \\
causative: \>   {}-\textit{paatag}    \> ‘cause/let give’ \\
applicative: \>   {}-\textit{atagan}/-\textit{tagan} \> ‘give to s.o.’ \\
pluraction:  \>  {}-\textit{ngatag} \>   ‘give together/several times’ \\
continuative:  \>  {}-\textit{atag-atag}   \> ‘keep on giving’ \\
distributive: \>   {}-\textit{tagtag}    \> ‘give to several/distribute’ \\
reciprocal: \>   {}-\textit{atagay}    \> ‘give to each other’ \\
causative+applicative: \> {}-\textit{paatagan}  \>  ‘cause/let give to s.o.’ \\
causative+continuative: \> {}-\textit{paatag-atag} \> ‘cause/let keep on giving’ \\
continuative+reciprocal: \> {}-\textit{atag-atagay} \> ‘to keep giving to each other’ \\
. . .
\end{tabbing}

\z

The Kagayanen words in \REF{bkm:Ref59345903} are all verb stems, ready to be inflected with one of the possible inflectional affixes (indicated by the dash in front of each form). They all have different, though related, meanings. Note that the morphological processes that give rise to these forms (\textit{pa}-, \textit{ng}-, -\textit{an}, -\textit{ay} and reduplication) may combine with one another to produce even more complex forms, as illustrated in the last three examples in \REF{bkm:Ref59345903}. These  stem-forming morphological processes\is{stem-forming morphology}, and others, are a rich resource available to Kagayanen speakers to create complex and nuanced meanings.

We hope this discussion has clarified the important distinction between  inflectional and  stem-forming morphological processes. In the rest of this chapter, we will describe how  inflectional and  stem-forming processes contribute to the structure of verbal predicates, including preliminary examples in \tabref{tab:someexamplesofverbalpredicates}. In \sectref{bkm:Ref62720957} we discuss overall verb structure. In \sectref{bkm:Ref59351027} we briefly describe roots in Kagayanen. In \sectref{bkm:Ref59351065} we discuss and exemplify  inflectional affixes, and in \sectref{bkm:Ref119308716} we describe certain discourse conditions under which  inflectional prefixes may be omitted.
\is{inflection|)}
\section{Overall verb structure}
\is{verb structure|(}

\label{bkm:Ref62720957}
A verbal predicate\is{verbal predicates} in Kagayanen consists minimally of a \textit{stem} \is{stem}(the main part of the verb), plus one and only one \isi{inflectional affix} (see \tabref{tab:inflectionalaffixes}).


\ea
Verbal predicate = Infl-Stem
\z

Optional \isi{adverbial elements}, including notably second-position \isi{adverbial particles}, may occur before or after this basic structure. See \chapref{chap:modification}, \sectref{sec:adverbs} on adverbial modifiers.

\hspace*{-1.5pt}A  stem consists minimally of a \textit{root},\is{roots} and may contain one or more \isi{stem-forming processes} (see \chapref{chap:stemformingprocesses}):

\ea
Stem = (SF)-Root-(SF)
\z

\tabref{tab:someexamplesofverbalpredicates} provides some preliminary examples of \isi{verbal predicates} with the obligatory  inflectional\is{inflection} (transitivity/modality) and root\is{root} elements, along with various combinations of optional  stem-forming (SF) affixes. Beginning with \sectref{sec:root}, each part of this template is described and illustrated in more detail.

\begin{table}
\caption{Some examples of verbal predicates}
\label{tab:someexamplesofverbalpredicates}
\fittable {
\begin{tabular} {
    p{1.2cm}     %INFL
    p{1cm}    %SF3
    p{1.1cm}  %SF2
    p{1.2cm}    %SF1
    p{2.9cm}    %ROOT
    p{1.1cm}    %SF
    >{\RaggedRight\arraybackslash}p{4cm}    %Free translation
                }
\lsptoprule
INFL & SF3 & SF2 & SF1 \newline
(redup.) & ROOT & SF & Free translation \\
\midrule
pa- \newline
\textsc{t.r} &  &  &  & luto \newline
‘cook’ &  & ‘cooks/cooked X’ \\
\tablevspace
ga- \newline
\textsc{i.r} &  &  &  & luto &  & ‘X cooks/cooked’ (intr.) \\
\tablevspace
pa- &  &  &  & lutu & {}-an \newline
\textsc{apl} & ‘cook(ed) with/in/for X’ \\
\tablevspace
pa- & pa- \newline
\textsc{caus} &  &  & luto &  & ‘cause(d) to cook X.’ \\
\tablevspace
na- \newline
\textsc{a.hap.r} &  &  &  & luto &  & ‘was-able-to/could cook X, have cooked X.’ \\
\tablevspace
ga- &  &  &  & palangga \newline
‘have affection’ & {}-ay \newline
\textsc{rec} & ‘have affection for each other’ \\
\tablevspace
{}-en\footnote{Though most inflectional affixes are prefixes, the transitive, irrealis member of the paradigm, \nobreakdash-\textit{en}/-\textit{on}, is a suffix, hence the full form of this Inflected Verb is \textit{palanggaen} ‘will have affection for X’. This analysis is exemplified and justified in \sectref{bkm:Ref448124627}.} \newline
\textsc{t.ir} &  &  &  & palangga &  & ‘will have affection for X’ \\
\tablevspace
ga- &  & ng- \newline
\textsc{pl.act} &  & agaļ \newline
cry &  & ‘each of several actors cry’ (intr.) \\
\tablevspace
ga- &  &  & luto & luto &  & ‘sort of/pretend to cook, keep on cooking’ \\
\tablevspace
i- \newline
\textsc{t.deon} &  &  &  & luto &  & ‘must/should cook X, have to cook X’ \\
\tablevspace
na- \newline
\textsc{a.hap.r} &  & ng- \newline
\textsc{pl.act} &  & uli \newline
‘go home’ &  & ‘each of several had gone home’ \\
\tablevspace
m- \newline
\textsc{i.v.ir} &  &  &  & uli &  & ‘will go home.’ \\
\tablevspace
pa- &  &  &  & lutu & {}-a \newline\textsc{xc} & ‘cooks/cooked X!’ (exclamatory) \\
\tablevspace
pa- &  &  &  & atag \newline
‘give’ & {}-i \newline
\textsc{xc.apl} & ‘gave to X!’ (exclamatory) \\
\lspbottomrule
\end{tabular}
}
\end{table}
\is{verb structure|)}

\section{The root} 
\label{bkm:Ref59351027}\label{sec:root} \is{roots|(}
A \textit{root}\is{root} is the main element that expresses the general situation described in a verbal clause. It is the “foundation” on which speakers build complex and nuanced \isi{discourse scenes}\is{scenes}. Substantive roots are two or more syllables in length. Roots with more than two syllables are sometimes borrowings, or involve archaic stem-forming processes that are not used productively to accomplish communicative work by modern speakers. In \chapref{chap:stemformingprocesses} we describe \isi{stem-forming processes} that are still “live”, and \isi{productive}.

Most lexical roots in Kagayanen are “precategorial”\is{precategoriality} in that they are not absolutely categorized as verbs, nouns, adjectives, and so on. Rather they \textit{function as} \isi{predicators} (verbs), \isi{Referring Expressions} (nouns), or \isi{modifiers} (adjectives/adverbs) as they are integrated into grammatical constructions. Because of their meanings, some roots are more likely to function as Referring Expressions, others as Modifiers, others as predicators, and so on, but in few cases is such categorization absolute.

For convenience, we will use the term “verb"\is{verbs} to refer to roots that typically function as predicators, though this must be understood as a descriptive conve\-nience-substantive lexical items are not inherently categorized as verbs, nouns or modifiers (see \citealt{oyzon2021} for discourse-based argumentation for a similar claim in Waray\is{Waray language}, another Philippine language).
\is{roots|)}

\section{Inflectional affixation}
\label{bkm:Ref58564378} \label{bkm:Ref59351065} \label{sec:verbinflection}\is{inflection|(}
 Inflectional affixes\is{inflectional affixes} consist of a set of prefixes, and one suffix, -\textit{en} (sometimes realized as -\textit{on} or zero). These affixes simultaneously express two major dimensions, which we term \isi{grammatical  transitivity} (\isi{transitive}, \isi{intransitive}, and \isi{ambitransitive}) and \isi{modality} (\isi{realis}\is{modality!realis} and \isi{irrealis}\is{modality!irrealis}, dynamic\is{dynamic modality}\is{modality!dynamic}, \isi{happenstantial modality}\is{modality!happenstantial}, \isi{external motivation}\is{modality!external motivation}, and \isi{deontic modality}\is{modality!deontic}) as displayed in \tabref{tab:inflectionalaffixes}. All other \isi{morphological processes} in verbs, including two kinds of \isi{reduplication}, other prefixes and suffixes, and one infix belong to the \isi{stem-forming} group. These are described in \chapref{chap:stemformingprocesses}.

As discussed in the introduction to this chapter, the inflectional affixes constitute a \isi{paradigm}---a finite set of affixes, one and only one of which is required for a form to function as the main predicator in an independent verbal clause. A discusssion of the terms in \tabref{tab:inflectionalaffixes} immediately follows. In \sectref{bkm:Ref124921711} through \sectref{bkm:Ref64362346}, examples of each of these affixes in context are provided.

\begin{table}
\small
\caption{Inflectional affixes} \is{inflectional affixes!table}
\label{tab:inflectionalaffixes}
\begin{tabular}{ll ll ll}
	\lsptoprule
	Modality &  Transitivity & \multicolumn{2}{c}{Realis} & \multicolumn{2}{c}{Irrealis} \\\cmidrule(lr){3-4}\cmidrule(lr){5-6}
	&  & General & Narrow & General & Narrow \\\midrule
	Dynamic & intransitive & ga- & ag-\footnote{(repetitive/habitual)} & mag- & m-\footnote{(volitional)} \\
	& transitive & pa- & ag- & {}-en/-on/-0 & i-\footnote{(deontic)} \\
	\tablevspace
	Happenstantial & intransitive\footnote{(agent-preserving verbs)} & naka- & ka-\footnote{(external motivation)} & maka- & ka-\footnote{(external enablement/inference)}, ma-\footnote{(hypothetical/polite)} \\
	& ambitransitive\footnote{(transitive, and patient-preserving verbs)} & na- &  & ma- & \\
	\lspbottomrule
\end{tabular}
\end{table}

\subsection{Grammatical transitivity}
\label{sec:grammaticaltransitivity}\is{grammatical transitivity|(}\is{transitivity!grammatical|(}

There is a difference between \isi{semantic transitivity}\is{transitivity!semantic} and grammatical transitivity. Semantic transitivity has to do with situations being communicated, whereas grammatical transitivity has to do with grammatical structures used to depict those situations. The two types of transitivity do not necessarily coincide. For example, English constructions such as \textit{she already ate}, or \textit{he drank of the water of bitterness} present semantically transitive\is{semantic transitivity}\is{transitivity!semantic} situations (someone eating or drinking something) in grammatically intransitive constructions (there is no direct object). Something similar occurs in Kagayanen. Situations that seem to require an Undergoer and a distinct Controller may be expressed in an intransitive construction, with the Undergoer simply omitted or presented in the non-absolutive case, preceded by \textit{ta} or \textit{ki}. We describe such constructions as \textit{detransitive}\is{detransitive constructions}. Such constructions are used when there is an Undergoer, but it is less salient in the communicative situation than the Actor in some way. The Actor is presented as the most affected participant, and the Undergoer, if present at all, is presented as less central in some way, either by being indefinite, non-specific, incompletely affected, and so on. Of course, the big difference between English and Kagayanen is that English does not mark grammatical transitivity on the verb, whereas Kagayanen does. The verb \textit{eat} is the same whether it occurs in a transitive (SUBJ \textit{eats} OBJ) or intransitive (SUBJ \textit{eats}) frame. In Kagayanen, however, the verb in a detransitive construction is explicitly marked as intransitive. Examples \REF{bkm:Ref118364790} and \REF{bkm:Ref125017724} illustrate detransitive constructions from the corpus. Additional examples are found throughout this grammar.
\ea
\label{bkm:Ref118364790}
Pagtapos  nay  igma,  \textbf{gatan-aw  kay  ta  sini}. \\\smallskip

\gll Pag-tapos  nay  igma,  \textbf{ga-tan-aw}  \textbf{kay}  \textbf{ta}  \textbf{sini}. \\
\textsc{nr.act}-finish 1p{excl.gen} lunch \textsc{t.r}-look 1p{excl.abs}  \textsc{nabs}  movie \\
\glt ‘After finishing our lunch, \textbf{we watched a movie}.’ [AGWN-L-01 5.5]
\z
\ea
\label{bkm:Ref125017724}
Piro  tama  man  na  \textbf{mga  inay  na  galuag  man  ta  kaanlao}... \\\smallskip

\gll Piro  tama  man  na  \textbf{mga}  \textbf{inay}  \textbf{na}  \textbf{ga-luag}  \textbf{man}  \textbf{ta} \textbf{kaanlao}... \\
but  many  also  \textsc{lk}  \textsc{pl}  mother  \textsc{lk}  \textsc{i.r}-watch  \textsc{emph}  \textsc{nabs}  lunar.eclipse \\
\glt ‘But there were many \textbf{mothers who watched a lunar eclipse} ...’ [JCOE-C-03 5.4]
\z

Example \REF{bkm:Ref118364790} is part of a story about what the Actors did. It is not about any particular movie.  In terms of \citealt{halliday2004}, \textit{sini}, `movie', simply describes the ``range" over which the watching took place; the Actors engaged in ``movie watching". Therefore the Actor is presented as absolutive (\textit{kay} `absolutive, first person plural exclusive'), and the verb is inflected as intransitive (\textit{ga}- `intransitive, realis'). Similarly, example \REF{bkm:Ref125017724} is about the mothers, and not any particular lunar eclipse. In this case, the mothers are the Head of a modifying (relative) clause. As will be discussed in \chapref{chap:clausecombining}, the Head of a relative clause must be the absolutive of that clause. Therefore, the verb inside the relative clause must be in an intransitive form with the Actor understood as the absolutive.

As discussed in \chapref{chap:voice}, Grammatical  transitivity in Kagayanen is crucially involved in the expression of \textit{voice}\is{voice}. That is to say, the transitivity values of the inflectional affixes reflect the Grammatical  transitivity of the clause depending on the voice being expressed. Since voice is a global property of constructions rather than individual verbs (i.e., in addition to verb marking, voice also involves case marking of Referential Expressions, and is dependent on discourse considerations), we defer a discussion of how Grammatical  transitivity is deployed in the voice system until \chapref{chap:voice}.

A \textit{macrorole}\is{macroroles} is a cover term for a group of \isi{semantic roles} that tends to be treated grammatically in a similar way in a particular language \citep{vanvalin2000}. The macroroles that are most relevant to Kagayanen syntax are the \textit{Actor}\is{actor} (or \textit{Starting point}\is{starting point}) and the \textit{Undergoer}\is{undergoer} (or \textit{Endpoint}\is{endpoint}) of the situation described in the predicate. Specific semantic roles included within these macroroles are listed in \tabref{tab:macrorolesandsemanticroles}. However, the precise semantic role of the absolutive and other nominals depends to a large extent on the context and the situation type expressed by the stem. In \chapref{chap:verbclasses-1} and \chapref{chap:verbclasses-2}, we discuss a range of situation types and the semantic roles that they represent.

\begin{table}
\caption{Macroroles and semantic roles}
\label{tab:macrorolesandsemanticroles}\is{macroroles and semantic roles, table}
\begin{tabularx}{\textwidth}{lQl}
\lsptoprule
Macroroles\is{macroroles}: & Actor\is{actor} (Starting point, \newline controller, initiator) & Undergoer (Endpoint) \\
\midrule
Specific Semantic roles: & Agent & Patient \\
& Cognizer\is{cognizer} & Instrument\is{instrument} \\
& Force\is{force} & Theme\is{theme} \\
& etc. &  Location\is{location} \\
&  & Goal\is{goal} \\
&  & Beneficiary\is{beneficiary} \\
&  & Recipient\is{recipient} \\
&  & etc. \\
\lspbottomrule
\end{tabularx}
\end{table}
\is{transitivity!grammatical|)}\is{grammatical transitivity|)}

\subsection{Modality}
\label{sec:modality}\is{modality|(}

The terms \textit{dynamic}\is{modality!dynamic}, \textit{happenstantial modality}\is{modality!happenstantial}, \textit{realis}\is{modality!realis}, \textit{irrealis}\is{modality!irrealis}, \textit{external motivation}\is{modality!external motivation} and \textit{deontic}\is{modality!deontic}\is{deontic modality} refer to particular grammatical values in the  inflectional paradigm of Kagayanen (see \tabref{tab:inflectionalaffixes}). These terms are based on what we see as the most general functions of these values, most of which seem to belong to a category best described as \textit{modality}, that is, the speaker’s perspective on the truth, likelihood, or necessity of a situation, as well as the speaker’s assessment of responsibility and control exercised by various referents or external forces. This is a very broad characterization of the functions of these values. As will become clear, some specific usages may seem to fall outside these notions, or even outside the domain of “modality” altogether as traditionally defined (see, e.g., \citealt{kiefer1987}, \citealt{nuyts2016}). Modality, as with all other components of the verbal system, is used creatively by speakers to create complex and nuanced meanings that sometimes defy categorical analysis by linguists. 

As mentioned above, the  inflectional values illustrated in \tabref{tab:inflectionalaffixes} sometimes have different effects depending on the situation type being expressed. For example, \isi{dynamic modality} is the common form for situations involving volitional motion, development, and/or change of state. In these situations, dynamic modality may be understood as \textit{perfective}\is{perfective aspect} in aspect, and in narrative discourse as main storyline (or \textit{foreground}\is{foreground} \citealt{hopper1980}) events. Happenstantial modality\is{happenstantial modality|(} may be used to express these same situation types, while presenting them as accidental, abilitative, opportunitive or perfect aspect, depending on the context.

On the other hand, happenstantial modality is the common form for situations involving sensory, emotional\footnote{One apparent emotion verb stands out as an exception to this generalization. The root \textit{gilek} ‘to get angry’ is usually expressed in dynamic modality, \textit{gagilek} rather than happenstantial \textit{?nagilek.} It is possible that \textit{gilek} is perceived as a volitional reaction to something, rather than an emotion, in which case the English gloss ‘get angry’ may be misleading.} or cognitive experience, and intransitive, \isi{non-volitional situations} (\textit{melt, die, drift off}, etc.). In these situations, happenstantial modality may be understood as perfective\is{perfective aspect} in aspect, and in narrative discourse as depicting foreground events\is{foreground}. Dynamic modality may be used for these same situation types, but presents them as ongoing processes or inchoatives\is{inchoative} (with intransitive inflections), or causatives\is{causative} (with transitive inflection). \tabref{tab:Relationshipbetweensituationtypesanddynamicvshappenstantialmodality} summarizes the relationship between general situation types and dynamic versus happenstantial modality. Specific examples are provided below.\is{happenstantial modality|)}

\begin{table}
\caption{Relationship between situation types and dynamic versus happenstantial modality}\is{dynamic and happenstantial modality, table}
\label{tab:Relationshipbetweensituationtypesanddynamicvshappenstantialmodality}
\begin{tabularx}{\textwidth}{QQQ}
\lsptoprule
Situation type & Effect of dynamic affixes & Effect of happenstantial affixes \\
\midrule
Volitional motion, development and/or change, and some non–volitional change of state processes.  & Perfective, main event line in narrative. & Accidental, abilitative, perfect aspect, opportunitive (‘get to do’) \\
\tablevspace
Experience, emotion, cognition, and other non-volitional states and situations. & Inchoative or ongoing situation (with intransitive inflection). Causative (with transitive inflection). & Perfective, main event line in narrative. \\
\lspbottomrule
\end{tabularx}
\end{table}

\is{irrealis modality|(}\is{modality!irrealis|(}The values described as “irrealis”\is{irrealis} have several usages, including \isi{future} time, imperative\is{imperative modality}, \isi{optative modality}, and others. What the main uses of the irrealis forms have in common is that they describe unrealized situations, therefore irrealis is a reasonable descriptive cover term.\is{irrealis modality|)}\is{modality!irrealis|)}

For some of the categories in \tabref{tab:inflectionalaffixes} we have made a distinction between “general”\is{general inflectional values} and “narrow”\is{narrow inflectional values} values. The general values are those that are the default in everyday discourse, and which express the widest range of meanings. The narrow values, while preserving the transitivity and modality values of their general counterparts, zero in on and enforce more precise meanings. Repetitive\is{repetitive}, \isi{volitional}, \isi{deontic modality}, \isi{hypothetical}, \isi{habitual}, and \isi{external motivation} meanings may be inferred from verbs with the general affixes, but they are directly expressed by verbs with the narrow affixes.

Volitional situations\is{volitional situations} are those in which a semantic Agent instigates and controls the situation consciously and with intent, for example, the basic senses of \textit{go, come, jump, kneel, grumble, wink, cook, read}, etc. Non-volitional situations are those for which there is no controller or initiator, for example, \textit{die, melt, come untied, drift off, sink}, etc. The prototypical absolutive argument of a non-volitional intransitive verb is a semantic Patient, or a theme (in the sense of \citealt{gruber1965}).

Deontic modality\is{deontic modality}\is{modality!deontic} is a kind of irrealis that asserts that the normally human Agent is compelled by some external force to perform the activity described by the verb. We will use the English modal \textit{have to} to approximate the meaning of deontic forms cited out of context. Deontic modality receives explicit expression in the dynamic transitive prefix \textit{i}{}-. For grammatically intransitive constructions, and happenstantial modality constructions, deontic is not formally distinguished from other irrealis modalities in verbal morphology. When necessary, a deontic sense can be specified for irrealis constructions using adverbials such as \textit{kinangļan}, ‘necessary/need’, and \textit{dapat}, ‘must’ (see \chapref{chap:modification}, \sectref{sec:adverbs}).

It is clear that the \isi{dynamic affixes}\is{dynamic modality}\is{modality!dynamic} constitute the most straightforward and easily understood verbal inflections. For this reason, when describing the inflections, we omit mention of modality for the dynamic mode affixes. For example, “transitive irrealis” is a shorthand way of referring to the dynamic, transitive, irrealis inflections \textit{-en/-on/-0.}

Examples of constructions that employ each of the inflectional affixes are given in the following subsections, with the relevant verb forms bolded. Short explanations are sometimes provided as justification for the terminology chosen. A description of the classes of roots that are distinguished in terms of the way they interact with inflectional morphology is provided in \chapref{chap:verbclasses-1}.
\is{modality|)}

\subsection{Intransitive realis \textit{ga}{}-}
\label{bkm:Ref124921711}\label{sec:intransitiverealis} \is{intransitive realis inflection|(}

The inflectional prefix \textit{ga}{}- is the most common form for \isi{dynamic intransitive events} on the storyline in narratives. In this usage, the aspectual interpretation is perfective\is{perfective aspect}, as in the first clause of example \REF{bkm:Ref396718263}:

\ea
\label{bkm:Ref396718263}
\textbf{Gapit}  kay  anay  ta  dawisan  tak  \textbf{ganampara} pa  mga  gurmiti  ta  sid-anan. \\\smallskip
 \gll \textbf{Ga-apit}  kay  anay  ta  dawis-an  tak  \textbf{ga-ng-tampara} pa  mga  gurmiti  ta  sid-anan. \\
\textsc{i.r}-stop.off  1\textsc{p.excl.abs}  first/for.a.while  \textsc{nabs}  point-\textsc{nr}  because  \textsc{i.r-pl}-goggles \textsc{inc}  \textsc{pl}  crew  \textsc{nabs}  viand \\
\glt `We stopped off first at the point (of the island) because the crew were still spear fishing for viand.’ [VAWN-T-18 2.8]
\z

However, predicates inflected with \textit{ga}{}- may be understood as imperfective\is{imperfective aspect}. For example, the second occurrence of \textit{ga-} (\textit{ganampara}) in \REF{bkm:Ref396718263} is within an adverbial clause expressing a backgrounded, non-storyline event. The imperfective aspect interpretation is reenforced by the use of the incompletive adverbial \textit{pa}.

Example \REF{bkm:Ref396718461} illustrates one use of \textit{ga-} in which the interpretation may be either \isi{inceptive} or imperfective. The event of walking is ongoing in the story. After the sentence illustrated in \REF{bkm:Ref396718461}, the story describes crossing a river on stepping stones and other events that happened on the way to the community of Teresa. Therefore, example \REF{bkm:Ref396718461} may be understood as a (perfective) inceptive “we began to walk again” or it could mean “we were walking again.” In either case it does not describe the entire event of walking to Teresa, and therefore is not perfective:

\largerpage[2]
\ea
\label{bkm:Ref396718461}
Pag-abot  nay  ta  liyo,  \textbf{gapanaw} kay eman  munta  ta  Teresa. \\\smallskip
 \gll Pag-abot  nay  ta  liyo,  \textbf{gapanaw} kay eman  m-punta  ta  Teresa. \\
\textsc{nr.act}-arrive  1\textsc{p.excl.gen}  \textsc{nabs}  other.side  \textsc{i.r}-go/walk  1\textsc{p.excl.abs}
as.before  \textsc{i.v.ir}-go  \textsc{nabs}  Teresa \\
\glt `When we arrived on the other side (of a river), we were/began walking again as before going to Teresa.’ [BGON-L-01 1.10]
\z

With roots that describe properties or states, \textit{ga}{}- expresses the process of entering the state (\textit{inchoative})\is{inchoative}. Examples \REF{bkm:Ref119937645} and \REF{bkm:Ref416513591} are common greetings, both of which may be considered compliments, or fixed expressions of greeting after long separation:

\ea
\label{bkm:Ref119937645}
\textbf{Gatambek} ka. \\\smallskip
 \gll \textbf{Ga-tambek}  ka. \\
\textsc{i.r}-fat/healthy  2\textsc{s.abs} \\
\glt ‘You are becoming/became fat/healthy.’
\z
\ea
\label{bkm:Ref416513591}
\textbf{Ganiwang}  ka. \\\smallskip
 \gll \textbf{Ga-niwang}  ka. \\
\textsc{i.r}-skinny  2\textsc{s.abs} \\
\glt ‘You are becoming/became skinny.’
\z

The following are additional examples of inchoative \textit{ga}- from the corpus:

\ea
Daw  isya  na  bai  \textbf{gabagnes}  dili  kanen  magpanaw  daw  kilem. \\\smallskip
 \gll Daw  isya  na  bai  \textbf{ga-bagnes}  dili  kanen  mag-panaw  daw  kilem. \\
 if/when  one  \textsc{lk}  woman  \textsc{i.r}-pregnant  \textsc{neg.ir}  3\textsc{s.abs}  \textsc{i.ir}-go/walk  if/when  night \\
\glt ‘When a woman becomes pregnant, she will not go (anywhere) at night.’ [VAOE-J-05 1.1]
\z
\ea
\textbf{Gabakod}  yaken  i  tenged  ta  iran  na  pagsupurta, labi  na  gid  ta  iran  na  pagpalangga. \\\smallskip
 \gll \textbf{Ga-bakod}  yaken  i  tenged  ta  iran  na  pag-supurta, labi  na  gid  ta  iran  na  pag-palangga. \\
\textsc{i.r}-big  1\textsc{s.abs}  \textsc{def.n}  because  \textsc{nabs}  3\textsc{p.gen}  \textsc{lk}  \textsc{nr.act}-support
especially  \textsc{lk}  \textsc{int}  \textsc{nabs}   3\textsc{p.gen}  \textsc{lk}  \textsc{nr.act}-love/affection \\
\glt `I grew up (became big over time) because of their support, especially their love/affection.’ [JBON-J-01 1.11]
\z

These examples illustrate why we consider that \textit{ga}{}- primarily expresses realis modality, and dynamicity, rather than aspect or tense. It is used when the situation expressed involves a change in state, completed or in process, currently relevant or in the past.
\is{intransitive realis inflection|)}

\subsection{Transitive realis \textit{pa}{}-}
\label{sec:transitiverealis}\is{transitive realis inflection|(}

The inflectional prefix \textit{pa-} is the basic form for dynamic, grammatically transitive clauses on the storyline of narratives.\footnote{There is also a  stem-forming prefix \textit{pa}- that is a morphological causative. We provide arguments that there are indeed two distinct \textit{pa}- prefixes in \chapref{chap:stemformingprocesses}, \sectref{sec:morphologicalcausative}.} As such its interpretation is normally perfective\is{perfective aspect} in aspect, with no reference to any internal temporal structure:

\ea
\label{bkm:Ref395019310}
Pag-abot  danen  naan  ta  suba,  \textbf{palubbas}  danen  iran  na bayo  daw  \textbf{pabatang}  danen  naan  ta  kilid  ta  suba  daw  maglangoy.\footnotemark{} \\\smallskip
 \gll Pag-abot  danen  naan  ta  suba,  \textbf{pa-lubbas}  danen  iran  na bayo  daw  \textbf{pa-batang}  danen  naan  ta  kilid  ta  suba  daw  mag-langoy.\footnotemark{} \\
\textsc{nr.act}-arrived  3\textsc{p.gen}  \textsc{spat.def}  \textsc{nabs}  river  \textsc{t.r}-undress  3\textsc{p.erg}  3\textsc{p.gen}  \textsc{lk}
clothes  and  \textsc{t.r}-put  3\textsc{p.erg}  \textsc{spat.def}  \textsc{nabs}  side  \textsc{nabs} river  and  \textsc{i.ir}-bathe {} \\
\footnotetext{This is the “culminative” use of irrealis inflection discussed below.}
\glt `Arriving at the river, they took off their clothes and put (them) at the side of the river and began to bathe.’ (This is a story about a cow and a water-buffalo who accidentally switch skins that they call clothes.) [CBWN-C-25 5.4]
\z
\ea
\textbf{Paibitan}  nay  ta  timbang  mama  na  duma nay. \\\smallskip
 \gll \textbf{Pa-ibit-an}  nay  ta  timbang  mama  na  duma\footnotemark{}  nay. \\
\textsc{t.r}-hold-\textsc{apl}  1\textsc{p.excl.erg}  \textsc{nabs}  balance  man  \textsc{lk}  companion  1\textsc{p.excl.gen} \\
\footnotetext{The word \textit{duma} has several functions, including as a comitative preposition ‘with’. Since it is followed by an enclitic genitive pronoun, \textit{nay} ‘our’, it has to mean ‘companion’ in this context.}
\glt ‘We held on both sides to the man who was our companion.’ (CBWN-C-11 4.8]
\z

In rare cases, a verb inflected with \textit{pa}{}- may be understood as imperfective\is{imperfective aspect}:

\ea
\textbf{Pabunakan}  din  pa  mga  bayo  din. \\\smallskip
 \gll \textbf{Pa-bunak-an}  din  pa  mga  bayo  din. \\
\textsc{t.r}-wash-\textsc{apl}  3\textsc{s.erg}  \textsc{inc}  \textsc{pl}  clothes  3\textsc{s.gen} \\
\glt ‘S/he is still washing his/her clothes.’
\z

In this case the adverbial particle \textit{pa}, ‘still’ (glossed \textsc{inc} for ‘incompletive’ in this context), forces an imperfective interpretation of this clause. Again, we conclude that aspect is secondary to the main function of the inflectional prefix \textit{pa-}. It always expresses dynamicity (movement or change), grammatical transitivity, and realis modality. This results in a perfective aspect interpretation in most cases, but this strong tendency may at times be overridden by other factors. Like \textit{ga}-, the main function of \textit{pa}- is to assert that a dynamic situation described by the verb is actually true in the discourse world. In contrast to \textit{ga}{}-, \textit{pa}{}- always indicates that the predicate has two core arguments either overtly expressed or strongly understood – an Undergoer, or endpoint, in the absolutive case, and a separate Actor, or starting point, in the ergative case.
\is{transitive realis inflection|)}
\subsection{Realis repetitive \textit{ag}{}-}
\label{sec:realisrepetitive}\is{realis repetitive inflection|(}

The prefix \textit{ag}{}- occurs only once in the corpus (example \ref{ex:ag} below) and is not used often in everyday speech. It may be used in a transitive or intransitive case frame, and therefore may be considered \textit{ambitransitive}\is{ambitransitive}. It expresses the idea that the situation has happened more than one time, a few times, once in a while or occasionally. It neither asserts the occurrence of one particular event, nor a habitual (often/on a regular basis) sense. Here we gloss it as \textsc{rep.r} for “repetitive, realis”. However, further research is needed to determine the precise aspectual or modal range of this relatively uncommon prefix.

\ea
Prefix \textit{ag-} in an intransitive frame: \\
\textbf{Ag-iling}  kay  ta  Cawili. \\\smallskip
 \gll \textbf{Ag-iling}  kay  ta  Cawili. \\
\textsc{rep.r}-go  1\textsc{p.excl.abs}  \textsc{nabs}  Cawili \\
\glt ‘We have gone at times to Cawili.’
\z
\ea
\textbf{Agsud-o}  kay  ki  danen  daw  uļa  ubra. \\\smallskip
 \gll \textbf{Ag-sud-o}  kay  ki  danen  daw  uļa  ubra. \\
\textsc{rep.r}-visit  1\textsc{p.excl.abs}  \textsc{obl.p}  3\textsc{p}  if/when  \textsc{neg.r}  work \\
\glt ‘We visited them if/when there was no work.’
\z

\ea
\label{ex:ag}
Prefix \textit{ag-} in a transitive frame (inside a relative clause): \\
Apo  Arey  isya  na  manakem  na  \textbf{ag-insaan}  ta  mga  ittaw. \\\smallskip
 \gll Apo  Arey  isya  na  manakem  na  \textbf{ag-insa-an}  ta  mga  ittaw. \\
ancestor  Arey  one  \textsc{lk}  older  \textsc{lk}  \textsc{rep.r}-ask-\textsc{apl}  \textsc{nabs}  \textsc{pl}  person \\
\glt ‘Ancestor Arey is an older person that people asked/requested things of.’ [EFWE-T-06 2.1]
\z

This prefix may be considered “narrow”, since it occurs far less often than the general, or default, forms \textit{ga}{}- and \textit{pa}{}-, and it expresses a more precise meaning, since a repetitive inference is possible with \textit{ga}{}- and \textit{pa}{}-, but is enforced with \textit{ag}{}-. This distinction between general and narrow dynamic realis prefixes mirrors a similar distinction in the dynamic irrealis prefixes, and happenstantial modalities (see \sectref{sec:transitiveirrealis} and \sectref{sec:intransitivehappenstantialirrealis}).
\is{realis repetitive inflection|)}

\subsection{Intransitive irrealis \textit{mag}{}- and \textit{m-}}
\label{sec:intransitiveirrealis}\is{intransitive irrealis inflection|(}

The inflectional prefix \textit{mag}{}- is the general  inflection for most dynamic intransitive irrealis predicates. It neither requires nor excludes volitionality. The prefix \textit{m}{}- is the more narrow form, which always indicates volitional events. Most verbs only take \textit{mag}{}-, while some allow either \textit{mag}{}- or \textit{m}{}-. As far as we know, there are no verb roots that take only \textit{m}{}-. Both \textit{mag}{}- and \textit{m}{}- are irrealis counterparts of \textit{ga}{}-. Long lists of verb roots that take only \textit{mag}- versus those that take \textit{mag}- or \textit{m}- are provided in \chapref{chap:verbclasses-1}, \sectref{sec:irrealisinflections}. Here we provide a few examples from the corpus.

Example \REF{bkm:Ref392932747} illustrates a majority class root, \textit{ayad} ‘be careful’, in the irrealis form in a construction with a verb in happenstantial modality (\textit{matellek} ‘might get stuck’):

\ea
\label{bkm:Ref392932747}
Pambaļan  ko  ake  na  mga  duma  na  \textbf{mag-ayad} tak  basi  matellek  danen  an. \\\smallskip
 \gll Pa-ambaļ-an  ko  ake  na  mga  duma  na  \textbf{mag-ayad} tak  basi  ma-tellek  danen  an. \\
\textsc{t.r}-say-\textsc{apl} 1\textsc{s.erg} 1\textsc{s.gen}  \textsc{lk}  \textsc{pl} companion \textsc{lk} \textsc{i.ir}-be.careful because  perhaps  \textsc{a.hap.ir}-stick  3\textsc{p.abs}  \textsc{def.m} \\
\glt `I warned my companions to be careful because maybe they might (happen to) get stuck (by a nail in a piece of wood on the ground).’ [VAWN-T-16 2.10]
\z

For minority class verbs, \textit{m}- is the unmarked form expressing a relatively likely irrealis, or \isi{immediate future} situation, while \textit{mag}- expresses the idea of hypothetical or distant future situations. For example, the form \textit{manaw}, based on the root \textit{panaw} ‘to go’, is used in simple future contexts, as when taking leave of someone:

\ea
\textbf{…} \textbf{manaw}  kay  en. \\\smallskip
 \gll \textbf{…} \textbf{m-panaw}  kay  en. \\
 {} \textsc{i.v.ir}-walk/go  1\textsc{p.excl.abs}  \textsc{cm} \\
\glt ‘…we will leave now.’ [RMWN-L-01 3.2]
\z
\ea
\textbf{Manaw}  ki  nang  isab  kani  ta  baybay  i. \\\smallskip
 \gll \textbf{M-panaw}  ki  nang  isab  kani  ta  baybay  i. \\
textsc{i.v.ir}-walk/go 1\textsc{p.incl.abs} only/just  again  later  \textsc{nabs}  beach  \textsc{def.n} \\
\glt ‘We will just walk again later on the beach.’ [BGON-L-01 12.6]
\z

In contrast, \textit{magpanaw} expresses a more distant future \REF{bkm:Ref63688177}, or more hypothetical irrealis action \REF{bkm:Ref63688194}:

\ea
\label{bkm:Ref63688177}
\textbf{Magpanaw}  danen   magsimba  ta  Leganes. \\\smallskip
 \gll \textbf{Mag-panaw}  danen   mag-simba  ta  Leganes. \\
\textsc{i.ir}-go/walk  3\textsc{p.abs}  \textsc{i.ir}-worship  \textsc{nabs}  Leganes \\
\glt ‘They will go to worship in Leganes.’
\z
\ea
\label{bkm:Ref63688194}
Para  dili  mabao   bai  i  daw  gusto  din \textbf{magpanaw}  daw  deļem  magdaļa  kanen  ta  uling,  asin  daw luy-a  ugsak  ta  bulsa  ta  bai  na  gabagnes. \\\smallskip
 \gll Para  dili  ma-bao\footnotemark{}   bai  i  daw  gusto  din \textbf{mag-panaw}  daw  deļem  mag-daļa  kanen  ta  uling,  asin  daw luy-a  ugsak  ta  bulsa  ta  bai  na  ga-bagnes. \\
for \textsc{neg.ir}  \textsc{a.hap.ir}-painful.pregnancy  woman \textsc{def.n}  if/when  want  3\textsc{s.erg}
\textsc{i.ir}-go/walk  if/when  dark  \textsc{i.ir}-carry  3\textsc{s.abs}  \textsc{nabs} charcoal  salt  and
ginger  inside  \textsc{nabs}  pocket  \textsc{nabs}  woman  \textsc{lk}  \textsc{i.r}-pregnant \\
\footnotetext{There are two words spelled \textit{bao} in Kagayanen. They are homographs, but not homophones. \textit{Bao} meaning “painful pregnancy” is stressed on the second syllable and there is a final glottal stop: [baʔóʔ]. \textit{Bao} meaning ‘odor’ is stressed on the first syllable, and lacks the final glottal stop: [báʔo]. Neither glottal stop nor stress is represented consistently in the writing system, though both are contrastive.}
\glt `In order for the woman not to have a painful pregnancy, if she wants to go somewhere when dark, she will carry charcoal, salt, and ginger inside the pocket of the woman who is pregnant.’ [VAOE-J-05 1.7]
\z

In example \REF{bkm:Ref392932169} the prefix \textit{mag}{}- is used twice on minority class verbs in grammatically intransitive constructions, and the prefix \textit{m}{}- is used once on a minority class verb. In this example, the verbs with \textit{mag}{}- describe situations that that may or may not happen (\isi{deontic modality} in this case, signaled by the adverbial \textit{kinangļan}), whereas the verb with \textit{m}{}- describes the much more inevitable event that the baby will come out.

\ea
\label{bkm:Ref392932169}
Daw  bagnes  en  isya  na  nanay  kinangļan  kanen  \textbf{magkaan} ta  gulay  daw  \textbf{mag-inem}  kanen  ta  bitamina  agod  daw \textbf{mwa}  bata  an  biskeg  iya  na  lawa  daw  madyo  ta masakit. \\\smallskip
 \gll Daw  bagnes  en  isya  na  nanay  kinangļan  kanen  \textbf{mag-kaan} ta  gulay  daw  \textbf{mag-inem}  kanen  ta  bitamina  agod  daw \textbf{m-gwa}  bata  an  biskeg  iya  na  lawa  daw  madyo  ta masakit. \\
if/when  pregnant  \textsc{cm}  one  \textsc{lk}  mother  need  3\textsc{s.abs}  \textsc{i.ir}-eat
\textsc{nabs} vegetables  and  \textsc{i.ir}-drink  3\textsc{s.abs}  \textsc{nabs}  vitamins  so.that  if/when \textsc{i.v.ir}-go.out  child  \textsc{def.m}  strong  3\textsc{s.gen}  \textsc{lk}  body  and  far  \textsc{nabs} sick \\
\glt `When a mother is pregnant she should eat vegetables and take (lit. drink) vitamins so that when the baby comes out, his/her body is strong and (s/he) is far from sickness.’ [LBOP-C-03 11.3]
\z

Example \REF{bkm:Ref396719550} illustrates what we call the \textit{culminative}\is{culminative} usage of the irrealis mood. Even though the events of walking and returning home are semantically realis in the story, they are presented as the culminating events in a string of main line events that are closely linked. The events previous to this were about chopping wood and preparing it to carry away. The development in example \REF{bkm:Ref396719550} is about how the Actor carried the wood home, at which point the story ends. This usage is very common following the conjunction \textit{daw}, and any of the irrealis affixes may function in this way.
\ea
\label{bkm:Ref396719550}
Tapos  na  pag-ipid  din  ta  iya  na  apin, papas-an  din  daw  \textbf{manaw}  \textbf{muli}. \\\smallskip
 \gll Tapos  na  pag-ipid  din  ta  iya\footnotemark{}  na  apin, pa-pas-an  din  daw  \textbf{m-panaw}  \textbf{muli}. \\
then  \textsc{lk}  \textsc{nr.act}-arrange  3\textsc{s.gen}  \textsc{nabs}  3\textsc{s.gen}  \textsc{lk} protective.cloth
\textsc{t.r}-carry.on.shoulder  3\textsc{s.erg}  and  \textsc{i.v.ir}-go/walk  \textsc{i.v.ir}-go.home \\
\footnotetext{Recall from Chapter 3 that there are both enclitic pronouns that follow their heads, and free pronouns that normally precede their heads. The form \textit{din} in the first line of this example is the 3\textsc{sg.gen} enclitic pronoun that follows the nominalized verb \textit{pag-ipid} ‘folded/folding’, while iya is the 3\textsc{sg.gen} free pronoun that precedes \textit{apin} ‘protective cloth,’ with the linker \textit{na} intervening. The genitive forms are the same as the ergative forms that occur with verbs inflected as transitive, as is the case with the use of \textit{din} in the second line of example \REF{bkm:Ref396719550}.}
\glt `Then when s/he arranged his/her protective cloth, s/he carried (wood) on his/her shoulders and walked going home.’ (The protective cloth goes on the shoulders under the wood.) [NFWN-T-01 2.21]
\z

Example \REF{bkm:Ref395019310} above also includes a verb with \textit{mag}{}- in this culminative function: \textit{daw maglangoy} ‘and began bathing.’ Verbs in the culminative usage definitely express realis events, though they appear in irrealis form. The functional or historical scenario under which this usage could have developed is a topic for future investigation.
\is{intransitive irrealis inflection|)}

\subsection{Transitive irrealis -\textit{en/-on/-0}}
\label{bkm:Ref448124627} \label{sec:transitiveirrealis}\is{transitive irrealis inflection|(}

The inflectional affix -\textit{en} and allomorphs -\textit{on} and \textit{\emptyset}- are the basic forms for dynamic irrealis transitive clauses. The occurrence of -\textit{en/on} versus \textit{\emptyset}- depends on the lexical class of the verbal root, and the presense versus absense of an applicative suffix. \tabref{tab:rootsthattaketwodistincttransitiveirrealisinflections-1} in \chapref{chap:verbclasses-1} lists a large subset of roots that are distinguished on the basis of the transitive irrealis inflections that they occur with. Here we will describe in prose how these root classes are related.

First, most verbal roots take the -\textit{en/-on} forms, for example, the second bolded verb in \REF{bkm:Ref119939353}, and the bolded verb in \REF{bkm:Ref119939420}. In \chapref{chap:verbclasses-1} these are designated as  ``Class VI" roots. There is also a minority class of roots that always take \emptyset{}- as the transitive, irrealis inflection. These are designated as ``Class VII" roots.

Second, whenever a stem-forming suffix (applicative -\textit{an} or the exclamatory -\textit{a}, or -\textit{i}) appears, the form -\textit{en/-on} is excluded for all roots, leaving \emptyset{}- as the exponent of the transitive, irrealis inflection. Thus one could say that the stem-forming suffixes form a Class VII stem from a Class VI root.

Third, the minority class consists of two subclasses, Class VIa and Class VIIb. Class VIIa roots may or may not take an applicative suffix in a grammatically transitive frame, while Class VIIb roots always take an applicative suffix in a grammatically transitive frame. Semantically, Class VIIa consists mostly of verbs of transfer, for example the first bolded verb in \REF{bkm:Ref119939353}, while Class VIIb consists of verbs that always take an applicative suffix in their basic, transitive form (\ref{ex:yourthings} and \ref{ex:yourschooling}). They may occur without the applicative, but only in a grammatically intransitive frame (i.e., detransitive constructions, see example \ref{bkm:Ref447101267} further below, and \chapref{chap:voice}, \sectref{sec:grammaticalrelations}). Semantically, Class VIIb verbs tend strongly to describe activities that involve superficial, incomplete, or invisible effect on the Patient.    

Finally, for Class VI roots, the form -\textit{on} occurs whenever the last vowel in the stem is /u/. Otherwise the default allomorph -\textit{en} occurs:

\ea
\label{bkm:Ref119939353}
… daw  pangallo  \textbf{batang}  no  kamuti  an  na  agi  asod  naan ta  kaldiro  daw  tapos  \textbf{lutuon}. \\\smallskip
 \gll … daw  pang-tallo  \emptyset{}-\textbf{batang}  no  kamuti  an  na  agi  asod  naan ta  kaldiro  daw  tapos  \textbf{luto-en}. \\
{} and  \textsc{ord}-three  \textsc{t.ir}-put  2\textsc{s.erg}  cassava  \textsc{def.m}  \textsc{lk}  pass  pound  \textsc{spat.def}
\textsc{nabs}  pot  and  then  cook-\textsc{t.ir} \\
\glt `…and third, put the cassava that has been pounded in the pot and then cook (it).’ [BCWE-T-07 2.9]
\z
\ea
\label{bkm:Ref119939420}
Uļa  ki  nang  lugay  dya  daw  \textbf{lubbasen}  ta nang  bayo  ta  i  aged  dili  mabasa ... \\\smallskip
 \gll Uļa  ki  nang  lugay  dya  daw  \textbf{lubbas-en}  ta nang  bayo  ta  i  aged  dili  ma-basa{}... \\
\textsc{neg.r} 1\textsc{p.incl.abs}  only/just  long.time  \textsc{d}4\textsc{loc}  and  undress-\textsc{t.ir}  1\textsc{p.incl.erg}
only/just  clothes  1\textsc{p.incl.gen}  \textsc{def.n}  so.that  \textsc{neg.ir}  \textsc{a.hap.ir}-wet \\
\glt `We won’t be long there and we will just take off our clothes so that (they) will not be wet.’ [CBWN-C-25 4.8]
\z
\ea
\label{ex:yourthings}
Dili ka maglibeg tak \textbf{ambligan} ko gid imo na gamit di… \\\smallskip
 \gll Dili	ka	mag-libeg	tak	\emptyset{}-\textbf{amblig-an}	ko	gid	imo	na	gamit	di… \\
\textsc{neg.ir}	2\textsc{s.abs}	\textsc{i.ir}-worry	because	\textsc{t.ir}–take.care-\textsc{apl}	1\textsc{s.erg}	\textsc{int}	2\textsc{s.gen}	\textsc{lk}	use/thing	\textsc{d1loc} \\
\glt `Do not worry because I will really take care of your things here….' [BCWL-C-03 6.2] \\\smallskip

* … \textbf{amblig/ambligen} ko gid imo na gamit di …
\z

\largerpage[2]
\ea
\label{ex:yourschooling}
Magpakabeet kaw pirmi daw mangamuyo ta Dyos na \textbf{tabangan} kaw din ta inyo na pag-iskwela. \\\smallskip
 \gll Mag-pa-ka-beet	kaw	pirmi	daw	ma-ngamuyo	ta	Dyos	na	\emptyset{}-\textbf{tabang-an} kaw	din	ta	inyo	na	pag-iskwila. \\
\textsc{i.ir}-\textsc{caus}-\textsc{nr}-behaved/kind	2\textsc{p.abs}	always	and	\textsc{a.hap.ir}-pray	\textsc{nabs}	God	\textsc{lk}	\textsc{t.ir}-help-\textsc{apl}
2\textsc{p.abs}	3\textsc{s.erg}	\textsc{nabs}	2\textsc{p.gen}	\textsc{lk}	\textsc{nr.act}-school \\
\glt ‘Always make yourself behaved/kind and pray to God that he will help you in your schooling.’ [ICWL-T-05 6.1] \\\smallskip

* … \textbf{tabang/tabangen} kaw din ta inyo na pag-iskwela.
\z\clearpage

As mentioned above, for Class VI roots, the applicative -\textit{an} overrides -\textit{en/-on}, thus leaving “zero” (a conspicuous absence of affixation) as the indicator of the transitive, irrealis inflection (see \chapref{chap:stemformingprocesses}, \sectref{sec:applicative-an} for further discussion). Thus the applicative suffix forms a stem that belongs to Class VII. This makes some semantic sense since the applicative indicates a superficially, invisibly or partially affected Undergoer (see \chapref{chap:voice}, \sectref{sec:applicativevoice}), and most verbs in Class VII also involve less than fully affected Patients as Undergoers (see \tabref{tab:rootsthattaketwodistincttransitiveirrealisinflections-1} in \chapref{chap:verbclasses-1}). Example \REF{bkm:Ref447101267} illustrates the Class VII root \textit{akid}, ‘to serve food’, in the basic transitive form (realis modality), and in the irrealis detransitive form:

\ea
\label{bkm:Ref447101267}
Ti  \textbf{paakiran}  din  en  bataan  ya  tak  ambaļ ta  manakem  ya,  “\textbf{Kiran}  no  bataan  no. \\\smallskip
 \gll Ti  \textbf{pa-akid-an}  din  en  bata-an  ya  tak  ambaļ ta  manakem  ya,  “\emptyset{}-\textbf{akid-an}  no  bata-an  no. \\
so  \textsc{t.r}-serve.food-\textsc{apl}  3\textsc{s.erg}  \textsc{cm}  child-\textsc{nr}  \textsc{def.f}  because  say
\textsc{nabs}  older  \textsc{def.f}  \textsc{t.ir}-serve.food-\textsc{apl}  2\textsc{s.erg}  child-\textsc{nr}  2\textsc{s.gen} \\
\glt `So she served food to the children because the older woman said, “Serve food to your children.”' [AION-C-01 7.7]
\z

Example \REF{bkm:Ref499555402} illustrates the root \textit{legem} ‘to dye’ that normally takes -\textit{en} as its transitive, irrealis form, but in this example it occurs with the applicative suffix, which overrides -\textit{en}, thus leaving \emptyset{}- as the exponent of the transitive, irrealis inflection:

\ea
\label{bkm:Ref499555402}
Daw  gusto  no  na  ikam  i  na  paļaļa  no  \textbf{betangan} legem,  pwidi  no  \textbf{legeman}  ta  minog,  grin  daw  violet. \\\smallskip
 \gll Daw  gusto  no  na  ikam  i  na  pa-ļaļa  no  \emptyset{}\textbf{-betang-an} legem,\footnotemark{}  pwidi  no  \emptyset{}\textbf{-legem-an}  ta  minog,  grin  daw  violet. \\
if/when  want  2\textsc{s.erg}  \textsc{lk}  mat  \textsc{def.n}  \textsc{lk}  \textsc{t.r}-weave  2\textsc{s.erg}  \textsc{t.ir}-put-\textsc{apl}
dye  can  2\textsc{s.erg}  \textsc{t.ir}-dye-\textsc{apl}  \textsc{nabs}  red  green  and  violet \\
\footnotetext{In this example, the non-absolutive \textit{ta} has dropped out between \textit{betangan} and \textit{legem}. This sometimes happens when a nominal has lost its status as absolutive as a result of the applicative derivation. In Relational Grammar terms \citep{perlmutter1986}, this may be a characteristic of “chômeurs.” Speakers agree that \textit{ta} could be added here, but that it sounds more natural to leave it out.}
\glt `If you want to put dye on the mat you wove, you can dye (it) with red, green and violet.’ [DBOP-C-12 1.11]
\z

Note that -\textit{an} itself cannot be the marker of the transitive, irrealis inflection because it also occurs in realis contexts (see \ref{bkm:Ref120012965} and many other examples thoughout this grammar). On the other hand, “zero” as a marker of transitive, irrealis inflection is independently attested for a large lexical class of roots (Class VII, see \chapref{chap:verbclasses-1}, \tabref{tab:rootsthattaketwodistincttransitiveirrealisinflections-1}). 

The following is an example from the corpus of the culminative use of transitive irrealis inflections:

\ea
Lugar  na  gatago  kanen  i  nakita  din  iya  na  maguļang  na  galebbeng  naan  ta  bļawan  daw  padakep  ta mga  ittaw maguļang  ya  daw  \textbf{gapuson}  daw  \textbf{patayen}. \\\smallskip
 \gll Lugar  na  ga-tago  kanen  i  na-kita  din  iya  na  maguļang  na  ga-lebbeng  naan  ta  bļawan  daw  pa-dakep  ta mga  ittaw maguļang  ya  daw  \textbf{gapos-en}  daw  \textbf{patay-en}. \\
then  \textsc{comp}  \textsc{i.r}-hide  3\textsc{s.abs}  \textsc{def.n}  \textsc{a.hap.r}-see  3\textsc{s.erg} 3\textsc{s.gen}  \textsc{lk} elder.sibling
\textsc{lk}  \textsc{i.r}{}-bury  \textsc{spat.def}  \textsc{nabs}  gold  and  \textsc{t.r}-capture  \textsc{nabs} \textsc{pl} person elder.brother  \textsc{def.f} and  tie-\textsc{t.ir} and  kill-\textsc{t.ir} \\
\glt `Then when he hid, he saw his older-brother who was-being-buried in gold and the people captured the older brother and tied (him) up and killed him.’ [CBWN-C-22 12.4]
\z

In addition to the culmination of a series of events, this example also exhibits high tension in this part of the story. When there is higher tension or climax, the culminative usage of irrealis mood tends to be used, as well as fewer nouns and pronouns. A full discourse study is needed to corroborate this observation.
\is{transitive irrealis inflection|)}

\subsection{Transitive deontic \textit{i-}}
\label{sec:transitivedeontic}\is{transitive deontic inflection|(}

The prefix \textit{i}{}- usually expresses or reports an admonition, telling someone what they should, must or ought to do. We see this \textit{deontic}\is{deontic modality} meaning as related to the ``conveyance", ``instrumental" or ``benefactive" voice meanings that have been ascribed to cognate \textit{i-} prefixes in other Philippine languages (see, e.g., \citealt[79]{wolff1973} on \isi{Cebuano}). Briefly, a beneficiary, an instrument and a requirement all participate in the \textit{instigation} of an action in some way. A beneficiary is a motivation for the action, an instrument assists or enables the actor to accomplish the action, and a requirement or necessity acts as an unseen force spurring the actor to action. Our hypothesis is that \textit{i}- evokes an image in which there is such an external motivating or enabling force in the discourse scene\is{scenes} depicted in the clause. \citep[344]{payneoyzon2022}, in a description of Waray, refer to this motivating force as a ``co-actant". Sometimes the co-actant can be mentioned as the absolutive of the clause, in which case \textit{i}- expresses a benefactive or instrumental applicative. In Waray, as in Kagayanen, the co-actant need not be overt at all, but remains ``behind the scenes", motivating the activity. In Kagayanen, \textit{i}- always expresses irrealis modality, and is usually understood as expressing deontic modality. We see its use as an instrumental applicative marker as a reasonable extension of this ``external enablement" meaning (see \chapref{chap:voice}, \sectref{sec:applicativevoice} for further discussion and corpus examples of -\textit{i} functioning as an instrumental applicative). 

Prefixation with \textit{i-} is possible for some Class VI verbs as displayed in \tabref{tab:rootsthattaketwodistincttransitiveirrealisinflections-1} in \chapref{chap:morphosyntacticallydefinedverbclasses} and all Class VII verbs. None of the Class VIII verbs (those that take -\textit{an} in their basic transitive form) can occur with the \textit{i}{}- prefix. For a few Class VI verbs, \textit{i}{}- functions as an instrumental or benefactive/recipient applicative, in which case deontic modality is a possible, but not necessary, implication (see \chapref{chap:voice}, \sectref{sec:applicativevoice} for a definition of applicative). The prefix \textit{i-} is not used very often, and when it does occur, the Actor is usually 2\textsuperscript{nd} person.


Examples \REF{bkm:Ref447002651} and \REF{bkm:Ref119941331} illustrate the verb \textit{beļad} ‘to dry X in the sun’. In \REF{bkm:Ref447002651} it occurs with the normal transitive irrealis suffix -\textit{en} and in example \REF{bkm:Ref119941331} the same verb occurs with \textit{i}{}- and the meaning is that it is necessary to dry the cassava after soaking it so that it does not go bad.

\ea
\label{bkm:Ref447002651}
Dayon  kamangen  sinaksak  an  na  derse  daw  \textbf{beļaren.} \\\smallskip
 \gll Dayon  kamang-en  s<in>aksak  an  na  derse  daw  \textbf{beļad-en.} \\
right.away  get-\textsc{t.ir}  <\textsc{nr.res}>chop  \textsc{def.m}  \textsc{lk}  small.\textsc{pl}  and  dry.in.sun-\textsc{t.ir} \\
\glt ‘Right away get the chopped parts that are small and dry (them in the sun).’ [JCWE-L-32 6.6]
\z
\ea
\label{bkm:Ref119941331}
Ta  katallo  na  adlaw  kinangļan  aw-asen  naan  ta silian  daw  \textbf{ibeļad}  naan  ta  bansada ta  adlaw. \\\smallskip
 \gll Ta  ka-tallo  na  adlaw  kinangļan  aw-as-en  naan  ta sili-an  daw  \textbf{i-beļad}  naan  ta  bansada ta  adlaw. \\
\textsc{nabs}  \textsc{ord}-three  \textsc{lk} sun/day  need  remove.from-\textsc{t.ir}  \textsc{spat.def}  \textsc{nabs}
change-\textsc{nr} and \textsc{t.deon}-dry.in.sun \textsc{spat.def}  {nabs}  exposed/open.place
\textsc{nabs} sun/day \\
\glt `On the third day (the cassava) needs to be removed from there (the container where the water for soaking was changed) and has to be dried in an open place in the sun.’ [DBWE-T-28 2.14]
\z

All Class VII roots (mostly verbs of transfer) may occur with the \textit{i}{}- with the same deontic meaning illustrated above. The following are some examples from the corpus. Examples \REF{bkm:Ref447005121} and \REF{bkm:Ref119941404} illustrate the verb \textit{batang} ‘to put X somewhere’. In example \REF{bkm:Ref447005121} the same root occurs in the unmarked transitive irrealis form meaning ‘will place the liver (absolutive) of the chick on the diaphragm of the person who is sick’ and in \REF{bkm:Ref119941404} it occurs with the \textit{i}{}- prefix meaning ‘I must put the thorns (absolutive) on the banana plant.’
\ea
\label{bkm:Ref447005121}
Ta,  daw  may  sabid  ittaw  an o naswang, atay  yan  ta  piyak  kamangen  ta  surano  daw  \textbf{batang} ta  ginawaan… \\\smallskip
 \gll Ta,  daw  may  sabid  ittaw  an o na-aswang,\footnotemark{} atay  yan  ta  piyak  kamang-en  ta  surano  daw  \emptyset{}-\textbf{batang} ta  ginawaan… \\
so  if/when  \textsc{ext.in}  sickness.from.spirit  person  \textsc{def.m}  or 
  \textsc{a.hap.r}-bewitch
liver  \textsc{def.m}  \textsc{nabs}  chick  get-\textsc{t.ir}  \textsc{nabs}  healer  and
\textsc{t.ir}-put \textsc{nabs}  diaphragm \\
\footnotetext{The root \textit{aswang} is the \isi{Tagalog} noun meaning ‘ghost’, or ‘evil spirit’, but it is used as an inflected verb in Kagayanen meaning ‘to bewitch/curse’. There is a Kagayanen word, \textit{maļbaļ}, that is similar in meaning to \textit{aswang}, but this root cannot be used as an inflected verb.}
\glt `So, if a person has a sickness from the spirit or is bewitched, the liver of a chick, the healer will get and put (it) on the diaphragm (of the sick person)…’ [CBWE-T-07 5.1]
\z
\ea
\label{bkm:Ref119941404}
Mamang  a  ta  mga  tellek  daw  \textbf{ibatang}  ko  naan  ta lawa  ta  saging  aged  dili  ka  kapanaog. \\\smallskip
 \gll M-kamang  a  ta  mga  tellek  daw  \textbf{ibatang}  ko  naan  ta lawa  ta  saging  aged  dili  ka  ka-panaog. \\
\textsc{i.v.ir}-get  1\textsc{s.abs}  \textsc{nabs}  \textsc{pl}  thorn  and  \textsc{t.deon}-put  1\textsc{s.erg}  \textsc{spat.def} \textsc{nabs} body  \textsc{nabs}  banana  so.that  \textsc{neg.ir}  2\textsc{s.abs}  \textsc{i.hap}-come.down \\
\glt `I need to get some thorns and put (them) on the trunk of the banana plant so that you can’t come down.’ [CBWN-C-16 8.9]
\z

The implication in \REF{bkm:Ref119941404} is that the speaker is compelled by the need to prevent the addressee from climbing down the banana plant.

Examples \REF{bkm:Ref460145354} through \REF{bkm:Ref447269492} illustrate verbs that can occur with \textit{i-} functioning in an applicative frame, in which case a deontic sense is possible, but not necessary. This function is not very productive, and only occurs with certain verbs that take -\textit{en} as their basic, transitive irrealis form-mostly verbs that involve some special instrument. For example, the verb \textit{asod} ‘to pound grain’ implies a specific instrument, \textit{aļļo} ‘pestle’. For this and similar verbs, \textit{i-} selects the instrument as the absolutive argument (instrumental voice-see \chapref{chap:voice}, \sectref{sec:instrumentalvoice}), though with \textit{{}-en} and the other transitive affixes, this verb selects the theme as the absolutive:

\ea
\label{bkm:Ref460145354}Instrumental voice (applicative) \\
\textbf{Iasod}  ko  aļļo  i  na  buat  din. \\\smallskip
 \gll \textbf{I-asod}  ko  aļļo  i  na  buat  din. \\
\textsc{t.deon}-pound  1\textsc{s.erg}  pestle  \textsc{def.n}  \textsc{lk} make  3\textsc{s.erg} \\
\glt ‘I will pound (something) with the pestle that s/he made.’
\z

Similarly, the verb \textit{akid} ‘to serve food’ strongly implies a specific serving implement, normally a spoon:

\ea
\textbf{Iakid}  no  luag  an  na  bag-o  ta  kan-en. \\\smallskip
 \gll \textbf{I-akid}  no  luag  an  na  bag-o  ta  kan-en. \\
\textsc{t.deon.apl}-serve.food  2\textsc{s.erg}  serving.spoon  \textsc{def.m}  \textsc{lk}  new  \textsc{nabs}  cooked.rice \\
\glt ‘Use the new serving spoon to serve rice.’
\z

The verb \textit{akid} also describes transfer (‘serve X to Y’). Therefore, for this verb, \textit{i}{}- may also select a beneficiary/recipient as the absolutive argument, as illustrated in \REF{bkm:Ref447269492}:
\ea
\label{bkm:Ref447269492}
Beneficiary/recipient voice applicative \\
\textbf{Iakid}  a  no  ta  kan-en. \\\smallskip
 \gll \textbf{I-akid}  a  no  ta  kan-en. \\
\textsc{t.deon.apl}-serve.food  1\textsc{s.abs}  2\textsc{s.erg}  \textsc{nabs}  rice \\
\glt ‘Serve some rice for/to me.’
\z

For verbs that allow both types of applicative (such as \textit{akid}), the instrumental versus benefactive meanings can always be distinguished-instruments are inanimate tools, while beneficiaries or recipients are always animate and usually human. There are additional verbs that can occur with the \textit{i}{}- instrumental and beneficiary applicative meanings, other than Class VI and VII listed in \chapref{chap:verbclasses-1}, \tabref{tab:rootsthattaketwodistincttransitiveirrealisinflections-1}. Furthermore, some speakers use \textit{i}{}- with more verbs than other speakers do, so there is variation in this system. However, the functions of \textit{i}{}- are constant: It only occurs in grammatically transitive constructions, and always expresses deontic modality. It can at times occur in an applicative frame, if the constructional context allows it.

The following are a few examples of the \textit{i-} transitive deontic applicative construction in context:

\ea
Instrumental voice applicative: \\
\textbf{Igeļet}  no  bari  i  naan  ta  karni. \\\smallskip
 \gll \textbf{I-geļet}  no  bari  i  naan  ta  karni. \\
\textsc{t.deon.apl}-cut  2\textsc{s.erg}  knife  \textsc{def.m}  \textsc{spat.def}  \textsc{nabs}  meat \\
\glt ‘Use the knife to cut meat.’ (*‘Cut some meat for the knife.’)
\z
\ea
\textbf{Ikidlas}  no  anay  kayan  an  na  ubra  ko. \\\smallskip
 \gll \textbf{I-kidlas}  no  anay  kayan  an  na  ubra  ko. \\
\textsc{t.deon.apl}-cut.in.strips  2\textsc{s.erg}  first/for.a.while  cutting.guide  \textsc{def.m}
\textsc{lk}  make  1\textsc{s.erg} \\
\glt ‘Use the guide that I made to cut long narrow strips of leaves (usually pandan or buli leaves).’
\z

\ea
Benefactive voice applicative: \\
\textbf{Igeļet}  a  no  ta  karni. \\\smallskip
 \gll \textbf{I-geļet}  a  no  ta  karni. \\
\textsc{t.deon.apl}-cut  1\textsc{s.abs}  2\textsc{s.erg}  \textsc{nabs}  meat \\
\glt ‘Cut some meat for me.’ (*‘Use me to cut some meat.’)
\z
\ea
\textbf{Ikidlas}  a  no  anay  ta  pandan. \\\smallskip
 \gll \textbf{I-kidlas}  a  no  anay  ta  pandan. \\
\textsc{t.deon.apl}-cut.in.strips  1\textsc{s.abs}  2\textsc{s.erg} first/for.a.while  \textsc{nabs} pandan.leaves \\
\glt ‘Cut for me some pandan leaves in long narrow strips.’
\z

There are only two examples in the text corpus of \textit{i}{}- functioning as an applicative, and both of these select instruments as the absolutive:

\ea
Daw  may  kwarta,  \textbf{ipalit}  ta  sabon,  puspuro,  agas, daw  tanan  na  mga  gamit  ta  baļay. \\\smallskip
 \gll Daw  may  kwarta,  \textbf{i-palit}  ta  sabon,  puspuro,  agas, daw  tanan  na  mga  gamit  ta  baļay. \\
if/when  \textsc{ext.in}  money  \textsc{t.deon.apl}-buy  \textsc{nabs}  soap  matches  kerosene
and  all  \textsc{lk}  \textsc{pl}  use  \textsc{nabs}  house \\
\glt `If there is money, (it) should be used to buy soap, matches, kerosene, and all that is used in the house.’ [NWE-L-01 2.9]
\z
\ea
\label{bkm:Ref119940330}
Gapasalamat  a  ta  ate  na  Dios  tak  uļa a  natabo  ta  laod  parti  ta  ake  na  pagpangita ta  \textbf{isagod}  ko  ta  ake  na  pamilya. \\\smallskip
 \gll Ga-pa-salamat  a  ta  ate  na  Dios  tak  uļa a  na-tabo  ta  laod  parti  ta  ake  na  pag-pangita ta  \textbf{i-sagod}  ko  ta  ake  na  pamilya. \\
\textsc{i.r}-\textsc{caus}-thank  1\textsc{s.abs}  \textsc{nabs}  1\textsc{p.incl.gen}  \textsc{lk}  God  because  \textsc{neg.r}
\textsc{inj}  \textsc{a.hap.r}-happen \textsc{nabs}  deep.sea  concerning  \textsc{nabs}  1\textsc{s.gen}  \textsc{lk}  \textsc{nr.act-}search
\textsc{nabs}  \textsc{t.deon.apl}-take.care  1\textsc{s.erg}  \textsc{nabs}  1\textsc{s.gen}  \textsc{lk}  family \\
\glt `I give thanks to our God because nothing happened in the deep sea concerning my searching for something I need to take care of my family.’ (The speaker was fishing so that his family would have food.) [MCWN-L-01 2.21]
\z

Furthermore, in example \REF{bkm:Ref119940330} \textit{i-} is functioning in a nominalized dependent clause, rather than the predicate in a main clause (as indicated by the prenominal case marker \textit{ta} occurring immediately before the verb). These are the only examples in the corpus of \textit{i-} occuring in an applicative frame.
\is{transitive deontic inflection|)}

\subsection{Ambitransitive happenstantial realis \textit{na}{}-}
\label{bkm:Ref107902303}\is{ambitransitive happenstantial realis inflection|(}
\label{sec:ambitransitivehappenstantialrealis}
The term “happenstantial” is sometimes used in descriptions of Philippine languages for a verbal category that describes non-deliberate, abilitative or coincidental situations. This is generally accurate for Kagayanen. However, it must be kept in mind that for some non-volitional intransitive verbs, and transitive experiential verbs (verbs of non-volitional perception, emotion and cognition) the happenstantial prefixes are the basic narrative forms, and as such can be understood as perfective in aspect. For other verb classes, happenstantial modality implies happenstance, for example, \textit{they happened to meet}, \textit{he happened to pass by}; ability, \textit{we were able/managed to hold on}; opportunitive, \textit{he} \textit{got to listen to a story}; or perfect aspect, \textit{I have been to Manila}, \textit{he has used up all the rice}. The extension of happenstantial modality to perfect aspect may be motivated by the fact that perfect aspect is always non-volitional, since it asserts the state that results from some (volitional or non-volitional) event, and not the event itself. An Actor may carry out an event on purpose, but the resultant state is just a fact of the new condition of the world. This is why English expressions like “She wanted to have seen that movie”, or “I intend to have been to Manila” are pragmaticaly odd at best. Having seen a particular movie is just the expression of a new state of the world that is the result of an event of seeing the movie. It is not something that one can volitionally intend or want to occur.

The happenstantial prefixes \textit{na-} and \textit{ma-} may occur in grammatically transitive or intransitive frames, but in all cases, except the special hypothetical/polite usage of \textit{ma}{}- described in \sectref{bkm:Ref125306234}, the situation is understood as not under the control of the absolutive argument. For this reason, we say these prefixes are “Undergoer oriented” – the absolutive is always an Undergoer. In order to detransitivize a transitive construction in happenstantial modality, the forms \textit{naka-} or \textit{maka}{}- must be employed (see \sectref{bkm:Ref447381050} and \sectref{bkm:Ref122761448}). The following examples illustrate this pattern with the realis, happenstantial \textit{na-}:

\ea
    \ea    Transitive frame  (\textit{ta bata an} = ergative) \\
     Nasugat  a  ta  bata  an. \\\smallskip
 \gll Na-sugat  a  ta  bata  an. \\
     \textsc{a.hap.r}-meet  1\textsc{s.abs}  \textsc{nabs}  child  \textsc{def.m} \\
    \glt ‘The child happened to meet me.’ \\\smallskip
    (Neither ‘*I happened to have met the child’, nor ‘*I hypothetically met the child.’)
    \ex
    Detransitive frame (\textit{ta bata an} = downplayed/demoted Undergoer) \\
    Nakasugat  a  ta  bata  an. \\\smallskip
\gll Naka-sugat  a  ta  bata  an. \\
      \textsc{i.hap.r}-meet  1\textsc{s.abs}  \textsc{nabs}  child  \textsc{def.m} \\
    \glt  ‘I happened to meet the child.’
    \ex
    Intransitive frame (\textit{ta kabaw an} = Oblique) \\
    Nuļog  a  ta  kabaw  an. \\\smallskip
\gll Na-uļog  a  ta  kabaw  an. \\
    \textsc{a.hap.r}-fall  1\textsc{s.abs}  \textsc{nabs}  carabao  \textsc{def.m} \\
    \glt ‘I fell off the carabao (accidentally).’
    \z
\z

Example \REF{bkm:Ref392922279} illustrates the happenstantial realis prefix \textit{na-} in its basic usage on a non-volitional intransitive verb describing an event on the story line of a narrative:

\ea
\label{bkm:Ref392922279} \label{bkm:Ref447862443}
\textbf{Nuļog}  a  ta  kabaw  tak  gadļagan  tudo  kabaw  an. \\\smallskip
 \gll \textbf{Na-uļog}  a  ta  kabaw  tak  ga-dļagan  tudo  kabaw  an. \\
\textsc{a.hap.r}-fall  1\textsc{s.abs}  \textsc{nabs}  carabao  because  \textsc{i.r}-run  intense  carabao  \textsc{def.m.}\textsc{} \\
\glt ‘I fell off the carabao because the carabao was running hard.’ [RCON-L-02 3.6]
\z

Examples \REF{bkm:Ref448318853} and \REF{bkm:Ref120012965} illustrate \textit{na}{}- in its usage on transitive experiential verbs of perception and cognition on the story line of a narrative:

\ea
\label{bkm:Ref448318853}
\textbf{Nakita}  din  ake  na  mangngod  na  nalemmes. \\\smallskip
 \gll \textbf{Na-kita}  din  ake  na  mangngod  na  na-lemmes. \\
\textsc{a.hap.r}-see  3\textsc{s.erg}  1\textsc{s.gen}  \textsc{lk}  younger.sibling  \textsc{lk}   \textsc{a.hap.r}-drown \\
\glt ‘He saw my younger sibling who drowned.’ [LCWN-T-01 2.8]
\z
\ea
\label{bkm:Ref120012965}
\textbf{Nademdeman}  din  arey  din  Lisga. \\\smallskip
 \gll \textbf{Na-demdem-an}  din  arey  din  Lisga.\footnotemark{} \\
\textsc{a.hap.r}-remember-\textsc{apl}  3\textsc{s.erg}  friend  3\textsc{s.gen}  fire.ant \\
\footnotetext{Usually this word is pronounced and spelled \textit{lasga}, but this author has written \textit{Lisga}. This may be an error, code mixing, or idiolectal variation.}
\glt ‘S/he remembered his friend Fire Ant.’ [VBWN-T-01 3.5]
\z

Example \REF{bkm:Ref508296047} illustrates an emotion verb in its default usage:

\ea
\label{bkm:Ref508296047}
Kanen  i  \textbf{nadlek}  gid  tak  uļa  gid  duma. \\\smallskip
 \gll Kanen  i  \textbf{na-adlek}  gid  tak  uļa  gid  duma. \\
3\textsc{s.abs}  \textsc{def.n}  \textsc{a.hap.r}-aftaid  \textsc{int} because  \textsc{neg.r}  \textsc{int}  companion \\
\glt ‘He was really afraid because there was no companion.’ [MBON-T-05 3.6]
\z

In example \REF{bkm:Ref396387524} the happenstantial \textit{na}{}- expresses a realized dynamic event that just happened without anyone intentionally trying to make it happen. This sentence could also express a realized capability, ‘they were able to (intentionally) meet’, but the discourse context makes it clear that the intended meaning is happenstance. It is also not perfect aspect because it asserts the event rather than a state resulting from an earlier event:

\ea
\label{bkm:Ref396387524}
\textbf{Nasugat}  danen  bata  i  ta  iran   na  maistro  na  ngaran din Pedro  na  galin  Puerto… \\\smallskip
 \gll \textbf{Na-sugat}  danen  bata  i  ta  iran   na  maistro  na  ngaran din Pedro  na  ga-alin  Puerto… \\
\textsc{a.hap.r}-meet  3\textsc{p.erg}  child  \textsc{def.n}  \textsc{nabs} 3\textsc{p.gen}  \textsc{lk}  teacher  \textsc{lk}  name
 3\textsc{s.gen} Pedro \textsc{lk}  \textsc{i.r}-from  Puerto \\
\glt `They happened to meet the son of their teacher whose name is Pedro who came from Puerto.’ [JCWN-T-20 21.2]
\z

In examples \REF{bkm:Ref396387548} and \REF{bkm:Ref119941722} \textit{na}{}- expresses a realized capability: ‘was able to.’ In this case, happenstance is not likely since “catching the ball” and “carrying the child” are not situations that occur without intention. And again, they are not perfect in aspect since they assert events rather than resultant states:

\ea
\label{bkm:Ref396387548}
Paglabyog  ta  pangka  ta  bula  ya,  \textbf{nasaļap}  dayon  ta  princesa. \\\smallskip
 \gll Pag-labyog  ta  pangka  ta  bula  ya,  \textbf{na-saļap}  dayon  ta  princesa. \\
\textsc{nr.act}-throw  \textsc{nabs} frog  \textsc{nabs} ball  \textsc{def.f}  \textsc{a.hap.r}-catch  immediately \textsc{nabs} princess \\
\glt `When the frog threw the ball (lit. the frog’s throwing of the ball), the princess immediately was able to catch it.’ [CBWN-C-17 4.1]
\z

\ea
\label{bkm:Ref119941722}
Tapos  bata  ya  \textbf{nadaļa}  gid  ta  tallo  ya  na  ayam… \\\smallskip
 \gll Tapos  bata  ya  \textbf{na-daļa}  gid  ta  tallo  ya  na  ayam… \\
then  child  \textsc{def.f}  \textsc{a.hap.r}-take/carry  \textsc{int}  \textsc{nabs}  three  \textsc{def.f}  \textsc{lk}  dog \\
\glt ‘Then as for the child the three dogs were able to carry (him) away.’ [CBWN-C-20 7.1]
\z

The extension of happenstantial to abilitative modality makes sense in that the ability to do something is non-volitional, even though the act of doing it may be volitional.

Example \REF{bkm:Ref392922782} illustrates \textit{na-} in a dynamic transitive construction expressing perfect aspect. Note that the Actor, the parrot fish, is marked as non-absolutive. This indicates that the absolutive argument is the Patient, the butterfly fish’s mouth. Happenstantial is used in this case to indicate a state that resulted from an event that happened earlier in the order of events on the storyline, and therefore may be understood as perfect in aspect:

\ea
\label{bkm:Ref392922782}
Kalibangbang  an  a  daw  makita  ta  man  en  anduni ļangkaw  en  baba  din  an  tak  \textbf{nakemes}  ta  tuļ{}-ungan. \\\smallskip
 \gll Kalibangbang  an  a  daw  ma-kita  ta  man  en  anduni ļangkaw  en  baba  din  an  tak  \textbf{na-kemes}  ta  tuļ{}-ungan. \\
butterfly  \textsc{def.m}  \textsc{ctr} if/when  \textsc{a.hap.ir}-see  1\textsc{p.incl.erg}  too  \textsc{cm} now/today
long  \textsc{cm}  mouth  3\textsc{s.gen}  \textsc{def.m}  because  \textsc{a.hap.r}-squeeze  \textsc{nabs} parrot.fish \\
\glt `The butterfly fish if we could see (it/them) indeed now/today, its mouth is long because the parrot fish has squeezed (it).’ [JCON-L-07 21.2]
\is{ambitransitive happenstantial realis inflection|)}
\z

\subsection{Ambitransitive happenstantial irrealis \textit{ma}{}-}
\label{sec:ambitransitivehappenstantialirrealis}\is{ambitransitive happenstantial irrealis inflection|(}

As with the happenstantial realis prefix \textit{na}{}-, the prefix \textit{ma}{}- occurs in intransitive non-volitional constructions and in certain transitive constructions. In its primary usage it expresses potential non-volitional situations. Like the realis counterpart, \textit{na}{}-, \textit{ma}{}- normally occurs with Patient-preserving intransitive verbs. In other words, the absolutive argument is a Patient/Undergoer rather than an Actor. If the absolutive is an Actor, either the form \textit{maka}{}- occurs, or the construction is understood as the hypothetical/polite usage described in \sectref{bkm:Ref125306276}. The following examples illustrate the usages of \textit{ma}{}- versus \textit{maka}{}-: 

\ea
    \ea Transitive frame (\textit{ta bata an} = ergative) \\ 
    Masugat  a  ta  bata  an. \\\smallskip
\gll Ma-sugat  a  ta  bata  an. \\
    \textsc{a.hap.ir}-meet  1\textsc{s.abs}  \textsc{nabs}  child  \textsc{def.m} \\
    \glt ‘The child will possibly meet me.’ Or ‘The child will be able to meet me.’    (Not ‘*I will possibly/be able to meet the child’)

    \ex Detransitive frame (\textit{ta bata an} = downplayed/demoted Undergoer) \\
    Makasugat  a  ta  bata  an. \\\smallskip
\gll Maka-sugat  a  ta  bata  an. \\
    \textsc{i.hap.ir}-meet  1\textsc{s.abs}  \textsc{nabs}  child  \textsc{def.m} \\
    \glt ‘I will possibly/be able to meet the child.’
    
    \ex Intransitive frame (\textit{ta kabaw an} = Oblique) \\
    Muļog  kanen  kani  ta  kabaw  an. \\\smallskip
\gll Ma-uļog  kanen  kani  ta  kabaw  an. \\
    \textsc{a.hap.ir}-fall  3\textsc{s.abs} later  \textsc{nabs}  carabao  \textsc{def.m} \\
    \glt ‘S/he will possibly fall off the carabao (accidentally).’
    \ex
    Mawigit  kani  kwarta  no  an  ta  bulsa  no. \\\smallskip
\gll Ma-wigit  kani  kwarta  no  an  ta  bulsa  no. \\
    \textsc{a.hap.ir}-fall.unnoticed  later  money  2\textsc{s.gen}  \textsc{def.m}  \textsc{nabs}  pocket  2\textsc{s.gen}\\
    \glt ‘Your money will later fall unnoticed from your pocket.’ 
    \newpage
    \ex Detransitive frame, polite usage (see \sectref{bkm:Ref125306312}) \\
    Masugat  ka  ta  bata  an. \\\smallskip
\gll Ma-sugat  ka  ta  bata  an. \\
    \textsc{a.hap.ir}-meet  2\textsc{s.abs}  \textsc{nabs}  child  \textsc{def.m} \\
    \glt ‘You should meet the child (polite).’    
    \z
\z

Examples \REF{bkm:Ref440036963} and \REF{bkm:Ref119941820} illustrate intransitive clauses with \textit{ma}{}- in its primary usage from the corpus:

\ea
\label{bkm:Ref440036963}
\textbf{Marwad}  ate  na  ambaļ  na  Kagayanen  ta  pila  na  adlaw. \\\smallskip
 \gll \textbf{Ma-duwad}  ate  na  ambaļ  na  Kagayanen  ta  pila  na  adlaw. \\
\textsc{a.hap.ir}-lose  1\textsc{p.incl.gen}  \textsc{lk}  say  \textsc{lk}  Kagayanen  \textsc{nabs}  few  \textsc{lk}  sun/day \\
\glt ‘Our Kagayanen language \textbf{will} \textbf{be} \textbf{lost} in a short time (lit. in a few days).’ [TTOB-J-01 8.3]
\z
\ea
\label{bkm:Ref119941820}
Daw  sikad  man  an  inog  nangka  i,  lugay  \textbf{makamang} iya  an  na  bao. \\\smallskip
 \gll Daw  sikad  man  an  inog  nangka  i,  lugay  \textbf{ma-kamang} iya  an  na  bao. \\
if/when  very  \textsc{emph}  \textsc{att}  ripe  jackfruit  \textsc{def.n}  long.time  \textsc{a.hap.ir}-remove 3\textsc{s.gen}  \textsc{def.m}  \textsc{lk}  odor \\
\glt `When the jackfruit is really ripe, (it takes) a long time for its odor \textbf{to} \textbf{dissipate}.’ [VPWE-T-02 2.2]
\z

\hspace*{-.7pt}Examples \REF{bkm:Ref119353744} and \REF{bkm:Ref119353747} illustrate semantically transitive\is{semantic transitivity}\is{transitivity!semantic} experiential verbs with \textit{ma}{}- in its primary usage:

\ea
\label{bkm:Ref119353744}
Tama  man  \textbf{makita}  no  na  mga  klasi  mga  ayep  naan  ta  patag. \textbf{Makita}  no  baka,  kabayo  daw  kanding. \\\smallskip
 \gll Tama  man  \textbf{ma-kita}  no  na  mga  klasi  mga  ayep  naan  ta  patag. \textbf{Ma-kita}  no  baka,  kabayo  daw  kanding. \\
many  too  \textsc{a.hap.ir}-see  2\textsc{s.erg}  \textsc{lk}  \textsc{pl}  kinds  \textsc{pl}  animal  \textsc{spat.def}  \textsc{nabs} plain
\textsc{a.hap.ir}-see  2\textsc{s.erg}  cow  horse  and  goat \\
\glt `The kinds of animals that you \textbf{will/might/can} \textbf{also} \textbf{see} on the plain are many. You \textbf{will/might/can} \textbf{see} cows, horses and goats.’ [VAWN-T-15 4.6-7]
\z

\newpage
\ea
\label{bkm:Ref119353747}
Yi  na  suļat  para  nang  kaon  agod  \textbf{mademdeman}  no man  yaken  i. \\\smallskip
 \gll Yi  na  suļat  para  nang  kaon  agod  \textbf{ma-demdem-an}  no man  yaken  i. \\
\textsc{d}1\textsc{abs}  \textsc{lk}  letter  for  only/just  2\textsc{s.abs}  so.that   \textsc{a.hap.ir}-remember-\textsc{apl}  2\textsc{s.erg}
too  1\textsc{s.abs}  \textsc{def.n} \\
\glt `This letter is just for you so that you \textbf{will} \textbf{remember} me too.’ [ATWL-J01 5.1]
\z

Examples \REF{bkm:Ref448320715} and \REF{bkm:Ref123671980} illustrate the happenstantial irrealis prefix \textit{ma-} expressing a potential capability: `will be able to' or `can':

\ea
\label{bkm:Ref448320715}
Daw  \textbf{madakep}  ta  kaw,  tekteken  ta  kaw. \\\smallskip
 \gll Daw  \textbf{ma-dakep}  ta  kaw,  tektek-en  ta  kaw. \\
if/when  \textsc{a.hap.ir}-catch  1\textsc{s.erg} 2\textsc{s.abs}  chop.fine-\textsc{t.ir} 1\textsc{s.erg}  2\textsc{s.abs} \\
\glt ‘If \textbf{I} \textbf{am} \textbf{able} \textbf{to} \textbf{catch} \textbf{you}, I will chop you up fine.’ [CBWN-C-16 9.17]
\z
\ea
\label{bkm:Ref123671980}
Manang,  daw  pwidi  nang,  inta  atagan  a  no  ta address  Maria  para  \textbf{masuļatan}  ko  kanen  ya. \\\smallskip
 \gll Manang,  daw  pwidi  nang,  inta  \emptyset{}-atag-an  a  no  ta address  Maria  para  \textbf{ma-suļat-an}  ko  kanen  ya. \\
older.sister  if/when  can  only  \textsc{opt}  \textsc{t.ir}-give-\textsc{apl}  1\textsc{s.abs} 2\textsc{s.erg}  \textsc{nabs}
address  Maria  for \textsc{a.hap.ir}-write-\textsc{apl}  1\textsc{s.erg}  3\textsc{s.abs}  \textsc{def.f} \\
\glt `Older sister, if/when just possible, give me the address of Maria so that \textbf{I} \textbf{can} \textbf{write} to her.’ [BCWL-C-01 3.20]
\is{ambitransitive happenstantial irrealis inflection|)}
\z

\subsection{Intransitive happenstantial hypothetical/polite \textit{ma}{}-}
\label{bkm:Ref125306312}\label{bkm:Ref125306276}\label{bkm:Ref125306234}\is{intransitive happenstantial hypothetical/polite inflection|(}

Examples \REF{bkm:Ref119942002} through \REF{bkm:Ref89078422} illustrate what we are calling the hypothetical usage of the prefix \textit{ma-}. In this usage, predicates inflected with \textit{ma}{}- describe situations that may occur from time to time, but no specific actual event. This usage occurs frequently in expository texts.

\ea
\label{bkm:Ref119942002}
\textbf{Masimba}  anay  danen  an  daw  magprusisyon  na  daļa a  santos  na  Santo  Ninyo  palibot  ta  banwa… \\\smallskip
 \gll \textbf{Ma-simba}  anay  danen  an  daw  mag-prusisyon  na  daļa a  santos  na  Santo  Ninyo  pa-libot  ta  banwa… \\
 \textsc{a.hap.ir}-church  first/for-a-while  3\textsc{p.abs}  \textsc{def.m}  and  \textsc{i.ir}-procession  \textsc{lk} take/carry
\textsc{inj}   statue  \textsc{lk}  Holy  Child  \textsc{t.r}-around  \textsc{nabs}  town/country \\
\glt `They \textbf{go} \textbf{to} \textbf{church} and go in a procession carrying the Santo Ninyo statue taking it around the town…’ [VAOE-J-09 2]
\z
\ea
Daw  miling  kani  ta  altar  en,  \textbf{mauna}  abay  i punta  ta  tengnga  ya  ta  unaan.  \textbf{Masunod}  mga  flowergirl daw  ringbearer. \\\smallskip
 \gll Daw  m-iling  kani  ta  altar  en,  \textbf{ma-una}  abay  i punta  ta  tengnga  ya  ta  una-an.  \textbf{Ma-sunod}  mga  flowergirl daw  ringbearer. \\
 if/when  \textsc{i.v.ir}-go  later  \textsc{nabs}  alter  \textsc{cm}  \textsc{a.hap.ir}-first  wedding.attendant  \textsc{def.n}
go  \textsc{nabs}  middle  \textsc{def.f}  \textsc{nabs} first-\textsc{nr}  \textsc{a.hap.ir}-follow  \textsc{pl}   flowergirl and  ringbearer \\
\glt `When (the wedding party) is going to the altar, the wedding attendant (bridesmaids and groomsmen) \textbf{go} \textbf{first} going to the middle of the front (of the church). The flowergirl and ringbearer \textbf{follow}.’ [DBOE-C-01 6.1]
\z
\ea
\label{bkm:Ref89078422}
Kan-o  na  timpo  daw  \textbf{malarga}  mga  bļangay  o lansa  man  galiid  gid  ti  na  puļo… \\\smallskip
 \gll Kan-o  na  timpo  daw  \textbf{ma-larga}  mga  bļangay  o lansa  man  ga-liid  gid  ti  na  puļo… \\
 previous  \textsc{lk}  time/season  if/when  \textsc{a.hap.ir}-depart  \textsc{pl}  2.masted.boat  or
launch  also  \textsc{t.r}-pass.close.by  \textsc{int}  \textsc{d}1\textsc{nabs}  \textsc{lk}  island \\
\glt `In previous times, if/when two-masted boats or launches \textbf{would} \textbf{depart}, (they) passed close by this island…’ [VAWN-T-17 3.2]
\z

If the first verb in the construct illustrated in \REF{bkm:Ref89078422} were \textit{marga}, which consists of \textit{m}{}- ‘dynamic, intransitive, irrealis’ plus the root \textit{larga}, then the sentence would imply a more immediate event ‘(the launch) will purposely depart’. The verb form \textit{maglarga} is the general irrealis form, open to multiple interpretations, or some more distant future event. The form \textit{malarga} describes events that happened many times in the past. Irrealis modality is appropriate because no particular event is asserted.
\is{intransitive happenstantial hypothetical/polite inflection|)}

\subsection{Intransitive happenstantial realis \textit{naka}{}-}
\label{bkm:Ref447381050}\label{sec:intransitivehappenstantialrealis}\is{intransitive happenstantial realis inflection|(}

The happenstantial realis prefix \textit{naka}- only occurs in grammatically intransitive constructions in which the controller of the situation is expressed in the absolutive case. Normally \textit{naka}- and its irrealis counterpart \textit{maka}- describe animate activity or activity that is “animate-like.” Example \REF{bkm:Ref447862310} illustrates \textit{naka}{}- expressing perfect aspect (a state resulting from an earlier event). If these two verbs, \textit{agi} ‘to pass’ and \textit{tinir} ‘to stay’, were on the storyline they would have the dynamic intransitive prefix \textit{ga}{}- instead of \textit{naka}\nobreakdash-. This is a fairly prototypical usage of \textit{naka-}:

\ea
\label{bkm:Ref447862310}
… may  gakuyog  na  Japonis  na \textbf{nakaagi}  ta  dagsa daw  \textbf{nakatinir}  di  ta  pila  na  buļan. \\\smallskip
 \gll … may  ga-kuyog  na  Japonis  na \textbf{naka-agi}  ta  dagsa daw  \textbf{naka-tinir}  di  ta  pila  na  buļan. \\
{} \textsc{ext.in}  \textsc{i.r}-come.with  \textsc{lk}  Japanese  \textsc{lk}  \textsc{i.hap.r}-pass  \textsc{nabs}  shore
and  \textsc{i.hap.r}-stay  \textsc{d}1\textsc{loc}  \textsc{nabs}  few  \textsc{lk}  month \\
\glt `…there was a Japanese (person) who came with (someone) who had passed by way of the shore and had stayed here a few months.’ [HEWE-L-02 4.1]
\z

Example \REF{bkm:Ref447862310} contrasts with the usage of \textit{na-} in, for example, \REF{bkm:Ref447862443} above in that the events of passing through and staying in \REF{bkm:Ref447862310} are volitional---they are something the absolutive participant does, rather than something that “happens to” the absolutive. For this reason, we say that \textit{naka}{}- (and \textit{maka}{}-) are “Actor-oriented”. Falling off the carabao in \REF{bkm:Ref447862443}, on the other hand, is “Undergoer-orien\-ted”---it is something that “happens to” the absolutive marked participant. However, in \REF{bkm:Ref447862310}, neither situation is on the storyline of the narrative. They are both “flashbacks” to earlier events that have relevance to the current narrative, but which do not themselves advance the storyline. For this reason a perfect aspect interpretation is appropriate.  Furthermore, \REF{bkm:Ref447862310} contrasts with \REF{bkm:Ref448318853} through \REF{bkm:Ref392922782} in that these earlier examples are in transitive frames, whereas \textit{naka}- always appears in an intransitive or detransitive frame in which the absolutive refers to the Controller.

Concerning example \REF{bkm:Ref448309255}, the event expressed by \textit{sangga}, ‘bump into’, occurred previously in the story expressed with the form \textit{nasanggaan} ‘happened to bump into’. In that case, the implication is that the fish accidentally bumped into a limestone formation after having swum off quickly without regard to the direction he was going. Then example \REF{bkm:Ref448309255} occurs at the end of the story with \textit{naka-} indicating that the parrot fish had some responsibility for running into the rock because he had so mindlessly swum off. The moral of the story is overtly stated as “think before reacting”. We have attempted to capture this assertion of impulsiveness on the part of the parrot fish in the free translation with the English expression “had gone and banged . . .”:

\ea
\label{bkm:Ref448309255}
A,  yon  en  na  anduni  tuļ-ungan  i  daw  makita ta  naan  dagat,  takong  din  an  buksoļ,  unti, tenged  na  \textbf{nakasangga}  kanen  ta  manunggoļ. \\\smallskip
 \gll A,  yon  en  na  anduni  tuļ{}-ungan  i  daw  ma-kita ta  naan  dagat,  takong  din  an  buksoļ,  unti, tenged  na  \textbf{naka-sangga}  kanen  ta  manunggoļ. \\
\textsc{inj}  \textsc{d}3\textsc{abs}  \textsc{cm}  \textsc{lk}  now/today  parrot.fish  \textsc{def.n}  if/when  \textsc{a.hap.ir}-see 1\textsc{p.incl.erg}  \textsc{spat.def} sea  forehead  3\textsc{s.gen}  \textsc{def.m} lump \textsc{d}l\textsc{loc.pr} because  \textsc{lk}  \textsc{i.hap.r}-bump.into  3\textsc{s.abs}  \textsc{nabs}  limestone \\
\glt `Well, that (is the reason) now/today that the parrot fish when we see (it/them) in the sea, his forehead is lumped, here, because he (has gone and) banged into a limestone formation.’ [JCON-L-07 21.1]
\z

In example \REF{bkm:Ref392929987} \textit{nakabantaw} expresses a realized ability: ‘we were able to see’:

\ea
\label{bkm:Ref392929987}
Ta  seled  ta  limma  na  adlaw  \textbf{nakabantaw}  \textbf{kay} ta  isya  na  bukid  ta  Panay. \\\smallskip
 \gll Ta  seled  ta  limma  na  adlaw  \textbf{naka-bantaw}  \textbf{kay} ta  isya  na  bukid  ta  Panay. \\
\textsc{nabs}  within  \textsc{nabs}  five  \textsc{lk}  day/sun  \textsc{i.hap.r}-look.far.away  1\textsc{p.excl.abs}
\textsc{nabs}  one  \textsc{lk}  mountain  \textsc{nabs}  Panay \\
\glt `Within five days we were able/managed to spot far away one mountain on Panay.’ [VPWN-T-06 2.4]
\z

The forms \textit{naka}{}- and \textit{maka}{}- may be used in some situations in which the central participant is not animate or volitional. For example, the verb \textit{abot} ‘arrive’ usually takes an animate, volitional Agent as its only required argument. However, it can also be used for typhoons, telegrams, boats, sicknesses, suffering and so on. In such usages, it may take the \textit{naka}{}- and \textit{maka}{}- prefixes:

\ea
Ake  na  masakit,  \textbf{nakaabot}  gid  isyam  na  buļan  pinitinsya  ko. \\\smallskip
 \gll Ake  na  masakit,  \textbf{naka-abot}  gid  isyam  na  buļan  pinitinsya  ko. \\
1\textsc{s.gen}  \textsc{lk}  sick  \textsc{i.hap.r}-arrive  \textsc{int}  nine  \textsc{lk}  month  suffering  1\textsc{s.gen} \\
\glt ‘My sickness, my suffering arrived (i.e. lasted) for nine months.’ [PEWN-T-01 2.15]
\z

The verb \textit{idad} ‘(person/animal) to age’ (from the Spanish noun \textit{edad} ‘age’) may also take \textit{naka}{}- or \textit{maka}{}-, though it is hard to construe “aging” as a volitional process:\footnote{Perhaps this figure of speech is appropriate in Kagayanen because it takes a lot of effort to age successfully on an isolated island without much in the way of medical help, stores, or other amenities of life. To live to an old age is a highly respected accomplishment on Cagayancillo.  In any case, even if ‘to age’ is not intentional, the only argument of this verb is always an animate participant.}

\ea
\textbf{Nakaidad}  a  ta  27  na  anyos. \\\smallskip
 \gll \textbf{Naka-idad}  a  ta  27  na  anyos. \\
\textsc{i.hap.r}-age  1\textsc{s.abs}  \textsc{nabs}  27  \textsc{lk}  years \\
\glt ‘I was 27 years old.’ (lit. ‘I had managed to age for 27 years.’)
\z

Finally, some sensory verbs, \textit{kita} ‘see’, \textit{mati} ‘hear’,  \textit{batyag} ‘feel’, \textit{plamao} `smell', and \textit{lasa} ‘taste’, normally occur with the \textit{naka}{}- and \textit{maka}{}- prefixes in happenstantial intransitive constructions, though they are clearly experiential rather than volitional.

\ea
Busa,  ta  pagpatay  Mambeng,  pangaranan  na  Lungag  Mambeng ta  iya  kadengegan  na  kanen  \textbf{nakakita}  ta  nyan  na  lungag. \\\smallskip
 \gll Busa,  ta  pag-patay  Mambeng,  pa-ngaran-an  na  Lungag  Mambeng ta  iya  ka-dengeg-an  na  kanen  \textbf{naka-kita}  ta  nyan  na  lungag. \\
so  \textsc{nabs}  \textsc{nr.act}-dead  Mambeng  \textsc{t.r}-name-\textsc{apl}  \textsc{lk}  hole/cave  Mamben
\textsc{nabs} 3\textsc{s.gen}  \textsc{nr}-honor-\textsc{nr}  \textsc{lk}  3\textsc{s.abs}  \textsc{i.hap.r}-see  \textsc{nabs}  d3\textsc{abs}  \textsc{lk}  hole/cave \\
\glt `So, when Mambeng died, (the cave) was named Cave of Mambeng in his honor for \textbf{he} \textbf{had} \textbf{found} \textbf{(saw)} that cave.’ [SAWE-L-02 4.3]
\z

\ea
\textbf{Nakabatyag}  a  ta  ļettem  tak  uļa  a  kamaaw. \\\smallskip
 \gll \textbf{Naka-batyag}  a  ta  ļettem  tak  uļa  a  ka-maaw. \\
\textsc{i.hap.r}-feel  1\textsc{s.abs}  \textsc{nabs}  hunger  because  \textsc{neg.r}  1\textsc{s.abs}  \textsc{i.hap}-breakfast \\
\glt ‘\textbf{I} \textbf{felt} \textbf{hungry} because I did not eat breakfast.’
\z

Example \REF{bkm:Ref447897963} is particularly interesting in that it illustrates the same verb, \textit{mati} ‘to hear/listen’ both in the dynamic form, and the intransitive happenstantial:

\ea
\label{bkm:Ref447897963}
Kami  darwa  kay  Maria  \textbf{gamati}  ta  drama  iran  Mama. Kami  i  \textbf{nakamati}  kay  ta  isya  na  singgit  ta  isya na  bai  na  ame  man  na  katagsa. \\\smallskip
 \gll Kami  darwa  kay  Maria  \textbf{ga-mati}  ta  drama  iran  Mama. Kami  i  \textbf{naka-mati}  kay  ta  isya  na  singgit  ta  isya na  bai  na  ame  man  na  katagsa. \\
1\textsc{p.excl.abs}  two  1\textsc{p.excl.abs}  Maria  \textsc{i.r}-hear  \textsc{nabs}  drama  3\textsc{p.gen}  Mother/Aunt
1\textsc{p.excl.abs}  \textsc{def.n}  \textsc{i.hap.r}-hear  1\textsc{p.excl.abs}  \textsc{nabs}  one  \textsc{lk}  shout  \textsc{nabs}  one
\textsc{lk}  woman  \textsc{lk}  1\textsc{p.excl.gen}  too  \textsc{lk}  cousin \\
\glt `We both Maria and I \textbf{were} \textbf{listening} to a drama (program on the radio) at the place of Mama (aunt). \textbf{We} \textbf{heard} a shout of a woman who was indeed our cousin.’ [RZWN-T-02 2.2-3]
\z

In the first instance of the verb \textit{mati} in \REF{bkm:Ref447897963}, \textit{gamati}, the dynamic form implies the Actors were intentionally \textit{listening} to the drama. In the second instance, \textit{nakamati}, they happened to hear something, not intentionally, but only experientially. English expresses this intentional/non-intentional distinction in the two lexical verbs, \textit{listen} and \textit{hear}, whereas Kagayanen accomplishes the same semantic distinction by exploiting the morphological contrast between dynamic and happenstantial modalities. The form \textit{namatian} would require the Actor to be in the ergative case, and the Undergoer in the absolutive, and since this excerpt is a series of events describing what the main protagonists were doing, the Actor-oriented intransitive form \textit{nakamati} is more appropriate.

Example \REF{bkm:Ref448405038} illustrates the cognition verbs \textit{intindi} ‘understand’, and \textit{sat-em} ‘comprehend’ in their normal usages with \textit{naka}. Other cognition verbs in this group include \textit{isip} ‘think’, \textit{demdem} ‘remember’, and \textit{lipat} ‘forget’:

\ea
\label{bkm:Ref448405038}
A  ginikanan  an  a  na  uļa  \textbf{nakaintindi},  uļa  \textbf{nakasat-em,} mambaļ  en  dayon,  “Indya  kwarta  no  ya?" \\\smallskip
 \gll A  ginikanan  an  a  na  uļa  \textbf{naka-intindi},  uļa  \textbf{naka-sat-em,} m-ambaļ  en  dayon,  “Indya  kwarta  no  ya?" \\
\textsc{inj}  parent  \textsc{def.m}  \textsc{ctr}  \textsc{lk}  \textsc{neg.r}  \textsc{i.hap.r}-understand  \textsc{neg.r}  \textsc{i.hap.r}-comprehend
\textsc{i.v.ir}-say  \textsc{cm}  immediately  where  money  2\textsc{s.gen}  \textsc{def.f} \\
\glt `Well, parents who \textbf{do} \textbf{not} \textbf{understand}, \textbf{do} \textbf{not} \textbf{comprehend} will say immediately, “Where is your money?”' (In the context of this sentence high school graduates tell their parents that they want to go to college, and this is the parents’ response.) [JCOB-L-02 13.3]
\z

Finally, the happenstantial affixes may be used to downplay one’s personal achievements. In the following example, the speaker worked very hard to graduate, but uses the happenstantial, as though it were something that “just happened”:

\ea
Kaluoy  man  ta  Dyos  \textbf{nakatapos}  a  man  ta  elementarya. \\\smallskip
 \gll Kaluoy  man  ta  Dyos  \textbf{naka-tapos}  a  man  ta  elementarya. \\
\textsc{nr}-mercy  \textsc{emph}  \textsc{nabs}  God  \textsc{i.hap.r}-finish  1\textsc{s.abs}  \textsc{emph}  \textsc{nabs}  elementary. \\
\glt ‘By the mercy of God I was able to finish elementary school.’ [DDWN-C-01 3.13]
\is{intransitive happenstantial realis inflection|)}
\z

\subsection{Intransitive happenstantial irrealis \textit{maka}{}-}
\label{bkm:Ref122761448}\label{sec:intransitivehappenstantialirrealis}
\is{intransitive happenstantial irrealis inflection|(}

The prefix \textit{maka}{}- is the irrealis counterpart of \textit{naka-}. It also occurs only in intransitive frames in which the single argument is animate, or an inanimate entity that acts in an animate way. Example \REF{bkm:Ref447962976} is a near prototypical example of the use of \textit{maka}{}-:

\ea
\label{bkm:Ref447962976}
Kanen  \textbf{makabali}  ta  sundang  daw  iya  na  makagat. \\\smallskip
 \gll Kanen  \textbf{maka-bali}  ta  sundang  daw  iya  na  ma-kagat. \\
3\textsc{s.abs}  \textsc{i.hap.ir}-break  \textsc{nabs}  machete  if/when  3\textsc{s.gen}  \textsc{lk}  \textsc{a.hap.ir}-bite \\
\glt ‘\textsc{he} (coconut crab) can break a machete if (it is the thing) he bites.’ [DBWE-T-27 2.5]
\z

If the prefix \textit{ma}{}- were used here, it would be extremely awkward, and could only mean “\textsc{him} the machete broke,” since with \textit{ma}{}- the absolutive has to be the Undergoer. To preserve the basic semantics using \textit{ma}{}- the argument structure would need to change in order to make it clear that the machete is the Patient. Thus, the first part of \REF{bkm:Ref447962976} can be considered a detransitive counterpart of \REF{bkm:Ref447963964} (see \chapref{chap:voice} for a discussion of detransitive constructions):

\ea
\label{bkm:Ref447963964}
\textbf{Mabali}  din  sundang  an. \\\smallskip
 \gll \textbf{Ma-bali}  din  sundang  an. \\
\textsc{a.hap.ir}-break  3\textsc{s.erg} machete  \textsc{def.m} \\
\glt ‘S/he can break the machete.’
\z

Example \REF{bkm:Ref447963006} illustrates another fairly prototypical use of \textit{maka-}:

\ea
\label{bkm:Ref447963006}
Piro  daw  may  ubra  ka  en  daw  may  usto  na  swildo \textbf{makatabang}  ka  ta  imo  na  mga  ginikanan  ta pagpaiskwila  ta  imo  na  mga  mangngod. \\\smallskip
 \gll Piro  daw  may  ubra  ka  en  daw  may  usto  na  swildo \textbf{maka-tabang}  ka  ta  imo  na  mga  ginikanan  ta pag-pa-iskwila  ta  imo  na  mga  mangngod. \\
but  if/when  \textsc{ext.in}  work  2\textsc{s.abs}  \textsc{cm}  and  \textsc{ext.in}  sufficient  \textsc{lk}  wage
\textsc{i.hap.ir}-help  2\textsc{s.abs}  \textsc{nabs}  2\textsc{s.gen}  \textsc{lk}  \textsc{pl}  parent  \textsc{nabs}
\textsc{nr.act}-\textsc{caus}-go.to.school  \textsc{nabs}  2\textsc{s.gen}  \textsc{lk}  \textsc{pl}  younger.sibling \\
\glt `But if you have work and have sufficient wages, you can help your parents to support your younger siblings in school.’ [DBWL-T-30 5.3]
\z

The verb \textit{tabang}, ‘to help’, is one that requires the applicative suffix in its basic transitive usage (see \chapref{chap:verbclasses-1}, \sectref{sec:volitionaltransitiveroots}). Since there is no applicative on \textit{tabang} in \REF{bkm:Ref447963006}, it must be detransitive, and therefore is the counterpart of the transitive construction illustrated in \REF{bkm:Ref448035920}. Example \REF{bkm:Ref120014772} illustrates the same verb form in the text corpus:

\ea
\label{bkm:Ref448035920}
\textbf{Matabangan}  no  imo  na  mga  ginikanan. \\\smallskip
 \gll \textbf{Ma-tabang-an}  no  imo  na  mga  ginikanan. \\
\textsc{hap.ir}-help-\textsc{apl}  2\textsc{s.erg} 2\textsc{s.gen}  \textsc{lk}  \textsc{pl}  parent \\
\glt ‘You can help your parents.”
\z
\ea
\label{bkm:Ref120014772}
… Pwidi   ta  kaw  \textbf{matabangan}  daw  ino  gusto  no. \\\smallskip
 \gll … Pwidi   ta  kaw  \textbf{ma-tabang-an}  daw  ino  gusto  no. \\
{} can  1\textsc{s.erg}  2\textsc{s.abs}  \textsc{a.hap.ir}-help-\textsc{apl} if/when  what  want  2\textsc{s.erg} \\
\glt ‘… I can help you with whatever you want.’ [LGON-L-01 12.3]
\z

Examples \REF{bkm:Ref392930212} and \REF{bkm:Ref120014811} illustrate \textit{maka}{}- with the experiential verbs \textit{kita} ‘to see/find’ and \textit{singngot} ‘to smell s.t.’ With perception and cognition verbs, \textit{naka}{}- and \textit{maka}{}- are the normal detransitive (or “Actor voice”) forms that correspond to \textit{na}{}-/\textit{ma}{}- in a transitive frame:\footnote{For this reason, the \textit{ka} component of \textit{naka}{}- and \textit{maka}{}- may be considered an explicit marker of an antipassive construction. However, we find scant independent evidence that this \textit{ka} is a separate morpheme, for example, it is not used in any other modalities to express an antipassive (our detransitive) construction.}.

\ea
\label{bkm:Ref392930212}
Pangamuyuan  ta  kaw  nang  na  kabay  \textbf{makakita} ka  man  ta  ubra  na  dayad. \\\smallskip
 \gll \emptyset{}-Pangamuyo-an  ta  kaw  nang  na  kabay  \textbf{maka-kita} ka  man  ta  ubra  na  dayad. \\
\textsc{t.ir}-pray-\textsc{apl}  1\textsc{s.erg}  2\textsc{s.abs}  ony/just  \textsc{lk} may.it.be \textsc{i.hap.ir}-see 2\textsc{s.abs} \textsc{too} \textsc{nabs}  work  \textsc{lk}  good \\
\glt `I will just pray for you that you may be able to find work that is good.’ [DBWL-T-20 8.8]
\z
\ea
\label{bkm:Ref120014811}
Tak, daw naan a agi ta dapit abagat i, \textbf{makasingngot} kanen ta bao ko. \\\smallskip
 \gll Tak, daw naan a agi ta dapit abagat i, \textbf{maka-singngot} kanen ta bao ko. \\
because  if/when  \textsc{spat.def} 1\textsc{s.abs}  pass  \textsc{nabs}  direction   south(west)  \textsc{def.n}\textsc{}
\textsc{i.hap.ir-}smell  3\textsc{s.abs}  \textsc{nabs}  odor  1\textsc{s.gen} \\
\glt `For, if I pass in the direction of south(west), he (a wild pig) will be able to smell my odor.’ [RCON-L-01 2.7]
\z

Examples \REF{bkm:Ref447964968} through \REF{bkm:Ref120014968} illustrate \textit{maka}{}- with semantically intransitive\is{semantic transitivity}\is{transitivity!semantic} volitional verb stems. These cannot be considered detransitives, because there is no Undergoer, and no transitive counterpart in which the Actor is ergative. They are semantically intransitive\is{semantic transitivity}\is{transitivity!semantic} concepts in which the only participant is an Actor, rather than an Undergoer. The dynamic irrealis affixes for these verbs are \textit{m}{}- or \textit{mag}{}- as in \textit{megbeng/magtegeng,} \textit{miling/mag-iling,} and \textit{magbakasyon}. In these examples the happenstantial \textit{maka}{}- expresses potential ability:

\ea
\label{bkm:Ref447964968}
Uļa  aren  timpo  na  \textbf{makategbeng}  naan  ki  kaon. \\\smallskip
 \gll Uļa  aren  timpo  na  \textbf{maka-tegbeng}  naan  ki  kaon. \\
\textsc{neg.r}  1\textsc{s.abs} time \textsc{lk}  \textsc{i.hap.ir}-go.down \textsc{spat.def}  {obl.p} 2s \\
\glt ‘I have no time that I can go down to you.’ [MAWL-C-03 4.5]
\z
\ea
May  anen  kay  man  en  landingan  na  pwidi  kaw  en \textbf{makailing}  di  ta  ame  i  na  lugar. \\\smallskip
 \gll May  anen  kay  man  en  landingan  na  pwidi  kaw  en \textbf{maka-iling}  di  ta  ame  i  na  lugar. \\
\textsc{ext.in}  \textsc{ext.g} 1\textsc{p.excl.abs}  too  \textsc{cm}  airstrip  \textsc{lk}  can  2\textsc{p.abs}  \textsc{cm}
\textsc{i.hap.ir}-go  \textsc{d}1\textsc{loc  nabs}  1\textsc{p.excl.gen}  \textsc{def.n}  \textsc{lk}  place \\
\glt `We now have an airstrip so that you will be able to come here to our place.’ [DBWL-T-19 9.7]
\z
\ea
\label{bkm:Ref120014968}
Kabay  pa  na  \textbf{makabakasyon}  ka  unti ... \\\smallskip
 \gll Kabay  pa  na  \textbf{maka-bakasyon}  ka  unti ... \\
may.it.be  \textsc{inc}  \textsc{lk}  \textsc{i.hap.ir}-vacation  2\textsc{s.abs}  \textsc{d}1\textsc{loc.pr} \\
\glt ‘May it be that you will be able to vacation here ...’ [DBWL-T-19 9.7]
\z

The following are some additional examples from the corpus of \textit{maka}{}- occuring with inherently transitive verb stems in detransitive constructions:

\ea
Piro  anduni  pabawalan  en  na  magdakep  ta  pawikan tak  kanen  i  \textbf{makatabang}  man  ta  mga  ittaw. \\\smallskip
 \gll Piro  anduni  pa-bawal-an  en  na  mag-dakep  ta  pawikan tak  kanen  i  \textbf{maka-tabang}  man  ta  mga  ittaw. \\
but  now/today  \textsc{t.r}-forbid-\textsc{apl}  \textsc{cm}  \textsc{lk}  \textsc{i.ir}-catch  \textsc{nabs}  sea.turtle
because  3\textsc{s.abs}  \textsc{def.n}  \textsc{i.hap.ir}-help  too  \textsc{nabs}  \textsc{pl} person \\
\glt `But now it is forbidden to catch sea turtles because s/he can help people.’ [YBWE-T-05 2.6]
\z
\ea
Sarang-sarang  en  anduni  tak  duma  an  na  may  kwarta \textbf{makapalit}  ta  bitamaks  daw  mga  tibi. \\\smallskip
 \gll Sarang-{}\sim{}sarang  en  anduni  tak  duma  an  na  may  kwarta \textbf{makapalit}  ta  bitamaks  daw  mga  tibi. \\
\textsc{red}\sim{}better  \textsc{cm}  now/today  because  some  \textsc{def.m}  \textsc{lk}  \textsc{ext.in}  money
\textsc{i.hap.ir}-buy  \textsc{nabs}  betamax  and  \textsc{pl}  TV \\
\glt `It is getting better because some who have money can buy betamaxes and TVs.’ [DBWL-T-19 9.9]
\z
\ea
\label{bkm:Ref120017342}
Dili  man  danen  \textbf{makatanem}  daw  sigi  pelles.  Ta panguma  a  lugay  pa  man  danen  \textbf{makatubbas} ta  iran  na  kamuti  tak  derse  pa  mga  bunga  din  an. \\\smallskip
 \gll Dili  man  danen  \textbf{maka-tanem}  daw  sigi  pelles.  Ta pang-uma  a  lugay  pa  man  danen  \textbf{maka-tubbas} ta  iran  na  kamuti  tak  derse  pa  mga  bunga  din  an. \\
\textsc{neg.ir}  too  3\textsc{p.abs}  \textsc{i.hap.ir}-plant  if/when  continually  strong.wind  \textsc{nabs}
farming  \textsc{ctr}  long.time  \textsc{inc}  too  3\textsc{p.abs}  \textsc{i.hap.ir}-harvest
\textsc{nabs}  3\textsc{p.gen}  \textsc{lk}  cassava  because  small.\textsc{pl}  \textsc{inc}  \textsc{pl}  fruit  3\textsc{s.gen}  \textsc{def.m} \\
\glt `They will not be able to plant (agar seaweed) if (there are) continually strong winds. In the season of farming, it will be a long time too before they will be able to harvest their cassava because its fruit is still small.’ [DBWL-T-19 9.12-13]
\is{intransitive happenstantial irrealis inflection|)}
\z

\subsection{External motivation \textit{ka}{}-}
\label{sec:externalmotivation}
\label{bkm:Ref64362346}\is{external motivation|(}

The syllable \textit{ka} has a number of functions in Kagayanen inflectional and stem-forming morphology. In this section, we describe the inflectional prefix \textit{ka}-, which we gloss as \textsc{exm} for \textit{external motivation}\is{external motivation}. This is a specialized intransitive happenstantial inflection that expresses the idea that there is some external factor that motivates or enables either the situation expressed by the stem, or the speaker’s knowledge of that situation. This \textit{ka}- may be considered the specialized counterpart of \textit{na}-, \textit{ma}-, \textit{naka}- and \textit{maka}- since a) it occurs in realis or irrealis contexts, and b) it occurs in all situation types (recall that \textit{naka}- and \textit{maka}- describe situations in which the absolutive argument is in control, whereas \textit{na}- and \textit{ma}- describe situations in which the absolutive argument is not in control).

In the external motivation usage, \textit{ka}- expresses that there is some external factor that motivates, or enables (prevents in the case of negative situations) the controller to carry out the activity. For example, in \REF{bkm:Ref389056631} a wooden cane enables the actor to walk, and in \REF{bkm:Ref396746244} fear prevents the actor from speaking: 
\ea
\label{bkm:Ref389056631}
Dayon  a   kamang   ta  kaoy  na  sungkod para  \textbf{kapanaw}  a. \\\smallskip
 \gll Dayon  a   ...-kamang\footnotemark{}   ta  kaoy  na  sungkod para  \textbf{ka-panaw}  a. \\
right.away  1\textsc{s.abs}  \textsc{i.r}-get  \textsc{nabs}  wood  \textsc{lk}  cane
for \textsc{i.abl}-go/walk  1\textsc{s.abs} \\
\footnotetext{In this sentence, the intransitive, realis prefix \textit{ga}{}- has been omitted. This is a common feature of daily conversation, and is discussed in \sectref{bkm:Ref120017222}.}
\glt `Then I got a wooden cane so that I could walk.’ [EFWN-T-11 15.2]
\z
\ea
\label{bkm:Ref396746244}
Pag-abot  nay  ta  baļay,  uļa  \textbf{kambaļ}  katagsa  ko ya tak  nakulbaan … \\\smallskip
 \gll Pag-abot  nay  ta  baļay,  uļa  \textbf{ka-ambaļ}  katagsa  ko ya tak  na-kulba-an … \\
\textsc{nr.act}-arrive  1\textsc{p.exl.gen}  \textsc{nabs}  house  \textsc{neg.r}  \textsc{i.hap}-speak  cousin  1\textsc{s.gen} \textsc{def.f}
because  \textsc{a.hap.r}-frightened-\textsc{apl} \\
\glt `When we arrived at the house, my cousin could not speak because of being frightened (of something) ...' [CBWN-C-19 8.1]
\z
In example \REF{bkm:Ref396747208}, catching fish is enabled by good luck:  
\ea
\label{bkm:Ref396747208}
Ta  dayad  na  panwirtién  asta  a  magpailis ta  Barangay  Nusa,  \textbf{kasubbad}  pa  ta  darwa  buok... \\\smallskip
 \gll Ta  dayad  na  pang-swirti-én  asta  a  mag-pa-ilis ta  Barangay  Nusa,  \textbf{ka-subbad}  pa  ta  darwa  buok... \\
\textsc{nabs} good  \textsc{lk}  \textsc{nr}-luck-\textsc{nr}  until  1\textsc{s.abs}  \textsc{i.ir}-\textsc{caus}-shallow.sea
\textsc{nabs} community  Nusa  \textsc{i.hap}-catch.fish  \textsc{inc}  \textsc{nabs} two  piece \\
\glt `With much good luck, even when I was beginning to go to the shallow sea (along the shore) of the community of Nusa, I was able to catch two more pieces (fish)...’ [EFWN-T-11 8.1]
\z
In example \REF{bkm:Ref447898317}, \textit{ka}{}-, expresses the idea that his feeling lonely is motivated by the fact that the addressee is absent.
\ea
\label{bkm:Ref447898317}
Gusto  ko  gid  man  en  na  magbakasyon  dyan  ta  lugar ta  yan,  tak  kis-a  \textbf{kabatyag}  a  man ta  kapung-aw  ki  kyo. \\\smallskip
 \gll Gusto  ko  gid  man  en  na  mag-bakasyon  dyan  ta  lugar ta  yan,  tak  kis-a  \textbf{ka-batyag}  a  man ta  ka-pung-aw  ki  kyo. \\
want  1\textsc{s.erg}  \textsc{int}  too  \textsc{cm}  \textsc{lk}  \textsc{i.ir}-vacation  \textsc{d}2\textsc{loc}  \textsc{nabs}  place
1\textsc{p.incl.gen}  \textsc{def.m} because  sometimes  \textsc{i.hap}-feel  1\textsc{s.abs} too
\textsc{nabs}  \textsc{nr}-lonely  \textsc{obl.p}  2p \\
\glt `I really want to go on vacation there in our place because sometimes I feel loneliness too for you.’ [MBON-T-07a 3.3]
\z

Example \REF{bkm:Ref120018997} illustrates external motivation \textit{ka}- in a negative context. The ability to go home is prevented by strong winds and large waves.
\ea
\label{bkm:Ref120018997}
Gani  ta  iran  na  pagnubig  kaysan  danen magapon. \textbf{Dili}  pa  \textbf{kuli}  tenged  ta  biskeg na  angin  daw  darko  na mga  baļed. \\\smallskip
 \gll Gani  ta  iran  na  pag-nubig  kaysan  danen magapon. \textbf{Dili}  pa  \textbf{ka-uli}  tenged  ta  biskeg na  angin  daw  darko  na mga  baļed. \\
so  \textsc{nabs} 3\textsc{p.gen}  \textsc{lk}  \textsc{nr.act}-haul.water  sometimes  3\textsc{p.abs}
all.day.long
 \textsc{neg.ir}  \textsc{inc}  \textsc{i.hap}-go.home  because  \textsc{nabs} strong
\textsc{lk} wind  and  large.\textsc{pl}  \textsc{lk} \textsc{pl}  wave \\
\glt `So in their hauling water sometimes they stayed all day long. (They) will not be able to come home (all day until late in the afternoon) because of strong winds and large waves.’ [VPWE-T-01 2.8]
\z

The prefix \textit{ka}- also has an \textit{inferential}\is{inferential} usage. In the context of example \REF{bkm:Ref447986985}, a fish sees a ship coming and tells all the other fish to hide. After the fish have hidden, example \REF{bkm:Ref447986985} occurs, expressing the inference that the boat most likely has passed by. In the sentence after that, the fish all look up and see that the boat in fact is moving away fast. An enablement interpretation of \REF{bkm:Ref447986985} is excluded by native speakers. In this context, the speaker (the fish) infers that the situation expressed by the root is true because of the external fact that much time has passed.
\ea
\label{bkm:Ref447986985}
Anduni  \textbf{kalambay}  en  bļangay  ya  en. \\\smallskip
 \gll Anduni  \textbf{ka-lambay}  en  bļangay  ya  en. \\
now/today  \textsc{i.hap}-pass.by  \textsc{cm} 2.masted.boat  \textsc{def.f}  \textsc{cm} \\
\glt ‘Now the boat (must have) passed by.’ (Or ‘Now the boat most-likely/feasibly/predictably has passed by.’) [JCON-L-07 5.1]
\z

With the prefix \textit{maka}- this construction would express a different meaning, namely that the ship ‘will/can/may possibly pass by’, and \textit{naka}- would mean it ‘has already passed by.’

In the narrative from which \REF{bkm:Ref396802277} is extracted, Pedro’s father was killed, and since Pedro is the son, of course he had heard about it. This is clearly inferential modality rather than abilitative.

\ea
\label{bkm:Ref396802277}
Bata  din  na  Pedro  naan  ta  Puerto  \textbf{kabati} ta  sakit  na  kamatayen  ta  iya  na  palangga  na  tatay. \\\smallskip
 \gll Bata  din  na  Pedro  naan  ta  Puerto  \textbf{ka-bati} ta  sakit  na  kamatayen  ta  iya  na  palangga  na  tatay. \\
child  3\textsc{s.gen}  \textsc{lk}  Pedro  \textsc{spat.def}  \textsc{nabs} Puerto  \textsc{i.hap}-hear.news
\textsc{nabs} painful  \textsc{lk} death  \textsc{nabs}  3\textsc{s.gen}  \textsc{lk}  love  \textsc{lk} father \\
\glt `His son Pedro in Puerto must have heard the news about the painful death of his beloved father.’ (Or most-likely/feasibly/predictably he has heard the news…) [BEWN-T-01 5.1]
\is{external motivation|)}
\is{inflection|)}
\z

\section{Conversational omission of inflectional affixes}
\label{bkm:Ref120017222}\label{bkm:Ref119308716} \label{sec:omissionofaffixes} \label{sec:omissionofprefixes} \is{affix omission|(}

There is a common tendency in Kagayanen conversation for inflectional prefixes (\textit{pa-, ga-, na-, mag-, ma}{}-, and \textit{ka-}) to be omitted in certain “candidate environments”. These are the environments in which prefixes may be omitted:

\begin{enumerate}
\item
Following adverbs such as \textit{tapos} ‘after’, \textit{ubos} ‘all used up’, \textit{dayon} ‘right away’, \textit{diritso} ‘straight away,’ \textit{listo} ‘hastily’, \textit{sigi} ‘continuously’, \textit{tudo} ‘with all-out effort’, \textit{primi} ‘always’, \textit{uļa} ‘negative, realis’, and perhaps others.
\item 
In fast or excited speech.
\item
At peaks of narratives, or other contexts that involve tension.
\item
In relative clauses especially when they occur before the head nominal.
\item
In habitual actions.
\end{enumerate}

It is clear that this omission of prefixes is “optional” for two reasons. First, such omission never results in ambiguity—speakers are always able to supply the missing form based on contextual factors. Second, sometimes inflectional prefixes do appear in these contexts, with no apparent variation in meaning. A full discourse study may be needed to determine whether there is a consistent pattern as to when prefixes are omitted versus when they are retained. Since most of the examples in this grammar are from natural texts, many have omitted prefixes. We have not edited or flagged these in any way, unless the omission is important to the point being illustrated in the example. Examples \REF{bkm:Ref119307770} and \REF{bkm:Ref119307774} illustrate affix omission, with the ellipsis character (…) in place of the omitted affix:

\ea
\label{bkm:Ref119307770}
Indi  no  imo  \textbf{kamanga}  mga  blawan  an  Pedro? \\\smallskip
 \gll Indi  no  imo  …\textbf{-kamang-a}  mga  blawan  an  Pedro? \\
where  2\textsc{s.erg}  \textsc{emph}  \textsc{t.r}-get-\textsc{xc}  \textsc{pl}  gold  \textsc{def.m}  Pedro \\
\glt ‘Where did you get the gold, Pedro!?' [CBWN-C-22 8.3]
\z

The following examples illustrate the root, \textit{dala}, ‘take/carry’, both with \REF{bkm:Ref394992438} and without \REF{bkm:Ref394992440} the overt transitive realis prefix. The argument structure frame (Actor-ergative and Undergoer-absolutive) makes it clear that both examples are transitive, and the contexts make it clear that both are realis:

\ea
\label{bkm:Ref394992438}
\textbf{Padaļa}  din  iya  na  bata  naan  ta   iran  na  baļay. \\\smallskip
 \gll \textbf{Pa-daļa}  din  iya  na  bata  naan  ta   iran  na  baļay. \\
\textsc{t.r}-bring/take  3\textsc{s.erg}  3\textsc{s.gen}  \textsc{lk}  child  \textsc{spat.def}  \textsc{nabs}  3\textsc{p.gen}  \textsc{lk}  house \\
\glt ‘She took her child to their house.’ (Her child had died, so she carried her back to their house.) [VAWN-T-20 4.5]
\z
\ea
\label{bkm:Ref119307774}\label{bkm:Ref394992440}
\textbf{Daļa}  nay  sawa  din  daw  sawa  Lola  Maria na  patay … \\\smallskip
 \gll …-\textbf{Daļa}  nay  sawa  din  daw  sawa  Lola  Maria na  patay … \\
\textsc{\textsc{t.r}}-take/carry  1\textsc{p.excl.erg}  spouse  3\textsc{s.gen}  and  spouse  grandmother  Maria \textsc{lk}  dead \\
\glt `We took her husband and the husband of Grandmother Maria who were dead …’
[BCWN-C-04 9.1]
\z

It appears that in everyday speech \textit{pa}{}- may be freely omitted, though \textit{pa}{}- is much more likely to occur in foregrounded clauses (in the sense of \citealt{hopper1980}) in narrative. Of course, \textit{ga}{}- is the corresponding foreground form for grammatically intransitive predicates. The following are examples of affix omission in various candidate environments.

\ea
\label{ex:yourdish}
Prefix \textit{m}{}- omitted following locational \textit{naan}: \\
Atag  ko  ki  kaon  daw  miyag  ka  sawaen  ta  kaw. Naan  a  \textbf{tunuga}  ta  katri  no  daw  naan  a  \textbf{kaan} ta  pinggan  no. \\\smallskip
 \gll Atag  ko  ki  kaon  daw  miyag  ka  sawa-en  ta  kaw. Naan  a  \textbf{…-tunuga}  ta  katri  no  daw  naan  a  \textbf{…-kaan} ta  pinggan  no. \\
give  1\textsc{s.erg}  \textsc{obl.p}  2s  if/when  agree/want  2\textsc{s.abs}  spouse-\textsc{t.ir}  1\textsc{s.erg+}2\textsc{s.abs}
\textsc{spat.def}  1\textsc{s.abs}  \textsc{i.v.ir}-sleep  \textsc{nabs}  bed  2\textsc{s.gen}  and  \textsc{spat.def}  1\textsc{s.abs}  \textsc{i.v.ir}-eat
\textsc{nabs}  dish  2\textsc{s.gen} \\
\glt `I will give (it) to you if you agree that I marry you. I will sleep in your bed and I will eat from your dish.’ [CBWN-C-17 3.14] 
\z

Example \REF{ex:yourdish} is from a story about a man who became a frog. The frog wants to mary a young princess. She is playing a ball and the ball gets away from her. The frog gets the ball and says he will give it back if she marries him. The word \textit{tunuga} ‘to sleep’ can be replaced with \textit{munuga} or \textit{magtunuga}, and the word \textit{kaan} ‘to eat’ can be replaced with \textit{maan} or \textit{magkaan}, though in the context of this story \textit{munuga} and \textit{maan} are more likely. 

\ea
\label{spouse}
Prefix \textit{naka}{}- or \textit{ka}{}- omitted following negative \textit{uļa}: \\
Sawa  din  ya  uļa  \textbf{kita}  ki  kanen. \\\smallskip
 \gll Sawa  din  ya  uļa  \textbf{…-kita}  ki  kanen. \\
spouse  3\textsc{s.gen}  \textsc{def.f}  \textsc{neg.r}  \textsc{i.hap.r}-see  \textsc{obl.p}  3s \\
\glt ‘His spouse did not see him.’ [AMWN-T-01 2.14]
\z

In example \REF{spouse} the word \textit{kita} ‘to see’ can be \textit{nakakita} or \textit{kakita} depending on whether there was some external motivation that enabled the event. In example \REF{syudad}, \textit{kita} can only be replaced by \textit{kakita}, since ‘not going with them’ is presented as the external motivation for not seeing the cities.
\ea
\label{syudad}
Daw  uļa  yaken  na  gakuyog  danen,  uļa  yaken  i  \textbf{kita} iling  ti  na  mga  syudad. \\\smallskip
 \gll Daw  uļa  yaken  na  ga-kuyog  danen,  uļa  yaken  i  \textbf{…-kita} iling  ti  na  mga  syudad. \\
if/when  \textsc{neg.r}  1\textsc{s.abs}  \textsc{lk}  \textsc{i.r}-go.with  3s  \textsc{neg.r}  1\textsc{s.abs}  \textsc{def.n}  \textsc{i.exm}-see
like  \textsc{d}1\textsc{nabs}  \textsc{lk}  \textsc{pl}  city \\
\glt `If I did not go with them I would not see like these cities.’ [BMON-C-05 10.7]
\z
\ea
\label{ex:hermitcrabvs}
Omitted \textit{na}{}- and \textit{pa-} at points of heightened tension: \\
Nalambayan  din  umang  ya.  \textbf{Kita}  din  matuod umang  an  \textbf{lambayan}  din. \\\smallskip
 \gll Na-lambay-an  din  umang  ya.  \textbf{…-Kita}  din  matuod umang  an  \textbf{…-lambay-an}  din. \\
\textsc{a.hap.}\textsc{r}-pass.by-\textsc{apl}  3\textsc{s.erg}  hermit.crab  \textsc{def.f}  \textsc{a.hap.}\textsc{r}-see  3\textsc{s.erg}  true
hermit.crab  \textsc{def.m}  \textsc{t.r}-pass.by-\textsc{apl}  3\textsc{s.erg} \\
\glt `He (Sea Turtle) passed the hermit crab. He saw that truly he passed the hermit crab.’ [JCON-L-08 43.10]
\z

Example \REF{ex:hermitcrabvs} describes a race between Sea Turtle and Hermit Crab. Hermit Crab tricks Sea Turtle by having many hermit crabs distributed along the race course, so there is always a hermit crab ahead of Sea Turtle and the one that he was racing is waiting for him at the finish line. But Sea Turtle does not know there are many hermit crabs and he can not understand why Hermit Crab is always ahead of him. The verb \textit{kita} ‘to see’ can be replaced with \textit{nakita}, and the verb \textit{lambayan} ‘to pass by’ can be replaced with \textit{palambayan} or \textit{nalambayan}, though \textit{palambayan} is more appropriate in this story.

\ea
\label{ex:pirates}
Omitted \textit{maka}{}- or \textit{ka}{}- following irrealis negative \textit{dili}: \\
Pirate  ya  di  mata  din  dili  \textbf{kita}  en. \\\smallskip
 \gll Pirate  ya  di  mata  din  dili  \textbf{…-kita}  en. \\
pirate  \textsc{def.f}  \textsc{inj}  eye  3\textsc{s.gen}  \textsc{neg.ir}  \textsc{exm/a.hap.ir}-see  \textsc{cm} \\
\glt ‘The pirate, his eyes can’t see anymore.’ [BBOE-C-02 1.23]
\z

Example \REF{ex:pirates} is from a story describing an attack by pirates on the island of Cagayancillo long ago. Just before this example, the Kagayanens had thrown sand into the eyes of the pirates so they couldn’t see. The word \textit{kita} ‘to see’ may be replaced with \textit{makakita} or \textit{kakita}, though \textit{kakita} is more appropriate because the sand in their eyes was the external motivation for the pirates not being able to see. 
\ea
Omitted \textit{ma}{}- following the verb \textit{miyag}, `agree/want': \\\smallskip

Dili  kanen  miyag  na  \textbf{matian}  din  pulong  ta  Ginuo. \\\smallskip
 \gll Dili  kanen  miyag  na  \textbf{…-mati-an}  din  pulong  ta  Ginuo. \\
\textsc{neg.ir}  3\textsc{s.abs}  agree/want  \textsc{lk}  \textsc{a.hap.}\textsc{ir}-hear-\textsc{apl}  3\textsc{s.erg}  message  \textsc{nabs}  Lord \\
\glt ‘He did not want to hear the message of the Lord.’ (The verb \textit{matian} ‘to hear’ may be replaced with \textit{mamatian}.) [ETON-C-07 3.9]
\z
\ea
Omitted \textit{ga}{}- in an emphatic utterance: \\\smallskip

Uyi  na  ingkantada  pirmi  nang  \textbf{bisita}  ta  ake  na  katagsa. \\\smallskip
 \gll Uyi  na  ingkantada  pirmi  nang  \textbf{…-bisita}  ta  ake  na  katagsa. \\
\textsc{emph}-\textsc{d}1\textsc{abs}  \textsc{lk}  fairy  always  only/just  \textsc{i.r}-visit  \textsc{nabs}  1\textsc{s.gen}  \textsc{lk}  cousin \\
\glt ‘As for this fairy, (she) always visited my cousin.’ [MBON-T-03 3.9]
\z
The following are a few additional examples of omission of inflectional prefixes from the corpus.
\ea
 … pirmi  a  nang  en  na  \textbf{pangamuyo}  ta  Dios… \\\smallskip
 \gll … pirmi  a  nang  en  na  \textbf{…-pangamuyo}  ta  Dios… \\
  {} always  1\textsc{s.abs}  only/just  \textsc{cm}  \textsc{lk}  \textsc{i.r}-pray  \textsc{nabs}  God \\
\glt ‘…I kept on praying to God…’ (The verb \textit{pangamuyo} can be \textit{gapangamuyo}.) [EFWN-T-10 4.7]
\z
\ea
Piro  pirmi  aren  en  \textbf{simba},  a,  \textbf{kuyog}  ta  ake  na  anti \\\smallskip
 \gll Piro  pirmi  aren  en  \textbf{…-simba},  a,  \textbf{…-kuyog}  ta  ake  na  anti. \\
but  always  1\textsc{s.abs}  \textsc{cm}  \textsc{i.r}-worship  \textsc{inj}  \textsc{i.r}-go.with  \textsc{nabs}  1\textsc{s.gen}  \textsc{lk}  aunt. \\
\glt `But I kept on going to church, well going with my aunt.’ [JCOE-C-04 7.3]
\z
\ea
Ayaw  kay  \textbf{daļa}  ta  libon  na  ugsakan. \\\smallskip
 \gll Ayaw  kay  \textbf{…-daļa}  ta  libon  na  ugsak-an. \\
each.one’s.own  1\textsc{p.abs}  \textsc{i.r}-take/carry  \textsc{nabs}  basket  \textsc{lk}  put.inside-\textsc{nr} \\
\glt ‘Each one had our own basket we were carryng where to put (the fruit they will pick).’ [JCWN-L-38 7.1]
\z
\ea
Dayon  \textbf{kaļ-ay}  ta  iya  na  laya. \\\smallskip
 \gll Dayon  \textbf{…-kaļ-ay}  ta  iya  na  laya. \\
right.away  \textsc{i.r}-carry.dangling.down  \textsc{nabs}  3\textsc{s.gen}  \textsc{lk}  cast.net \\
\glt ‘Right away he carried his cast net dangling down.' (The verb \textit{kaļ-ay} ‘to carry dangling down’ can be \textit{gakaļ-ay}.) [EDWN-T-03 2.6]
\z
\ea
Gabalik  iruplano  ta  Puerto   na  \textbf{daļa} ta  duktor    a  sampel ta  iksamin  danen. \\\smallskip
 \gll Ga-balik  iruplano  ta  Puerto   na  …-\textbf{daļa} ta  duktor    a  sampel ta  iksamin  danen. \\
\textsc{i.r}-return  airplane  \textsc{nabs}  Puerto  \textsc{lk}  \textsc{t.r}-take/carry  \textsc{nabs}  doctor    \textsc{inj}  sample
\textsc{nabs}  examine  3\textsc{s.erg} \\
\glt `The airplane returned to Puerto taking to the doctor the sample for them to examine.’ (The verb \textit{daļa} ‘to take/carry’ can be \textit{padaļa} or \textit{nadaļa}.)  [JCWN-T-21 6.5]
\z

\newpage
\ea
\textit{Mag}{}- omitted: \\
Muli  ka  en.  \textbf{daļa}  ka  man  suman. \\\smallskip
 \gll M-uli  ka  en.  \textbf{…-daļa}  ka  man  suman. \\
\textsc{i.v.ir}-go.home  2\textsc{s.abs}  \textsc{cm}  \textsc{i.ir}-take/carry  2\textsc{s.abs}  also  sticky.rice.cake \\
\glt ‘You will go home. You will take some sticky rice cake.’ [EDOP-T-02 5.6]
\z
\ea
\label{ex:tidlak}
Daw  oras  en  \textbf{tanem}  ta  kamuti  a  tag-iya  ta  uma pagtidlak  ta  iran  na  uma. \\\smallskip
 \gll Daw  oras  en  \textbf{…-tanem}  ta  kamuti  a  tag-iya  ta  uma pag-tidlak  ta  iran  na  uma. \\
if/when  hour/time  \textsc{cm}  \textsc{i.ir}-plant  \textsc{nabs}  cassava  \textsc{inj}  owner  \textsc{nabs}  field
\textsc{nr.act}-first.planting.tradition  \textsc{nabs}  3\textsc{p.gen}  \textsc{lk}  field \\
\glt `When it is the time to plant cassava, the owner of the field is performing \textit{tidlak}.' [VAOE-J-07 1.3]
\z

Example \REF{ex:tidlak} describes \textit{tidlak}, traditional practices for beginning to plant a field. The verb \textit{tanem} ‘to plant’ can be \textit{magtanem}. 
\is{affix omission|)}

% \begin{verbatim}%%move bib entries to  localbibliography.bib
% \end{verbatim}warning
