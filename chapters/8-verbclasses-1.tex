\chapter{Morphosyntactically defined verb classes}
\label{chap:morphosyntacticallydefinedverbclasses}
\label{chap:verbclasses-1}
\section{Introduction}
\label{sec:introduction-8}

In \chapref{chap:verbstructure} we argued for a distinction between inflectional and stem-forming morphological processes, and described each of the inflectional affixes. We showed that inflection in Kagayanen consists of a paradigmatic grid with two major dimensions: transitivity and modality. In \chapref{chap:stemformingprocesses} we described eight stem-forming morphological processes. In this chapter and \chapref{chap:verbclasses-2} we discuss \textit{verb classes}\is{verb classes}, that is, groups of stems functioning as verbal predicates based on how they characteristically pattern in constructions, including inflectional and certain stem-forming morphological processes. In all cases, the classes are distinguished in terms of their morphosyntactic behavior, though the members of individual classes tend to exhibit semantic coherence. For example, verbs describing situations that normally occur without control or intention pattern one way (non-volitional situations), while those descibing controlled, intentional situations pattern differently (volitional situations). 

On the other hand, semantics does not mechanistically determine verb classes. Sometimes a verb class is characterized purely in terms of its form. For example, verbs beginning with the syllable \textit{li}{}- have their own distinctive inflectional possibilities, regardless of their semantics. Finally, the class membership of some verbs appears to be random--their meanings would seem to be consistent with one class, but their morphosyntactic behavior puts them in another.

Verb classes identified in this chapter are, for the most part, motivated in terms of four semantic dimensions. These are \textit{Aktionsarten} (or \textit{situation type}, \citealt{vendler1957}) (\sectref{bkm:Ref148441557}), \textit{volitionality} (\sectref{bkm:Ref148441557}), semantic transitivity\textit{} (\sectref{bkm:Ref148441557}), and \textit{change of state}\is{change of state} (\sectref{sec:changeofstate}).

In \chapref{chap:verbclasses-2} we discuss certain semantically motivated verb classes that also exhibit characteristic morphosyntactic behavior, but are less directly sensitive to the semantic dimensions discussed in the present chapter. These chapters are necessarily incomplete, since we can’t possibly describe every possible morphosyntactic property that may unite two or more roots. Nor can we discuss every semantically characterized situation that Kagayanen speakers care to talk about. Furthermore, there is variation from speaker to speaker and situation to situation. Speakers use the grammatical resources of their language in any way they can to express the ideas they need to express in a way they believe their audience will understand. We cannot possibly describe or predict how those resources will be used in every situation. We hope, however, to describe certain recurring patterns that, taken as a whole, provide a general picture of how the grammatical resources of Kagayanen are deployed to craft and communicate complex meanings. 

\section{Situation types (\textit{Aktionsarten})}
\label{bkm:Ref148854086}
There are several methods for categorizing situations in terms of their semantics. In this chapter, it will be helpful to employ the well-known categorization scheme suggested by \citet{vendler1957}. This is a useful framework within which to categorize and understand the semantics of predicates in any language. Vendler described situations in terms of what he called \textit{Aktionsarten}, or inherent aspectual properties. Although he couched his discussion in terms of individual verbs, his categorization is more insightfully understood in terms of situations as presented by speakers in communicative contexts. Verbs and grammatical constructions are simply tools that speakers use to communicate situations. As described by Vendler himself, any given verb may be used to express multiple situation types depending on the communicative needs of speakers. The communication of ideas is logically prior to and influences to a large extent the grammatical behavior of particular verb roots and stems in discourse. This is particularly true in Philippine languages, in which robust verb morphology may adjust the situation type expressed by a particular root, thus allowing speakers to create and represent a wide range of complex and nuanced discourse scenes.

\citet{vendler1957} described situation types as \textit{states}\is{states}, \textit{achievements}\is{achievements}, \textit{accomplishments}\is{accomplishments} and \textit{activities}\is{activities}. States are situations in which there is no change, for example, \textit{Melvin is a doctor}, \textit{she is tall for her age}, \textit{she loves you}, or \textit{the vase is broken}. A state may result from an earlier event, but an assertion of a resultant state does not assert the event that led to the state. For example, a sentence like \textit{the vase is broken} simply describes the static condition of the vase. The state of being broken is necessarily the result of an event of breaking, but the sentence does not specifically assert the event--only the resultant state.

Achievements, accomplishments and activities are all dynamic in that they involve motion and/or change. Achievements and accomplishments additionally involve an inherent end point (\textit{the vase shattered}, \textit{the ice} \textit{melted}, \textit{we reached the summit}), whereas activities do not (\textit{we danced}, \textit{they approached the summit}, \textit{we ate pizza}). In the accomplishment of reaching the summit, there is an inherent point at which the summit is reached. Until that point, it is not true that “we reached the summit”. Whereas at any point in the activity of eating pizza it is true that “we ate pizza”. Another way of saying this is that achievements and accomplishments are \textit{telic}\is{telic}--they have an inherent endpoint, or “end in view.” Activities, on the other hand, are not telic--they have no inherent endpoint.

The difference between achievements and accomplishments is that achievements occur in an instant in time, that is, they are \textit{punctual} (\textit{the vase shattered, the balloon burst}). Unless we are physicists looking at the shattering of a vase from a nanosecond by nanosecond view, a vase cannot “begin to shatter” or “be shattering”. Accomplishments, on the other hand occur over time (\textit{the ice melted}). Melting is a \textit{durative} process which leads to a new resultant state – melted ice. In ordinary conversation, it is perfectly normal to assert that ice “begins to melt” or “is melting”.

In addition to the Vendler classification of situations into states, achievements, accomplishments and activities, there are three other semantic dimensions that we find useful for understanding Kagayanen grammar. These are volitionality, semantic transitivity and change of state.

\section{Volitionality and semantic transitivity}
\label{bkm:Ref148441557}
Our use of the term \textit{basic} in the following discussion refers to the simplest situations depicted by verb-like roots with no stem-forming morphology or external modifiers such as adverbials or oblique elements. For example, the root \textit{leddang} ‘to sink’ is “basically intransitive” because in a situation of sinking, only one participant is required: the item that sinks. This root, along with many others, may appear in a grammatically transitive frame with no stem-forming morphology, in which case the meaning is causative: ‘to make/let sink’. However, a situation describable as \textit{leddang} does not need to involve a causer--something can just sink with no necessary causal Actor. Therefore the causative meaning is not basic. On the other hand, a situation describable as \textit{kaan} ‘to eat’ is “basically transitive” because in order to count as a situation of \textit{kaan}, two participants must be involved: an eater and an eaten thing. This root, along with many others, may appear in a grammatically intransitive frame with no stem-forming morphology, but it is always understood that two participants are involved, even though one is downplayed or not present on the scene at all.

Volitionality and semantic transitivity are semantic dimensions that Vendler’s classification of situation types does not directly address. Nevertheless, we find them useful in the description of various Kagayanen verb classes discussed in this chapter.

As mentioned in \chapref{chap:verbstructure}, some basically intransitive situations are \textit{non-volitional}\is{non-volitional} in the sense that the only participant does not consciously control the situation. The basic senses of English verbs such as \textit{die, collapse, melt, tumble, shiver, sink} and \textit{trip} fall into this category. These verbs describe situations that \textit{happen to} the only participant. In terms of macroroles, we say that the single participant of such situations is an Undergoer. Other basically intransitive situations are \textit{volitional} in that they are normally accomplished on purpose by a conscious Actor. The basic senses of English verbs such as \textit{go, come, jump, linger, grunt, roar, lie down, stand, sit, walk, swim} and many others normally describe situations that the only participant \textit{does}, therefore they fall into the volitional category.

Volitionality also plays a role in basically transitive situations. For example, simple perception verbs such as \textit{see}, \textit{hear} and \textit{sense} usually describe non-volitio\-nal situations in which one participant is a Stimulus and the other is an Experiencer. Some cognition, emotion and other experiential situations are also treated grammatically as non-volitional (see \chapref{chap:verbclasses-2}). Most other transitive situations are volitional in that they involve someone or something that acts, and something else that is acted upon. The basic meanings of English verbs such as \textit{eat, hug, build, read, raise, push, pull, examine} and many others fall into the volitional, transitive category. Volitionality is another semantic dimension that helps us understand the morphosyntactic properties of certain classes of roots in Kagayanen.

\section{Change of state}
\label{sec:changeofstate}
As mentioned in \sectref{bkm:Ref148854086}, a state is an unchanging condition. From this it follows that a \textit{change of state} involves a change in such a condition. For example, situations describable by English verbs such as \textit{break}, \textit{collapse}, \textit{crumble}, \textit{eat}, \textit{burn} and \textit{grow} all involve something that changes its physical state. For the most part, these verbs have attributive forms (known as “past participles” in traditional English grammar) that refer to a referent that has undergone such a change in state: \textit{a broken glass, a collapsed house, crumbled cheese}, and so on. Other situations do not involve a change in state. For example, situations describable by English verbs such as \textit{sing}, \textit{see}, \textit{run}, \textit{enjoy} and \textit{view} do not involve a change in state. Therefore, for the most part, the past participles of these verbs are not used attributively: *\textit{a sung aria}, *\textit{a seen airplane}, *\textit{a run child}, and so on. Change of place is also often treated grammatically in the same way as a change in state. For example, the past participles of certain change of place verbs in English may be used attributively: \textit{an escaped prisoner}, \textit{a fallen log}, \textit{a returned veteran}, \textit{a given assumption} and so on.

Change of state is another semantic dimension that Vendler’s classification of situation types does not incorporate. Some achievements and accomplishments assert events that produce a new resultant state (\textit{the earthquake shattered the window, the ice melted}, \textit{Felnor burst my balloon}), whereas others do not (\textit{they spotted the airplane}, \textit{we walked to the park}, \textit{they found the entrance}). In Kagayanen, change of state is a major dimension in the use of the happenstantial modality inflections. As we will show in \sectref{bkm:Ref148856716}, among non-volitional intransitive situations, those that involve a change in state take the happenstantial inflections, while those that don’t involve a change in state for the most part disallow the happenstantial inflections. Whether or not a situation involves a change of state also influences other parts of the inflectional system, as will be seen in the rest of this chapter.

\section{Overview of eight classes of verbal roots}
\label{bkm:Ref149367537} \label{sec:overviewofeightverbclasses}

In this section we describe eight root classes which are largely motivated by the semantic parameters of \textit{Aktionsart}, volitionality, semantic transitivity and change of state (\tabref{tab:eightverbclasses}). Recall that just about any root may optionally take a causative prefix and/or an applicative suffix. In many cases, these affixes form a stem that belongs to a different class than the bare root. For example, the bare root \textit{sayaw}, ‘to dance’, normally describes a volitional intransitive activity (Class IV). The addition of the causative \textit{pa}{}- creates the stem \textit{pasayaw}, ‘make/let dance’ which is a volitional transitive accomplishment (Class VI). \tabref{tab:eightverbclasses} describes the properties of bare roots in their basic argument structure frames, with no modifying elements such as causative prefixes or locational phrases.

Following \tabref{tab:eightverbclasses} we give inflectional paradigms for each of the sample roots. In \sectref{bkm:Ref148856716}-\sectref{sec:irrealisinflections} we provide lists of roots that fall into each of these classes. We then present and discuss corpus examples of verbs from each class.

\begin{table}[ht]
\caption{Root classes based on volitionality, Aktionsart, semantic transitivity and change of state}
\label{tab:eightverbclasses}
\begin{tabular} {
    >{\RaggedRight\arraybackslash}p{.85cm}     %Root Class
    >{\RaggedRight\arraybackslash}p{2.3cm}    %Basic semantic transitivity
    >{\RaggedRight\arraybackslash}p{2.8cm}  %Volitionality and Aktionsart
    >{\RaggedRight\arraybackslash}p{1.4cm}    %Change of state
    >{\RaggedRight\arraybackslash}p{2.4cm}    %Sample root
                }
\lsptoprule
\textbf{Root Class} & \textbf{Semantic\newline transitivity} & \textbf{Volitionality and Aktionsart} & \textbf{Change of state} & \textbf{Sample root} \\
\midrule
I & Intransitive & Non-volitional Achievement & yes & \textit{buong} ‘shatter’ \\
\tablevspace
II & Intransitive & Non-volitional Accomplishment & yes & \textit{leddang} ‘sink’ \\
\tablevspace
III & Intransitive & Non-volitional activity & no & \textit{bagting} ‘ring’ \\
\tablevspace
IV & Intransitive & Volitional\newline activity & no & \textit{sayaw} ‘dance’ \\
\tablevspace
V & Transitive & Non-volitional accomplishment & no & \textit{kita} ‘see’ \\
\tablevspace
VI & Transitive & Volitional\newline  accomplishment & yes  & \textit{inem} ‘drink’ \\
\tablevspace
VII & Transitive & Volitional\newline accomplishment  & yes (change of place) & \textit{atag} ‘give’ \\
\tablevspace
VIII & Transitive & Volitional\newline activity & no & \textit{arek} ‘kiss’ \\
\lspbottomrule
\end{tabular}
\end{table}

As mentioned often in this grammar, transitivity of argument structure frames (sometimes referred to as \textit{grammatical transitivity}) is independent of basic (semantic) transitivity of roots or stems. Recall that an intransitive argument structure frame is one in which the only argument is an absolutive (\textsc{abs}). That absolutive argument may be an Undergoer or an Actor. A transitive argument structure frame is one that contains an absolutive Undergoer and a distinct ergative Actor (\textsc{erg}). Finally, a detransitive argument structure frame contains an absolutive Actor and a non-absolutive Undergoer (\textsc{nabs}) either understood (omitted), or preceded by a prenominal case marker, \textit{ta} or \textit{ki}. The following paradigms illustrate each of the roots mentioned in \tabref{tab:eightverbclasses} in their characteristic argument structure frames. The translations given are only approximate. They are meant to provide a general sense of the meanings of each construction, not an inclusive semantic analysis. More detailed descriptions of the usages of each construction are provided in the remainder of this chapter.

Examples \REF{bkm:Ref148453300}-\REF{bkm:Ref148453302} illustrate \textit{buong}, ‘shatter’, a Class I root in intransitive, transitive and detransitive frames. Because Class I roots are punctual achievements, they do not take dynamic affixes in the basic intransitive frame. This is because dynamicity involves change over time--in Kagayanen, an object cannot “begin to shatter” or “be shattering”. In a transitive frame, these roots express direct causation, without a causative prefix. They may also take a causative prefix, in which case they express indirect causation (see \chapref{chap:voice}, \sectref{sec:causatives}).\footnote{Direct causation refers to situations in which a causer exerts direct control over the caused situation, for example, \textit{Melvin broke the window} (\textit{caused the window to break}), \textit{Shelly} \textit{sank the boat} (\textit{caused the boat to sink}). Indirect causation refers to situations in which the agent of the caused situation (sometimes referred to as the \textit{causee}), retains some control over the situation, for example, \textit{Melvin let Peter run wild}, \textit{Shelly had Peter get a haircut}.} In a detransitive frame, these verbs also express direct causation, but with the causee downplayed.

\ea
\begin{tabbing}
\hspace{4.5cm} \= \kill
\label{bkm:Ref148453300}
Intransitive frames: \\
a.  \textit{buong} \textsc{abs} \>    ‘\textsc{abs} is shattered’  (Non-verbal predicate \\ \> expressing resultant state) \\
b.  \textit{nabuong} \textsc{abs} \> ‘\textsc{abs} shattered’ \\
     \> ‘\textsc{abs} has already shattered’ (perfect aspect) \\
c.  \textit{mabuong} \textsc{abs} \> ‘\textsc{abs} will shatter’ \\
d.  *\textit{gabuong} \textsc{abs} \> (‘\textsc{abs} began to shatter/was/is shattering') \\
e.  *\textit{magbuong} \textsc{abs} \> (‘\textsc{abs} will begin to shatter, will be shattering')
\end{tabbing}
\z
\ea
\begin{tabbing}
\hspace{4.5cm} \= \kill
Transitive frames: \\
a.  \textit{nabuong} \textsc{erg} \textsc{abs}  \> ‘\textsc{erg} accidentally/carelessly shattered \textsc{abs}’ \\
\>      ‘\textsc{erg} was able to shatter \textsc{abs}’ \\
\>      ‘\textsc{erg} has already shattered \textsc{abs}’ \\
b.  \textit{mabuong} \textsc{erg} \textsc{abs} \> ‘\textsc{erg} will accidentally/carelessly shatter \textsc{abs}’ \\
\>      ‘\textsc{erg} will be able to shatter \textsc{abs}’ \\
\>      ‘\textsc{erg} will have shattered \textsc{abs}’ \\
c.  \textit{pabuong} \textsc{erg} \textsc{abs} \> ‘\textsc{erg} shattered \textsc{abs}’ \\
d.  \textit{buungon} \textsc{erg} \textsc{abs} \> ‘\textsc{erg} will shatter \textsc{abs}’
\end{tabbing}
\z
\ea
\label{bkm:Ref148453302}
\begin{tabbing}
\hspace{4.5cm} \= \kill
Detransitive frames: \\
a.  \textit{gabuong} \textsc{abs} (\textsc{nabs}) \> ‘\textsc{abs} shattered \textsc{nabs}’ \\
\>       ‘\textsc{abs} was/is shattering \textsc{nabs}’ \\
\>       ‘\textsc{abs} began/was/is beginning to shatter \textsc{nabs}’ \\
b.  \textit{magbuong} \textsc{abs} (\textsc{nabs}) \> ‘\textsc{abs} will shatter \textsc{nabs}’ \\
\>       ‘\textsc{abs} will begin to shatter \textsc{nabs}’ \\
c.  \textit{nakabuong} \textsc{abs} (\textsc{nabs}) \> ‘\textsc{abs} accidentally/carelessly shattered \textsc{nabs}’ \\
\>       ‘\textsc{abs} was able to shatter \textsc{nabs}’ \\
\>      ‘\textsc{abs} has already shattered \textsc{nabs}’ \\
d.  \textit{makabuong} \textsc{abs} (\textsc{nabs}) \> ‘\textsc{abs} will accidentally/carelessly shatter \textsc{nabs}’ \\
\>      ‘\textsc{abs} will be able to shattered \textsc{nabs}’ \\
\>      ‘\textsc{abs} will have already shattered \textsc{nabs}’
\end{tabbing}
\z

Examples \REF{bkm:Ref148516780}-\REF{bkm:Ref148516783} illustrate the root \textit{leddang} ‘to sink’, a Class II root. These roots occur in the same set of argument structure frames as Class I roots. However, since they express durative accomplishments, they may take dynamic modality in the basic intransitive frame:
\ea
\label{bkm:Ref148516780}
\begin{tabbing}
\hspace{4.5cm} \= \kill
Intransitive frames: \\
a.  \textit{leddang} \textsc{abs} \> ‘\textsc{abs} is sunken.’ (non-verbal predicate) \\
b.  \textit{naleddang} \textsc{abs} \> ‘\textsc{abs} sank’ \\
\>       ‘\textsc{abs} has already sunk’ \\
c.  \textit{maleddang} \textsc{abs} \> ‘\textsc{abs} will sink’ \\
d.  \textit{galeddang} \textsc{abs} \> ‘\textsc{abs} was/is sinking’ \\
\>       ‘\textsc{abs} began/was/is beginning to sink’ \\
e.  \textit{magleddang} \textsc{abs} \> ‘\textsc{abs} will begin to sink’ \\
\>      ‘\textsc{abs} will sink/be sinking’ \\
\>      ‘\textsc{abs} sinks (habitual/generic/infinitive)’
\end{tabbing}
\z
\ea
\begin{tabbing}
\hspace{4.5cm} \= \kill
Transitive  frames: \\
a.  \textit{naleddang} \textsc{erg} \textsc{abs} \> ‘\textsc{erg} accidentally/carelessly sank \textsc{abs}’ \\
\>      ‘\textsc{erg} was able to sink \textsc{abs}’ \\
\>      ‘\textsc{erg} has already sunk \textsc{abs}’ \\
b.  \textit{maleddang} \textsc{erg} \textsc{abs} \> ‘\textsc{erg} will accidentally/carelessly sink \textsc{abs}’ \\
\>      ‘\textsc{erg} will be able to sink \textsc{abs}’ \\
\>      ‘\textsc{erg} will have sunk \textsc{abs}’ \\
c.  \textit{paleddang} \textsc{erg} \textsc{abs} \> ‘\textsc{erg} sank \textsc{abs}’ \\
d.  \textit{leddangen} \textsc{erg} \textsc{abs} \> ‘\textsc{erg} will sink \textsc{abs}’
\end{tabbing}
\z
\ea
\label{bkm:Ref148516783}
\begin{tabbing}
\hspace{4.5cm} \= \kill
Detransitive frames: \\
a.  \textit{galeddang} \textsc{abs} (\textsc{nabs}) \> ‘\textsc{abs} sank \textsc{nabs}’ \\
\>       ‘\textsc{abs} was/is sinking \textsc{nabs}’ \\
\>       ‘\textsc{abs} began/was/is beginning to sink \textsc{nabs}’ \\
b.  \textit{magleddang} \textsc{abs} (\textsc{nabs}) \> ‘\textsc{abs} will sink \textsc{nabs}’ \\
\>       ‘\textsc{abs} will begin to sink \textsc{nabs}’ \\
\>      ‘\textsc{abs} sinks \textsc{nabs}’ \\
c.  \textit{nakaleddang} \textsc{abs} (\textsc{nabs}) \> ‘\textsc{abs} acccidentally/carelessly sank \textsc{nabs}’ \\
\>      ‘\textsc{abs} was able to to sink \textsc{nabs}’ \\
\>      ‘\textsc{abs} has already sank \textsc{nabs}’ \\
d.  \textit{makaleddang} \textsc{abs} (\textsc{nabs}) \> ‘\textsc{abs} will acccidentally/carelessly sink \textsc{nabs}’ \\
\>      ‘\textsc{abs} will be able to sink \textsc{nabs}’ \\
\>      ‘\textsc{abs} will have sunk \textsc{nabs}’
\end{tabbing}
\z

Examples \REF{bkm:Ref148517624}{}-\REF{bkm:Ref148517629} illustrate the root \textit{bagting} ‘to ring’, a Class III root. Since these roots do not involve a change in state (ringing a bell does not change the bell in any way), they do not allow a resultant state usage. For the same reason, the bare roots do not easily occur in the happenstantial forms. For inherently intransitive non-volitional roots, happenstantial forms imply a change of state.

\ea
\label{bkm:Ref148517624}
\begin{tabbing}
\hspace{4.5cm} \= \kill
Intransitive frames: \\
a.  \textit{*bagting} \textsc{abs}  \>  (‘\textsc{abs} is rung’) \\
b.  \textit{*nabagting} \textsc{abs}  \>  (‘\textsc{abs} rang, \textsc{abs} has already rung’) \\
c.  \textit{*mabagting} \textsc{abs}  \>  (‘\textsc{abs} will ring, \textsc{abs} will have rung’) \\
d.  \textit{gabagting} \textsc{abs}  \>  ‘\textsc{abs} rang/rings’ \\
\>        ‘\textsc{abs} began/begins to ring’ \\
\>        ‘\textsc{abs} was/is ringing’ \\
e. \textit{magbagting} \textsc{abs}  \>  ‘\textsc{abs} will ring’ \\
\>        ‘\textsc{abs} will begin to ring’ \\
\>        ‘\textsc{abs} will be ringing’ \\
\>        ‘\textsc{abs} rings’
\end{tabbing}
\z
\ea
\begin{tabbing}
\hspace{4.5cm} \= \kill
Transitive frames: \\
a.  \textit{*nabagting} \textsc{erg} \textsc{abs}  \>  (‘\textsc{erg} accidentally/carelessly rang \textsc{abs}’) \\
b.  \textit{*mabagting} \textsc{erg} \textsc{abs}  \>  (‘\textsc{erg} will accidentally/carelessly ring \textsc{abs}’) \\
c.  \textit{pabagting} \textsc{erg} \textsc{abs}  \>  ‘\textsc{erg} rang \textsc{abs} on purpose’ \\
d.  \textit{bagtingen} \textsc{erg} \textsc{abs}  \>  ‘\textsc{erg} will ring \textsc{abs} on purpose’
\end{tabbing}
\z
\ea
\label{bkm:Ref148517629}
\begin{tabbing}
\hspace{4.5cm} \= \kill
Detransitive frames: \\
a.  \textit{gabagting} \textsc{abs} (\textsc{nabs})  \>  ‘\textsc{abs} rang \textsc{nabs}’ \\
\>         ‘\textsc{abs} is/was ringing \textsc{nabs}’ \\
\>         ‘\textsc{abs} began/was/is beginning to ring \textsc{nabs}’ \\
b.  \textit{magbagting} \textsc{abs} (\textsc{nabs}) \> ‘\textsc{abs} will ring \textsc{nabs}’ \\
\>         ‘\textsc{abs} will begin to ring \textsc{nabs}’ \\
\>        ‘\textsc{abs} rings \textsc{nabs}’ \\
c.  \textit{nakabagting} \textsc{abs} (\textsc{nabs}) \> ‘\textsc{abs} accidentally/carelessly rang \textsc{nabs}’ \\
\>       ‘\textsc{abs} was able to ring \textsc{nabs}’ \\
\>        ‘\textsc{abs} has already rung \textsc{nabs}’ \\
d.  \textit{makabagting} \textsc{abs} (\textsc{nabs}) \> ‘\textsc{abs} will accidentally/carelessly ring \textsc{nabs}’ \\
\>       ‘\textsc{abs} will be able to ring \textsc{nabs}’ \\
\>        ‘\textsc{abs} will have rung \textsc{nabs}’
\end{tabbing}
\z

The examples in \REF{bkm:Ref148518761} illustrate the root \textit{sayaw} ‘to dance’, a Class IV root. Semantically, Class IV roots describe volitional, basically intransitive situations. They do not entail a change in state, therefore there is no resultant state form. Grammatically, Class IV roots do not occur in transitive frames without causative or applicative morphology, therefore the inflected root does not appear in detransitive frames either:

\ea
\label{bkm:Ref148518761}
\begin{tabbing}
\hspace{4.5cm} \= \kill
Intransitive frames: \\
a.  \textit{*sayaw} \textsc{abs}  \>  (‘\textsc{abs} is danced’) \\
b.  \textit{gasayaw} \textsc{abs} \> ‘\textsc{abs} danced’ \\
\>      ‘\textsc{abs} was/is dancing’ \\
\>       ‘\textsc{abs} began/was/is beginning to dance’ \\
c.  \textit{magsayaw} \textsc{abs} \> ‘\textsc{abs} will dance’ \\
\>       ‘\textsc{abs} will begin to dance’ \\
\>      ‘\textsc{abs} dances’ \\
d.  \textit{mayaw} \textsc{abs}  \>  ‘\textsc{abs} will soon dance’\footnotemark{} \\
e.  \textit{nakasayaw} \textsc{abs} \> ‘\textsc{abs} carelessly danced’ \\
\>      ‘\textsc{abs} is/was able to dance’ \\
\>      ‘\textsc{abs} has already danced’ \\
f.  \textit{makasayaw} \textsc{abs} \> ‘\textsc{abs} will carelessly dance’ \\
\>      ‘\textsc{abs} will be able to dance’ \\
\>      ‘\textsc{abs} will have danced’ 
\end{tabbing}
\footnotetext{See \sectref{sec:irrealisinflections} below for the subtle difference in meaning between \textit{mag}{}- and replacive \textit{m}{}- as the intransitive irrealis inflection for those roots, such as \textit{sayaw}, that allow both.}
\z

Examples \REF{bkm:Ref150170077} and \REF{bkm:Ref150170080} illustrate the transitive, non-volitional root \textit{kita} ‘to see’ (Class V). Note that this verb, along with other members of its class, does not easily occur in the dynamic forms. This makes semantic sense, since dynamicity involves movement and/or change, and seeing something does not involve either of these semantic features.

\ea
\label{bkm:Ref150170077}
\begin{tabbing}
\hspace{4.5cm} \= \kill
Transitive frames: \\
a.  \textit{nakita} \textsc{erg} \textsc{abs}  \>  ‘\textsc{erg} saw \textsc{abs}’ \\
b.  \textit{makita} \textsc{erg} \textsc{abs}  \>  ‘\textsc{erg} will see \textsc{abs}’ \\
c.  *\textit{pakita}\footnotemark{} \textsc{erg} \textsc{abs}  \>  (‘\textsc{erg} saw \textsc{abs}’) \\
d.  *\textit{kitaen} \textsc{erg} \textsc{abs}  \>  (‘\textsc{erg} will see \textsc{abs}’)
\end{tabbing}
\footnotetext{This is a grammatical causative stem meaning ‘to show.’ There are no dynamic modality forms of the root \textit{kita}.}  
\z
\ea
\label{bkm:Ref150170080}
\begin{tabbing}
\hspace{4.5cm} \= \kill
Detransitive frames: \\
a.  *\textit{gakita} \textsc{abs} (\textsc{nabs})  \>  (‘\textsc{abs} is/was seeing \textsc{nabs}’) \\
\>         (‘\textsc{abs} began/was/is beginning to see \textsc{nabs}’) \\
b.  *\textit{magkita} \textsc{abs} (\textsc{nabs})  \>  (‘\textsc{abs} will see \textsc{nabs}’) \\
\>         (‘\textsc{abs} will begin to see \textsc{nabs}’) \\
c.  \textit{nakakita} \textsc{abs} (\textsc{nabs})  \>  ‘\textsc{abs} saw \textsc{nabs}’ \\
d.  \textit{makakita} \textsc{abs} (\textsc{nabs})  \>  ‘\textsc{abs} will see \textsc{nabs}’
\end{tabbing}
\z

Class VI can be considered the majority class for roots describing semantically transitive\is{semantic transitivity}\is{transitivity!semantic} situations in which there is a change in state. Like classes V, VII and VIII, Class VI roots expressing transitive situations do not occur in basic intransitive frames.\footnote{Some roots fall into more than one class. The situation types described by root Classes V-VIII are basically (semantically) transitive, so for these meanings basic intransitive forms are not available. Some of these roots also fall into other classes with different meanings. For example, \textit{agi} as a volitional intransitive activity means ‘to pass through’ (Class IV). This is the only meaning available for this root in a basic intransitive frame. As a non-volitional transitive activity it means ‘to experience X’ (Class V). For this meaning, the root does not appear in a basic intransitive frame.} All intransitive inflections for the transitive meanings are detransitive. Examples \REF{bkm:Ref148531334} and \REF{bkm:Ref148531341} illustrate the Class VI root \textit{inem} ‘to drink’:

\ea
\label{bkm:Ref148531334}
\begin{tabbing}
\hspace{4.5cm} \= \kill
Transitive frame: \\
a.  \textit{painem} \textsc{erg} \textsc{abs} \>  ‘\textsc{erg} drank \textsc{abs}’ \\
b.  \textit{inemen} \textsc{erg} \textsc{abs} \> ‘\textsc{erg} will drink \textsc{abs}’ \\
c.  \textit{nainem} \textsc{erg} \textsc{abs} \>  ‘\textsc{erg} accidentally/carelessly drank \textsc{abs}’ \\
\>      ‘\textsc{erg} was able to drink \textsc{abs}’ \\
\>      ‘\textsc{erg} has already drunk \textsc{abs}’ \\
d.  \textit{mainem} \textsc{erg} \textsc{abs} \>  ‘\textsc{erg} will accidentally/carelessly drink \textsc{abs}’ \\
\>      ‘\textsc{erg} will be able to drink \textsc{abs}’ \\
\>      ‘\textsc{erg} will have drunk \textsc{abs}’
\end{tabbing}
\z

\largerpage[2]
\ea
\begin{tabbing}
\hspace{4.5cm} \= \kill
\label{bkm:Ref148531341}Detransitive frame: \\
a.  \textit{gainem} \textsc{abs} (\textsc{nabs}) \> ‘\textsc{abs} drank/drinks (\textsc{nabs})’ \\
b.  \textit{mag{}-inem} \textsc{abs} (\textsc{nabs}) \> ‘\textsc{abs} will drink (\textsc{nabs})’ \\
\>      ‘\textsc{abs} drinks (\textsc{nabs})’ \\
c.  \textit{minem} \textsc{abs} (\textsc{nabs}) \> ‘\textsc{abs} will soon drink (\textsc{nabs})’ \\
d.  \textit{nakainem} \textsc{abs} (\textsc{nabs}) \> ‘\textsc{abs} accidentally/carelessly drank (\textsc{nabs})’ \\
\>      ‘\textsc{abs} was able to drink (\textsc{nabs})’ \\
\>      ‘\textsc{abs} has already drunk (\textsc{nabs})’ \\
c.  \textit{makainem} \textsc{abs} (\textsc{nabs}) \> ‘\textsc{abs} will accidentally/carelessly drink (\textsc{nabs})’ \\
\>      ‘\textsc{abs} will be able to drink (\textsc{nabs})’ \\
\>      ‘\textsc{abs} will have drunk (\textsc{nabs})’
\end{tabbing}
\z\clearpage

Examples \REF{bkm:Ref148532039} and \REF{bkm:Ref148532042} illustrate the root \textit{atag} ‘to give’, a Class VII root. Most Class VII roots describe situations of transfer (sometimes called “ditransitive verbs”) in which the item transferred appears in the Absolutive, and the Recipient in the Non-absolutive case. There are no basic intransitive forms of Class VII roots – all intransitive forms are detransitives, with the item transferred downplayed or omitted. Classes VII and VIII roots are also characterized by the fact that the transitive, irrealis inflection is expressed by the bare verb form (or a “zero” affix; see \ref{bkm:Ref148532039}b):

\ea
\label{bkm:Ref148532039}
\begin{tabbing}
\hspace{4.3cm} \= \kill
Transitive frame: \\
a.  \textit{paatag} \textsc{erg} \textsc{abs} (\textsc{nabs}) \> ‘\textsc{erg} gave \textsc{abs} (to \textsc{nabs})’ \\
b.  \textit{atag} \textsc{erg} \textsc{abs} (\textsc{nabs}) \> ‘\textsc{erg} will give \textsc{abs} (to \textsc{nabs})’ \\
c.  \textit{naatag} \textsc{erg} \textsc{abs} (\textsc{nabs}) \> ‘\textsc{erg} accidentally/carelessly gave \textsc{abs} (to \textsc{nabs})’ \\
\>      ‘\textsc{erg} was able to give \textsc{abs} (to \textsc{nabs})’ \\
 \>     ‘\textsc{erg} has already given \textsc{abs} (to \textsc{nabs})’ \\
d.  \textit{maatag} \textsc{erg} \textsc{abs} (\textsc{nabs}) \> ‘\textsc{erg} will accid./carelessly give \textsc{abs} (to \textsc{nabs})’ \\
\>      ‘\textsc{erg} will be able to give \textsc{abs} (to \textsc{nabs})’ \\
\>      ‘\textsc{erg} will have given \textsc{abs} (to \textsc{nabs})’  
\end{tabbing}
\z
\ea
\begin{tabbing}
\hspace{4.4cm} \= \kill
\label{bkm:Ref148532042}Detransitive frame: \\
a.  \textit{gaatag} \textsc{abs} (\textsc{nabs}) (\textsc{nabs}) \\
\> ‘\textsc{abs} gave/gives (\textsc{nabs}) (to \textsc{nabs})’ \\
b.  \textit{mag-atag} \textsc{abs} (\textsc{nabs}) (\textsc{nabs}) \\
\> ‘\textsc{abs} will give (\textsc{nabs}) (to \textsc{nabs})’ \\
\>        ‘\textsc{abs} gives (\textsc{nabs}) (to \textsc{nabs})’ \\
c.  \textit{matag} \textsc{abs} (\textsc{nabs}) (\textsc{nabs}) \> ‘\textsc{abs} will soon give (\textsc{nabs}) (to \textsc{nabs})’ \\
d.  \textit{nakaatag} \textsc{abs} (\textsc{nabs}) (\textsc{nabs}) \\
\> ‘\textsc{abs} accid./carel. gave (\textsc{nabs}) (to \textsc{nabs})’ \\
\>        ‘\textsc{abs} was able to give (\textsc{nabs}) (to \textsc{nabs})’ \\
\>        ‘\textsc{abs} has already given (\textsc{nabs}) (to \textsc{nabs})’ \\
e.  \textit{makaatag} \textsc{abs} (\textsc{nabs}) (\textsc{nabs}) \\
\> ‘\textsc{abs} will accid./carel. give (\textsc{nabs}) (to \textsc{nabs})’ \\
\>        ‘\textsc{abs} will be able to give (\textsc{nabs}) (to \textsc{nabs})’ \\
\>        ‘\textsc{abs} will have given (\textsc{nabs}) (to \textsc{nabs})’
\end{tabbing}
\z

Finally, examples \REF{bkm:Ref148532212} and \REF{bkm:Ref148532215} illustrate the Class VIII root \textit{arek} ‘to kiss’. Roots of this class obligatorily take the applicative suffix -\textit{an} in basic transitive constructions in both realis and irrealis modalities. The -\textit{an} drops in detransitive constructions. Like Class VII roots, the transitive, irrealis inflection is expressed by the bare verb stem, which we represent in glossed examples as a “zero” prefix. These roots describe situations in which the Undergoer is a less than prototypical Patient in that it does not undergo a visible change of state. Rather, the Undergoer is invisibly, partially or superficially affected by the activity described by the root. The examples in \REF{bkm:Ref148532212} illustrate forms of the stem \textit{arekan}, since the applicative -\textit{an} is required for basic transitive constructions for this class of verbs. 

\ea
\label{bkm:Ref148532212}
\begin{tabbing}
\hspace{4.5cm} \= \kill
Transitive frame: \\
a.  \textit{paarekan} \textsc{erg} \textsc{abs} \>  ‘\textsc{erg} kissed \textsc{abs}’ \\
b.  \textit{arekan} \textsc{erg} \textsc{abs} \> ‘\textsc{erg} will kiss \textsc{abs}’ \\
c.  \textit{naarekan} \textsc{erg} \textsc{abs} \>  ‘\textsc{erg} accidentally/carelessly kissed \textsc{abs}’ \\
\>      ‘\textsc{erg} was able to kiss \textsc{abs}’ \\
\>      ‘\textsc{erg} has already kissed \textsc{abs}’ \\
d.  \textit{maarekan} \textsc{erg} \textsc{abs} \>  ‘\textsc{erg} will accidentally/carelessly kiss \textsc{abs}’ \\
\>      ‘\textsc{erg} will be able to kiss \textsc{abs}’ \\
\>      ‘\textsc{erg} will have kissed \textsc{abs}’
\end{tabbing}
\z
\ea
\label{bkm:Ref148532215}
\begin{tabbing}
\hspace{4.5cm} \= \kill
Detransitive frame: \\
a.  \textit{gaarek} \textsc{abs} (\textsc{nabs}) \> ‘\textsc{abs} kissed/kisses (\textsc{nabs})’ \\
b.  \textit{mag{}-arek} \textsc{abs} (\textsc{nabs}) \> ‘\textsc{abs} will kiss (\textsc{nabs})’ \\
\>      ‘\textsc{abs} kisses (\textsc{nabs})’ \\
c.  \textit{marek} \textsc{abs} (\textsc{nabs}) \> ‘\textsc{abs} will soon kiss (\textsc{nabs})’ \\
d.  \textit{nakaarek} \textsc{abs} (\textsc{nabs}) \> ‘\textsc{abs} accidentally/carelessly kissed (\textsc{nabs})’ \\
\>      ‘\textsc{abs} was able to kiss (\textsc{nabs})’ \\
\>      ‘\textsc{abs} has already kissed (\textsc{nabs})’ \\
e.  \textit{makaarek} \textsc{abs} (\textsc{nabs}) \> ‘\textsc{abs} will accidentally/carelessly kiss (\textsc{nabs})’ \\
\>      ‘\textsc{abs} will be able to kiss (\textsc{nabs})’ \\
\>      ‘\textsc{abs} will have kissed (\textsc{nabs})’ \\
\end{tabbing}
\z

\section{Roots describing non-volitional intransitive situations (Classes I – III)}
\label{bkm:Ref148856716}
Root Classes I through III described above consist of verbal roots that express non-volitional semantically intransitive\is{semantic transitivity}\is{transitivity!semantic} situations. Roots in Classes I and II additionally involve a change of state in the absolutive arguments. The difference between Class I and Class II is that Class I roots express punctual situations (achievements), whereas Class II roots express durative situations (accomplishments). Both Class I and II roots allow a “zero” form (no inflectional affixation) that describes a state that is the result of the situation expressed in the root, for example, \textit{the glass is shattered}, \textit{the sail is torn}. Since Class III roots don’t involve a change in state, there is no resultant state form for these roots.

For Class I and II roots, the happenstantial forms express a semantic perfective aspect, for example, \textit{the glass shattered/will shatter, the sail tore/will tear}. The dynamic forms (\textit{ga}{}- realis and \textit{mag}{}- irrealis) occur in Class II and Class III roots. For Class II roots, the dynamic forms describe an inceptive, \textit{the sail began to tear}, or an imperfective, \textit{the sail is tearing}, sense. Because Class I roots describe achievements (i.e., they are punctual), they do not allow the inceptive or imperfective senses. 

The dominant pattern for Class III roots is that they only occur in the dynamic forms. These verbs for the most part describe activities, for example, \textit{the bells are ringing}, \textit{the fire is flaring up}, \textit{the weather is getting better}. These do not occur in the zero form nor in the happenstantial forms because they do not describe resultant states, or situations that result in a change in state. Occasionally some of these roots may occur in the irrealis happenstantial form, but we have not investigated the conditions under which this may occur.

As will be seen below, volitional activities (Class IV) take both happenstantial and dynamic affixes, but the dynamic affixes seem to be the default forms in basic intransitive frames, and commonly express a semantic perfective aspect. For these verbs, the happenstantial forms express more subtle aspectual or modal senses such as accidentally/carelessly, ability, happenstance, or opportunitive (see \chapref{chap:verbstructure}, \sectref{sec:verbinflection}). 

Because of the patterns described above and illustrated in \REF{bkm:Ref148453300} through \REF{bkm:Ref148532215}, we conclude that non-volitionality and change of state are the core semantic features that underlie the use of the happenstantial modality. At times these core features may be extended into other domains (e.g. change of state may include change of place, as in \textit{to fall off accidentally}). The inherent semantics of Class I roots so strongly favor these features that happenstantial modality is (almost) obligatory. Class II roots, on the other hand, may be interpreted as involving either a change of state, or an activity, depending on the context. Such situations may be expressed in either happenstantial or dynamic modalities, with corresponding differences in meaning. Finally, Class III roots so strongly favor the activity feature that dynamic modality is (almost) required.

\largerpage
\tabref{tab:rootsexpressingnon-volitionalintransitivesituations} lists some of the non-volitional, intransitive verbs that fall into these three classes. While there is individual variation throughout the Kagayanen community, and particular roots may “shift” from one Class to another, depending on the communicative needs and creativity of speakers, the classification given here is our best determination of the default, normal usages of these roots.\footnote{For an English analogy, one may consider such common assertions as “stative verbs do not occur in the progressive aspect.” However, speakers may certainly use normally stative verbs in non-stative senses by putting them in the progressive, for example, “I’m not understanding what you are saying,” or ``she’s believing more and more that ...” Such creative usages on the part of speakers do not negate the general fact that stative verbs in English have distinct grammatical properties from non-stative verbs. Analogous class “shifting” probably occurs in every language.} Full examples of a few of these from the corpus follow.

\begin{table}
    \caption{Roots expressing non-volitional intransitive situations}
    \label{tab:rootsexpressingnon-volitionalintransitivesituations}
    \fittable {
    \begin{tabular}{
        >{\RaggedRight\arraybackslash}p{4cm}
        >{\RaggedRight\arraybackslash}p{4.1cm}
        >{\RaggedRight\arraybackslash}p{4cm}
                    }
\lsptoprule
Class I & Class II & Class III \\
\midrule
buong  ‘to break/shatter’ \newline
dugan ‘to be pressed under something heavy’ \newline
igo ‘to be hit’ \newline
ipit ‘to get pinned’ \newline
gubba ‘to be ruined/to break/malfunction’ \newline
samad ‘to be ruined, destroyed’ \newline
biak ‘to split in half’ \newline
patay ‘living things to die' \newline
bugto ‘rope/string to break’ \newline
uļog ‘to fall’ \newline
anad ‘to be used to’ \newline
ļabo ‘to capsize’ \newline
ļao ‘to be thirsty’ \newline
ļettem  ‘to be hungry’ \newline
lukat ‘to be uprooted’ \newline
duwad ‘to disappear, be lost’ \newline
pessa ‘to break into many pieces’ \newline
tao ‘to be born’ \newline
taring ‘to lose one’s way’ \newline 
tuog ‘to be pierced' &
anod ‘to drift off' \newline
bali ‘to break off’ \newline
basa ‘to get wet' \newline
gisi ‘to tear/rip’ \newline
kamang ‘to dissipate’ \newline
badbad ‘to come untied’ \newline
ayad ‘to get well’ \newline
dagdag ‘to fall off (as leaves)’ \newline
daeb ‘to turn face down' \newline
dagsa ‘to wash ashore’ \newline
ukab ‘to be loose/pull away (as floor boards or toenails) \newline
lattik ‘to crack’ \newline
lunot ‘to disintegrate/decay’ \newline
lapta ‘to spread out covering surface of something’ \newline
kay-ag ‘to scatter, be messy’ \newline
lasik ‘to splatter, splash’ \newline
lekkep ‘to cover entirely’ \newline
leddang ‘to sink’ \newline
patay ‘fire to extinguish’ \newline
pudpod ‘to crumble to powder’ \newline
sangit  ‘to get snagged’ \newline
tangtang ‘to fall apart/off’ \newline
tumba ‘to fall over’ \newline
tunaw ‘to melt, dissolve' \newline
upod ‘to wear away’ \newline
uyak ‘to spill out’ \newline
wigit ‘to fall unnoticed' \newline
wili ‘to be engrossed' &
dabadaba ‘fire to flare' \newline
ļegļeg ‘for fire to flame’ \newline
sangsarang ‘to improve (weather, or sick person)’ \newline
kereg ‘to shiver’ \newline
narem ‘to have sleep paralysis' \newline
dagbeng ‘to rumble/thud’ \newline
bagting ‘bells to ring’ \newline
bengngel ‘to be deaf‘ \newline
bukļad ‘to open’ \newline
baog ‘to bend’ \newline
agdaw ‘flame to reduce’ \newline
sikaļ  ‘water to boil’ \newline
luslos ‘color to fade’ \newline
ubļo ‘to buldge out’ \newline
enneb ‘to absorb, infuse’ \newline
ilig ‘to flow' \newline
lassot ‘things to loosely fit together’ \newline
luttaw ‘to float’ \newline
pilit ‘to stick to/on’ \newline
sablig ‘to splash’ \newline
sampaw ‘to be on top of something’ \newline
takļap  ‘to get covered’ \newline
tubo ‘to grow’ \newline
umpok ‘to bounce’ \\
\lspbottomrule
    \end{tabular} 
    }
\end{table}
In the following examples, the verbs being illustrated are bolded, in Kagayanen and the English translations. 
\subsection{Non-volitional achievements (Class I)}

Examples \REF{bkm:Ref398301852} and \REF{bkm:Ref398301854} illustrate the root \textit{buong} ‘shatter’, which is a non-volitio\-nal root of Class I.  With no  inflectional affixes this root describes a simple state \REF{bkm:Ref398301852}. This may be considered a non-verbal, modification predicate (see \chapref{chap:non-verbalclauses}, \sectref{sec:predicatemodifiers}). In \REF{bkm:Ref398301854}, with happenstantial marking it describes the perfective achievement of shattering. As with all Class I verbs, this root does not occur in dynamic modality (*\textit{gabuong/*magbuong}). This makes sense since shattering is a punctual event, therefore the root that expresses this concept cannot describe an activity, since activities are inherently non-punctual:

\ea
\label{bkm:Ref398301852}
\textbf{Buong} baso  ya.\\\smallskip
\gll \textbf{Buong}    baso  ya. \\
shatter  glass  \textsc{def.f} \\
\glt `That glass \textbf{is shattered}.' \hspace{1cm}STATE
\z
\ea
\label{bkm:Ref398301854}
\textbf{Nabuong}  baso  ya. \textbf{shattered}.’ \\\smallskip
\gll \textbf{Na-buong}  baso  ya. \\
\textsc{a.hap.r}-shatter  glass  \textsc{def.f} \\
\glt `That glass \textbf{shattered}.'\hspace{1cm}ACHIEVEMENT \\\smallskip
*gabuong baso ya. \\
(‘That glass was/is shattering/began to shatter.’) *ACTIVITY
\z

Examples \REF{bkm:Ref440375076}{}-\REF{bkm:Ref119566298} illustrate additional Class I verbs from the corpus:


\ea
\label{bkm:Ref440375076}
Tiset  a  nang  \textbf{naipit}  ta  jip. \\\smallskip
\gll Tiset  a  nang  \textbf{na-ipit}  ta  jip. \\
small  1\textsc{s.abs} only/just  \textsc{a.hap.r}-pin  \textsc{nabs}  jeep \\
\glt ‘A jeep almost \textbf{pinned} me.’ [HBWN-T-01 3.26]
\z

\newpage

\ea
Daw  uļa  galin  suguon  din,  ta  dili  \textbf{mapatay}  bata  din. \\\smallskip
\gll Daw  uļa  ga-alin  suguon  din,  ta  dili  \textbf{ma-patay}  bata  din. \\
if/when  \textsc{neg.r}  \textsc{i.r}-from  servant  3\textsc{s.gen}  so  \textsc{neg.ir}  \textsc{a.hap.ir}-die  child  3\textsc{s.gen} \\
\glt ‘If her servant had not quit work, then her child would not \textbf{have} \textbf{died}.’ [MOOE-C-01 225.1]
\z
\ea
\textbf{Napatay}  kanen  ta  kadalok. \\\smallskip
\gll \textbf{Na-patay}  kanen  ta  ka-dalok. \\
\textsc{a.hap.r}-die  3\textsc{s.abs}  \textsc{nabs}  \textsc{nr}-greedy \\
\glt ‘he \textbf{died} from greed.’ [MBON-T-04 13.7]
\z
\ea
Tanan  na  \textbf{napatay}  ta  masakit  na ya  20  gid  na  ittaw. \\\smallskip
\gll Tanan  na  \textbf{na-patay}  ta  masakit  na ya  20  gid  na  ittaw. \\
all  \textsc{lk}  \textsc{a.hap.r}-die  \textsc{nasb}  sickness  \textsc{lk} \textsc{d3adj}  20  \textsc{int}  \textsc{lk}  people \\
\glt ‘All who \textbf{died} from that sickness were really 20 people.’ [JCWN-T-21 18.1]
\z
\ea
Yi  na  manakem  uļa  inta  kanen  \textbf{napatay}  tak  kanen  gatago naan  ta  kasilyas  ya. \\\smallskip
\gll Yi  na  manakem  uļa  inta  kanen  \textbf{na-patay}  tak  kanen  ga-tago naan  ta  kasilyas  ya. \\
\textsc{d1adj}  \textsc{lk}  older  \textsc{neg.r}  \textsc{opt}  3\textsc{s.abs}  \textsc{a.hap.r}-die  because  3\textsc{s.abs}  \textsc{i.r}-hide \textsc{spat.def}  \textsc{nabs}  bathroom  \textsc{def.f} \\
\glt ‘This older person should not \textbf{have} \textbf{died} because he hid in the bathroom.’ [BCWN-C-04 6.7]
\z

\ea
Manong  ya  \textbf{patay}  en. \\\smallskip
\gll Manong  ya  \textbf{patay}  en. \\
Older.brother  \textsc{def.f}  die  \textsc{cm} \\
\glt ‘Older Brother \textbf{is dead} now.’ [CBWN-C-22 13.2]
\z
\ea
\textbf{Nabugto}  en  ate  na  pagpari. \\\smallskip
\gll \textbf{Na-bugto}  en  ate  na  pag-pari. \\
\textsc{a.hap.r}-break  \textsc{cm}  1\textsc{p.incl.gen}  \textsc{lk}  \textsc{rel}-friend \\
\glt ‘Our friendship \textbf{has} broken now.’ [RBWN-T-02 5.5]
\z

\newpage
\ea
Basi  \textbf{gubba}  en  mga  papers  ya. \\\smallskip
\gll Basi  \textbf{gubba}  en  mga  papers  ya. \\
perhaps  ruin  \textsc{cm}  \textsc{pl}  papers  \textsc{def.f} \\
\glt ‘Perhaps the papers \textbf{are ruined} now.’ [PTOE-T-01 207.1]
\z
\ea
\label{bkm:Ref119566298}
\textbf{Naigo}  kanen  ta  granada  tak  palimpyuan  din. \\\smallskip
\gll \textbf{Na-igo}  kanen  ta  granada  tak  pa-limpyo-an  din. \\
\textsc{a.hap.r}-hit  3\textsc{s.abs}  \textsc{nabs}  grenade  because  \textsc{t.r}-clean-\textsc{apl}  3\textsc{s.erg} \\
\glt ‘He \textbf{happened} \textbf{to} \textbf{be} \textbf{hit} by a grenade (explosion), because he cleaned it’ or ‘The grenade (explosion) happened to hit him because he cleaned it.’ [MBON-T-07a 14.3]
\z

\subsection{Non-volitional accomplishments (Class II)}

Examples \REF{bkm:Ref398301059}{}-\REF{bkm:Ref398301061} illustrate the root \textit{gisi} ‘torn/tear’, which is a non-volitional root of Class II. In \REF{bkm:Ref398301059} with no inflectional affixes it describes a state. In \REF{bkm:Ref367348391} with happenstantial marking it describes the perfective accomplishment of tearing. In \REF{bkm:Ref398301061} with dynamic marking the same root describes an imperfective activity, with no resultant change in state implied:

\ea
\label{bkm:Ref398301059}
\textbf{Gisi}  layag  i. \\\smallskip
\gll \textbf{Gisi}  layag  i. \\
tear  sail  \textsc{def.n} \\
\glt ‘The sail is \textbf{torn}.’\hspace{1cm}State
\z
\ea
\label{bkm:Ref367348391}
\textbf{Nagisi}  layag  i. \\\smallskip

\gll \textbf{Na-gisi}  layag  i. \\
\textsc{a.hap.r}-tear  sail  \textsc{def.n} \\
\glt ‘The sail \textbf{tore}.’\hspace{1cm}Accomplishment
\z
\ea
\label{bkm:Ref398301061}
Galayag  kay  nang  en  na  uļa  nay  nļami   daw  indi  kay  punta  asta  nang  en  na  \textbf{gagisi}  layag  i. \\\smallskip
\gll Ga-layag  kay  nang  en  na  uļa  nay  na-aļam-i   daw  indi  kay  punta  asta  nang  en  na  \textbf{ga-gisi}  layag  i.\\
\textsc{i.r}-sail  1\textsc{p.excl.abs}  only/just  \textsc{cm}  \textsc{lk}  \textsc{neg.r}  1\textsc{p.excl.erg}  \textsc{a.hap.r}-know-\textsc{xc.apl} if/when  where  1\textsc{p.excl.abs}  go  until  only/just  \textsc{cm}  \textsc{lk}   \textsc{i.r}-tear  sail  \textsc{def.n} \\
\glt ‘We sailed without knowing where we were going until the sail \textbf{was tearing/began to tear}.’ [VAWN-T-18 5.1]\hspace{1cm}Activity
\z

We hypothesize that the reason this verb (and others in Class II) may occur in the dynamic form is that it describes an event that is not necessarily punctual. It may take time. Something can “begin to tear”, it can “be tearing,” or it can “be torn.” This is in contrast to Class I verbs such as \textit{buong} ‘shatter’ which may only describe a state, “it is shattered” or a punctual achievement “it shattered.” A glass cannot “be shattering” or “begin to shatter,” therefore the root \textit{buong} may not describe an activity.

The following are some additional examples of Class II verbs from the corpus. Example \REF{bkm:Ref388967079} illustrates \textit{leddang}, ‘to sink’, in its basic sense:

\ea
\label{bkm:Ref388967079}
Na magsaļep  en  adlaw  an,  naan  kay  en  ta  tetenga \\\smallskip
\gll Na\footnotemark{}  mag-saļep  en  adlaw  an,  naan  kay  en  ta  te-tenga ta  saļangan  daw  \textbf{na-leddang}  kay  tak  sikad  na  selleg. \\
\textsc{lk}  \textsc{i.ir}-sunset  \textsc{cm}  sun/day  \textsc{def.m}  \textsc{spat.def}  1\textsc{p.excl.abs}  \textsc{cm}  \textsc{nabs}  \textsc{red}-middle \textsc{nabs}  passageway  and  \textsc{a.hap.r}-sink  1\textsc{p.excl.abs}  because  very  \textsc{lk}  current \\
\footnotetext{The \textit{na} at the beginning of this sentence is the linker that introduces adverbial clauses.}
\glt ‘When the sun was about to set, we were kind of in the middle of the passageway (through rocks/corals) and \textbf{we} \textbf{sank} because (it was) a strong current.’ [CBWN-C-11 4.1]
\z

Example \REF{bkm:Ref388967386} illustrates the same verb in dynamic modality, expressing an imperfective, inceptive sense:

\ea
\label{bkm:Ref388967386}
Paibitan  nay  ta  timbang  mama  na  duma  nay daw  muoy  piro  mama  an  \textbf{galeddang}  daw  nadaļa  din  man kami  tak  mama  i  bakod  kis-a  ki  kami.\\\smallskip
\gll Pa-ibit-an  nay  ta  timbang  mama  na  duma  nay daw  m-luoy  piro  mama  an  \textbf{ga-leddang}  daw  na-daļa  din  man kami  tak  mama  i  bakod  kis-a  ki  kami.\\
\textsc{t.r}-hold.on-\textsc{apl}  1\textsc{p.excl.erg}  \textsc{nabs}  balance  man  \textsc{lk}  companion  1\textsc{p.excl.gen} and  \textsc{i.v.ir}-swim  but  man  \textsc{def.m}  \textsc{i.r}-sink  and  \textsc{a.hap.r}-take  3\textsc{s.erg}  also  1\textsc{p.excl.abs}	because	man	\textsc{def.n}	big	than	\textsc{obl.p}	1\textsc{p.excl} \\
\glt ‘We held on to both sides of the man that was our companion and swam, but the man \textbf{was} \textbf{sinking} (or ‘\textbf{began} \textbf{to} \textbf{sink}’) and took us also (down with him) because the man was bigger than us.' [CBWN-C-11 4.8]
\z

The following illustrate the root \textit{ayad} ‘well/be well’ in the stative \REF{bkm:Ref148778216}, dynamic irrealis \REF{bkm:Ref148778256}, and happenstantial irrealis \REF{bkm:Ref148778286} forms: 

\ea
\label{bkm:Ref148778216}
Gapangamuyo  kay  man  na  kabay  na  \textbf{ayad}  en  Manang  ya aged  makabalik  man  kanen  ya  ta  Pilipinas  i.\\\smallskip
\gll Ga-pangamuyo  kay  man  na  kabay  na  \textbf{ayad}  en  Manang  ya aged  maka-balik  man  kanen  ya  ta  Pilipinas  i.\\
\textsc{i.r}-pray  1\textsc{p.excl.abs} too \textsc{lk}  may.it.be  \textsc{lk}   well  \textsc{cm}  Older.sister  \textsc{def.f} so.that  \textsc{i.hap.ir}-return  too  3\textsc{s.abs}  \textsc{def.f}  \textsc{nabs}  Philippines \textsc{def.n} \\
\glt ‘We prayed too that Manang may \textbf{be} \textbf{well} so that she can return to the Philippines.’ [PBWL-C-04 4.4]
\z

\ea
\label{bkm:Ref148778256}
Mag-ubra  danen  ta  duļot  agod  \textbf{mag-ayad}  ka. \\\smallskip
\gll Mag-ubra  danen  ta  duļot  agod  \textbf{mag-ayad}  ka. \\
\textsc{i.ir}-work/make  3\textsc{p.abs}  \textsc{nabs}  food.offering  so.that  \textsc{i.ir}-well  2\textsc{s.abs} \\
\glt ‘They will do a food offering so that you \textbf{will} \textbf{get} \textbf{well}.’ [SAWE-T-01 3.12]
\z
\ea
\label{bkm:Ref148778286}
Ambaļ  din  en  bisan  ino  pa  kon  ayuon  din  atag  din kon  basta  \textbf{mayad}  nang  kon  kanen  an. \\\smallskip
\gll Ambaļ  din  en  bisan  ino  pa  kon  ayo-en  din  \emptyset{}-atag  din kon  basta  \textbf{ma-ayad}  nang  kon  kanen  an. \\
say  3\textsc{s.erg}  \textsc{cm}  any  what  \textsc{inc}  \textsc{hsy}  request-\textsc{t.ir}  3\textsc{s.erg}  \textsc{t.ir}-give  3\textsc{s.erg} \textsc{hsy}   just.so.that  \textsc{a.hap.ir}-well  only/just  \textsc{hrs}  3\textsc{s.abs}  \textsc{def.m} \\
\glt ‘He said whatever else he asks for, he will give just so that he \textbf{will} \textbf{get} \textbf{well}.’ [PBON-T-01 4.2] 
\z

The following are additional Class II verbs in various forms:

\ea
Piro  ta,  parti  ta  mga  Kagayanen  nļaman  en  danen  daw  ino  iran  na  buaten  tak  kaļat  na  usaren  danen   \textbf{gabok}  en ...\\\smallskip
\gll Piro  ta,  parti  ta  mga  Kagayanen  na-aļam-an  en  danen  daw  ino  iran  na  buat-en  tak  kaļat  na  usar-en  danen   \textbf{gabok}  en ...\\
but  \textsc{nabs}  about  \textsc{nabs}  \textsc{pl}  Kagayanen  \textsc{a.hap.r}-know-\textsc{apl}  \textsc{cm}  3\textsc{p.erg}  if/when
what  3\textsc{p.gen}  \textsc{lk}  do/make-\textsc{t.ir}  because  rope  \textsc{lk}  use-\textsc{t.ir}  3\textsc{p.erg}  rotten  \textsc{cm} \\
\glt  ‘But about/concerning the Kagayanens, they knew what they will do, because the rope they will use \textbf{was rotten}...’  [EMWN-T-07 3.7]
\z
The story goes on to explain that when the Kagayanens did tug-of-war with the rotten rope, it would break and the unsuspecting enemies would fall over and be vulnerable to attack.

\ea
\textbf{Nabali}  gid  sanga  i  na  ake  i  na  patungtungan. \\\smallskip
\gll \textbf{Na-bali}  gid  sanga  i  na  ake  i  na  pa-tungtong-an. \\
\textsc{a.}\textsc{hap.r}-break  \textsc{int}  branch  \textsc{def.n}  \textsc{lk}  1\textsc{s.gen}  \textsc{def.n}  \textsc{lk}  \textsc{t.r}-on.top-\textsc{apl} \\
\glt ‘The branch on which I was (sitting) really \textbf{broke}.’ [DBWN-T-2 3.7]
\z

\ea
Ta  oras  ta  tag-uran  di  kan-o  gabaa  suba  an daw  sikad  tama  na  mga  batang  na  nadaļa  ta  baa  na  naan galin  ta  bukid  daw  \textbf{nadagsa}  naan  ta  baybay.\\\smallskip
\gll Ta  oras  ta  tag-uran  di  kan-o  ga-baa  suba  an daw  sikad  tama  na  mga  batang  na  na-daļa  ta  baa  na  naan ga-alin  ta  bukid  daw  \textbf{na-dagsa}  naan  ta  baybay.\\
\textsc{nabs}  hour/time  \textsc{nabs}  \textsc{nr}-rain  \textsc{d1loc}  previously  \textsc{i.r}-flood  river  \textsc{def.m}
and  very  many  \textsc{lk}  \textsc{pl}  driftwood  \textsc{lk} \textsc{a.}\textsc{hap.r}-carry  \textsc{nabs}  flood  \textsc{lk} \textsc{spat.def} \textsc{i.r}-from  \textsc{nabs}  mountain  and  \textsc{a.hap.r}-wash.ashore  \textsc{spat.def}  \textsc{nabs}  beach \\
\glt ‘In the times of rainy season here previously the river flooded and there were very many (pieces of) driftwood which were brought by the flood from the mountain and \textbf{washed} \textbf{ashore} on the beach.’ [DDWN-C-01 2.4]
\z

\ea
Antipara  din  an  yaan  a  \textbf{nadagsa}  ta  mga  tallo mitros  ta  lawa  din. \\\smallskip
\gll Antipara  din  an  yaan  a  \textbf{na-dagsa}  ta  mga  tallo mitros  ta  lawa  din. \\
goggles  3\textsc{s.gen}  \textsc{def.m}  \textsc{spat.def}  \textsc{inj} \textsc{a.hap.r}-wash.ashore  \textsc{nabs}  \textsc{pl} three
meters  \textsc{nabs}  body  3\textsc{s.gen} \\
\glt ‘His goggles were, well, \textbf{washed ashore} about three meters from his body.’ (A fisherman was missing and they found his body and goggles washed ashore.) [JCWN-T-26 15.8]
\z

\ea
Tapos  na  \textbf{gupod}  bagoļ    an  en  mangen  baga din  an,  ugsak  ta  plantsa.  Tapos  ugsak  ta  plantsa may  nasama  na  uling  tampekan  ta  pantad para  dili  \textbf{mupod}.\\\smallskip
\gll Tapos  na  \textbf{ga-upod}  bagoļ    an  en  kamang-en  baga din  an,  \emptyset{}-ugsak  ta  plantsa.  Tapos  …-ugsak  ta  plantsa may  na-sama  na  uling  \emptyset{}-tampek-an  ta  pantad para  dili  \textbf{ma-upod}.\\
after  \textsc{lk}  \textsc{i.r}-consumed  coconut.shell  \textsc{def.m}  \textsc{cm}  get-\textsc{t.ir}  coal
3\textsc{s.gen}  \textsc{def.m}  \textsc{t.ir}-put.inside \textsc{nabs} iron  after \textsc{t.r}-put.inside  \textsc{nabs} iron
\textsc{ext.in} \textsc{a.}\textsc{hap.r}-leftover  \textsc{lk}  coals  \textsc{t.ir}-pack.on-\textsc{apl}  \textsc{nabs}  sand
in.order  \textsc{neg.ir}  \textsc{a.hap.ir}-consumed \\
\glt ‘Then after the coconut shell is \textbf{being consumed} (getting smaller and smaller by the fire), get some of its coals and put (them) inside the iron. Then after putting it inside the iron, where there are some leftover coals, pack some sand on top so that (it) \textbf{will} \textbf{not} \textbf{be} \textbf{consumed} (by burning to nothing).’ (This is a text about how to iron clothes with an iron that uses coals.) [BMOP-C-07 2.4-5]
\z

\ea
Sikad  dessen  en  mga  basak  na  \textbf{naukab}  ta  \textbf{nalattik}. \\\smallskip
\gll Sikad  dessen  en  mga  basak  na  \textbf{na-ukab}  ta  \textbf{na-lattik}. \\
very  hard  \textsc{cm}  \textsc{pl}  soil/ground  \textsc{lk}  \textsc{a.hap.r}-loose  \textsc{nabs}  \textsc{a.hap.r}-crack \\
\glt ‘The ground that \textbf{came} \textbf{loose} from \textbf{having} \textbf{cracked} was very hard now.’ [JCWE-T-14 3.4]
\z

\ea
Ta  pagtikang  din  en  na  magtakkad  ta  basak, gulpi  nang  \textbf{galattik}  basak  ya  daw  kanen  galunod  dya daw  uļa  en  danen  nakita. \\\smallskip
\gll Ta  pag-tikang  din  en  na  mag-takkad  ta  basak, gulpi  nang  \textbf{ga-lattik}  basak  ya  daw  kanen  ga-lunod  dya daw  uļa  en  danen  na-kita. \\
\textsc{nabs}  \textsc{nr.act}-step  3\textsc{s.gen}  \textsc{cm}  \textsc{lk}  \textsc{i.ir}-step.on  \textsc{nabs} soil/ground suddenly  only/just  \textsc{i.r}-crack  soil/ground  \textsc{def.f}  and  3\textsc{s.abs}  \textsc{i.r}-drop.in  \textsc{d4loc} and  \textsc{neg.r}  \textsc{cm}  3\textsc{p.erg}  \textsc{a.hap.r}-see \\
\glt ‘When he took a step stepping on the ground, suddenly the ground \textbf{began} \textbf{to} \textbf{crack} and he dropped in there and they never found (him).’ [PBWN-C-12 22.1]
\z

Several roots, such as \textit{daeb} ‘to turn face down’ may present non-volitional situations with an inanimate Undergoer, or volitional situations with an animate Actor. As such these roots fall logically into Class II and Class IV. When the absolutive is an inanimate Undergoer they exhibit the affixation pattern of Class II roots, and when the absolutive is an animate Actor, they exhibit the pattern of Class IV roots. Here we give some examples of the Class II usage. The following examples are from the same text, and describe the same discourse event. However, in \REF{bkm:Ref148966026} happenstantial modality is used, while in \REF{bkm:Ref148965908} dynamic modality occurs. In \REF{bkm:Ref148966026} the act of turning over is non-volitional and is on the event line of the narrative. In this context, happenstantial modality is expected, and can be understood as expressing a semantic perfective aspect. 

\ea
\label{bkm:Ref148966026}
Pag-abot  ta  trisi  na  pagkaan  ki  yaken  na  sikad  gid  biskeg,   sakayan  ko  ya  \textbf{nadaeb}. \\\smallskip
\gll Pag-abot  ta  trisi  na  pag-kaan  ki  yaken  na  sikad  gid  biskeg,   sakay-an  ko  ya  \textbf{na-daeb}. \\
{nr.act}-arrive  \textsc{nabs}  thirteen  \textsc{lk}  \textsc{nr.act}-eat  \textsc{obl.p}  1s  \textsc{lk} very  \textsc{int}  strong
ride-\textsc{nr}  1\textsc{s.gen}  \textsc{def.f}  \textsc{a.hap.r}-turn.over \\
\glt  ‘When it reached the thirteenth (time of a fish) eating (the bait) from me which was very strong, my boat \textbf{turned} \textbf{over}.’ [EFWN-T-10 4.4]
\z

In \REF{bkm:Ref148965908} the same verb occurs, but this time it is not on the main event line of the narrative. It simply describes the condition of the boat that resulted from the event of turning over narrated earlier in the story. In this case, dynamic modality is appropriate. If happenstantial were used again in this context, the intention would be that the boat turned over a second time, constituting another event on the main event line of the story. This would have been possible, but highly unusual inside a relative clause. 

\ea
\label{bkm:Ref148965908}
Piro  naan  aren  ta  tudtod  ta  pambot  na  \textbf{gadaeb} na  sigi  en  anod. \\\smallskip
\gll Piro  naan  aren  ta  tudtod  ta  pambot  na  \textbf{ga-daeb} na  sigi  en  anod. \\
but  \textsc{spat.def}  1\textsc{s.abs}  \textsc{nabs}  back  \textsc{nabs}  boat  \textsc{lk}  \textsc{i.r}-turn.over
\textsc{lk} continuously  \textsc{cm}  drift \\
\glt ‘But I was on the back of the boat which had turned over that was continuously now adrift.’ [EFWN-T-11 13.3]
\z

In \sectref{bkm:Ref149230430} we discuss examples of \textit{daeb} as a volitional, Class IV, root.  

\subsection{Non-volitional activities (Class III)}
\label{sec:non-volitionalactivities-classIII}
Class III intransitive verbs describe atelic activities that result in no change of state. Therefore, these verbs do not normally occur in the resultant state or happenstantial forms:

\ea
*\c{L}egļeg apoy an.  (‘The fire is flamed.’)  (STATE) \\
*Naļegļeg apoy an.  (‘The fire flamed.’)  (ACHIEVEMENT)
\z

Examples \REF{bkm:Ref442966831} through \REF{bkm:Ref444436899} illustrate Class III verbs from the corpus:  

\ea
\label{bkm:Ref442966831}
Mga  mata  ko  na  galuag  paryo  ta  apoy  na  \textbf{gaļegļeg}. \\\smallskip
\gll Mga  mata  ko  na  ga-luag  paryo  ta  apoy  na  \textbf{ga-ļegļeg}. \\
\textsc{pl} eye  1\textsc{s.gen}  \textsc{lk}  \textsc{i.r}-watch  like  \textsc{nabs}  fire  \textsc{lk}  \textsc{i.r}-flame \\
\glt ‘My eyes that were watching were like a fire which \textbf{is} \textbf{flaming}.’ [JCOO-T-11 10.1]
\z

\ea
Pagkita  nay  na  apoy  an  \textbf{gaubļo}  na  sikad  gid  bakod   dabadaba,  dayon  kay  panaog  daw  mļagan  naan  punta   ta   Sintro. \\\smallskip
\gll Pag-kita  nay  na  apoy  an  \textbf{ga-ubļo}  na  sikad  gid  bakod   dabadaba,  dayon  kay  …-panaog  daw  m-dļagan  naan  punta   ta   Sintro. \\
\textsc{nr.act}-see  1\textsc{p.excl.gen}  \textsc{lk}  fire  \textsc{def.m}  \textsc{i.r}-flare.up  \textsc{lk}  very  \textsc{int}  big
big.flames  right.away  1\textsc{p.excl.abs}  \textsc{i.r}-go.down  and  \textsc{i.v.ir}-run  \textsc{spat.def} going
\textsc{nabs}  Central \\
\glt ‘When we saw the fire \textbf{flaring} \textbf{up}  with very big flames, we immediately went down and ran going to Central.’ [RZWN-T-02 2.9]
\z

\ea
Daw  gagilek  sikad  bakod  kagi  na  taning  daw  \textbf{gakereg}  paryo gid  ta  naakad  blengngan  din  an.\\\smallskip
\gll Daw  ga-gilek  sikad  bakod  kagi  na  taning  daw  \textbf{ga-kereg}  paryo gid  ta  na-akad  blengngan  din  an.\\
if/when  \textsc{i.r}-angry  very  big  voice  \textsc{lk}  high.pitch  and  \textsc{i.r}-shake  like
\textsc{int}  \textsc{nabs}  \textsc{a.hap.r}-come.apart  throat  3\textsc{s.gen}  \textsc{def.m} \\
\glt ‘When getting angry, it is a very loud high sound and \textbf{it} \textbf{shakes} really like its throat has come apart.’ (This is a description of the sound of a cat.) [JCWE-T-14 13.2]
\z

\ea
Bilang  pabugtawen  no  gid  kon  anay  ittaw  an  na \textbf{ganarem}  bag-o  no  tuturan  lampraan. \\\smallskip
\gll Bilang  pa-bugtaw-en  no  gid  kon  anay  ittaw  an  na \textbf{ga-narem}  bag-o  no  \emptyset{}-tutod-an  lampraan. \\
for.example  \textsc{caus}-wake-\textsc{t.ir}  2\textsc{s.erg}  \textsc{int}  \textsc{hsy}  first/for.a.while  person  \textsc{def.m}  \textsc{lk}
\textsc{i.r}-have.ISP  before  2\textsc{s.erg}  \textsc{t.ir}-light-\textsc{apl}  lamp \\
\glt ‘For example, first wake up the person who \textbf{has} \textbf{Isolated} \textbf{Sleep} \textbf{Paralysis} (ISP) before you light a lamp.’ [ETON-C-07 4.5]
\z

\ea
Daw  \textbf{gabagting}  langganay  ta  simbaan  ta  miad, tanda  na  anen  en  mga  gubat. \\\smallskip
\gll Daw  \textbf{ga-bagting}  langganay  ta  simba-an  ta  miad, tanda  na  anen  en  mga  gubat. \\
if/when  \textsc{i.r}-ring  bell  \textsc{nabs}  worship-\textsc{nr} \textsc{nabs}  well sign  \textsc{lk}  \textsc{ext.g}  \textsc{cm}  \textsc{pl}  raider \\
\glt ‘Whenever the bell of the church \textbf{was} \textbf{ringing} hard, it was the sign that the raiders are here.’ [JCWN-T-20]
\z

\ea
\label{bkm:Ref444436899}
Daw  may  \textbf{gadagbeng}  na  nuļog,  isipen  no tak  yon nan  niog  na  naipo  ko. \\\smallskip
\gll Daw  may  \textbf{ga-dagbeng}  na  na-uļog,  isip-en  no tak  {yon} {nan}  niog  na  na-ipo  ko. \\
if/when  \textsc{ext.in}  \textsc{i.r}-thud  \textsc{lk}  \textsc{a.hap.r}-fall  think/count-\textsc{t.ir}  2\textsc{s.erg} because  \textsc{d}3\textsc{abs} \textsc{d}3\textsc{abs.pr}  coconut  \textsc{lk}  \textsc{a.hap.r}-pick  1\textsc{s.erg} \\
\glt ‘If something falls \textbf{making} \textbf{a} \textbf{thud} \textbf{sound}, count it because that very one is the coconut I have picked.’ (The speaker climbs a tree to get coconuts and he tells the blind guy on the ground to count how many thuds he hears because that will be the coconuts that he picks and lets fall to the ground. But the one who climbs the tree keeps falling out and so the blind guy counts each time the other guy falls out of the tree thinking it is a coconut.) [CBWN-C-15 4.4]
\z

\subsection{Class I-III roots in transitive and detransitive frames}
\label{bkm:Ref148791529}
Most Class I-III verbs can be used in a transitive frame with no transitivizing stem-forming morphology (applicative or causative). In this case they express direct causation. Example \REF{ex:hasdissolved} illustrates the Class II root \textit{tunaw} ‘to dissolve/melt’ in its basic intransitive frame, while example \REF{bkm:Ref395126196} illustrates the same root in a transitive, causative, frame:

\ea
\label{ex:hasdissolved}
\textbf{Natunaw}  tubuyong  an. \\\smallskip
\gll \textbf{Na-tunaw}  tubuyong  an. \\
\textsc{a.hap.r}-dissolve  manioc.flour  \textsc{def.m} \\
\glt ‘This manioc flour \textbf{has} \textbf{dissolved}.’  
\z
\ea
\label{bkm:Ref395126196}
\textbf{Tunawen}  no  anay  tubuyong  an  para  pangmiroļ. \\\smallskip
\gll \textbf{Tunaw-en}  no  anay  tubuyong  an  para  pang-miroļ. \\
dissolve-\textsc{t.ir}  2\textsc{s.erg}  first/for.a.while  manioc.flour  \textsc{def.m}  for  \textsc{inst}-clothes.starch \\
\glt ‘\textbf{Dissolve} please the manioc flour for use as starch.’
\z

Similarly, example \REF{bkm:Ref398385527} illustrates the verb \textit{tumba} ‘to fall over’ in its basic, intransitive frame, while \REF{bkm:Ref398385530} illustrates the same verb in a transitive, causative frame:

\ea
\label{bkm:Ref398385527}
\textbf{Natumba}  kaoy  ya  naan  ta  tinanem nay  na  mga  gulay. \\\smallskip
\gll \textbf{Na-tumba}  kaoy  ya  naan  ta  t<in>anem nay  na  mga  gulay. \\
\textsc{a.hap.r}-fall.over  tree  \textsc{def.f}  \textsc{spat.def}  \textsc{nabs}  <\textsc{nr.res}>plant
1\textsc{p.excl.gen}  \textsc{lk}  \textsc{pl}  vegetables \\
\glt ‘The tree fell over on our planted vegetables.’ 
\z

\ea
\label{bkm:Ref398385530}
\textbf{Patumba}  din  kaoy  ya  naan  ta  tinanem nay  na  mga  gulay. \\\smallskip
\gll \textbf{Pa-tumba}  din  kaoy  ya  naan  ta  t<in>anem nay  na  mga  gulay. \\
\textsc{t.r}-fall.over  3\textsc{s.erg}  tree  \textsc{def.f}  \textsc{spat.def}  \textsc{nabs}  <\textsc{nr.res}>plant
1\textsc{p.excl.gen}  \textsc{lk}  \textsc{pl}  vegetables \\
\glt ‘S/he felled the tree (on our planted vegetables).’
\z

Example \REF{bkm:Ref119509106} illustrates \textit{leddang} ‘to sink’ in a transitive form, meaning ‘cause to sink’: 

\ea
\label{bkm:Ref119509106}
\textbf{Paleddang}  ta  pirata  pambot  nay  ya. \\\smallskip
\gll \textbf{Pa-leddang}  ta  pirata  pambot  nay  ya. \\
\textsc{t.r}-sink  \textsc{erg}  pirate  motor.boat  1\textsc{p.excl.gen}  \textsc{def.f} \\
\glt ‘The pirates \textbf{sank} our motor boat.’
\z

We know that the prefix \textit{pa}{}- in  \REF{bkm:Ref398385530} and \REF{bkm:Ref119509106} is the transitive, realis \textit{pa}{}- rather than the causative for several reasons. First, if \textit{pa}{}- in these examples were the causative, the predicates would imply indirect causation, as though the boat retained some responsibility for its own sinking. Second, this \textit{pa}{}- is not retained in the irrealis (\ref{bkm:Ref395126196}, \ref{bkm:Ref123281189}) or detransitive (Actor voice) form \REF{bkm:Ref123281290}:


\ea
\label{bkm:Ref123281189}
\textbf{Leddangen}  nyo  lunday  an  tak  sikad  bao. \\\smallskip
\gll \textbf{Leddang-en}  nyo  lunday  an  tak  sikad  bao. \\
sink-\textsc{t.ir}  2\textsc{p.erg}  outrigger.canoe  \textsc{def.m}  because  very  odor \\
\glt ‘Sink the outrigger canoe because it smells bad.’ (The outrigger canoe probably has the odor of rotten fish and sinking it will wash it out.)
\z

\ea
\label{bkm:Ref123281290}
\textbf{Galeddang}  mga  pirata  ta  pambot  nay  ya. \\\smallskip
\gll \textbf{Ga-leddang}  mga  pirata  ta  pambot  nay  ya. \\
\textsc{i.r}-sink  \textsc{pl}  pirate  \textsc{nabs}  motor.boat  1\textsc{p.excl.gen}  \textsc{def.f} \\
\glt ‘The pirates sank our motor boat.’\smallskip

* gapaleddang ...
\z

Finally, there is a causative form of this root, expressed with the causative prefix \textit{pa}{}- in addition to the transitive inflections (realis in \ref{bkm:Ref123281737} and irrealis in \ref{bkm:Ref148779410}):

\ea
\label{bkm:Ref123281737}
\textbf{Papaleddang}  din  ki  yaken  pambot  ya. \\\smallskip
\gll \textbf{Pa-pa-leddang}  din  ki  yaken  pambot  ya. \\
\textsc{t.r}-\textsc{caus}-sink  3\textsc{s.erg}  \textsc{obl.p}  1s  motor.boat  \textsc{def.f} \\
\glt ‘S/he caused/let/allowed me to sink the motor boat.’
\z

\ea
\label{bkm:Ref148779410}
\textbf{Paleddangen}  din  ki  yaken  pambot  ya. \\\smallskip
\gll \textbf{Pa-leddang-en}  din  ki  yaken  pambot  ya. \\
\textsc{caus}-sink-\textsc{t.ir}  3\textsc{s.erg}  \textsc{obl.p}  1s  motor.boat  \textsc{def.f} \\
\glt ‘S/he will cause/let/allow me to sink the motor boat.’
\z

Most or all verbs that describe transitive situations may occur in an intransitive frame in which the absolutive is the controller of the situation and the Undergoer is either omitted or placed in an oblique role, marked by a pre-nominal particle \textit{ta} or \textit{ki}. These are similar in function to English expressions such as \textit{Frodo already ate}, \textit{she drank of the water}, \textit{he kicked at the ball} and so on. These are all grammatically intransitive expressions of semantically transitive\is{semantic transitivity}\is{transitivity!semantic} situations. In each example the meaning involves an Undergoer, but because of the communicative context, the speaker chooses to “downplay” or omit reference to the Undergoer. Analogous constructions exist in Kagayanen, with the difference being that the verbs are explicitly marked morphologically as intransitive. We have described such constructions as \textit{detransitive constructions} because they present semantically transitive\is{semantic transitivity}\is{transitivity!semantic} situations in grammatically intransitive forms (see \chapref{chap:verbstructure}, \sectref{sec:grammaticaltransitivity}, and \chapref{chap:voice}, \sectref{sec:actorvoice}).

Since Class I, II and III roots may occur in transitive frames, expressing direct causation, they may also occur in detransitive frames, with the causer as the absolutive and the Undergoer either omitted or expressed in the non-absolutive case:

\ea
Class I: \textit{uļog} ‘to fall’: \\
Ta  pagpadayon,  \textbf{gauļog}  a  pa  gid  ta  ake  na paan  daw  naylon.\\\smallskip
\gll Ta  pag-pa-dayon,  \textbf{ga-uļog}  a  pa  gid  ta  ake  na paan  daw  naylon.\\
\textsc{nabs}  \textsc{nr.act}-\textsc{caus}-continue  \textsc{i.r}-fall  1\textsc{s.abs}  \textsc{inc}  \textsc{int}  \textsc{nabs} 1\textsc{s.gen}  \textsc{lk}
bait    and  fish.line \\
\glt ‘In continuing, I dropped yet again my bait and fish line.’ [EFWN-T-11 6.1]
\z

\ea
Class II basa ‘to get wet’ and Class III \textit{sablig} ‘to splash’ and \textit{takļap} ‘to  get covered’: \\
Duma  an  \textbf{gasablig}  ta  waig  daw  \textbf{gabasa}  ta  sako daw  labyog  ki  Pedro  na  \textbf{takļap}  ta  atep  aged  mapatay apoy  an.\\\smallskip
\gll Duma  an  \textbf{ga-sablig}  ta  waig  daw  \textbf{ga-basa}  ta  sako daw  …-labyog  ki  Pedro  na  …-\textbf{takļap}  ta  atep  aged  ma-patay apoy  an.\\
others  \textsc{def.m}  \textsc{i.r}-splash  \textsc{nabs} water  and  \textsc{i.r}-wet  \textsc{nabs}  sack and  \textsc{t.r}-throw  \textsc{obl.p}  Pedro  \textsc{lk}  \textsc{i.r}-cover.over  \textsc{nabs}  roof  so.that  \textsc{a.hap.ir}-extinguish fire  \textsc{def.m} \\
\glt ‘Others splashed water and wetted some sacks and threw (them) to Pedro to cover over the roof so that the fire will exinguish.’ [RZWN-T-02 3.14]
\z

\section{Roots describing volitional intransitive situations (Class IV)}
\label{bkm:Ref149230430}

The next major Class of semantically intransitive\is{semantic transitivity}\is{transitivity!semantic} verbs involve volition on the part of the only obligatory argument. In other words, the absolutive argument refers to the participant that controls the situation. These include the following:
\ea
\label{bkm:Ref120784669}
\begin{multicols}{2}
amag ‘to want to go with s.o.’\\
ampang ‘to play’\\
anges ‘to pant’\\
arti ‘to act’\\
batok ‘to rebel’\\
bayli ‘to slow dance’\\
bugtaw ‘to wake up’\\
bukot ‘to stay inside’\\
bunak 'to do laundry' \\
daag ‘to win’\\
daik ‘to crawl on stomach’\\
dayon ‘to stay somewhere’\\
demel ‘to lower the head, look downwards’ \\
dļagan ‘to run’\\
duļak ‘for animals to fight’\\
dungka ‘to dock’\\
eseb ‘to go underwater’\\
giba ‘to sit on a lap’\\
gumod ‘to mumble/grumble’\\
iling ‘to go’\\
isoļ ‘to go backward’\\
istar ‘to reside’\\
iyod ‘to stretch oneself’\\
kangkang ‘to extend’\\
kawas ‘to disembark’\\
kilip ‘to look out the corner  of the   eye’\\
kingking ‘to hop’\\
kipat ‘to wink’\\
laeg ‘to joke’\\
lagaw ‘to go out visiting/strolling’\\
lagmi ‘to speak loudly, angrily’\\
lakas ‘to return home from distant place’\\
lambay ‘to go past’\\
langoy ‘to bathe, swim’\\
laog ‘to go out without permission’\\
larga ‘to depart, travel’\\
layas ‘to run away, flee’\\
leeb ‘to bow’\\
ļekkep ‘to spread out on the surface’ \\
lenge ‘to move head right and left’\\
liad ‘to arch back’\\
libot ‘to go around’\\
limos ‘to beg’\\
luko ‘for an animal to lie down’\\
lumba ‘to race’\\
luod ‘to kneel’\\
luoy ‘to swim’\\
lusko/lukso ‘to jump’\\
negga ‘to lie down’\\
ngusmod ‘to frown’\\
panaw ‘to leave, go,  walk’\\
pasyar ‘to go out visiting/strolling’\\
pattik ‘to flick with finger’\\
pikpik ‘to tap, pat’\\
plastar ‘to take one’s position’\\
puay ‘to rest’\\
puļaw ‘to stay up late’\\
pungko ‘to sit down’\\
pyangka ‘to sit cross-legged’\\
pyungkot ‘to lie in fetal position’\\
sakay ‘to ride’\\
sampet ‘to arrive briefly’\\
sandig ‘to lean’\\
sayaw ‘to dance’\\
segseg ‘to move over’\\
seļ{}-et ‘to squeeze into a small space’\\
suko ‘to surrender’\\
tagad ‘to wait’\\
takas ‘to go up a hill’\\
tambong ‘to attend’\\
tangkis ‘to grin’\\
tawa ‘to smile, laugh’\\
tegbeng ‘to go down a hill’\\
tindeg ‘to stand (up)’\\
tinir ‘to stay temporarily’\\
tipon ‘to gather together’\\
tanuga/tinuga/tunuga/nuga ‘to go to sleep’\\
tuwad ‘to bend over’\\
uli ‘to go home’\\
untat ‘to stop’\\
utad ‘to step on’\\
uyok ‘to whistle’\\
\end{multicols}
\z

Note that all of the roots in \REF{bkm:Ref120784669} describe activities; that is volitional situations that do not result in a change of state. Therefore, it is not surprising that the basic affixes for these verbs are the dynamic \textit{ga}{}- realis and \textit{m}{}- or \textit{mag}{}- for irrealis modality. In contrast to the non-volitional intransitive verbs, many of these verbs, but not all, can occur with the \textit{m}{}- prefix, in which case the root-initial consonant drops out. Another characteristic of these volitional intransitive verbs is that in happenstantial modality, they occur with  \textit{naka}{}-/\textit{maka}{}-, \textit{ka}{}- (and \textit{ma}{}-  in the hypothetical/polite usage), but do not occur with the ambitransitive forms \textit{na-} or \textit{ma-}. Thus we can make the generalization that in happenstantial modality, constructions in which the absolutive is an Actor take \textit{naka-/maka}{}- and those in which the absolutive is an Undergoer take \textit{na-/ma}{}-.
\ea
Dynamic: \\
\textbf{Gapuay}  kay  uļa  nang  lugay  daw  mag-igma. \\\smallskip
\gll  \textbf{Ga-puay}  kay  uļa  nang  lugay  daw  mag-igma. \\
\textsc{i.r}-rest  1\textsc{p.excl.abs}  \textsc{neg.r} only/just  long.time  and  \textsc{i.ir}-lunch \\
\glt ‘We were resting/rested for a short time and ate lunch.’ [DBWN-T-24 3.2]
\z
\ea
Happenstantial: \\
\textbf{Nakapuay}  kay  uļa  nang  lugay. \\\smallskip
\gll \textbf{Naka-puay}  kay  uļa  nang  lugay. \\
  \textsc{i.hap.r}-rest  1\textsc{p.excl.abs}  \textsc{neg.r} only/just  long.time \\
\glt  ‘We happened to rest for a short time.’ \\
  ‘We were able to rest for a short time.’ \\
  ‘We got to rest for a short time.’ \\
  ‘We have rested for a short time.’ \\\smallskip
  *Napuay kay uļa nang lugay.
\z

\ea
Dynamic: \\
\textbf{Gatinir}  a  annem  buļan  ta  maistra  i. \\\smallskip
\gll \textbf{Ga-tinir}  a  annem  buļan  ta  maistra  i. \\
\textsc{i.r}-stay  1\textsc{s.abs}  six  month  \textsc{nabs}  teacher  \textsc{def.n} \\
\glt ‘I stayed for six months with the teacher.’ [DBWN-T-21 2.6]
\z
\ea
Happenstantial: \\
\textbf{Nakatinir}  a  annem  buļan. \\\smallskip
\gll \textbf{Naka-tinir}  a  annem  buļan. \\
\textsc{i.hap.r}-stay  1\textsc{s.abs}  six  month \\
\glt ‘I happened to stay for six months.’ \\
  ‘I was able to stay for six months.’ \\
  ‘I got to stay for six months.’ \\
  ‘I have stayed for six months.’ \\\smallskip
*Natinir a annem buļan ta maistra i.
\z

Another grammatical difference between non-volitional and volitional intransitive roots is that non-volitional roots may easily occur in a transitive frame to express direct causation (see \sectref{bkm:Ref148856716} above). Class IV roots, on the other hand, almost always require explicit transitivizing stem-forming morphology, either causative (\textit{pa}{}-) or applicative (-\textit{an}, -\textit{i}) in order to occur in a transitive frame. This is because the causee is often animate, and therefore retains some control over the situation. However, if the causee is inanimate or otherwise incapable of controlling the situation, some of these roots may express direct causation in a transitive frame without the stem-forming causative prefix. For example, roots such as \textit{balik} ‘return’, \textit{tago} ‘hide’ and others that describe motion to a destination (return to a place, hide in a place) when used intransitively describe volitional situations \REF{bkm:Ref148946728}. When the same roots describe caused situations in which the causee is capable of exercising some control over the situation, the causative prefix is used, as in \REF{bkm:Ref148968257}. However, the same verbs may occur in a transitive \REF{bkm:Ref148946793} or detransitive \REF{bkm:Ref148946893} construction with no causative prefix, in which case they imply that the causee is probably inanimate, and therefore has no control or volition. In other words, as with root Classes I-III, this construction expresses direct causation:

\ea
\label{bkm:Ref148946728}
Intransitive (volitional): \\
Lugay  na  tapos  kay  ipo  atis  en \textbf{gabalik}  kami  naan  baybay. \\\smallskip
\gll Lugay  na  tapos  kay  ipo  atis  en \textbf{ga-balik}  kami  naan  baybay. \\
long.time  \textsc{lk}  finish  1\textsc{p.excl.abs}  pick  sugar.apple  \textsc{cm}
\textsc{i.r}-return  1\textsc{p.excl.abs}  \textsc{spat.def}  beach \\
\glt ‘After some time when we finished picking sugar apples, we returned to the beach.’ [DBON-C-06 6.1]
\z

\ea
\label{bkm:Ref148968257}
Indirect causative (volitional causee): \\
  \textbf{Papabalik} ko kanen an naan ta baļay danen. \\\smallskip
\gll  Pa-pa-balik  ko  kanen  an  naan  ta  baļay  danen. \\
  \textsc{t.r}-\textsc{caus}-return  1\textsc{s.erg}  3\textsc{s.abs}  \textsc{def.m}  \textsc{spat.def}  \textsc{nabs}  house  3\textsc{p.gen} \\
\glt  ‘I made/let him/her return to their house.’
\z
\ea
\label{bkm:Ref148946793}
Direct causation (non-volitional causee):\\
\textbf{Pabalik}  ko  en  kuguran  din  ya  naan  ta  baļay  danen. \\\smallskip
\gll \textbf{Pa-balik}  ko  en  kugod-an  din  ya  naan  ta  baļay  danen. \\
\textsc{t.r}-return  1\textsc{s.erg}  \textsc{cm}  grate.coconut-\textsc{nr}  3\textsc{s.gen}  \textsc{def.f}  \textsc{spat.def}  \textsc{nabs}  house  3\textsc{p.gen} \\
\glt ‘I returned his/her coconut grater to their house.’
\z

\ea
\label{bkm:Ref148946893}
Detransitive of causative: \\
\textsc{Gabalik}  aren  ta  kuguran  din  ya  naan  ta baļay  danen. \\\smallskip
\gll \textsc{Ga-balik}  aren  ta  kugod-an  din  ya  naan  ta baļay  danen. \\
\textsc{i.r}-return  1\textsc{s.abs}  \textsc{nabs}  grate.coconut-\textsc{nr}  3\textsc{s.gen}  \textsc{def.f}  \textsc{spat.def}  \textsc{nabs}  house  3\textsc{p.gen} \\
\glt ‘I \textbf{returned} his/her coconut grater to their house.’
\z

Other volitional motion verbs of Class IV, such as \textit{larga} ‘depart’, never take a location as an oblique or Undergoer argument (\ref{bkm:Ref148793705}). For such verbs, a destination is expressed in a complement clause \REF{bkm:Ref148793741}, preceded by the linker/complementizer \textit{na}:

\ea
\label{bkm:Ref148793705}
Ambaļ  ta  rais  ya  \textbf{marga}  gid  kon  tak \textbf{galarga}  kon  en  duma  ya  na  bļangay. \\\smallskip
\gll Ambaļ  ta  rais  ya  \textbf{m-larga}  gid  kon  tak \textbf{ga-larga}  kon  en  duma  ya  na  bļangay. \\
say  \textsc{nabs}  captain  \textsc{def.f}  \textsc{i.v.ir}-depart  \textsc{int}  \textsc{hsy}  because
\textsc{i.r}-depart  \textsc{hsy}  \textsc{cm}  other  \textsc{def.f}  \textsc{lk}  2.mast.boat \\
\glt ‘The captain said to really \textbf{depart} because the other two-mast boats are \textbf{already} \textbf{departing}.’ [VAWN-T-18 3.5]
\z

\ea
\label{bkm:Ref148793741}
\textbf{Galarga}  kay  kan-o  na  miling  Iloilo. \\\smallskip
\gll \textbf{Ga-larga}  kay  kan-o  na  m-iling  Iloilo. \\
\textsc{i.r}-depart  1\textsc{p.excl.abs}  previously  \textsc{lk}  \textsc{i.v.ir}-go  Iloilo \\
\glt ‘\textbf{We} \textbf{departed} previously to go to Iloilo.’
\z

The volitional intransitive verb, \textit{tago} ‘to hide’, may express a location optionally as an oblique (\ref{bkm:Ref148793817}), but this is not a detransitive of direct causation because it cannot mean ``the younger sibling hid a well":

\ea
\label{bkm:Ref148793817}
Piro  mangngod  i  \textbf{gatago}  naan  ta  bubon para  dili  makita.\\\smallskip
\gll Piro  mangngod  i  \textbf{ga-tago}  naan  ta  bubon para  dili  ma-kita.\\
but  younger.sibling  \textsc{def.n}  \textsc{i.r}-hide  \textsc{spat.def}  \textsc{nabs}  well
for  \textsc{neg.ir}  \textsc{a.hap.ir}-see \\
\glt ‘But the younger sibling \textbf{hid} in the well in order not to be seen.’ [BCWN-C-04 6.8]
\z

The location of hiding may be absolutive, but only with the addition of the applicative -\textit{an}:

\ea
\textbf{Pataguan}  din  bubon  ya. \\\smallskip
\gll \textbf{Pa-tago-an}  din  bubon  ya. \\
\textsc{t.r}-hide-\textsc{apl}  3\textsc{s.erg}  well  \textsc{def.f} \\
\glt ‘S/he \textbf{hid} in the well.’
\z

Example \REF{bkm:Ref148947015} illustrates the volitional intransitive motion root, \textit{iling} ‘to go’, with an obligatory destination expressed as an oblique \REF{bkm:Ref148947015} or as absolutive with applicative, \REF{bkm:Ref148947027} and \REF{bkm:Ref148947052}:

\ea
\label{bkm:Ref148947015}
Kanen  i  \textbf{giling}  ta  yi  na  puļo  paagi  ta pagsakay  ta  lunday  para  manglaya.\\\smallskip
\gll Kanen  i  \textbf{ga-iling}  ta  yi  na  puļo  paagi  ta pag-sakay  ta  lunday  para  ma-ng-laya.\\
3\textsc{s.abs}  \textsc{def.n}  \textsc{i.r}-go  \textsc{nabs}  \textsc{d}1\textsc{adj}  \textsc{lk}  island  by.means  \textsc{nabs} \textsc{nr.act}-ride  \textsc{nabs}  outrigger.canoe  for  \textsc{a.hap.ir-pl}-cast.net \\
\glt ‘He \textbf{went} to this island by means of riding an outrigger canoe in order to cast-net fish.’ [VAWN-T-17 2.2]
\z

\ea
\label{bkm:Ref148947027}
Pamiro,  first  time,  ko  na  \textbf{pailingan}  Puerto. \\\smallskip
\gll Pamiro,  first  time,  ko  na  \textbf{pa-iling-an}  Puerto. \\
first  first  time  1\textsc{s.erg}  \textsc{lk}  \textsc{t.r}-go-\textsc{apl}  Puerto. \\
\glt ‘First, first time, I \textbf{went} to Puerto.’ [BMON-C-05 1.5]
\z
\ea
\label{bkm:Ref148947052}
Kaysan  bisan  puon  ta  mga  darko  na  mga  kaoy  paryo  ta  baliti, kumpang  o  bugo  \textbf{pailingan}  daw  ambaļen  dya  na, “Anen  en  bata  an.”\\\smallskip
\gll Kaysan  bisan  puon  ta  mga  darko  na  mga  kaoy  paryo  ta  baliti, kumpang  o  bugo  \textbf{pa-iling-an}  daw  ambaļ-en  dya  na, “Anen  en  bata  an.”\\
sometimes  even  trunk  \textsc{nabs}  \textsc{pl}  large.\textsc{pl}  \textsc{lk}  \textsc{pl}  trees  same/like  \textsc{nabs}  ficus wild.almond  or  garuga.floribunda  \textsc{t.r}-go-\textsc{apl}  and  say-\textsc{t.ir}  \textsc{d}4\textsc{loc}  \textsc{lk} \textsc{ext.g}  \textsc{cm}  child  \textsc{def.m} \\
\glt ‘Sometimes even the trunks of large trees like ficus, wild almond or garuga floribunda (they) \textbf{go} to and say there, “Here is the child.”' [JCWE-T-15 4.4]
\z

Other volitional intransitive, mostly motion+manner  roots, may also occur in a transitive frame without valence increasing morphology, but with different semantic effect. For these verbs, the meaning in a grammatically transitive frame is that the Undergoer is the destination of the Actor’s motion, or something retrieved by the Actor. For example \textit{luoy} ‘to swim’ with transitive inflection,  \textit{paluoy} or \textit{luuyon}, means ‘to swim to a specific place’ or ‘to swim to get a specific object’. Other verbs that function similarly are \textit{daik} ‘to crawl’, \textit{eseb} ‘to go underwater’, \textit{panaw} ‘to go/walk’, and \textit{dļagan} ‘to run’.

\ea
Di  mugpa  ka  daw  \textbf{luuyon}  no  dya  ta  Nusa. \\\smallskip
\gll Di  m-tugpa  ka  daw  \textbf{luoy-en}  no  dya  ta  Nusa. \\
\textsc{inj}  \textsc{i.v.ir}-jump.down  2\textsc{s.abs}  and  swim-\textsc{t.ir}  2\textsc{s.erg}  \textsc{d4loc}  \textsc{nabs}  Nusa \\
\glt `So, what else, won’t you jump down (in the sea) and \textbf{swim} to get them, (shells that were left behind, known from context) there in Nusa?’ [DBWN-T-24 5.2] 
\z
\ea
\textbf{Paeseb}  din  sundang  an  na  nuļog na  gatebteb  ta  tilik. \\\smallskip
\gll \textbf{Pa-eseb}  din  sundang  an  na  na-uļog na  ga-tebteb  ta  tilik. \\
\textsc{t.r}-go.underwater  3\textsc{s.erg}  machete  \textsc{def.m}  \textsc{lk}  \textsc{a.hap.r}-fall \textsc{lk}  \textsc{i.r}-chop  \textsc{nabs}  sea.urchin \\
\glt ‘He went underwater to get the machete that fell when (he was) chopping sea urchin.’ 
\z

Most Class IV roots can take both happenstantial and dynamic inflections, depending on whether the scene being described involves volition or not. A short list of such roots is given in \REF{bkm:Ref148949595}:

\ea 
\label{bkm:Ref148949595}
\begin{multicols}{2} seddep  ‘to go into a small space or opening’ \\
  teneng ‘to cease’ \\
  tumpok ‘to heap/bunch up together’ \\
  ligid  ‘to roll’ \\
  tungtong  ‘to be on top’ \\
  tunton  ‘to hang down \\
  liped ‘to block a view or hide behind’ \\
  abot ‘to arrive and stay’ \\
  alin ‘come from’ \\
  atras  ‘to recede/go/move back’ \\
  balik  ‘to return somewhere’ \\
  daeb ‘to lie face down’ \\
  gwa ‘to go out of/outside’ \\
  selled ‘to go inside of/inside’ \\
  lapta ‘to disperse’ \\
  ligid ‘to roll over’ \\
  takilid ‘to lie on side’ \\
  tengeb ‘to be together’ \\
  unat ‘to stretch out’ \\
  mukļat ‘to open eyes’ \\
  peddeng ‘to close eyes’
\end{multicols}
\z

The following are a few examples of such roots in context. In example \REF{bkm:Ref148966154}, the root \textit{seddep} ‘go into a small space’ is used in a transitive frame with happenstantial modality. In this case the ants are presented as a “substance” that happened to enter, was able to enter, or had entered the speaker’s ear. In example \REF{bkm:Ref148966156} the same verb occurs in dynamic (volitional) modality. In this case, the ants are presented as “actors”, purposely entering the speaker’s ear:

\ea
\label{bkm:Ref148966154}
\textbf{Naseddepan}  ta  geyem  talinga  ko  i. \\\smallskip
\gll \textbf{Na-seddep-an}  ta  geyem  talinga  ko  i. \\
\textsc{a.hap.r}-go.in.small.space-\textsc{apl}  \textsc{nabs}  ant  ear  1\textsc{s.gen}  \textsc{def.n} \\
\glt ‘The ant(s) \textbf{happened} \textbf{to} \textbf{go} \textbf{inside} my ear.’ 
\z
\ea
\label{bkm:Ref148966156}
\textbf{Paseddepan}  ta  geyem  talinga  ko  i. \\\smallskip
\gll \textbf{Pa-seddep-an}  ta  geyem  talinga  ko  i. \\
\textsc{t.r}-go.in.small.space-\textsc{apl}  \textsc{nabs}  ant  ear  1\textsc{s.gen}  \textsc{def.n} \\
\glt ‘The ant(s) \textbf{went} \textbf{inside} my ear.’ 
\z

Examples \REF{bkm:Ref148966256} and \REF{bkm:Ref148965959} illustrate this same root, without the applicative suffix, in dynamic modality:

\ea
\label{bkm:Ref148966256}
Ta,  ake  na  duma  \textbf{gaseddep}  ta  baļas  an… \\\smallskip
\gll Ta,  ake  na  duma  \textbf{ga-seddep}  ta  baļas  an… \\
so  1\textsc{s.gen}  \textsc{lk}  companion  \textsc{i.r}-go.in.small.space  \textsc{nabs}  forest  \textsc{def.m} \\
\glt ‘So my companion \textbf{went} \textbf{into} the forest…’ [NFWN-T-01 2.10]
\z

Example \REF{bkm:Ref148965959} is from a story about a raid on the island of Cagayancillo. A group of Kagayanens were hiding in a cave (referred to as a “hole”), and the attackers set a fire at the entrance to the cave such that everyone inside died except one woman. This was a volitional act on the part of the raiders, though the absolutive argument, the smoke, is not itself volitional:

\ea
\label{bkm:Ref148965959}
Aso  ta  apoy  \textbf{gaseddep}  en  ta  lungag… \\\smallskip
\gll Aso  ta  apoy  \textbf{ga-seddep}  en  ta  lungag… \\
smoke  \textsc{nabs}  fire  \textsc{i.r}-go.in.small.space  \textsc{cm}  \textsc{nabs}  hole \\
\glt ‘The smoke of the fire \textbf{went} \textbf{inside} the hole….’ [JCWN-T-25 2.11]
\z

The following are some examples of \textit{daeb} ‘to turn face down’ expressing volitional intransitive situations:

\ea
\textbf{Gadaeb}  en  bata  an. \\\smallskip
\gll \textbf{Ga-daeb}  en  bata  an. \\
\textsc{i.r}-turn.over  \textsc{cm}  child  \textsc{def.m} \\
\glt ‘The child \textbf{turned} \textbf{over} face down.’
\z
\ea
\textbf{Padaeban}  a  din  na  gatanuga. \\\smallskip
\gll \textbf{Pa-daeb-an}  a  din  na  ga-tanuga. \\
\textsc{t.r}-turn.over-\textsc{apl}  1\textsc{s.abs}  3\textsc{s.erg}  \textsc{lk}  \textsc{i.r}-sleep \\
\glt ‘S/He \textbf{turned} \textbf{over} on me when sleeping.’
\z
\ea
\textbf{Nadaeban}  a  din  na  gatanuga. \\\smallskip
\gll \textbf{Na-daeb-an}  a  din  na  ga-tanuga. \\
\textsc{a.hap.r}-turn.over-\textsc{apl}  1\textsc{s.abs}  3\textsc{s.erg}  \textsc{lk}  \textsc{i.r}-sleep \\
\glt ‘S/He \textbf{happened} \textbf{to} \textbf{turn} \textbf{over} on me when sleeping.’
\z

As evidence that \textit{daeb} falls into both root Class II and root Class IV, we note that when the absolutive is inanimate, it is possible to use the bare root as a resultative form \REF{bkm:Ref149299861} (characteristic of Class II), but when the absolutive is animate, the same form is simply ungrammatical \REF{ex:childturnedover} (characteristic of Class IV):

\ea
    \ea[]{
    \label{bkm:Ref149299861}
    Daeb  pambot  an. \\\smallskip
\gll daeb  pambot  an \hspace{1cm} \\
    turn.over  motor.boat  \textsc{def.m} \\
    \glt ‘The motor boat is turned over.’
    }
    \ex[*]{
    \label{ex:childturnedover}
    \gll daeb  bata  an \\
    turn.over  child  \textsc{def.m} \\
    \glt (‘The child is turned over.’)  
    }
    \z
\z

\section{Roots describing non-volitional transitive situations (Class V)}

The last four classes of verbal roots we will discuss in this chapter can all be characterized as semantically transitive\is{semantic transitivity}\is{transitivity!semantic}. None of them occur in basic intransitive frames, their only intransitive forms being detransitives. Class V roots describe non-volitional transitive situations. For the most part, these roots do not easily occur in dynamic modality. Many of these describe perception, cognition and emotion. As such they are discussed in more detail in \chapref{chap:verbclasses-2}. There are two subclasses of Class V roots. Subclass Va do not require the applicative \textit{{}-an} in the basic transitive form (though
%I want to check which ones in the first subclass can take {}-an but still no help from Bebo.
%Carol Pebley
%November 10, 2023, 3:53 PM
%
%I will leave this comment here for future reference.
%Allen
%November 11, 2023, 9:59 AM
 some  allow it), while Subclass Vb roots all require the applicative suffix. Example \REF{bkm:Ref150247189} lists a subset of these roots. Examples in context follow. 

\ea
\label{bkm:Ref150247189}
\begin{tabbing}\hspace{5cm}\= \kill
\textbf{Subclass Va}    \>     \textbf{Subclass Vb} \\
(-\textit{an} sometimes allowed, but \>   (-\textit{an} required in basic, transitive frame) \\
never required) \\
kita ‘to see X’   \>     masmas ‘to notice X’ \\
kaļa ‘to recognize/know X.’ \> tingaļa ‘to be amazed/wonder about X’ \\
anad ‘to be used to X’  \>    lasa ‘to taste X’ \\
bunggo ‘to bump into X’   \>   abot ‘to happen to come upon X’ \\
igo ‘to strike/hit X’   \>   agi ‘to experience X’ \\
kaya ‘to be able to do X’ \>   largá  ‘to ignore X’ \\
\>          aļam ‘to know X’
\end{tabbing}
\z

\ea
Subclass Va: \\
Iling  din  \textbf{makaļa}  din  daw  \textbf{makita}  din. \\\smallskip
\gll …-Iling  din  \textbf{ma-kaļa}  din  daw  \textbf{ma-kita}  din. \\
\textsc{t.r}-say  3\textsc{s.erg}  \textsc{a.hap.ir}-know/recognize  3\textsc{s.erg}  if/when  \textsc{a.hap.ir}-see  3\textsc{s.erg} \\
\glt ‘He said that he will recognize (him) if he sees (him).' [JCWN-T-26 4.13] \\\smallskip
*gakaļa *pakaļa, *gakita *pakita
\z
\ea
Subclass Vb \\
Tay,  Nay,  \textbf{mļaman}  nyo  nidlaw  a  gid  en  ki  kyo tak  lugay  man  na  uļa  ki kitaay. \\\smallskip
\gll Tay,  Nay,  \textbf{ma-aļam-an}  nyo  na-idlaw  a  gid  en  ki  kyo tak  lugay  man  na  uļa  ki  …-kita-ay. \\
dad  mom  \textsc{a.hap.ir}-know-\textsc{apl}  2\textsc{p.erg}  \textsc{a.hap.r}-miss  1\textsc{s.abs}  \textsc{int}  \textsc{cm}  \textsc{obl.p}  2p because  long.time  \textsc{emph}  \textsc{lk}  \textsc{neg.r}  1\textsc{p.incl.abs}  \textsc{i.r}-see-\textsc{rec} \\
\glt ‘Dad, Mom, you should know that I really miss you because it is a long time that we have not seen each other.’ [BCWL-T-10 2.7]
\z

\ea
\textbf{Namasmasan}  ta  mama  ya  na  gambaļ  nangka  ya.\\\smallskip
\gll \textbf{Na-masmas-an}  ta  mama  ya  na  ga-ambaļ  nangka  ya.\\
\textsc{a.hap.r}-notice/realize-\textsc{apl}  \textsc{nabs}  man  \textsc{def.f}  \textsc{lk}  \textsc{i.r}-say  jackfruit  \textsc{def.f}\\
\glt ‘The man noticed/realized that the jackfruit was speaking.’ [YBWN-T-01 5.6] \\\smallskip
  *pamasmasan *gamasmas
\z

The root \textit{agi} may take dynamic affixes, in which case it expresses the meaning of ‘to pass somewhere’, and falls into Class IV. However, when occurring with happenstantial affixes, it has the somewhat idiomatic meaning of ‘to experience’, as a sickness (example \ref{bkm:Ref150244144}). In this case the experiencer is expressed as ergative, and the source of the experience as absolutive. Therefore in this usage it falls into Subclass Vb: 

\ea
\label{bkm:Ref150244144}
Ta  buļan  na  galigad  \textbf{naagian}  ko  a  masakit  na   swalem. \\\smallskip
\gll Ta  buļan  na  ga-ligad  \textbf{na-agi-an}  ko  a  masakit  na   swalem. \\
\textsc{nabs}  month/moon  \textsc{lk}  \textsc{i.r}-pass.by  \textsc{a.hap.r}-experience-\textsc{apl}  1\textsc{s.erg}  \textsc{inj}  sick  \textsc{lk}  chickenpox \\
\glt ‘In the month that passed by I experienced the sickness of chickenpox.’ [EMWN-T-05 3.1] \\\smallskip
*\textit{paagian} *\textit{gaagi} intended: ‘to experience X’.
\z

The verb \textit{igo} ‘hit/strike’ describes the non-volitional act of hitting an object, as a tree branch falling on a parked car, or a rock hitting a window. It may or may not be the result of a volitional act of throwing, shooting, etc. (as in \ref{bkm:Ref150345606}), but the event of hitting is presented as happenstantial. 

\ea
\label{bkm:Ref150345606}
\textbf{Naigo}  ko  kalilawan  i. \\\smallskip
\gll \textbf{Na-igo}  ko  kalilawan  i. \\
\textsc{a.hap.r}-hit  1\textsc{s.erg}  Philippine.oriole  \textsc{def.n} \\
\glt ‘I hit the Philippine oriole.’ (He was shooting at it with a slingshot.) [MEWN-T-02 4.1] \\\smallskip
*paigo *gaigo
\z

\section{Roots describing volitional transitive situations (Classes VI-VIII)}
\label{bkm:Ref150248001}
\label{sec:volitionaltransitiveroots}

Class VI is the majority class of semantically transitive\is{semantic transitivity}\is{transitivity!semantic} roots. It is distinguished by the use of the suffix -\textit{en}/-\textit{on} for the transitive, irrealis inflection. Classes VII and VIII employ the “zero” allomorph of the transitive, irrealis inflection. Semantically, Class VI contains, for the most part, roots that describe transitive volitional accomplishments in which the Undergoer is completely affected by the situation.  Classes VII and VIII describe situations in which the Undergoer does not undergo a complete change of state. Class VII for the most part contains roots that involve transfer of a theme (expressed as absolutive in a transitive non-applicative construction) to a recipient or destination. Class VIII is distinguished grammatically by requiring the applicative suffix \textit{{}-an} in its basic transitive construction. Roots in this class describe situations in which the Undergoer is only partially, superficially or invisibly affected by the situation.

\tabref{tab:rootsthattaketwodistincttransitiveirrealisinflections-1} lists several transitive, volitional verbs according to which of these three classes they fall into. \\

\begin{table}
    \caption{Verbal roots that take two distinct transitive, irrealis inflections}
\label{tab:rootsthattaketwodistincttransitiveirrealisinflections-1}
    \fittable{
    \begin{tabular} {
        >{\RaggedRight\arraybackslash}p{4.55cm}
        >{\RaggedRight\arraybackslash}p{4.1cm}
        >{\RaggedRight\arraybackslash}p{4.2cm}
                    }
\lsptoprule
Class VI (majority) & Class VII & Class VIII \\
\midrule
Roots that take -\textit{en/-on} & \multicolumn{2}{c}{Roots that never take \textit{-en/-on}} \\
\midrule
\multicolumn{2}{c}{} & {Roots that always take -\textit{an} in transitive frames} \\
\midrule
Mostly non-transfer verbs,\newline
absolutive = Patient & Mostly verbs of transfer,\newline absolutive = Theme & Mostly non-transfer verbs,\newline absolutive = partially, invisibly or superficially affected Undergoer. \\
\midrule
% Begin column 1, page 1
asod ‘to pound X with mortar and pestle’\newline 
agaw ‘to grab X away’\newline 
akid ‘to serve food’\newline 
anggat ‘to invite X’\newline 
ayos ‘to fix X’\newline 
badbad ‘to untie/unwind X’\newline 
baļad/beļad ‘to dry X in sun’\newline
basa ‘to read X’\newline 
basoļ ‘to blame/scold X’\newline
begkes ‘to bundle X’ \newline  
beļag ‘to separate X’\newline
betteng ‘to pull X’\newline 
bilang ‘to count X’\newline 
bingbing ‘to carry X by a handle, as a bucket’\newline 
buat ‘to make/do X’\newline 
dakep ‘to catch/arrest X’\newline 
daļa ‘to carry/take X’\newline
dumog ‘to wrestle/hand fight X’\newline 
gabot ‘to pull X out’\newline 
galing ‘to grind/mill X’\newline
gamit ‘to use X’\newline 
usar ‘to use X’\newline 
geļet ‘to cut/slice X’\newline
gisi ‘to tear/rip X’\newline 
gunting ‘to cut X with scissors’\newline 
guyod ‘to lead/guide/drag X’ %End Column 1 of page 1 % Begin column 2 of page 1
&
alad ‘to offer X to s.o.’\newline 
atag ‘to give X to s.o.’\newline 
basya ‘to splash X on s.o. or s.t.’\newline 
baligya ‘to sell X to s.o.’\newline 
baog ‘to feed X to animals’\newline 
bayad ‘to pay X to s.o.’\newline 
betang/batang ‘to put X s.w.’\newline 
bindisyon ‘to bless s.o. with X’\newline 
bubo ‘to pour out liquid into/onto X’\newline 
bubod ‘to pour out dry material into/onto X’\newline
bunyag ‘to water plants, to baptize s.o. with X’\newline
bwin/buin ‘to reduce the number/amount of s.t.’\newline
desek ‘to tamp X’\newline
dugang ‘to add X to s.t.’\newline
duļ-ong/duaļ ‘to take/accompany X somewhere’\newline
duoļ ‘to lift X to the head’\newline ellang ‘to put X s.w., to block the way’\newline 
igot ‘to tie X with s.t.’\newline 
intriga ‘to entrust X to s.o.’ %End column 2 of page 1 %Begin column 3 of page 1
&
amblig/amlig ‘to take care of X or be careful with X’\newline 
arek ‘to kiss X’\newline 
awid ‘to hold X back from going s.w.’\newline 
bagnes ‘to impregnate X’\newline 
batok ‘to rebel against X’\newline 
bantay ‘to watch/guard X’\newline 
banggod ‘to straighten out X’\newline 
demet ‘to hold a grudge against s.o.’\newline 
ibit ‘to hold X in hand or hold on to X’\newline 
kagon ‘to arrange for X to be married’\newline 
lalis ‘to disagree with X’\newline 
neļseļ ‘to regret X’\newline 
saliga ‘to insult X’\newline 
saplid ‘to brush lightly against X’\newline 
sirbi ‘to serve X’\newline 
taap ‘to winnow X’\newline 
tabang ‘to help X’\newline 
tagad ‘to wait for X’\newline 
takmi ‘to sip X’\newline 
tanod ‘to watch/care for a child’\newline 
tilaw ‘to taste X’\newline 
tawad ‘to ask for discount’ \\ % End Column 3 page 1
\lspbottomrule
\end{tabular}
    }     %end fittable
\end{table}

%Begin page 2
\begin{table}
    \caption*{Verbal roots that take two distinct transitive, irrealis inflections (cont.)}
    \fittable{
    \begin{tabular}{
        >{\RaggedRight\arraybackslash}p{4.55cm}
        >{\RaggedRight\arraybackslash}p{4cm}
        >{\RaggedRight\arraybackslash}p{4.2cm}
       }
\lsptoprule
Class VI  & Class VII & Class VIII \\
\midrule
% Begin column 1 of page 2
id-id ‘to grate X’\newline 
imes ‘to prepare X’\newline 
inem ‘to drink X’\newline 
ipid ‘to arrange/fold X up’\newline 
ipit ‘to pin/clip X’\newline 
ipo ‘to pick fruits/flowers/\newline coconuts’\newline 
isa ‘to raise X up’\newline 
iyaw ‘to slaughter X’\newline 
kaan ‘to eat X’\newline 
kabig ‘to treat/claim/\newline consider X as ones own'\newline 
kaddot ‘to pinch X’\newline 
kagat ‘to bite X’\newline 
kaig ‘to brush/swipe X off’\newline 
kamang ‘to get/take X’\newline 
kayos ‘to scratch X’\newline 
kemes ‘to squeeze X’\newline 
kitkit ‘to bite/nibble at X’\newline 
kugod ‘to grate coconut’\newline 
ļaļa ‘to weave X’\newline 
lagas ‘to chase X’\newline 
lagpat ‘to guess X’\newline 
laga ‘to boil X in water’\newline 
lam-ed ‘to swallow X’\newline 
legsek ‘to smash X’\newline 
luag ‘to look at/watch X’ 
lubag ‘to twist X’\newline 
lubbas ‘to undress’\newline 
lukot  ‘to roll up X’\newline 
lunod ‘to add X to food’\newline 
luto ‘to cook X’\newline 
luko ‘to trick X’\newline 
mukļat ‘to open eyes’\newline 
mara ‘to dry X’\newline 
mitlang ‘to pronounce word(s)’\newline 
mugmog ‘to gargle X’\newline 
munit ‘to skin X’% End of column 1 page 2 % Begin column 2 page 2
&
isoļ ‘to move X backwards’\newline 
lagnas ‘to clean with X (lots of water)’\newline 
lampes ‘to strike X on s.t. else’\newline 
langoy ‘to bathe s.o. with X’\newline 
layong ‘two or more to carry something with X’\newline 
lumba ‘to race against s.o.’\newline 
maket ‘to kindle a fire with X’\newline 
laswa ‘to pour X (hot liquid) on s.t. or s.o.’\newline 
laygay ‘to advise/preach about X to s.o.’\newline 
lebbeng ‘to bury X’\newline 
limas ‘to bail out X from s.t.’\newline 
lua ‘to spit X out of mouth’\newline 
paid ‘to wipe X off or on s.t.’\newline 
panaik ‘to bring/take X upstairs’\newline 
panaog ‘to bring/take X downstairs’\newline 
pilak ‘to throw X away’\newline 
tampak/tupak ‘to patch X on s.t.’ \newline 
tampek ‘to pack X on or in s.t.’ \newline
tanem ‘to plant X’\newline 
tudlo ‘to teach X to s.o.’\newline 
tuļod ‘to push X’\newline 
tungtong ‘to place X on
top of s.t.’ % End column 2 page 2 % Begin column 3 page 2
&
trangka ‘to lock X’\newline 
tutod ‘to light a lamp/candle’\newline 
tugaļ ‘to make holes in X for planting’\newline 
ugas ‘to wash X’\newline 
uyak ‘to waste/spill out X’\newline 
kisi ‘to rinse rice before cooking’\newline 
banlaw ‘to rinse’\newline 
bunak ‘to launder’\newline 
lakbay ‘to skip over something’\newline 
lakted ‘to cross over something’ \\
%End column 3, page 2
\lspbottomrule
\end{tabular}
} % fittable
\end{table}

%Begin page 3
\begin{table}
    \caption*{Verbal roots that take two distinct transitive, irrealis inflections (cont.)}
    \fittable{
    \begin{tabular}{
        >{\RaggedRight\arraybackslash}p{4.3cm}
        >{\RaggedRight\arraybackslash}p{4cm}
        >{\RaggedRight\arraybackslash}p{4.2cm}
       }
\lsptoprule
Class VI  & Class VII & Class VIII \\
\midrule
% Begin column 1 of page 3
nakaw/takaw ‘to steal X’\newline 
kiskis ‘to scrub’\newline 
nisnis ‘to scour X’\newline 
nubli ‘to inherit X’\newline 
ngamuyo ‘to pray for X’\newline 
ngaddi ‘to pray a ceremonial prayer for X’\newline 
ngali ‘to dig up root crops’\newline 
nganga ‘to open/gap X’\newline 
ngaoy ‘to gather wood’\newline 
ngita ‘to look for X’\newline 
padpad ‘to wipe/brush X off s.t. else.’\newline 
pakang ‘to hit X with s.t. held in hand’\newline 
pega ‘to wring X out’\newline 
pessa ‘to smash/crush/break X into pieces’\newline 
petpet ‘to chop down a plant/bamboo/small tree’\newline 
pilo ‘to fold X over’\newline 
pili ‘to choose X’\newline 
plantsa ‘to iron clothes’\newline 
priso ‘to imprison X’\newline 
prito ‘to fry X’\newline 
pudpod ‘to crush X into powder’\newline 
pukan ‘to chop down a big tree’\newline 
pukaw ‘to wake X up’\newline 
pukpok ‘to pound X’\newline 
samad ‘to destroy/break X’\newline 
sambit ‘to mention/refer to X’\newline 
sabat ‘to reply, answer X’
% Begin column 2 of page 3
&
tuto ‘to contribute X’\newline 
ugsak ‘to put X in s.t.’\newline 
uļog ‘to drop X’\newline 
ladlad ‘to lay something flat’\newline 
suoļ ‘to pay wages to s.o.’\newline 
laket ‘to add X to s.t.’ 
% End column 2, p. 3. 

% Begin column 3, page 3.
& \\
% End column 3, page 3.
\lspbottomrule
\end{tabular}
} % fittable
\end{table}

Another distinguishing factor among semantically transitive\is{semantic transitivity}\is{transitivity!semantic} roots has to do with the use of the transitive deontic prefix \textit{i-}. The roots listed in \REF{bkm:Ref447111587} are all Class VI verbs that can occur with \textit{i}{}- rather than -\textit{en}/-\textit{on}. Notice that the only difference between root+-\textit{en} and \textit{i-}+root is that with \textit{i-}+root there is a deontic meaning. The argument structures are the same—the Actor is ergative and Undergoer (X) is absolutive. Therefore for these, verbs \textit{i}{}- may be thought of as the deontic modality replacement for -\textit{en}/-\textit{on}:

\ea
\label{bkm:Ref447111587}
\label{bkm:Ref148965704}
Class VI verbs (\tabref{tab:rootsthattaketwodistincttransitiveirrealisinflections-1}) that occur with \textit{i}{}-: \begin{tabbing}
\hspace{2cm} \= \hspace{4cm} \= \kill
Root \> Root+-\textit{en} \> \textit{i}{}-+Root \\
\textit{baļad/beļad } \>  baļaren/beļaren \> \textit{ibaļad/ibeļad} \\
                     \>  ‘will dry X in sun’ \> ‘have to dry X in sun’ \\
\textit{basa } \>    \textit{basaen} ‘will read X’ \> \textit{ibasa} ‘have to read X’ \\
\textit{beļag } \>   \textit{beļagen} ‘will separate X’ \> ibeļag ‘have to separate X’ \\
\textit{galing } \>  \textit{galingen} \> igaling \\
                 \>  ‘will grind/mill X’ \> ‘have to grind/mill X’ \\
\textit{geļet } \>  \textit{geļeten} ‘will cut/slice X’ \> \textit{igeļet} ‘have to cut/slice X’ \\
\textit{laga } \>  \textit{lagaen} ‘will boil X’ \> \textit{ilaga} ‘have to boil X’ \\
\textit{lunod } \>  \textit{lunuron} \> \textit{ilunod} \\
               \>  ‘will add X into cooking’ \>  ‘have to add X into cooking’ \\
\textit{luto } \>  \textit{lutuon} ‘will cook X’ \> \textit{iluto} ‘have to cook X’ \\
\textit{padpad } \>  \textit{padparen} \> \textit{ipadpad} \\
                 \>  ‘will brush X off’ \> ‘have to brush X off’ \\
\textit{prito } \>  \textit{prituon} ‘will fry X’ \> \textit{iprito} ‘have to fry X’ \\
\textit{suot } \>  \textit{suuton} ‘will wear X’ \> \textit{isuot} ‘have to wear X’ \\
\textit{unat } \>  \textit{unaten} ‘will stretch X’ \> \textit{iunat} ‘have to stretch X’
\end{tabbing}
\z

For Class VI verbs that describe situations that typically involve an instrument, the prefix \textit{i}{}- may have an instrumental applicative meaning (see \chapref{chap:voice}, \sectref{sec:instrumentalvoice}), in addition to deontic modality. That is to say, the roots in \REF{bkm:Ref148965668} partially overlap with those in \REF{bkm:Ref447111587}. For these verbs, \textit{i-} may be considered the deontic modality “instrumental voice” replacement for -\textit{en}/-\textit{on}:

\ea
\label{bkm:Ref148965668}Verbs for which \textit{i}{}- selects an Instrument as the absolutive argument: \\
\textit{iasod}   ‘have to pound grain with X \\
\textit{iakid}   ‘have to serve food with X’ \\
\textit{ibegkes}   ‘have to bundle things with X’ \\
\textit{iid-id}   ‘have to grate something with X \\
\textit{igeļet}   ‘have to cut something with X’ \\
\textit{ikidlas}   ‘have to cut leaves in long strips with X’ \\
\textit{iluto}   ‘have to cook with X’ \\
\textit{ipakang}   ‘have to hit downward on something with X’ (X is a hard long     object) \\
\textit{ipukpok}   ‘have to pound/beat on something with X’
\z

Finally, for another subset of Class VI roots, \textit{i}{}- may also function as a deontic modality benefactive applicative marker. We can discern no relevant semantic commonality to this group of roots that is not also shared by many other roots.

\ea
Verbs for which \textit{i}{}- selects a Beneficiary or Recipient as the absolutive argument: \\
\textit{iakid}   ‘have to serve food to X’ \\
\textit{igeļet}   ‘have to cut something for X’ \\
\textit{ikidlas}   ‘have to cut leaves in long strips for X’ \\
\textit{iluto}   ‘have to cook something for X
\z

Independently of the potential applicative uses of \textit{i-}, it always expresses deontic modality. Therefore we conclude that deontic modality is its major function. 

The following are a few examples from the corpus of Class VI, VII and VIII roots.

\ea
Class VI realis \\
Tapos  gauli  danen  i  daw  \textbf{padakep}  ta  mga  pulis mama  ya  daw  papriso  danen. \\\smallskip
\gll Tapos  ga-uli  danen  i  daw  \textbf{pa-dakep}  ta  mga  pulis mama  ya  daw  pa-priso  danen. \\
then  \textsc{i.r}-go-home  3\textsc{p.abs}  \textsc{def.n}  and  \textsc{t.r}-catch/arrest  \textsc{nabs}  \textsc{pl}  police
man  \textsc{def.f}  and  \textsc{t.r}-prisoner  3\textsc{p.erg} \\
\glt ‘Then they (the police) went home and the police \textbf{arrested} the man and they imprisoned (him).' [MBON-C-01 5.2]
\z

\newpage
\ea
Class VI, irrealis \\
Daw  mambaļ  gani  na  uno,  dos,  tris,   \textbf{dumugon}  ta nang  en  danen  an  na  patayen  ta  nang  en. \\\smallskip
\gll Daw  m-ambaļ  gani  na  uno,  dos,  tris,   \textbf{dumog-en}  ta nang  en  danen  an  na  patay-en  ta  nang  en. \\
if/when  \textsc{i.v.ir}-say  truly  \textsc{lk}  one  two  three  hand.fight-\textsc{t.ir}  1\textsc{p.incl.erg}
only/just  \textsc{cm}  3\textsc{p.abs}  \textsc{def.m}  \textsc{lk}  die-\textsc{t.ir}  1\textsc{p.incl.erg}  only/just  \textsc{cm} \\
\glt ‘When (someone) says truly one, two, three, let’s \textbf{hand} \textbf{fight} them so that we will kill (them).’ [PTOE-T-01 15.12]
\z

\ea
Class VII, realis \\
\textbf{Paugsak}  din  kon  ta  duyan  darwa  ya  na  bata. \\\smallskip
\gll \textbf{Pa-ugsak}  din  kon  ta  duyan  darwa  ya  na  bata. \\
\textsc{t.r}-put.inside  3\textsc{s.erg}  \textsc{hsy}  \textsc{nabs}  hamock  two  \textsc{def.f}  \textsc{lk}  child \\
\glt ‘She \textbf{put} \textbf{inside} the hamock the two children.’ [MBON-T-05 3.8]
\z

\ea
Class VII irrealis \\
Daw  kan-o  kani  uli  sawa  no  ya  liso  ta  kalabasa  na darwa  na i  \textbf{tanem}  ta  kuron  na  busļot. \\\smallskip
\gll Daw  kan-o  kani  uli  sawa  no  ya  liso  ta  kalabasa  na darwa  na i  \emptyset{}-\textbf{tanem}  ta  kuron  na  busļot. \\
if/when  when  later  go.home  spouse  2\textsc{s.gen}  \textsc{def.f}  seed  \textsc{nabs}  squash  \textsc{lk}
two  \textsc{lk} \textsc{d}1\textsc{adj}  \textsc{t.ir}-plant  \textsc{nabs}  clay.pot  \textsc{lk}  hole \\
\glt ‘Whenever later your spouse comes home, these two seeds of the squash plant (them) in a clay pot with a hole.’ [AION-C-01 9.4]
\z

\ea
Class VIII realis \\
\textbf{Paugasan}  ta  nanligan  kabaļ  ya  daw  baļad  para gamiten  ta  bata  ta  iya  na  pagdako. \\\smallskip
\gll \textbf{Pa-ugas-an}  ta  nanligan  kabaļ  ya  daw  ...-baļad  para gamit-en  ta  bata  ta  iya  na  pag-dako. \\
\textsc{t.r}-wash-\textsc{apl}  \textsc{nabs}  midwife  vernix  \textsc{def.f}  and  \textsc{t.r}-dry.in.sun  for
use-\textsc{t.ir}  \textsc{nabs}  child  \textsc{nabs}  3\textsc{s.gen}  \textsc{lk}  \textsc{nr.act}-big (Hiligaynon) \\
\glt ‘The midwife washed the vernix caseosa and dry it in the sun for the child to use when growing up (lit. becoming big).’ [VAOE-J-06 3.3]
\z

\newpage
\ea
Class VIII irrealis \\
\textbf{Bantayan}  kay  no  ta  ame  na  ubra na  dili  kay  madaag  ta  panulay. \\\smallskip
\gll \emptyset{}-\textbf{Bantay-an}  kay  no  ta  ame  na  ubra na  dili  kay  ma-daag  ta  panulay. \\
\textsc{t.ir}-watch/guard-\textsc{apl}  1\textsc{p.excl.abs}  2\textsc{s.erg}  \textsc{nabs}  1\textsc{p.excl.gen}  \textsc{lk}  work/make
\textsc{lk}  \textsc{neg.ir}  1\textsc{p.excl.abs}  \textsc{a.hap.ir}-win  \textsc{nabs}  evil \\
\glt ‘Guard/watch us in our work so that evil can not win over us.’ (This is part of a prayer.) [ETOB-C-02 1.2]
\z

\ea
Class VIII irrealis \\
Daw  duma  ko  ya  ambaļ  din,  ``\textbf{Tagaran}  a  no  man anay  tak  pakaļa  a  no  ta Amirikano  ya  na  arey  no."   \\\smallskip
\gll Daw  duma  ko  ya  ambaļ  din,  ``\emptyset{}-\textbf{Tagad-an}  a  no  man anay  tak  pa-kaļa  a  no  ta Amirikano  ya  na  arey  no." \\
and  companion  1\textsc{s.gen}  \textsc{def.f}  say  3\textsc{s.erg}  \textsc{t.ir}-wait-\textsc{apl}  1\textsc{s.abs}  2\textsc{s.erg}  also
first/for.a.while  because  \textsc{caus}-know/recognize  1\textsc{s.abs}  2\textsc{s.erg}  \textsc{nabs}  American(m)  \textsc{def.f}  \textsc{lk}  friend  2\textsc{s.gen} \\
\glt ‘My companions said, “Wait for me for a while because (I want you to) introduce me to the American who is your friend.”' [RCON-L-03 20.6]
\z

\section{Roots distinguished by the dynamic, intransitive, irrealis inflections}
\label{sec:irrealisinflections}

As mentioned and exemplified in \chapref{chap:verbstructure}, \sectref{sec:verbinflection}, the dynamic, intransitive, irrealis inflections distinguish two groups of verb roots. Verbs in the majority group only allow the prefix \textit{mag}{}- while minority group verbs allow either \textit{mag}{}- or \textit{m-}.  There are no roots that we are aware of that take only \textit{m-}. These two groups correlate with the eight root classes discussed in \sectref{bkm:Ref148856716} through \sectref{bkm:Ref150248001} in the following way: Classes II, III, and V only allow \textit{mag}{}-, while some verbs in classes IV, VI, VII, and, VIII allow \textit{m-} in addition to \textit{mag-}. Recall that Class I roots do not take the dynamic affixes, therefore do not fall into either of these groups.

The distinction between verbs that allow \textit{m-} and those that do not (\tabref{tab:twogroupsofverbalroots}) is partly structural, partly semantic, and partly just lexical. On the structural side, verb roots that begin with the syllable \textit{li}, or nasal consonants (\textit{n} or \textit{ng}) do not allow \textit{m}{}-, regardless of their semantics. On the semantic side, the minority group includes semantically intransitive\is{semantic transitivity}\is{transitivity!semantic} motion verbs, and other volitional activity verbs (see \citealt{pebley1998}). These verbs take \textit{m}{}- as their basic form, and \textit{mag}{}- only to express intended/planned or more distant future situations (see below). Many semantically transitive\is{semantic transitivity}\is{transitivity!semantic} verbs in the minority group are usually expressed in a detransitive frame, and imply a particular Patient or a highly predictable set of possible Patients. For example, \textit{bunak} ‘to launder’ implies clothing as the Patient, \textit{kaan} ‘eat’ implies food as the Patient, \textit{inem} ‘drink’, implies liquid, \textit{asod} ‘pound (grain)’, and \textit{sillig} ‘sweep (floor or yard)’. The majority group also includes inherently transitive verbs that impose fewer semantic restrictions on the Patient, and which usually occur with an overt, often human, Patient. These include \textit{silot} ‘punish’, \textit{basoļ} ‘scold’, \textit{bat-eg} ‘choke’, \textit{biag} ‘capture’, \textit{bļebed} ‘wrap around’, \textit{biak} ‘split s.t.’, \textit{bilin} ‘leave behind’, \textit{bitay} ‘hang s.t. up’, \textit{bugbog} ‘hit’, and \textit{bugno} ‘greet’. Other semantic types that make up the minority group are listed in \tabref{tab:twogroupsofverbalroots}. These include verbs of position (\textit{tindeg} ‘to stand’, \textit{tuwad} ‘to bend over’, \textit{taliad} ‘to bend backwards’ etc.), verbs of intentional visual perception (\textit{sil-ing} ‘to peek inside’, \textit{tan\nobreakdash-aw} ‘to look at’, \textit{luag} ‘to watch’ etc.), verbs of utterance (\textit{singgit} ‘to shout’, \textit{isturya} ‘to talk’, \textit{sugid} ‘to tell’ etc.), and others. These semantic characterizations are very general and variable. Apparent “exceptions” exist in both groups. For this reason, we say that the groups are partially “just lexical” — there appear to be no synchronic motivations for some of their members. \tabref{tab:twogroupsofverbalroots} lists a few verbs in each of these groups. It must be kept in mind that any verb root can appear with intransitive affixation (see \chapref{chap:voice}, and \chapref{chap:verbstructure}, \sectref{sec:grammaticaltransitivity}).

\begin{table}
    \caption{Groups of verbal roots depending on the dynamic intransitive irrealis inflection}
\label{tab:twogroupsofverbalroots}
    % \fittable{
    \begin{tabular} {
        >{\RaggedRight\arraybackslash}p{5.6cm}
        >{\RaggedRight\arraybackslash}p{5.6cm}
                    }
\lsptoprule
\multicolumn{1}{>{\centering\arraybackslash}m{5.6cm}}{\hspace{.7cm}\textbf{Group 1\newline Roots that allow only \textit{mag}-\newline (majority group)}} 
    & \multicolumn{1}{>{\centering\arraybackslash}m{5.6cm}}{\hspace{.7cm}\textbf{Group 2\newline Roots that allow \textit{mag}- or \textit{m}- \newline
(minority group)}} \\
\midrule
{ \textbf{Roots} \textbf{beginning} \textbf{with} \textbf{\textit{li}}, \textbf{or} \textbf{nasal} \textbf{consonants:}\footnote{We do not include roots beginning with \textit{m}{}- in either of these lists because they are indeterminate as to their group membership. They all take \textit{mag-}, as do all roots, but because \textit{m}{}- is replacive, there is no difference between the bare form and what the form would be with \textit{m-}. For example, \textit{maket} ‘to kindle a fire’, \textit{maļ}\textit{a} ‘to change chothes’, \textit{maal} ‘to love/to be expensive’, \textit{mama} ‘to chew betel nut leaf, lime and tobacco’, \textit{mando} ‘to command’, \textit{mikaw} ‘to sacrifice for a new house’ \textit{miy}\textit{ag} ‘to agree/want’ \textit{mingaw} ‘to be sad, lonely or drunk’.}}\newline
% Begin column 1 page 1
liad ‘to arch ones back’\newline 
liag ‘to court someone romantically’\newline 
libak ‘to gossip/backbite’\newline 
libot ‘to go about/around s.t.’\newline 
ligid ‘to roll over’\newline 
ligis ‘to crush to powder’\newline 
likaw ‘to avoid’\newline 
likid ‘to roll up s.t. flat’\newline 
like ‘to turn’\newline 
limas ‘to bail out’\newline 
limbeng ‘to take cover from wind’\newline 
lipay ‘to rejoice/be joyful/be happy’\newline 
lipeng ‘to be dizzy/faint’\newline 
limpyo ‘to clean’\newline 
liped ‘to block the view’\newline 
narem ‘to have sleep paralysis’\newline 
neļseļ ‘to regret’\newline
nisnis ‘to scour s.t.’\newline 
ngaoy ‘to gather wood’\newline 
ngita ‘to  look for s.t.’\newline
ngamuyo ‘to pray'\newline
ngusmod ‘to frown’% End column 1 page 1. % Begin column 2 page 1.
& \textbf{Volitional, intransitive motion events:}\newline
panaw ‘to go/walk/leave’\newline 
gwa ‘to go/come out’\newline 
selled ‘to go inside’\newline 
iling ‘to go somewhere’\newline 
layog ‘to fly’\newline 
panaik ‘to go up (usually stairs)’\newline 
panaog ‘to go down (usually stairs)’\newline 
kawas ‘to disembark’\newline 
tunga ‘come up from under water’\newline 
tugpa ‘to jump down’\newline 
sugat ‘to go meet s.o.’\newline 
pasyar ‘to go for a walk/visit’\newline 
kuyog ‘to go with someone’\newline 
duksoļ ‘to approach some people who are already eating and eat with them’\newline
duaw ‘to stop by’\newline 
sud-o ‘to visit’\newline 
antos ‘to dive underwater’\newline 
abot ‘to arrive’\newline 
sayaw ‘to dance’\newline
tegbeng ‘to go downhill to a place’\newline
takas ‘to go up hill to a place’\\ 
% End of column 2, page 1
\lspbottomrule
\end{tabular}
% \footnotetext{}
\end{table}

\begin{table}
    \caption*{Groups of verbal roots depending on the dynamic intransitive irrealis inflection (cont.)}
    % \fittable{
    \begin{tabular} {
        >{\RaggedRight\arraybackslash}p{5.6cm}
        >{\RaggedRight\arraybackslash}p{5.6cm}
                    }
\lsptoprule
\multicolumn{1}{>{\centering\arraybackslash}m{5.6cm}}{\hspace{.7cm}\textbf{Group 1 (majority group)}} 
    & \multicolumn{1}{>{\centering\arraybackslash}m{5.6cm}}{\hspace{.7cm}\textbf{Group 2 (minority group)}} \\
\midrule
% begin column 1, page 2
\textbf{Other roots:}\newline 
ayad ‘to heal/be careful’\newline
baog ‘to feed an animal’\newline 
basa ‘to read’ \newline 
bayad ‘to pay’ \newline
biag ‘to capture’\newline 
bļebed ‘to wind/wrap around’\newline
bindisyon ‘to bless’\newline 
bubo ‘to pour out liquid’\newline 
bubod ‘to pour out dry’\newline 
bugno ‘to greet’\newline 
bui ‘to live’\newline 
kagat ‘to bite’\newline 
keme ‘to close your hand’\newline 
kingin ‘to slash \& burn in farming’\newline 
kingking ‘to hop’\newline 
kitkit ‘to nibble at’\newline 
kumpas ‘to wave arms’\newline 
kaddot ‘to pinch’\newline 
kusnit ‘to pinch’\newline 
daik ‘to crawl like a turtle’\newline 
dapa ‘to lie face down’\newline 
gyanap ‘to crawl (insects)’\newline 
gisa ‘to sauté s.t.’\newline 
giyo ‘to move’\newline 
gumod ‘to grumble’\newline 
lagaw ‘to go out visiting/strolling’\newline 
laóg ‘to go somewhere without permission or letting others know’\newline 
láog ‘to pass through/to connect’\newline 
laga ‘to boil s.t. in water’\newline
lamon ‘to pull  or cut weeds’\newline 
lawas ‘to go out of s.t.’% End column 1, page 2 % begin column 2 page 2
&
{\bfseries Semantically transitive, volitional situations with highly restricted objects:}\newline
kaan ‘to eat’\newline 
tumar ‘to take medicine’\newline
inem ‘to drink’\newline 
sakay ‘to ride’\newline 
asod ‘to pound grain with mortar and pestle’\newline 
sin-ad ‘to cook rice’\newline 
panggas ‘to plant grain’\newline 
sillig ‘to sweep’\newline 
suļat ‘to write’\newline
arek ‘to kiss’\newline 
akes ‘to hug’\newline 
bunak ‘to launder’\newline 
banyos ‘to rub medicine on body’\newline 
utang ‘to borrow money’\newline 
tukod ‘to build a house/building’\newline 
seyep ‘to slurp up, suck in’\newline 
padpad ‘to brush off’\newline 
tilaw ‘to taste/try/experience s.t.’\newline 
peddeng ‘to close eyes’\newline 
plantsa ‘to iron clothes’\newline 
pudyot ‘to pick up (s.t. small) with fingers’\newline 
petpet ‘chop down plant/bamboo/ small tree’\newline 
pukan ‘to chop down a big tree’\newline 
sagod ‘to take care of a person or animal’\newline
sekeb ‘to measure volume’\\ % End of column 2, p. 2. 
\lspbottomrule
\end{tabular}
\end{table}

\begin{table}
    \caption*{Groups of verbal roots depending on the dynamic intransitive irrealis inflection (cont.)}
    % \fittable{
    \begin{tabular} {
        >{\RaggedRight\arraybackslash}p{5.6cm}
        >{\RaggedRight\arraybackslash}p{5.6cm}
                    }
\lsptoprule
\multicolumn{1}{>{\centering\arraybackslash}m{5.6cm}}{\hspace{.7cm}\textbf{Group 1 (majority group)}} 
    & \multicolumn{1}{>{\centering\arraybackslash}m{5.6cm}}{\hspace{.7cm}\textbf{Group 2 (minority group)}} \\
\midrule
% Begin column 1 page 3
laygay ‘to preach or advise’\newline 
lubbas ‘to undress’\newline 
lumba ‘to race s.o.’\newline 
ani ‘to harvest’ (generic word) \newline
lungi ‘to harvest corn’\newline 
tubbas ‘to harvest grain’\newline 
tapas ‘to cut the stem of grain’\newline 
luod ‘to kneel’\newline 
lua ‘to spit out’\newline 
puļaw ‘to stay up late at night’\newline
puyat ‘to lack sleep’\newline 
sablig ‘to splash liquid on s.t.’\newline 
sili ‘to change something’\newline 
silot ‘to punish’\newline 
tuman ‘to obey/fulfill’\newline 
sikway ‘to reject’\newline 
kabig ‘to claim/possess/consider’\newline 
tekeb ‘to maul’\newline 
sigyet ‘to incite/encourage’\newline 
angken ‘to acquire’\newline 
nangken ‘to conceive/become pregnant’\newline 
taod ‘to respect’\newline 
desdes ‘to press down on’\newline 
singngot ‘to smell/sniff something’\newline 
peļļeg ‘to threaten’\newline
pati ‘to believe/obey’\newline 
kega ‘to choke’\newline 
bunaļ ‘to spank’\newline
lagas ‘to chase’\newline 
dugan ‘to be pinned under s.t.’\newline
dagammo ‘to dream’\newline 
lekkep ‘to cover over’ % End column 1 page 3, begin column 2 page 3
&
saļod ‘to catch s.t. falling or poured out into a container’\newline
sungit ‘to feed another by conveying food to their mouth’\newline 
sukoļ ‘to measure size’\newline
suot ‘to wear clothes/shoes’\newline
sandok ‘to fetch water’\newline 
saļok ‘to dip out liquid like water’\newline

{\bfseries Volitional intransitive situations of position}\newline

tindeg ‘to stand’\newline 
pungko ‘to sit’\newline
tuwad ‘to bend over’\newline
taliad ‘to bend backwards’\newline 
tungtong ‘to get on top of/perch on’\newline 
tubang ‘to turn to face’\newline 
tinir ‘to remain/stay/live temporarily s.w.’\newline
sandig ‘to lean against’\newline
leeb ‘to bow the head’\newline 

{\bfseries Utterance}\newline

singgit ‘to shout’\newline 
isturya ‘to talk/converse’\newline 
sugid ‘to tell’\newline 
ambaļ ‘to say’\newline
insa ‘to ask’\newline 
sabat ‘to answer/reply/respond to’\newline 
tugon ‘to instruct, give a message’\newline 
pangabay ‘to plead’\newline 
\textbf{Intentional} \textbf{visual} \textbf{perception:}\newline 
sil-ing ‘to peek inside’\newline 
tan-aw ‘to look at’\\ % End of column 2 p. 3. End of p. 3
\lspbottomrule
\end{tabular}
\end{table}

\begin{table}
    \caption*{Groups of verbal roots depending on the dynamic intransitive irrealis inflection (cont.)}
    % \fittable{
    \begin{tabular} {
        >{\RaggedRight\arraybackslash}p{5.6cm}
        >{\RaggedRight\arraybackslash}p{5.6cm}
                    }
\lsptoprule
\multicolumn{1}{>{\centering\arraybackslash}m{5.6cm}}{\hspace{.7cm}\textbf{Group 1 (majority group)}} 
    & \multicolumn{1}{>{\centering\arraybackslash}m{5.6cm}}{\hspace{.7cm}\textbf{Group 2 (minority group)}} \\
\midrule
% Begin Column 1 page 4
balikid ‘to look back over shoulders'\newline 
gugma ‘to love’\newline 
palangga ‘to give/show affection’\newline 
inggit ‘to envy’\newline 
kayab ‘to fan’\newline
kaļ-ay ‘to carry dangling down’\newline
laway ‘to salivate’\newline
pega ‘to squeeze’\newline 
betteng ‘to pull’\newline 
taap ‘to winnow’\newline 
bitbit ‘to hold in the hand’\newline 
tuwang ‘to carry two things on a pole’\newline 
kanyo ‘to enjoy (usually food)’\newline 
dampig ‘to side with’\newline 
ketteng ‘to cut through’\newline 
kereg ‘to shiver/tremble’\newline 
sangit ‘to snag on something’\newline 
palid ‘to blow away/off’’\newline 
takļap ‘to cover over’\newline
bis-ak ‘to split wood/coconut etc.’\newline 
tampaling ‘to slap in the face hard’\newline 
lagpi ‘to slap lightly’\newline 
demeļ ‘to lower the head, look downwards’\newline
kiyat ‘to wink’\newline
tande ‘to nod head for yes’\newline 
kulba ‘to be nervous/frightened’\newline
tangtang ‘to come apart’\newline 
kuslip ‘to insert into’\newline 
daag ‘to win’\newline
pirdi ‘to lose’\newline 
kiskis ‘to scrub s.t.’ % End column 1, page4. Begin column 2 page 4
&
luag ‘to watch’\newline 
sil-ip ‘to peek/watch’\newline 
angad ‘to look upwards’\newline 
panilag ‘to observe’\newline 
kilip ‘to glance out of corner of eye’\newline 
luaw ‘to look out an opening like a window’\newline 
{\bfseries Other volitional situations:}\newline
ubra ‘to work/make’\newline 
ugas ‘to wash’\newline 
kanta ‘to sing’\newline 
ampang ‘to play’\newline 
baļes ‘respond to/take revenge’\newline
atag ‘to give’\newline 
eļes ‘to borrow’\newline
batok ‘to rebel/resist/oppose/fight back’\newline 
baton ‘to receive/accept’\newline 
bugtaw ‘to wake up’\newline 
pukaw ‘to wake up s.o. else’\newline 
labet ‘to participate in’\newline 
palingki ‘to shop at a wet market’\newline 
palit ‘to buy’\newline 
pista ‘to attend a festival’\newline 
seey ‘to pout’\newline 
suka ‘to vomit’\newline 
tanuga/tunuga ‘to go to sleep’\newline 
anggat ‘to invite to accompany going somewhere’\newline
tubay ‘to greet/pay attention to s.o.’\newline
tudlo ‘to teach/point to/point out’\newline 
agaw ‘to grab something away’\newline
sudlay ‘to comb hair’\\
\lspbottomrule
% End of column 2 p. 4, end of p. 4.
\end{tabular}
\end{table}

\begin{table}
    \caption*{Groups of verbal roots depending on the dynamic intransitive irrealis inflection (cont.)}
    % \fittable{
    \begin{tabular} {
        >{\RaggedRight\arraybackslash}p{5.6cm}
        >{\RaggedRight\arraybackslash}p{5.6cm}
                    }
\lsptoprule
\multicolumn{1}{>{\centering\arraybackslash}m{5.6cm}}{\hspace{.7cm}\textbf{Group 1 (majority group)}} 
    & \multicolumn{1}{>{\centering\arraybackslash}m{5.6cm}}{\hspace{.7cm}\textbf{Group 2 (minority group)}} \\
\midrule
% Begin column 1, page 5
leged ‘to rub s.t. on’ \newline 
dagdag ‘to fall off’\newline 
dagbeng ‘to rumble/thud’\newline 
kusi ‘to pinch off’\newline 
tutod ‘to light a lamp/candle’\newline 
sepsep ‘to sip’\newline 
tuslok ‘to poke/stick s.t. inside’\newline 
batang/betang ‘to put’\newline 
paspas ‘to swipe away’\newline 
gaba ‘to be cursed for disrespecting an older person’\newline 
kagon ‘to be betrothed to someone’\newline 
kuno-kuno ‘to pretend’\newline 
pelles ‘strong winds/windy’\newline 
kinangļan ‘to need’\newline
lagtik ‘to make a small popping/ snapping sound’\newline 
lagtok ‘to make a big popping/ snapping sound’\newline 
timon ‘to steer a boat’\newline 
salamat ‘to thank’\newline 
daya ‘to deceive’\newline 
bula ‘to lie’\newline 
abri ‘to open/begin something’\newline 
bukas ‘to open something’\newline 
tuļo ‘to drip’\newline
busļo ‘to make a hole’\newline
tupak ‘to patch’\newline 
ngali ‘to dig up root crop’\newline 
dagyaw ‘to work cooperatively’\newline 
takilid ‘to be/turn on its side’\newline 
tuog ‘to penetrate through’\newline 
ļagok ‘to snore’% End column 1, page 5 % Begin column 2, page 5
&
pili ‘to choose something’\newline 
amag ‘to want to go with s.o.’\newline 
lakted ‘to cross over something'\newline
kablit ‘to touch/tap lightly’\newline 
lambay ‘to pass by’ \newline
ani ‘to harvest’ (generic word) \newline
kutkot ‘to dig’\newline 
layong ‘more than one carries s.t.’\newline 
tanem ‘to plant’\newline 
tampa ‘to slap the face hard’\newline 
tigbas ‘to chop with machete/axe’\newline 
lakkang ‘to step over something’\newline 
paļot ‘to peel’\newline
tambong ‘to attend/gather’\newline 
ikap ‘to touch, stroke or pet’\newline 
kaļot ‘to scratch at something itchy’\newline
ug-og ‘to mourn loudly’\newline 
\textbf{Verbs that take \textit{m}{}- but the initial dental approximate does not drop out}\newline
ļaļa ‘to weave’\newline
ļabo ‘to capsize’ \\ % End of column 2 p 5
\lspbottomrule
\end{tabular}
\end{table}

\begin{table}
    \caption*{Groups of verbal roots depending on the dynamic intransitive irrealis inflection (cont.)}
    % \fittable{
    \begin{tabular} {
        >{\RaggedRight\arraybackslash}p{5.7cm}
        >{\RaggedRight\arraybackslash}p{5.5cm}
                    }
\lsptoprule
\multicolumn{1}{>{\centering\arraybackslash}m{5.7cm}}{\hspace{.7cm}\textbf{Group 1 (majority group)}} 
    & \multicolumn{1}{>{\centering\arraybackslash}m{5.5cm}}{\hspace{.7cm}\textbf{Group 2 (minority group)}} \\
\midrule
% Beginning of column 1, p. 6
lam-ed ‘to swallow’\newline 
luib ‘to betray’\newline 
prusigir ‘to persevere’\newline 
pilak ‘to throw away’\newline
tanod ‘to watch children’\newline 
sirbi ‘to serve’\newline 
bangdan ‘to blame wrongly’\newline 
agdaw ‘to reduce fire’\newline 
saliga ‘to insult s.o.’\newline 
salig ‘to trust’\newline 
tuo ‘to believe in’\newline 
takmi ‘to sip a little’\newline 
bantay ‘to guard/watch over’\newline 
kuskos ‘to strum/scrub with hand’\newline 
kullaw ‘to wonder/puzzle about’\newline 
tingaļa ‘to be amazed’\newline
kudigo ‘to cheat in classroom’\newline 
kupya ‘to make a copy’\newline 
pirma ‘to sign one’s name’\newline 
demet ‘to hold a grudge’\newline 
disisyon ‘to decide’\newline 
luwas ‘to save from danger’\newline 
karagpa ‘to stumble’\newline 
tampek ‘to pack something on’\newline 
bantaw ‘to look at something from a distance’\newline 
balikid ‘to look back over shoulders’\newline 
siliring ‘to move head sideways’\newline 
tangkaeg ‘to crane  one’s neck’\newline 
lakbay ‘to skip over something’\newline 
...\newline 
(Most roots in the language).% End of column 1 page 6
& \\
\lspbottomrule
\end{tabular}
\end{table}

The following are a few examples of each of these major groups from the corpus. Examples \REF{ex:tobecomewell} through \REF{bkm:Ref150434151} illustrate Group 1 roots. The ungrammatical forms illustrate their non-occurrence with the \textit{m-} prefix:

\ea
Class II \textit{ayad} ‘to become well’ \\
\label{ex:tobecomewell}
Mag-ubra  danen  ta  duļot  agod  \textbf{mag-ayad}  ka. \\\smallskip
\gll Mag-ubra  danen  ta  duļot  agod  \textbf{mag-ayad}  ka. \\
\textsc{i.ir}-work/make  3\textsc{p.abs}  \textsc{nabs}  food.offering  so.that  \textsc{i.ir}-well  2\textsc{s.abs} \\
\glt ‘They will do a food offering so that you \textbf{will} \textbf{get} \textbf{well}.’ [SAWE-T-01 3.12] \\\smallskip
*mayad
\z

\ea
Class III \textit{tubo} ‘to grow’ \\
Daw  \textbf{magtubo}  en  batad  i,  oras-oras  man  na  lamunan. \\\smallskip
\gll Daw  \textbf{mag-tubo}  en  batad  i,  oras-oras  man  na  \emptyset{}-lamon-an. \\
if/when  \textsc{i.ir}-grow  \textsc{cm}  sorghum  \textsc{def.n}  \textsc{red}-hour/time  \textsc{emph}  \textsc{lk}  \textsc{t.ir}-weed-\textsc{apl} \\
\glt “When the sorghum grows, weed it all the time.’ [YBWE-T-04 2.2] \\\smallskip
*mubo
\z
\ea
Class IV \textit{lambay} ‘to pass by’ \\
\textbf{Maglambay}  kay  ta  isya  na  syudad. \\\smallskip
\gll \textbf{Mag-lambay}  kay  ta  isya  na  syudad. \\
\textsc{i.ir}-pass.by  1\textsc{p.excl.abs}  \textsc{nabs}  one  \textsc{lk}  city \\
\glt ‘We will pass by one city.’ [EMWN-T-09 5.7] \\\smallskip
*mambay
\z


\ea
Class VI \textit{buat} ‘to do/make something’: \\
Daw  yon  gusto  no  \textbf{magbuat}  ki  ta  imbitasyon  na tanan  na  ittaw  ta  ate  i  na  lugar  magtindir  ta  kasaļ  no. \\\smallskip
\gll Daw  yon  gusto  no  \textbf{mag-buat}  ki  ta  imbitasyon  na tanan  na  ittaw  ta  ate  i  na  lugar  mag-tindir  ta  kasaļ  no. \\
if/when  \textsc{d}3\textsc{abs}  want  2\textsc{s.abs}  \textsc{i.ir}-make  1\textsc{p.incl.abs}  \textsc{nabs}   invitation  \textsc{lk}
all  \textsc{lk}  person  \textsc{nabs}  1\textsc{p.incl.gen}  \textsc{def.n}  \textsc{lk}  place  \textsc{i.ir}-attend  \textsc{nabs}  wedding  2\textsc{s.gen} \\
\glt ‘If that is what you want, we will make invitations so that all the people in our place will attend your wedding.’ [CBWN-C-17 7.4] \\\smallskip
*muat
\z


\ea
Class VI \textit{taod} ‘to respect someone’ \\
Kyo  na  mga  kabataan,  dapat  gid  na  \textbf{magtaod}  kaw  ta  inyo na  mga  ginikanan  daw  inyo  na  mga  utod. \\\smallskip
\gll Kyo  na  mga  ka-bata-an,  dapat  gid  na  \textbf{mag-taod}  kaw  ta  inyo na  mga  ginikanan  daw  inyo  na  mga  utod. \\
2\textsc{p.abs}  \textsc{lk}  \textsc{pl}  \textsc{nr}-child- \textsc{nr}  must  \textsc{int}  \textsc{lk}  \textsc{i.ir}-respect  2\textsc{p.abs}  \textsc{nabs}  2\textsc{p.gen} \textsc{lk}  \textsc{pl}  parent  and  2\textsc{p.gen}  \textsc{lk}  \textsc{pl}  sibling \\
\glt ‘You children, you must really respect your parents and your siblings.’ [BCWL-T-12 2.1] \\\smallskip
*maod
\z

\ea
Class VII \textit{bayad} ‘to pay something to someone’ \\
Primiro  gid  na  pag-imuon  ta  ubra  ta  bļangay, \textbf{magbayad}  kay  ta  lisinsya  ta  pagkurti  ta  kaoy naan  ta  Bureau of Forestry  daw  pila  na  mitro  kubiko a  isya  na  bļangay. \\\smallskip
\gll Primiro  gid  na  pag-imo-en  ta  ubra  ta  bļangay, \textbf{mag-bayad}  kay  ta  lisinsya  ta  pag-kurti  ta  kaoy naan  ta  Bureau of Forestry  daw  pila  na  mitro  kubiko a  isya  na  bļangay. \\
first  \textsc{int}  \textsc{lk}  \textsc{nr.act}-do-\textsc{nr}  \textsc{nabs}  make/work  \textsc{nabs}  2.mast.boat
\textsc{i.ir}-pay  1\textsc{p.excl.abs}  \textsc{nabs}  license  \textsc{nabs}  \textsc{nr.act}-shape  \textsc{nabs}  woood
\textsc{spat.def}  \textsc{nabs}   Bureau of Forestry  if/when  how.many  \textsc{lk}  meter  cubic \textsc{inj}   one  \textsc{lk}  2.mast.boat \\
\glt ‘Really first when making a 2 mast boat, we pay for the license for cutting trees at the Bureau of Forestry for how many cubic meters is one 2 mast boat.’ [HCWE-J-01 1.1] \\\smallskip
*mayad
\z

\ea
\label{bkm:Ref150434151}
Class VIII \textit{tutod} ‘to light a candle or lamp’ \\
Gabangon   a  daw  \textbf{magtutod}  ta  pitrumaks. \\\smallskip
\gll Ga-bangon   a  daw  \textbf{mag-tutod}  ta  pitrumaks. \\
\textsc{i.r}-get.up  1\textsc{s.abs}  and  \textsc{i.ir}-light  \textsc{nabs}  petromax \\
\glt ‘I got up and lit a petromax.’ [JCWN-L-31 3.3] \\\smallskip
*mutod
\z

Examples \REF{bkm:Ref150434299} through \REF{bkm:Ref150430302} illustrate Group 2 verbs from Classes IV, VI, VII, and VIII that may take \textit{m}{}- as well as \textit{mag}{}- as the intransitive, irrealis inflection. Sometimes there is no discernible difference in meaning between the two forms, while other times the \textit{m}{}- forms describe more immediate, inevitable situations. The \textit{mag}{}- forms tend to describe more non-specific, habitual situations, often in combination with another clause in the same construction. Example \REF{bkm:Ref150434299} illustrates the culminative usage of irrealis modality (see \chapref{chap:verbstructure}, \sectref{sec:intransitiveirrealis}):

\ea
\label{bkm:Ref150434299}Class IV \textit{tunga} ‘to come up from under water’ \\
Apang  ta  pangallo  ko  na  pag-eseb \textbf{magtunga}  a  dagat  naan  tuman  ta  ake  na  ilek. \\\smallskip
\gll Apang  ta  pang-tallo  ko  na  pag-eseb \textbf{mag-tunga}  a  dagat  naan  tuman  ta  ake  na  ilek. \\
but  \textsc{nabs}  \textsc{ord}-three  1\textsc{s.gen}  \textsc{lk}  \textsc{nr.act}-go.underwater  \textsc{i.ir}-come.up.from.underwater  1\textsc{s.abs}  sea  \textsc{nabs}  until  \textsc{nabs}  1\textsc{s.gen}  \textsc{lk}  armpit \\
\glt ‘But on my third going under water I came up from under the sea to my armpits.’ [EFWN-T-11 14.7]
\z

In a different telling of the same story, the same speaker uses the \textit{m}{}- form, again in the culminative usage \REF{bkm:Ref150434555}. Here there does not seem to be any discernable difference in meaning:

\ea
\label{bkm:Ref150434555}Class IV \textit{tunga} ‘to come up from underwater’ \\
Pangallo  ko  na  eseb  \textbf{munga}  a  dagat naan  tuman  ta  ilek  i. \\\smallskip
\gll Pang-tallo  ko  na  eseb  \textbf{m-tunga}  a  dagat naan  tuman  ta  ilek  i. \\
\textsc{ord}-three  1\textsc{s.gen}  \textsc{lk}  go.underwater  \textsc{i.v.ir}-come.up.from.underwater  1\textsc{s.abs}  sea
\textsc{spat.def}   until  \textsc{nabs}  armpit  \textsc{def.n} \\
\glt ‘My third going under water I came up from under the sea until the armpits.’ [EFWN-T-10 4.17]
\z

\ea
Class VI \textit{kaan} ‘to eat’: \\
Daw  \textbf{magkaan}  kon  prinsisa  i  gatungtong  kon  pangka  i naan  ta  pinggan  din  na  gasalo. \\\smallskip
\gll Daw  \textbf{mag-kaan}  kon  prinsisa  i  ga-tungtong  kon  pangka  i naan  ta  pinggan  din  na  ga-salo. \\
If/whan  \textsc{i.ir}-eat  \textsc{hsy}  princess  \textsc{def.n}  \textsc{i.r}-got.on.top  \textsc{hsy}  frog  \textsc{def.n}
\textsc{spat.def}  \textsc{nabs}   dish  3\textsc{s.gen}  \textsc{lk}  \textsc{i.r}-eat.from.same.dish \\
\glt ‘When the princess was eating, the frog got on top of her dish eating from the same dish.’ [CBWN-C-17 5.4]
\z

\ea
Bisan  daw  imol  ki  nang  o  \textbf{maan}  ki  nang  ta  mga gamut  ta  kaoy  basta  gatingeb  ki  nang  magbata.\\\smallskip
\gll Bisan  daw  imol  ki  nang  o  \textbf{m-kaan}  ki  nang  ta  mga gamut  ta  kaoy  basta  ga-tingeb  ki  nang  mag-bata.\\
  even  if/when  poor  1\textsc{p.incl.abs}  only  or  \textsc{i.v.ir}-eat  1\textsc{p.incl.abs}  only  \textsc{nabs}  \textsc{pl}
root  \textsc{nabs}  tree  as.long.as  \textsc{i.r}-together  1\textsc{p.incl.abs}  only  \textsc{rel}-child \\
\glt ‘Even if we are only poor or we only eat roots of trees, (it is okay/it does not matter) just as long as we parents and children are together.’ [CBWE-C-05 4.2] 
\z

\largerpage
\ea
Class VII \textit{atag} ‘to give’: \\
Gasuļat  a  ki  kaon  tak  gusto  ko  man  na  \textbf{mag-atag} ta  lipo  nang  na  laygay  ki  kaon … \\\smallskip
\gll Ga-suļat  a  ki  kaon  tak  gusto  ko  man  na  \textbf{mag-atag} ta  lipo  nang  na  laygay  ki  kaon … \\
\textsc{i.r}-write  1\textsc{s.sabs}  \textsc{obl.p}  2s  because  want  1\textsc{s.erg}  also  \textsc{lk}  \textsc{i.ir}-give
\textsc{nabs}  short  only  \textsc{lk}  advice  \textsc{obl.p}  2s \\
\glt ‘I am writing to you because I want to give just some short advice to you …’ [YBWL-T-02 2.1]
\z

\ea
Daw  may  nanagat  na  kakamang  ta  sidda  \textbf{matag} gid  ki  kanen. \\\smallskip
\gll Daw  may  na-ng-dagat  na  ka-kamang  ta  sidda  \textbf{m-atag} gid  ki  kanen. \\
 if/when  \textsc{ext.in}  \textsc{a.hap.r}-\textsc{pl}-sea  \textsc{lk}  \textsc{i.exm}-get  \textsc{nabs}  fish  \textsc{i.v.ir}-give \textsc{int}  \textsc{obl.p}  3s \\
\glt ‘If there is someone fishing who got a/some fish, s/he really will give (some) to him.’ [JCWN-T-20 10.1]
\z

\ea
Class VIII \textit{tabang} ‘to help’ \\
Naan  ko  dya  nļaman  na  gangita  kanen  ta  duma na  \textbf{magtabang}  ki  kanen  naan  ta  translation. \\\smallskip
\gll Naan  ko  dya  na-aļam-an  na  ga-ngita  kanen  ta  duma na  \textbf{mag-tabang}  ki  kanen  naan  ta  translation. \\
\textsc{spat.def}  1\textsc{s.erg}  \textsc{d}4\textsc{loc}  \textsc{a.hap.r}-know-\textsc{apl}  \textsc{lk}  \textsc{i.r}-search  3\textsc{s.abs}  \textsc{nabs}  other
\textsc{lk}  \textsc{i.ir}-help  \textsc{obl.p}  3s  \textsc{spat.def}  \textsc{nabs}  translation \\
\glt ‘It was there I knew that she was searching for another to help her in translation.’ [SLWN-C-01 8.3]
\z

\ea
\textbf{Mabang}  a  gid  ki  kaon  a  tak  utod  ki. \\\smallskip
\gll \textbf{M-tabang}  a  gid  ki  kaon  a  tak  utod  ki. \\
  \textsc{i.v.ir}-help  1\textsc{s.abs}  \textsc{int}  \textsc{obl.p}  2s  \textsc{inj}  because  sibling  1\textsc{p.incl.abs} \\
\glt  ‘I will help you well because we are siblings.’ [RBON-T-01 3.8]
\z
\ea
Class VIII \textit{tudlo} ‘to teach/point to’ \\
  Tay,  Nay  miling  a  ta  Manila  tak  \textbf{magtudlo}  a  kon   ki  Pedro  ta  ambaļ  ta  na  Kagay-anen. \\\smallskip
\gll Tay,  Nay  m-iling  a  ta  Manila  tak  \textbf{mag-tudlo}  a  kon   ki  Pedro  ta  ambaļ  ta  na  Kagay-anen. \\
  dad  mom  \textsc{i.v.ir}-go  1\textsc{s.abs}  \textsc{nabs}  Manila  because  \textsc{i.ir}-teach  1\textsc{s.abs}  \textsc{hsy}
 \textsc{obl.p}   Pedro  \textsc{nabs}  say  1\textsc{p.incl.gen}  \textsc{lk}  Kagayanen \\
\glt  ‘Dad, Mom, I will go to Manila because I will teach Pedro our language Kagay-anen.’ [RCON-L-03 4.1]
\z

\newpage
\ea
\label{bkm:Ref150430302}
Uļa  baba,  uļa  irong,  uļa  mata,  piro  sikad  \textbf{mudlo}  daw  ano   dili  an  makita  ta  duma.  Sabat:  tudlo \\\smallskip
\gll Uļa  baba,  uļa  irong,  uļa  mata,  piro  sikad  \textbf{m-tudlo}  daw  ano\footnotemark{}   dili  an  ma-kita  ta  duma.  Sabat:  tudlo. \\
  \textsc{neg.r}  mouth  \textsc{neg.r}  noise  \textsc{neg.r}  eye  but  always  \textsc{i.v.ir}-teach  if/when  what
  \textsc{neg.ir}  \textsc{def.m}  \textsc{a.hap.ir}-see  \textsc{nabs}  others  answer finger \\
\footnotetext{This use of \textit{ano} is code-switching from \isi{Tagalog}.}
\glt  ‘It has no mouth, it has no nose, it has no eyes, but it’s always teaching/pointing to what others will not see. Answer: finger.’ (The Kagayanen word \textit{tudlo} can be either a verb meaning to teach or point to something or a noun meaning finger.) [SFWR-L-05 1.1-2]
\z

% \begin{verbatim} % % move bib entries to  localbibliography.bib
% \end{verbatim}
