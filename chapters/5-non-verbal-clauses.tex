\chapter{Non-verbal clauses}
\label{chap:non-verbalclauses}
\section{Introduction}
\label{sec:introduction-5}

Every language has constructions that express \textit{classification}\is{classification}\is{proper inclusion}, \textit{equation}\is{equative clauses}, \textit{attribution}\is{attributive clauses}\is{attribution}, \textit{location}\is{location}, \textit{existence}\is{existence} and \textit{possession}\is{predicative!possession}\is{possession!predicative} (defined below).  In Kagayanen these functions are typically expressed in clauses that do not involve an Inflected verb. Instead, the predicate in such constructions is a Basic Referring Phrase, a Modifier Phrase, a Genitive Referring Phrase, a Locative adverbial element (either a location word or a prepositional phrase), an Existential element, or a Manner adverbial. Consistent with the general typology of the language, non-verbal predicates normally precede their single argument, and optional adverbial or oblique elements. Example 1 illustrates the basic template for most non-verbal clauses. Existential and possessive constructions are also non-verbal, however many of them are represented by a slightly different template described in \sectref{sec:existentialconstructions} and \sectref{sec:predicativepossession}:

\ea
\label{bkm:Ref118885283}Predicate + Argument (Oblique)
\z

In Kagayanen there is no \isi{copula} or “coupling element” that intervenes between the predicate and the argument. Time adverbs, such as \textit{gibii} ‘yesterday’, \textit{kisyem} ‘yesterday,’ \textit{gina} ‘earlier’, \textit{kani} ‘later’, or \textit{basi} ‘maybe’,  are used to express the time and/or modal status of the situation if necessary (see examples below). As is generally true for all clause types, the basic Predicate+Argument order may be inverted for a variety of discourse functions, including \isi{emphasis}, new topic\is{topicalization}\is{topic}, or \isi{focal prominence}\is{focus} (see \chapref{chap:pragmaticallymarkedstructures}, \sectref{sec:constituentordervariation} for a discussion of these concepts). 

Constructions that belong to the family of non-verbal predicates are described below as predicate Nominals (\sectref{sec:predicatenominals}), predicate Modifiers (\sectref{sec:predicatemodifiers}), Locational Clauses (\sectref{sec:locationalclauses}), Existential Constructions (\sectref{sec:existentialconstructions}), Possessive Clauses (\sectref{sec:predicativepossession}), Manner Clauses (\sectref{sec:mannerclauses}), and Comparative and superlative constructions (\sectref{sec:comparativesuperlativeclauses}).

\section{Predicate nominals}
\label{bkm:Ref428962397}\label{sec:predicatenominals}
\is{predicate nominals|(}
The basic functions of predicate nominals are to express classification \is{classification|(} \is{proper inclusion} and \isi{equation}. Classification is the idea that a particular argument is properly included within a set of things, for example, “Pedro is a teacher,” or “the earth is a planet.” These propositions express the ideas that Pedro is a member of the class of teachers, and the earth is a member of the class of planets, respectively. Classification may be expressed with a “proper subset sign”: X${\subset}$Y (X is included within Y) in mathematical notation. Equation \is{equation}is the idea that the argument is equated, or identified with some specific item, for example, “Pedro is the teacher” or “the earth is the planet I live on”. In these cases, the predicate and argument can be related with an “equals sign”: X=Y (X and Y are identical). The following are some basic examples in Kagayanen. In these examples, the argument is bracketed:

\ea
Classification (Our vehicle ${\subset}$ two masted ship)  \\
Bļangay  ame  na  sakayan. \\\smallskip
Predicate\hspace{.8cm}  Argument \\
\gll Bļangay [ ame  na  sakay-an ]. \\
2.masted.boat  {} 1\textsc{p.excl.gen}  \textsc{lk}  ride\textsc{-nr} \\
\glt ‘Our vehicle is a two masted boat.’
\is{classification|)}
\z
\ea
\label{bkm:Ref442630001}\is{equation|(}
Equation (Pedro = my cousin) \\
Katagsa  ko  Pedro  an. \\\smallskip
Predicate\hspace{.8cm}  Argument \\ 
\gll Katagsa  ko [ Pedro  an{ }]. \\
cousin  1\textsc{s.gen} {} Pedro  \textsc{def.m} \\
\glt ‘Pedro is my cousin.’
\z

In the case of equative clauses, it is often difficult or impossible to determine which element is the argument and which is the predicate, since of course the equality relationship is symmetric. In general we can say that the argument is likely to be the \isi{discourse topic} (see, e.g., \citealt{givon2001-2}), and the predicate is a comment on that \isi{topic}. For example, if the topic of conversation is “Pedro,” then a sentence like \REF{bkm:Ref442630001} would be a Predicate+Argument structure. On the other hand, if the topic of conversation were “my cousin”, the same sentence would have the pragmatically marked Argument+Predicate structure ‘As for my cousin (he) is Pedro.’  This would likely be used if the speaker wished to contrast the equality of Pedro and the cousin with a different idea held by the hearer, for example, ‘Pedro is my brother,’ or ‘Juan is my cousin.’\is{equation|)}

The following are some additional examples of predicate nominal clauses from the text corpus:  

\newpage
\ea
Ame  na  sakayan  isya  na  bļangay  na  ngaran  din  Linda. \\\smallskip
Argument\hspace{2.7cm} Predicate \\
\gll [ Ame  na  sakay-an ]  isya  na  bļangay  na  ngaran  din  Linda. \\
{} 1\textsc{p.excl.gen}  \textsc{lk}  ride\textsc{-nr} {} one    \textsc{lk}  2.masted.boat  \textsc{lk}  name  3\textsc{s.gen}  Linda \\
\glt ‘Our vehicle was a 2 mast ship which was named Linda.’ [VAWN-T-19 2.4]
\z
\ea
Taga-Cagayancillo  yaken  i. \\\smallskip
Predicate\hspace{1.8cm}Argument \\
\gll Taga-Cagayancillo\footnotemark{} [  yaken  i{ }]. \\
from-{}-Cagayancillo {} 1\textsc{s.abs}  \textsc{def.n} \\
\footnotetext{The stem-forming prefix (\textit{taga}{}- derives a noun that means someone whose origin is the place named in the root. See \chapref{chap:referringexpressions}, \sectref{sec:taga}.)}
\glt ‘I am from Cagayancillo.’ [BMON-C-05 1.2]
\z
\ea
Cagayancillo  puļo  na  madyo  ta  duma  na  puļo. \\\smallskip
\hspace{.2cm}Argument\hspace{.8cm}Predicate \\
\gll [ Cagayancillo ] puļo  na  madyo  ta  duma  na  puļo{}. \\
{} Cagayancillo {} island  \textsc{lk}  far  \textsc{nabs}  other  \textsc{lk}  island \\
\glt ‘Cagayancillo is an island that is far from other islands.' [JCWN-L-33 38.1]
\z

Note that when the argument is an enclitic pronoun\is{enclitic pronouns}, it occurs after the first element in the predicate. This means that if the predicate includes more than one word, the pronominal argument “intrudes”\is{pronoun intrusion} within the predicate phrase, as in examples \REF{bkm:Ref442631395} and \REF{bkm:Ref428962681}: 

\ea
\label{bkm:Ref442631395}
... sawa a no. \\\smallskip
Predicate  Arg.\hspace {.1cm} Predicate \\
\gll ... sawa   [{ }a{ }] no. \\
{}  spouse  1\textsc{s.abs} 2\textsc{s.gen} \\
\glt ‘... I am your spouse.’ [PBON-T-01 2.23]
\z

Example \REF{bkm:Ref428962681} contains a phrase, \textit{ta highschool}, “of high school” modifying the RP that functions as the predicate. Thus “high school teachers” is the set, and the clause asserts that “s/he” is a member of that set:

\newpage
\ea
\label{bkm:Ref428962681}
Maistro  ta  highschool  kanen  an. \\\smallskip
Predicate\hspace{2.7cm}Argument \\
\gll Maistro  ta  highschool [ kanen  an\hspace{3pt}]. \\
teacher  \textsc{nabs}  highschool {}  3\textsc{s.abs}  \textsc{def.m} \\
\glt ‘S/he is a high school teacher.’
\z

While example \REF{bkm:Ref428962681} is grammatical, the more common order when the argument is instantiated as a pronoun is for the pronoun to occur after the first major constituent of the predicate, as in example \REF{bkm:Ref428962810}. As mentioned in \chapref{chap:modification}, \sectref{sec:attributivemps}, such “\isi{intrusive pronouns}”\is{pronoun intrusion} are formally \isi{second-position enclitics}\is{enclitics!second-position}, and as such do not usually take \isi{demonstrative determiners}:

\ea
\label{bkm:Ref428962810}
Maistro  kanen  ta  highschool. \\\smallskip
Predicate  Arg.\hspace{.5cm}  Predicate \\
\gll Maistro{  }[ kanen ] ta  highschool. \\
teacher 3\textsc{s.abs}  {}  \textsc{nabs}  high.school \\
\glt ‘S/he is a highschool teacher.’ 
\z
Any difference in meaning between \REF{bkm:Ref428962681} and \REF{bkm:Ref428962810} is subtle, and is not usually pertinent in conversation---the two may be considered paraphrases of one another, though \REF{bkm:Ref428962810} is clearly preferred.

A predicate nominal construction may contain a location phrase, as in the following.

\ea
Maistro  kanen  ta  highschool  naan  ta  Puerto. \\\smallskip
Predicate  Arg.\hspace{.5cm}Predicate\hspace{1.2cm}  Location \\
\gll Maistro [ kanen ] ta  highschool  naan  ta  Puerto. \\
teacher {} 3\textsc{s.abs} {} \textsc{nabs}  high.school  \textsc{spat.def}  \textsc{nabs}  Puerto \\
\glt ‘S/he is a high school teacher in Puerto.’ or ‘S/he is a teacher in the high school in Puerto.’ 
\z

As mentioned above, the order of predicate and argument may be inverted for various pragmatic purposes. This is consistent with the function of argument fronting throughout the language (see \chapref{chap:pragmaticallymarkedstructures}, \sectref{sec:constituentordervariation}). Small capitalization is used in free English translations of the following examples to approximate the contrastive sense of these marked constructions:

\newpage
\ea
Kanen  an  maistro  ta  highschool. \\\smallskip
\hspace{.2cm}Argument\hspace{.8cm}  Predicate \\
\gll [ Kanen  an{ }] maistro  ta  highschool. \\
{} 3\textsc{s.abs}  \textsc{def.m}  teacher  \textsc{nabs}  high.school \\
\glt ‘\textsc{s/he} is a highschool teacher.’
\z
\ea
Yi  en  inyo  na  apo. \\\smallskip
\hspace{.2cm}Arg.\hspace{.6cm}  Predicate \\
\gll [{ }Yi{ }]  en  inyo  na  apo. \\
\textsc{d1abs}   \textsc{cm}  2\textsc{p.gen}  \textsc{lk}  descendant \\
\glt ‘\textsc{this} now is your descendant.’ [JCWE-T-15 5.1]
\z

The predicate and argument may also be inverted when the predicate is particularly “heavy”.\is{heavy predicates} In example \REF{bkm:Ref442262389} the predicate contains a rather long modifying clause. In such cases, inversion is normal, and does not necessarily indicate \isi{contrastive focus}. 

\ea
\label{bkm:Ref442262389}
Lalong  klasi  ta  sidda  na  tama  tellek  iya  na  lawa. \\\smallskip
\hspace{.1cm}Argument  Predicate \\
\gll [ Lalong{ }] klasi  ta  sidda  na  tama  tellek  iya  na  lawa. \\
{} lion.fish type  \textsc{nabs}  fish  \textsc{lk}  many  prickles  3\textsc{s.gen}  \textsc{lk}  body \\
\glt ‘Lion fish is the type of fish that has many prickles on his/her body.’ [VAWE-T-10 9.1]
\is{predicate nominals|)}
\z

\section{Predicate modifiers}
\label{bkm:Ref444443349}\label{sec:predicatemodifiers}\is{predicate modifiers|(}\is{predicative!modification}

The basic function of predicate modifiers is to assert an attribute\is{attributive predicates} or property of some argument. The predicate of a predicate modifier clause is a Modifier Phrase (see \chapref{chap:modification}, \sectref{sec:modifierphrases}), examples \REF{bkm:Ref329869322}, \REF{bkm:Ref428973739}, and \REF{bkm:Ref428973723}, or a Referring Phrase (RP) with a modifier, as in \REF{bkm:Ref445618532}. The argument can be any definite RP, as in \REF{bkm:Ref329869322} and \REF{bkm:Ref428973723}, or a nominalized clause as in \REF{bkm:Ref428973739}. In the examples in this section, the argument will be surrounded by square brackets: 

\ea
\label{bkm:Ref329869322}\label{bkm:Ref442279471}
Piro  mga  salingan  nay  sikad  gid  dessen  iran  na  tagipusuon. \\\smallskip
\gll Piro  mga  salingan  nay  sikad  gid  dessen [ iran  na  tagipusuon ]. \\
but  \textsc{pl}  neighbor  1\textsc{p.excl.gen}  very  \textsc{int}  hard {} 3\textsc{p.gen}  \textsc{lk}  heart \\
\glt ‘But as for our neighbors their hearts are very hard.’ [ETON-C-07 2.9]
\z
\ea
\label{bkm:Ref428973739}
Sikad  gid  tudo  iran  na  pag-dļagan. \\\smallskip
\gll Sikad  gid  tudo [ iran  na  pag-dļagan{ } ]. \\
very  \textsc{int}  intense {} 3\textsc{p.gen}  \textsc{lk}  \textsc{nr.act}-run \\
\glt ‘Their running was really very intense.’
\z
\ea
\label{bkm:Ref428973723}\label{bkm:Ref442631809}
Miad    mga  Kagayanen  an. \\\smallskip
\gll Miad  [ mga  Kagayanen  an{ }]. \\
kind   {} \textsc{pl}  Kagayanen  \textsc{def.m} \\
\glt ‘The Kagayanen (people) are kind.’
\z
\ea
\label{bkm:Ref445618532}
Kagayanen  miad  na  mga  ittaw. \\\smallskip
\gll [ Kagayanen ] miad  na  mga  ittaw. \\
{} Kagayanen {} kind  \textsc{lk}  \textsc{pl}  person \\
\glt ‘Kagayanens are kind people.’ [JCWN-T-24 4.2]
\z

As with \isi{predicate nominals}, if the argument is a pronoun, it usually intrudes\is{pronoun intrusion} after the first element of the predicate, creating a \isi{discontinuous predicate}. Example \REF{bkm:Ref428973688} would be more likely in daily conversation than \REF{bkm:Ref442632390}: 

\ea
\label{bkm:Ref428973688}
Sikad kanen kon  miad  na  dļaga. \\\smallskip
\gll Sikad [ kanen ] kon  miad  na  dļaga. \\
very {} 3\textsc{s.abs} {} \textsc{hsy}  kind  \textsc{lk}  single.woman \\
\glt ‘She is a very kind single woman, they say.’ [EMWL-T-04 5.1]
\z
\ea
\label{bkm:Ref442632390}
Sikad  kon  miad  na  dļaga  kanen  an. \\\smallskip
\gll Sikad  kon  miad  na  dļaga [ kanen  an{ }]. \\
very  \textsc{hsy}  kind  \textsc{lk}  single.woman  {} 3\textsc{s.abs} \textsc{def.m} \\
\glt ‘She is a very kind single woman, they say.’
\z

As with predicate nominals, the order of predicate and argument may be inverted to express \isi{contrastive focus} (small capital letters in the free translations of the following examples indicate the sense of contrast):

\ea
Kanen  an  sikad  kon  miad  na  dļaga. \\\smallskip
\gll [ Kanen  an{ }] sikad  kon  miad  na  dļaga. \\
{} 3\textsc{s.abs}  \textsc{def.m} very  \textsc{hsy}  kind  \textsc{lk}  single.woman \\
\glt ‘\textsc{she} is a very kind single woman they say.’
\z

Example \REF{bkm:Ref442632516} is a particularly clear example of \isi{Argument+Predicate constructions} used for contrastive focus. In this example the speaker uses two predicate modifier constructions in a row to contrast the characters of two brothers.

\ea
\label{bkm:Ref442632516}
Mangngod  ya  a  dili  maisa.  Maguļang  ya  a  maisaen. \\\smallskip
\gll [ Mangngod  ya{ }] a  dili  ma-isa.  [ Maguļang  ya{ } ] a  ma-isa-en.\footnotemark{}  \\
{} younger.sibling  \textsc{def.f} \textsc{ctr}  \textsc{neg.ir}  \textsc{adj}-selfish {} older.sibing  \textsc{def.f} {} \textsc{ctr}  \textsc{adj}-selfish-\textsc{adj} \\
\footnotetext{The use of -\textit{én} in the second part of this contrastive pair of clauses indicates that the older brother had an inherent tendency toward selfishness. In the first part, the same root is used without -\textit{én}. This is because one can’t have an inherent tendency toward not being selfish---unselfishness is the norm. See \chapref{chap:modification}, \sectref{sec:adjectiveformingprocesses} for a discussion of \textit{ma-}, versus \textit{ma- . . . -én} as \isi{adjective-forming processes}.}
\glt ‘\textsc{the younger brother} was not selfish. \textsc{the older brother} was selfish.’ [MBON-T-04 13.4-5]
\is{predicate modifiers|)}
\z
\section{Locational clauses}
\label{bkm:Ref229903412}\label{sec:locationalclauses}\is{locational clauses|(}

A locational clause functions to express the location of some argument. In Kagayanen, any prepositional phrase may appear in the predicate position, while the located argument normally follows. The preposition is \textit{naan} (with the idiolectal variants \textit{yaan} and \textit{nyaan}) when the predicate is a definite location, as in \REF{bkm:Ref107331810} through \REF{bkm:Ref444841001}: 
\ea
\label{bkm:Ref107331810}
Naan  ki  kanen  bata  ko  ya. \\\smallskip
\gll Naan  ki  kanen  [ bata  ko  ya{ }]. \\
\textsc{spat.def}  \textsc{obl.p} 3s {} child  1\textsc{s.gen}  \textsc{def.f} \\
\glt ‘My child is there with her/him.’ \\
\z
\ea
Naan  ta  dagat  tama  mga  sidda. \\\smallskip
\gll Naan  ta  dagat [ tama  mga  sidda{ }]. \\
\textsc{spat.def}  \textsc{nabs}  sea {} many  \textsc{pl}  fish \\
\glt ‘Many fish are in the sea.’ (Even though the quantifier \textit{tama} appears here, it also has a plural marker possibly because there are many different kinds of fish.) [JCON-L-07 2.4]
\z

\newpage
\ea
Naan  ta  Guimaras  kabatuan  man  pario  ta  ate  na  banwa. \\\smallskip
\gll Naan  ta  Guimaras  [ ka-bato-an  man  pario  ta  ate  na  banwa{ }]. \\
\textsc{spat.def}  \textsc{nabs}  Guimaras  {} \textsc{nr}-rock-\textsc{nr}  also  like  \textsc{nabs}  1pl\textsc{incl.gen}  \textsc{lk}  town \\
\glt ‘Places with many rocks like in our town are in Guimaras.’ [SFOB-L-01 7.22]
\z
\ea
\label{bkm:Ref444841001}
Naan  ta  Manila  sawa  ko  ya. \\\smallskip
\gll Naan  ta  Manila  [ sawa  ko  ya{ }]. \\
\textsc{spat.def}  \textsc{nabs}  Manila {} spouse  1\textsc{s.gen}  \textsc{def.f} \\
\glt ‘My spouse is in Manila.’
\z

When the argument is a pronoun, the enclitic form normally intrudes\is{pronoun intrusion} within the predicate, and the \textit{ta} prenominal case marker\is{prenominal case markers} is optional, as in \REF{bkm:Ref429114698} through \REF{ex:yourplace}.

\ea
\label{bkm:Ref429114698}
Naan  kanen (ta)  Manila. \\\smallskip
\gll Naan  [ kanen{ }]  (ta)  Manila. \\
\textsc{spat.def} {} 3\textsc{s.abs} \textsc{nabs}  Manila \\
\glt ‘S/he is in Manila.’
\z

Examples \REF{ex:inpuerto} and \REF{ex:yourplace} are from the corpus. In both of these examples, the optional \textit{ta} is retained.

\ea
\label{ex:inpuerto}
... yaan  kay  ta  Puerto.  \\\smallskip
\gll ... yaan  [{ }kay{ }] ta  Puerto.  \\
{} \textsc{spat.def} 1\textsc{p.excl.abs}   \textsc{nabs}  Puerto \\
\glt ‘... we were in Puerto.’ [ETON-C-07 1.2]
\z
\ea
\label{ex:yourplace}
Salig  danen  kani  Papa  a  daw  naan  ki  pa  ta  inyo  ya. \\\smallskip
\gll Salig  danen  kani  Papa  a  daw  naan  [{ }ki{ }] pa  ta  inyo  ya. \\
think.wrongly  3\textsc{p.erg}  later  papa  \textsc{inj}  if/when  \textsc{spat.def}   1\textsc{p.incl.abs}  \textsc{inc}  \textsc{nabs}  2\textsc{p.gen}  \textsc{def.f} \\
\glt ‘Our papa and companions will think wrongly later that we are still at your (place).’ [CBWN-C-11 5.20]
\z

As with other non-verbal predicates, the order of predicate and argument may be inverted for \isi{contrastive focus}:

\ea
Kanen  ya  naan  ta  Manila. \\\smallskip
\gll [ Kanen  ya{ }] naan  ta  Manila. \\
{} 3\textsc{s.abs}  \textsc{def.f} \textsc{spat.def}  \textsc{nabs}  Manila. \\
\glt ‘\textsc{s/he} is in Manila.’
\z

Examples \REF{bkm:Ref442263283} through \REF{bkm:Ref444840137} illustrate locational clauses with locational demonstratives in the predicate:

\ea
\label{bkm:Ref442263283}
Naan  dya  Pedro   ya. \\\smallskip
\gll Naan  dya  [ Pedro   ya{ }]. \\
\textsc{spat.def}  \textsc{d4loc} {} Pedro  \textsc{def.f} \\
\glt ‘Pedro is over there.’
\z
\ea
Uļa  pa  tagaok  manok  ya  gina  nyaan  kay  di  en. \\\smallskip
\gll Uļa  pa  tagaok  manok  ya  gina  nyaan [{ }kay{ }] di  en. \\
\textsc{neg.r}  \textsc{inc}  crow  chicken  \textsc{def.f}  earlier  \textsc{spat.def}  1\textsc{p.excl.abs}  \textsc{d}1\textsc{loc}  \textsc{cm} \\
\glt ‘The chicken had not yet crowed earlier and we were here.’ [BGON-L-01 3.17]
\z
\ea
... nyaan  dya  iran  ya  na  lugar. \\\smallskip
\gll ... nyaan  dya  [ iran  ya  na  lugar{ }]. \\
{} \textsc{spat.def}  \textsc{d}4\textsc{loc} {} 3\textsc{p.abs}  \textsc{def.f}  \textsc{lk}  place \\
\glt ‘...their place was there.’ [TTOB-J-01 4.3]
\z
\ea
Sir,  melled  ka.  Nyaan  ka  ti selled  i. \\\smallskip
\gll Sir,  m-selled  ka.  Nyaan  [{ }ka{ }] ti\footnotemark{}  selled  i. \\
Sir  \textsc{i.v.ir}-inside  2\textsc{s.abs}  \textsc{spat.def}  2\textsc{s.abs}  \textsc{d}1\textsc{loc.pr}  inside  \textsc{def.n} \\
\footnotetext{The form \textit{unti} often reduces to \textit{ti}, as in this example and the next.}
\glt ‘Sir, come inside. You will be right here inside.’ [BGON-L-01 4.11-12]
\z

In example \REF{bkm:Ref444840137}, the argument (“I”) appears as an enclitic in second position, and then is recapitulated in its full form following the \isi{locational demonstrative}:

\ea
\label{bkm:Ref444840137}\label{bkm:Ref444864313}
Naan  a  ti  yaken  i. \\\smallskip
\gll Naan  [{ }a{ }] ti [ yaken  i{ }]. \\
\textsc{spat.def} 1\textsc{s.abs}  \textsc{d}1\textsc{loc.pr} {} 1\textsc{s.abs}  \textsc{def.n} \\
\glt ‘I am right here.’ [PBON-T-01 2.26]
\z

This doubling of the argument pronoun is quite common in conversation. It adds an emotional nuance, such as excitement, wonder, happiness, and so on. It may also be a way of attracting the hearer’s attention.

Other \isi{prepositions} may also occur in the predicate of a locational clause. However, we use the term “locational” in a general way to describe all clauses in which the predicate is a prepositional phrase or \isi{locational demonstrative}, even if the semantic relation is not, strictly speaking, locational. For example, prepositional phrases with \textit{parti} or \textit{tenged} ‘about’ may occur in a non-verbal clause (\ref{bkm:Ref246297918} and \ref{ex:fish}). In this case, the marked order Argument+Predicate\is{Argument+Predicate constructions} occurs quite frequently since this construction type functions to introduce important participants into the discourse, as in \REF{ex:fish} and \REF{bkm:Ref429136467}:

\ea
\label{bkm:Ref246297918}
Parti  ta  sidda  yi  na  libro. \\\smallskip
\gll Parti  ta  sidda  [ yi  na  libro{ }]. \\
about  \textsc{nabs}  fish  {} \textsc{d1adj}  \textsc{lk}  book \\
\glt ‘This book is about fish.’
\z
\ea
\label{ex:fish}
Isturya  na  i  tenged  a  ta  mga  sidda. \\\smallskip
\gll [ Isturya  na  i{ }] tenged  a  ta  mga  sidda. \\
 {} story  \textsc{lk} \textsc{d1abs} about  \textsc{inj}  \textsc{nabs}  \textsc{pl}  fish \\
\glt ‘This story is about fish.’ [JCON-L-07 2.3]
\z
\ea
\label{bkm:Ref429136467}
Yi  na  isturya  parti  ta  darwa  na  magsawa. \\\smallskip
\gll [ Yi  na  isturya ] parti  ta  darwa  na  mag-sawa. \\
{} \textsc{d1adj}  \textsc{lk}  story {} about  \textsc{nabs}  two  \textsc{lk}  \textsc{rel}-spouse \\
\glt ‘This story is about a married couple.’ [PBON-T-01 2.1]
\is{locational clauses|)}
\z
\section{Existential constructions}
\label{sec:existentialconstructions} \is{existential constructions|(}

Existential constructions predicate the existence of something, usually in some specific location. In Kagayanen, existential constructions normally have three constituents: an existential particle\is{existential particles} or particles functioning as the predicate, a Referring Phrase functioning as the argument (the item that exists), and an optional location, as shown in \REF{ex:existentialtemplate}. 
\ea
\label{ex:existentialtemplate}
Predicate\textsubscript{exist}  Argument  (Location)
\z

\subsection{Basic existential constructions}
\label{ex:basicexistentialconstructions}\is{existential constructions!basic|(}

There are two existential particles in Kagayanen which collaborate to define three existential constructions. The two forms are \textit{may} ‘indefinite existential particle’\is{indefinite existential particle}\is{existential particles!indefinite}, and \textit{anen}, the ‘given existential particle\is{given existential particle}\is{existential particles!given}’. \textit{May} without \textit{anen} defines the \textit{indefinite new} \is{indefinite new existential construction}\is{existential constructions!indefinite new}existential construction \REF{bkm:Ref329888024}. This construction introduces a participant for the first time into the discourse, “out of the blue” so to speak. \textit{Anen} without \textit{may} defines the \textit{definite given} existential construction\is{definite given existential construction}\is{existential constructions!definite given} \REF{bkm:Ref329888060}-\REF{bkm:Ref442643960}, which asserts the presence of a particular referent that the speaker assumes the hearer can identify, and is probably aware of at the moment of speaking. In addition, \textit{may} and \textit{anen} may co-occur to form what we identify as the \textit{indefinite given} existential construction\is{indefinite given existential construction}\is{existential constructions!indefinite given} \REF{bkm:Ref329888090}. This construction is used when the speaker presupposes that the hearer is thinking about, or is aware of some defined group of referents (e.g. ‘men’), but cannot identify the particular referent or subgroup of referents being singled out. In the following examples, the English free translations ‘some N’ or ‘some Ns’ are used to refer to the argument in an indefinite, given construction, though this is not a fully adequate translation. For instance, \REF{bkm:Ref329888090} below may be the answer to a question such as “Are there any men here?” The question refers to a set of possible referents, namely ‘men’, and the response implies that one or more of those are present, but the speaker does not think the hearer can identify which specific one or ones might be intended. Note that \textit{may} and \textit{anen} are dedicated \isi{existential particles}. They are not locational demonstratives meaning ‘here’ or ‘there’.
 
\ea
\label{bkm:Ref329888024} \is{indefinite new existential construction}\is{existential constructions!indefinite new}
\textbf{Indefinite}, \textbf{new}: May  mama  di. \\\smallskip
\glll \textnormal{Predicate}  \textnormal{Argument}  \textnormal{(Location)} \\
\textit{May}  \textit{mama}  \textit{di}. \\
\textsc{ext.in}  man  \textsc{d}1\textsc{loc} \\
\glt  ‘There is a man here.’
\z
\ea
\label{bkm:Ref329888060}
\textbf{Definite,} \textbf{given}:    Anen  a  di. \\\smallskip
\gll Anen  a  di. \\
\textsc{ext.g}  1\textsc{s.abs}  \textsc{d}1\textsc{loc} \\
\glt `I’m here.’
\z
\ea
\is{definite given existential construction}\is{existential constructions!definite given}
\textbf{Definite,} \textbf{given}:    Anen  di  yaken  i. \\\smallskip
\gll Anen  di  yaken  i. \\
\textsc{ext.g}  \textsc{d}1\textsc{loc}  1\textsc{s.abs}  \textsc{def.n} \\
\glt  ‘\textbf{\textsc{I’m}} here.’ (Contrasitive)
\z
\ea
\label{bkm:Ref442643960}
\textbf{Definite,} \textbf{given}:  Anen  di  mama  an. \\\smallskip
\gll Anen  di  mama  an. \\
\textsc{ext.g  d}1\textsc{loc}  man  \textsc{def.m} \\
\glt ‘The man is here.’ (we’ve been expecting a particular man.)
\z
\ea
\label{bkm:Ref329888090} \is{indefinite given existential construction}\is{existential constructions!indefinite given}
\textbf{Indefinite,} \textbf{given}:  May   anen  mama  di. \\\smallskip
\gll May   anen  mama  di. \\
\textsc{ext.in}  \textsc{ext.g}  man  \textsc{d}1\textsc{loc} \\
\glt ‘There’s some man/men here.’ (‘Men’ have been mentioned or otherwise presupposed, but the particular man or men have not been identified).
\z

\hspace*{-5.8pt}In the indefinite constructions (i.e., those with \textit{may}), the argument is a “stripped” Referring Phrase\is{stripped Referring Phrases} in that it takes no demonstrative determiners, genitive pronouns, or demonstratives \REF{bkm:Ref329889347}-\REF{bkm:Ref442938671}. Also, it may not be replaced by a pronoun (see \citealt{miner1986} on the notion of “noun stripping” and its similarity to noun incorporation). However, it may take quantifiers, relative clauses, and other modifiers. This stripped argument is clearly not absolutive because it does not take determiners, and does not participate in syntactic processes, such as fronting and relativizability, that are allowed for absolutives. While the stripped argument is not absolutive, neither is it Oblique in that it does not take any of the Oblique prepositions (including the non-absolutive case marker \textit{ta}) and it may not be fronted, as can Obliques. Thus the indefinite existential constructions are among the very few constructions in Kagayanen that do not require an absolutive constituent. Another salient characteristic of the two \isi{indefinite existential constructions} is that they serve as the bases for a type of predicative possession\is{predicative!possession} in which the possessed item is the stripped argument, and a possessor occurs as a distinct absolutive referring expression (see \sectref{sec:predicativepossession} below).

The following are examples of the indefinite, new existential construction:
 \is{indefinite new existential construction}\is{existential constructions!indefinite new}
\ea
\label{bkm:Ref329889347}
Argument with adjective: \\
May  manakem  na  mama. (Or: May mama na manakem.) \\\smallskip
\gll May  manakem  na  mama. \\
\textsc{ext.in}  older  \textsc{lk}  man \\
\glt ‘There’s an older man.’
\z
\ea
Argument with number: \\
May  tallo  na  mama. (*May mama na tallo.) \\\smallskip
\gll  May  tallo  na  mama. \\
\textsc{ext.in}  three  \textsc{lk}  man \\
\glt ‘There are three men.’ \\
\z
\ea
May tallo buok na mama. Or: May mama na tallo buok. \\\smallskip
\gll May tallo buok na mama. \\
\textsc{ext.in}  three  \textsc{cl} \textsc{lk}  man \\
\glt ‘There are three individual men.’
\z
\ea
Argument with location phrase:\footnotemark \\
May  manakem  na  mama  naan  ta  gwa.  \\\smallskip
\gll  May  manakem  na  mama  naan  ta  gwa.  \\
\textsc{ext.in}  older  \textsc{lk}  man  \textsc{spat.def}  \textsc{nabs}  outside \\
\footnotetext{We find no definitive evidence for analyzing this location phrase as being a constituent of the argument (‘a man outside’), or of the larger construct (‘outside is a man.’). However, “bare” indefinite existentials (‘there are men’) are extremely rare, since a location is normally added to such utterances. For this reason, the general template for the existential construction includes the possibility of a construction-level location element, though sometimes evidence for this occurring at the construction or RP level is unclear.}
\glt ‘There is an older man outside.’
\z
\ea
\label{bkm:Ref442938611}\textbf{Argument} \textbf{with} \textbf{nominalized} \textbf{(relative)} \textbf{clause}: \\
May   manakem  na  gatindeg  naan  ta  gangaan. \\\smallskip
\gll  May   manakem  na  ga-tindeg  naan  ta  gangaan. \\
\textsc{ext.in} older  \textsc{lk}  \textsc{i.r}-stand  \textsc{spat.def}  \textsc{nabs}  doorway \\
\glt ‘There’s an older (one) standing in the doorway.’ [YBWN-T-01 2.15]
\z

The next set of examples are of indefinite, given existential constructs: \is{indefinite given existential construction}\is{existential constructions!indefinite given}
\ea
\label{bkm:Ref442938671}
Argument with adjective: \\
May  anen  manakem  na  mama. (Or: May anen mama na manakem.) \\\smallskip
\gll  May  anen  manakem  na  mama. \\
\textsc{ext.in}  \textsc{ext.g}  older  \textsc{lk}  man \\
\glt ‘There’s some older man.’ (The conversation has evoked the notion of “men”, but the speaker assumes the hearer cannot identify 
which particular man is meant).
\z
\ea
Argument with number: \\
May  anen  tallo  na  mama. (*May anen mama na tallo.) \\\smallskip
\gll  May  anen  tallo  na  mama. \\
\textsc{ext.in  ext.g}  three  \textsc{lk}  man \\
\glt ‘There are some three men.’
\z
\ea
May  anen  tallo  buok  na  mama. (?May anen mama na tallo buok.) \\\smallskip
\gll May  anen  tallo  buok  na  mama. \\
\textsc{ext.in}  \textsc{ext.g}  three  \textsc{cl}    \textsc{lk}  man \\
\glt ‘There are some three individual men.’
\z
\ea
Argument with location phrase: \\
May  anen  manakem  naan  ta  gwa. (*May anen naan ta gwa manakem.) \\\smallskip
\gll  May  anen  manakem  naan  ta  gwa. \\
\textsc{ext.in}  \textsc{ext.g}  older  \textsc{spat.def}  \textsc{nabs}  outside \\
\glt ‘There’s some older man outside.’ \\
\z
\ea
Argument with relative clause: \\
May  anen  gatindeg  na  manakem  naan  ta    gangaan. \\\smallskip
\gll  May  anen  ga-tindeg  na  manakem  naan  ta    gangaan. \\
\textsc{ext.in}  \textsc{ext.g}  \textsc{i.r}-stand  \textsc{lk}  older    \textsc{spat.def}  \textsc{nabs}  doorway \\
\glt ‘There’s some older one who is standing in the doorway.’
\z

In the definite given existential construction, the argument may be an absolutive pronoun or Referring Phrase that includes the full range of possible determiners and modifiers \REF{ex:definitegivenexistentialwithadjective}-\REF{bkm:Ref329939931}. The following are examples of definite, given existential constructs:\is{definite given existential construction}\is{existential constructions!definite given}
 
\ea
\label{ex:definitegivenexistentialwithadjective}
Argument with adjective: \\
Anen  manakem  an  na  mama.  (?Anen mama an na manakem.) \\\smallskip
\gll  Anen  manakem  an  na  mama. \\
\textsc{ext.g}  older  \textsc{def.m}  \textsc{lk}  man \\
\glt ‘The older man is present.’ (Or: ‘There’s the older man.’) 
\z
\ea
\label{bkm:Ref442519677}
Argument with number: \\
Anen  tallo  an  na  mama. (*Anen mama an na tallo.) \\\smallskip
\gll  Anen  tallo  an  na  mama. \\
\textsc{ext.g}  three  \textsc{def.m  lk}  man \\
\glt ‘The three men are present.’ (Or: ‘There are the three men.’) \\\smallskip
Also possible: \\
Anen tallo buok na mama. \\
Anen mama an na tallo buok.
\z

\newpage
\ea
\label{ex:givendefiniteexistentialwithlocation}
Argument with location phrase: \\
Anen  manakem  an  naan  ta  gwa. (*Anen naan ta gwa manakem an.) \\\smallskip
\gll  Anen  manakem  an  naan  ta  gwa. \\
\textsc{ext.g}  older  \textsc{def.m} \textsc{spat.def}  \textsc{nabs}  outside \\
\glt ‘The older one is present outside.’ \\
\z
\ea
\label{bkm:Ref329939931}
\label{bkm:Ref442854660}
Argument with relative clause: \\
Anen  gatindeg  ya  na  manakem  naan  ta  gangaan. (Or: Anen manakem ya na gatindeg naan ta gangaan.) \\\smallskip
\gll  Anen  ga-tindeg  ya  na  manakem  naan  ta  gangaan. \\
\textsc{ext.g}  \textsc{i.r}-stand  \textsc{def.f}  \textsc{lk} older \textsc{spat.def}  \textsc{nabs}  outside \\
\glt ‘The older one is present who was standing in the doorway.’ \\
\z

Example \REF{bkm:Ref329939931} employs the distal deictic determiner\is{deictic determiners} \textit{ya} instead of the intermediate deictic determiner \textit{an} in the argument RP.  This can mean the older man is the one who previously was standing, or the one who previously was far away.
\is{existential constructions!basic|)}

\subsection{Head omission in the argument}
\label{sec:headomision}\is{existential constructions!head omission|(}

When the argument includes a relative clause\is{relative clauses}, the head of the clause can drop out, leaving just a clause or verb. This is a property of relative clauses in general (see \chapref{chap:clausecombining}, \sectref{sec:relativeclauses}), and not specifically of the existential construction. However, since such omission of the head occurs often in existential contexts, we are including a selection of such examples here.

Example \REF{ex:indefinitenewconstruct} illustrates an indefinite new existential construct, while \REF{ex:indefinitenewconstructwithheadomission} illustrates the same construct with omission of the head in the argument Referring Phrase:
 
\ea
\label{ex:indefinitenewconstruct}
May  manakem  na  gatindeg  naan  ta  gangaan. (Or: May  gatindeg  na  manakem  naan  ta  gangaan.) \\\smallskip
Pred\hspace{13pt}Argument\hspace{58pt}  Location \\
\gll May  manakem  na  ga-tindeg  naan  ta  gangaan. \\
\textsc{ext.in}  older \textsc{lk} \textsc{i.r}-stand  \textsc{spat.def}  \textsc{nabs}  doorway \\
\glt ‘There’s an older one standing in the doorway.’
\z
\ea
\label{ex:indefinitenewconstructwithheadomission}
May  gatindeg  naan  ta  gangaan. \\\smallskip
Pred\hspace{13pt}Argument  Location \\
\gll May  ga-tindeg  naan  ta  gangaan. \\
\textsc{ext.in}  \textsc{i.r}-stand  \textsc{spat.def}  \textsc{nabs}  doorway \\
\glt ‘There’s someone standing in the doorway.’
\z

Example \REF{ex:indefiniteexistentialclausewithafullrpargument} illustrates an indefinite existential clause\is{indefinite existential constructions} with a full RP argument:

\ea
\label{ex:indefiniteexistentialclausewithafullrpargument}
May  nakita  na  yupan  maguļang  ko  an. \\\smallskip
Pred\hspace{13pt}Argument  \\
\gll May  na-kita  na  yupan  maguļang  ko  an. \\
\textsc{ext.in}  \textsc{a.hap.r}-see  \textsc{lk}  bird  older.sibling  1\textsc{s.gen}  \textsc{def.m} \\
\glt `My older sibling saw a bird.’ (lit. `There was a bird that my older sibling saw')
\z

The head noun may drop out of the RP leaving just the verb:
 
\ea
May  nakita  maguļang  ko  an. \\\smallskip
Pred\hspace{13pt}Argument \\
\gll May  na-kita  maguļang  ko  an. \\
\textsc{ext.in}  \textsc{a.hap.r}-see  older.sibling  1\textsc{s.gen}  \textsc{def.m} \\
\glt ‘My older sibling saw something.’ (lit. ‘There was something my older sibling saw.’)
\z

The following are a few additional examples of head omission in existential clauses:
 
\ea
May  anen  nabilin. \\\smallskip
\gll  May  anen  na-bilin. \\
\textsc{ext.in}  \textsc{ext.g}  \textsc{a.hap.r}-leave \\
\glt ‘There is/was some left.’
\z
\ea
May  anen  daļa  maguļang  ko  an. \\\smallskip
\gll  May  anen  daļa  maguļang  ko  an. \\
\textsc{ext.in}  \textsc{ext.g}  take/carry  older.brother 1\textsc{s.gen}  \textsc{def.m} \\
\glt ‘My older sibling has something brought (with him).’
\z
\ea
Anen  gatindeg  ya  gina  naan  ta  gangaan. \\\smallskip
\gll  Anen  ga-tindeg  ya  gina  naan  ta  gangaan. \\
\textsc{ext.g}  \textsc{i.r}-stand  \textsc{def.f}  earlier  \textsc{spat.def}  \textsc{nabs}  doorway \\
\glt ‘Here’s the one standing earlier in the doorway.’
\z

\newpage
\ea
Anen  ki  kanen  pangita  no  ya. \\\smallskip
\gll  Anen  ki  kanen  pa-ng-kita  no  ya. \\
\textsc{ext.g}  \textsc{obl.p}  3s  \textsc{t.r}-\textsc{pl}-search\footnotemark{}  2\textsc{s.erg}  \textsc{def.f} \\
\footnotetext{The root \textit{kita}, is glossed `to see' when occuring in happenstantial mode. When occuring in dynamic mode, as in this example, it implies an active process rather than a non-volitional experience of perception. Combined with the pluractional stem-forming prefix, the image evoked is of someone actively ``seeing", that is, looking multiple times in multiple places. As such the English gloss ``search" is more accurate.}
\glt ‘Something that you are/were searching for is with him/her.’
\z

The following are examples of head omission in existential constructions from the corpus:

\ea
Yaan kanen ta iya na istaran. Gatingaļa kanen i tak may gatagbaļay. \\\smallskip
\gll Yaan  kanen  ta  iya  na  istar-an.  Ga-tingaļa  kanen i tak  may  ga-tag-baļay. \\
\textsc{spat.def}  3\textsc{s.abs}  \textsc{nabs}  3\textsc{s.gen}  \textsc{lk} live-\textsc{nr}  \textsc{i.r}-wonder  3\textsc{s.abs}  \textsc{def.n}
because  \textsc{ext.in}  \textsc{i.r}-\textsc{nr}-house \\
\glt ’He was at his house (his living place). He was wondering because (there was) someone calling the owner of the house.’ [MBON-T-03 3.2]
\z

\ea
May anen gapungko. May anen man gatindeg bilang gasakripisyo danen. \\\smallskip \gll May  anen  ga-pungko.  May    anen  man  ga-tindeg  bilang ga-sakripisyo  danen. \\
\textsc{ext.in}  \textsc{ext.g}   \textsc{i.r}-sit  \textsc{ext.in}  \textsc{ext.g}  also  \textsc{i.r}-stand    as \textsc{i.r}-sacrifice  3\textsc{p.abs} \\
\glt ‘There are some sitting. There are also some standing as they are sacrificing/suffering.’ [ABOE-L-01 6.7]
\z

\ea
Unso ki. Bistaan ta dya daw may anen dya. \\\smallskip \gll Unso  ki.  \emptyset-bista-an  ta    dya  daw  may  anen dya. \\
\textsc{d}4\textsc{loc.pr}  1\textsc{p.incl.abs}  \textsc{t.ir}-go.find.out-\textsc{apl}  1\textsc{p.incl.erg}  \textsc{d}4\textsc{loc}  if/when  \textsc{ext.in}  \textsc{ext.g}
\textsc{d}4\textsc{loc} \\
\glt ‘Let’s (go) right there (far away). Let’s go find out if there is some (food) there.’ [CBWN-C-16 3.11]
\z
 
\newpage
\ea
Daw may anen ki, mangatag ki ta duma ta. \\\smallskip \gll Daw  may  anen  ki  ma-ng-atag  ki  ta  duma ta. \\
if/when  \textsc{ext.in}  \textsc{ext.g}  1\textsc{p.incl.abs}  \textsc{a.hap.ir}-\textsc{pl}-give  1\textsc{p.incl.abs}  \textsc{nabs}  companion
1\textsc{p.incl.gen} \\
\glt ‘If we have (something) let us habitually be giving to others.’ (This is the lesson of a story about not being selfish or greedy. So in the context it is understood that the omitted part is ‘something’ or ‘anything.’) [PMWN-T-01 4.2] 
\z

\ea
Anduni anen nang en gatinir ta baļay danen bantay ta mangngod din tak bai dili dapat sigi nang na panaw. \\\smallskip \gll Anduni  anen  nang  en  ga-tinir  ta  baļay  danen  bantay  ta mangngod  din  tak  bai  dili  dapat  sigi  nang  na  panaw. \\
now/today  \textsc{ext.g}  only/just  \textsc{cm}  \textsc{i.r}-stay  \textsc{nabs}  house  3\textsc{p.gen}  watch/guard  \textsc{nabs}
younger.sibling  3\textsc{s.gen}  because  woman  \textsc{neg.ir}  must  continual  only/just  \textsc{lk}  go/walk \\
\glt ‘Now (the woman) is just there staying in their house watching her younger sibling(s) because women should just not keep gooing (places).’ [RZWE-J-01 9.5]
\z
\ea
Ambaļ kon ta amo ya, ``Anen ki di en. May nakita ka?" Sabat man ta bubuo ya, ``Uļa. Imo, may nakita ka?" Sabat man ta amo ya, “Uļa man.” Dayon kon ambaļ bubuo i na, ``Unso ki. Bistaan ta dya daw may anen dya." \\\smallskip \gll Ambaļ  kon  ta  amo  ya,   ``Anen  ki  di  en.   May  na-kita ka?"  Sabat  man  ta  bubuo  ya,  ``Uļa.  Imo,  may  na-kita  ka?" Sabat  man  ta  amo  ya,  “Uļa  man.”  Dayon  kon  ambaļ  bubuo  i    na, ``Unso  ki.  0-bista-an  ta  dya  daw  may  anen  dya." \\
say  \textsc{hsy}  \textsc{nabs}  monkey  \textsc{idef.f}  \textsc{ext.g}  1\textsc{p.incl.abs}  \textsc{d}1\textsc{loc}  \textsc{cm}  \textsc{ext.in}  \textsc{a.hap.r}-see
2\textsc{s.abs}  reply  also  \textsc{nabs}  tortoise  \textsc{def.f}  \textsc{neg.r}  2\textsc{s.gen}  \textsc{ext.in}  \textsc{a.hap.r}-see  2\textsc{s.abs}
reply  also  \textsc{nabs}  monkey  \textsc{def.f}  \textsc{neg.r}  also  right.away  \textsc{hsy}  say  tortoise  \textsc{def.n}  \textsc{lk}
\textsc{d}4\textsc{loc.pr}  1\textsc{p.incl.abs}  \textsc{t.ir}-go.find.out-\textsc{apl}  1\textsc{p.incl.erg}  \textsc{d}4\textsc{loc}  if/when  \textsc{ext.in}  \textsc{ext.g}
\textsc{d}4\textsc{loc} \\
\glt ‘The monkey said “We are here now. Do you see something (food)?”  The tortoise also replied, “No/Nothing. As for you (contrast), do you see something (food)?” The monkey also replied, “No/Nothing also.” Right away the tortoise said, “Let’s (go) right there (far away). Let’s go find out if there is some (food) there."’ [CBWN-C-16 3.5-11]
\is{existential constructions!head omission|)}
\z


\subsection{Definite non-specific individuals of a group}\is{existential constructions!definite|(}

When talking about definite but non-specific individuals of a group, as in ‘some of us’, ‘some of you’, or ‘some of them’, then \textit{may anen} is followed by a plural pronoun in the oblique case.
 
\ea
\textbf{May}  \textbf{anen}  \textbf{ki}  \textbf{kyo}  na  gasunod  pa  ta  mikaw. \\\smallskip \gll \textbf{May}  \textbf{anen}  \textbf{ki}  \textbf{kyo}  na  ga-sunod  pa  ta  mikaw. \\
\textsc{ext.in}  \textsc{ext.g}  \textsc{obl.p}  2p  \textsc{lk}  \textsc{i.r}-follow  \textsc{inc}  \textsc{nabs}  food.sacrifice \\
\glt ‘There are some of you who still follow food sacrifice.’
\z
\ea
Siguro \textbf{may} \textbf{anen} \textbf{ki} \textbf{kyo} dyan na gapati na a bayo nyo an iling tan unduni alin ta kaoy. \\\smallskip \gll Siguro  may  anen  ki  kyo  dyan  na  ga-pati  na  a  bayo  nyo  an iling  tan  unduni  alin  ta  kaoy. \\
perhaps  \textsc{ext.in}  \textsc{ext.g}  \textsc{obl.p}  2p  \textsc{d}2\textsc{loc}  \textsc{lk}  \textsc{i.r}-believe  \textsc{lk}  \textsc{inj}  clothes  2\textsc{p.gen}  \textsc{def.m}
like  \textsc{d}3\textsc{nabs}  now/today  from  \textsc{nabs}  tree/wood \\
\glt ‘Perhaps, there are some of you there (near addressees) who believe that your clothes like that now are from trees.’ [ROOB-T-01 7.7]
\is{existential constructions!definite|)}
\is{existential constructions|)}
\z

\section{Possessive clauses (predicative possession)}
\label{sec:predicativepossession} \is{possessive clauses|(}\is{predicative!possession|(}

\citet{stassen2009} described clauses that assert the notion that “X possesses Y” as “predicative posssion” (e.g., \textit{I have a book}), as distinct from \isi{adnominal possession}\is{possession!adnominal} (e.g., \textit{my book}).  In Kagayanen, there are three construction types that express predicative possession, all of which lack an Inflected Verb (see \chapref{chap:verbstructure}), and therefore belong to the family of non-verbal predicates. The first predicative possession construction is based on the \is{predicate nominals}predicate nominal template (\sectref{sec:predicativepossession-predicatenominal}), the second is based on the \isi{predicate locative} template (\sectref{sec:predicativepossession-predicatelocative}), and the third is based on the existential\is{existential constructions} template (\sectref{sec:existentialpossession}).

\subsection{Possessive clauses formed on a predicate nominal template}
\label{sec:predicativepossession-predicatenominal}\is{possessive clauses!predicate nominal|(}

Perhaps the most common means of expressing predicative possession in Kagayanen is the use of a  \isi{free genitive pronoun} (\ref{bkm:Ref429136651}-\ref{bkm:Ref246298287}) or \isi{genitive Referring Phrase} \REF{bkm:Ref429136604} as the predicate (see \chapref{chap:referringexpressions}, \sectref{sec:pronouns} and \sectref{sec:caseinreferringexpressions}). This construction normally expresses \isi{permanent possession}, and is structurally identical to predicate nominals:
 
\ea
\label{bkm:Ref429136651}
Ame  yan  na  baļay. \\\smallskip
Predicate\hspace{.5cm}Argument \\
\gll Ame  yan  na  baļay. \\
1\textsc{p.excl.gen}  \textsc{d2abs}  \textsc{lk}   house \\
\glt ‘That house is ours.’
\z
\ea
\label{bkm:Ref246298287}
Yan  na  baļay  ame. \\\smallskip
Argument\hspace{1cm}Predicate \\
\gll Yan  na  baļay  ame. \\
\textsc{d2abs}  \textsc{lk}  house  1\textsc{p.excl}.\textsc{gen} \\
\glt ‘\textsc{That house} is ours.’
\z
\ea
\label{bkm:Ref429136604}
Iya  (ta)  maguļang  ko  yon  na  baļay. \\\smallskip
Predicate\hspace{3.5cm}Argument \\
\gll Iya  (ta)  maguļang  ko  yon  na  baļay. \\
3\textsc{s.gen}  \textsc{nabs}  older.sibling  1\textsc{s.gen}  \textsc{d3abs}  \textsc{lk}  house \\
\glt ‘That house is my older sibling’s.’ (lit. ‘His/hers (of) my older sibling is that house.’)
\z
\ea
Yi  na  lugar  ni  a,  iya  ta  mga  patay. \\\smallskip
Argument\hspace{2.9cm}Predicate \\
\gll Yi  na  lugar  ni  a,  iya  ta  mga  patay. \\
\textsc{d1abs}  \textsc{lk}  place  \textsc{d1pr}  \textsc{inj}  3\textsc{s.gen}  \textsc{nabs}  \textsc{pl}  dead \\
\glt ‘\textsc{This very place} is the dead one’s (place).’ [PBON-T-01 2.15]
\z

This type of construction is not considered to be predicative possession by \citet{stassen2009}, because it is simply one usage of the predicate nominal template. In other words, there is no evidence that it is a dedicated possessive construction. However, it is very common in Kagayanen as a means of expressing permanent possession, and so warrents mention in this section of the grammar. 
\is{possessive clauses!predicate nominal|)}

\subsection{Possessive clauses formed on a predicate locative template}
\label{sec:predicativepossession-predicatelocative}\is{possessive clauses!locative|(}

The locational non-verbal construction serves as the basis for the second type of predicative possession. This is an instance of the “locational possessive”\is{locational possessive constructions} type \citep[47]{stassen2009}, which is the most common type exhibited in the languages in Stassen’s sample. In Kagayanen, the locational possessive construction normally expresses \isi{temporary possession}. The predicate of this construction consists of a locative phrase initiated by the spatial demonstrative \textit{naan}, with idiolectical variants \textit{yaan} and \textit{nyaan} ({\ref{bkm:Ref251934246}-\ref{ex:elementaryschool}}). \textit{Naan} is the most common variant of the \isi{spatial demonstrative}, followed by \textit{yaan}, and then \textit{nyaan}. There are no examples of \textit{nyaan} in locational possessive constructions in the corpus. All variants are equivalent in meaning and can be substituted for one another in the following examples. A thorough study is needed to determine any sociolinguistic or topolectical patterns to their usages.
\ea
\label{bkm:Ref251934246}
\textbf{Naan  ki  kanen}  kalaw  no  an. \\\smallskip
Predicate\hspace{2.2cm}Argument \\
\gll \textbf{Naan}  \textbf{ki}  \textbf{kanen}  kalaw  no  an. \\
\textsc{spat.def}  \textsc{obl.p}  3s  winnow.basket  2\textsc{s.gen}  \textsc{def.m} \\
\glt ‘S/he has your winnowing basket.’ (lit. ‘Your winnowing basket is with/on/at him/her.’)
\z
\ea
\label{bkm:Ref429141918}
Kaļaw  no  an  \textbf{naan  ki  kanen}. \\\smallskip
Argument\hspace{3cm}Predicate \\
\gll Kaļaw  no  an  \textbf{naan}  \textbf{ki}  \textbf{kanen}. \\
winnow.basket  2\textsc{s.gen}  \textsc{def.m}  \textsc{spat.def}  \textsc{obl.p}  3s \\
\glt ‘S/he has \textsc{your winnowing basket}.’
\z
\ea
Daw tenged ta libro na ambaļ \textbf{naan} \textbf{ki} \textbf{danen} \textbf{Maria} uļa pa danen gapadaļa… \\\smallskip \gll Daw  tenged  ta  libro  na  ambaļ  \textbf{naan}  \textbf{ki}  \textbf{danen}  \textbf{Maria}  uļa  pa  danen ga-pa-daļa… \\
and  about  \textsc{nabs}  book  \textsc{lk}  say  \textsc{spat.def}  \textsc{obl.p}  3p  Maria  \textsc{neg.r}  \textsc{inc}  3\textsc{p.abs}
\textsc{i.r}-\textsc{caus}-carry \\
\glt ‘And about the book that is said to be with Maria and companions, they have not yet sent (it)….’ [BCWL-C-02 5.1]
\z

\newpage
\ea
\label{ex:elementaryschool}
Tanan-tanan na mga tudlo \textbf{yaan ki danen}. \\\smallskip \gll tanan\sim{}-tanan	na	mga	tudlo		\textbf{yaan}		\textbf{ki}
\textbf{danen}. \\
\textsc{red}\sim{}all		\textsc{lk}	\textsc{pl}	teaching	\textsc{spat.def}	\textsc{obl.p}	3p \\
\glt `Absolutely all the teaching they have.' (This is from a speech at an elementary school graduation.) [SFOB-L-01 8.3]
\z

Occasionally, \textit{naan/yaan/nyaan} may be replaced by the \isi{given existential particle}\is{existential particles!given} \textit{anen} (\ref{ex:winnowingbasket}, \ref{ex:onethousandpesos}).
 
\ea
\label{ex:winnowingbasket}
\textbf{Anen}  \textbf{ki}  \textbf{kanen}  kaļaw  no  an. \\\smallskip
Predicate\hspace{1.7cm}Argument \\
\gll \textbf{Anen}  \textbf{ki}  \textbf{kanen}  kaļaw  no  an. \\
\textsc{ext.g}  \textsc{obl.p}  3s  winnowing.basket  2\textsc{s.gen}  \textsc{def.m} \\
\glt ‘S/he has your winnowing basket.’ (lit. ‘Your winnowing basket exists with/on/at him/her.’)
\z
\ea
\label{ex:onethousandpesos}
Daw gusto nyo, \textbf{anen} \textbf{man} \textbf{ki} \textbf{yaken} risibo i ta isya libo.  \\\smallskip \gll Daw  gusto  nyo,  anen  man  ki  yaken  risibo  i  ta  isya  libo.  \\
if/when  want  2\textsc{p.erg}  \textsc{ext.g}  also  \textsc{obl.p}  1s  receipt  \textsc{def.n}  \textsc{nabs}   one  thousand \\
\glt ‘If you want (to get it), there is with me a receipt (I have a receipt) for one thousand (pesos).’ [PBWL-T-06 7.3].
\z

 This variant of the locational possessive construction is a “bridge” to the \isi{existential possessive constructions} described in \sectref{sec:existentialpossession}. However the two constructions differ in their treatment of the possessor and argument. In all locational possessive constructions, including those with \textit{anen}, the possessor is treated grammatically as a Location, and the possessed item (the argument) is in the absolutive role. Whereas in the existential possessive construction described below, the possessor is presented in the absolutive role and the possessed item is a “stripped” Referring Phrase\is{stripped Referring Phrases} (see \sectref{sec:existentialpossession} below).
\is{possessive clauses!locative|)}

\subsection{Possessive uses of existential constructions}
\label{sec:existentialpossession}\is{possessive clauses!existential|(}

A third way of predicating possession in Kagayanen is to use an Existential construction (see \sectref{sec:existentialconstructions} above). In this usage, the possessor is expressed in the absolutive case and the possessed item is a “stripped” Referring Phrase\is{stripped Referring Phrases}, that is, a Referring Phrase lacking determiners, case marking and some modifiers as discussed in \sectref{sec:existentialconstructions}. All three subtypes of existential construction mentioned in \sectref{sec:existentialconstructions} may be employed  in such possessive constructions, though the \isi{definite given existential construction} overlaps with the locative template, as discussed in section \sectref{sec:predicativepossession-predicatelocative} above.

Examples \REF{ex:manhashouse} and \REF{ex:manhashouseonbeach} illustrate the \isi{indefinite new existential construction} expressing possession.
 
\ea
\label{ex:manhashouse}
May  baļay  mama  an. \\\smallskip \gll May  baļay  mama  an. \\
\textsc{ext.in}  house  man  \textsc{def.m} \\
\glt ‘The man has a house.’
\z
\ea
\label{ex:manhashouseonbeach}
May  baļay  mama  an  naan  ta  baybay. \\\smallskip \gll May  baļay  mama  an  naan  ta  baybay. \\
\textsc{ext.in}  house  man  \textsc{def.m}  \textsc{spat.def}  \textsc{nabs} beach \\
\glt ‘The man has a house on the beach.’
\z

Examples \REF{ex:manhassomehouse} and \REF{ex:manhassomehouseonbeach} illustrate the \isi{indefinite given existential construction}.

\ea
\label{ex:manhassomehouse}
May  anen  baļay  mama  an. \\\smallskip \gll May  anen  baļay  mama  an. \\
\textsc{ext.in}  \textsc{ext.g}  house  man  \textsc{def.m} \\
\glt ‘The man has some house.’
\z
\ea
\label{ex:manhassomehouseonbeach}
May  anen  baļay  mama  an  naan  ta  baybay. \\\smallskip
\gll May  anen  baļay  mama  an  naan  ta  baybay. \\
\textsc{ext.in}  \textsc{ext.g}  house  man  \textsc{def.m}  \textsc{spat.def}  \textsc{nabs} beach \\
\glt ‘The man has some house on the beach.’
\z

\hspace*{-0.6pt}Definite, given Existential predicates express \isi{temporary possession}. These constructs are related to the “locational possessive”\is{locational possessive constructions} type identified by \citet[49]{stassen2009}, and discussed in \sectref{sec:predicativepossession-predicatelocative} above:

\ea
Anen  pala  no  ya  ki  manong  ko. \\\smallskip \gll Anen  pala  no  ya  ki  manong  ko. \\
\textsc{ext.g}  shovel  2\textsc{s.gen}  \textsc{def.f}  \textsc{obl.p}  older.brother  1\textsc{s.gen} \\
\glt \textsc{‘}Your shovel is with my older brother.’
\z

In this construction the possessor appears as an oblique argument, rather than as an absolutive RP. This kind of possessive clause is identical to the given existential \textit{anen} with a location phrase in \REF{ex:givendefiniteexistentialwithlocation}. It also shares properties with the Locational construction with the spatial demonstrative \textit{naan/yaan/nyaan}, except that the “location” is animate, usually human. 

\subsubsection{Constituent orders in existential possessive clauses}\is{constituent orders in existential possessive clauses|(}\is{possessive clauses!constituent orders}

For possessive existential clauses, the possessor may be fronted before the predicate, but the argument (the possessed item) may not be fronted. This construction resembles the “topic possessive type”\is{topic possessive constructions} identified by \citet[58]{stassen2009}:
 
\ea
\textbf{Kami}  \textbf{i}  may  (anen)  baļay  naan  ta  Cawili. \\\smallskip \gll \textbf{Kami}  \textbf{i}  may  (anen)  baļay  naan  ta  Cawili. \\
1\textsc{p.excl.abs}  \textsc{def.n}  \textsc{ext.in}  \textsc{ext.g}   house  \textsc{spat.def}  \textsc{nabs}  Cawili \\
\glt ‘\textsc{as for us}, (we) have a/some house on Cawili (island).’
\z

The location may also be fronted:
 
\ea
    \ea
    \textbf{Naan}  \textbf{ta}  \textbf{Cawili}  may  (anen)  baļay  kami  i. \\\smallskip
\gll \textbf{Naan}  \textbf{ta}  \textbf{Cawili}  may  (anen)  baļay  kami  i. \\
    \textsc{spat.def}  \textsc{nabs}  Cawili  \textsc{ext.in}  \textsc{ext.g}   house  1\textsc{p.excl.abs}  \textsc{def.n} \\
    \glt ‘\textsc{on cawili (island)} we have a/some house.’
    \ex
    \textbf{Naan}  \textbf{ta}  \textbf{Cawili}  may  (anen)  kay  baļay. \\\smallskip
\gll \textbf{Naan}  \textbf{ta}  \textbf{Cawili}  may  (anen)  kay  baļay. \\
    \textsc{spat.def}  \textsc{nabs}  Cawili  \textsc{ext.in}  \textsc{ext.g}   1\textsc{p.excl.abs}   house \\
     \glt ‘\textsc{on cawili (island)} we have a/some house.’
    \z
\z

The ungrammatical examples in \REF{bkm:Ref143335735} show that the possessed item may not be fronted. This is consistent with the analysis of the possessed item as a stripped nominal\is{stripped Referring Phrases} that has been incorporated into the predicate:

\ea
    \label{bkm:Ref143335735}
    \ea
    *{ }\textbf{Baļay} naan ta Cawili may (anen) kay/kami i. \\
    \ex
    *{ }\textbf{Baļay} may (anen) naan ta Cawili kami i. \\
    \ex
    *{ }\textbf{Baļay} may (anen) kay naan ta Cawili.
    \z
\z

When the possessor RP of an indefinite new existential clause is a pronoun, it can be either the enclitic\is{enclitic pronouns} \REF{ex:wehavechickens-a} or the free\is{free pronouns} absolutive \REF{ex:wehavechickens-b}. 
 
\ea
    \ea
    \label{ex:wehavechickens-a}
    May  manok  \textbf{kay}. \\\smallskip
\gll May  manok  \textbf{kay}. \\
    \textsc{ext.in}  chicken  1\textsc{p.excl.abs} \\
    \glt \textsc{} ‘We have chicken(s).’
    \ex
    \label{ex:wehavechickens-b}
    May manok \textbf{kami} \textbf{i}. \\\smallskip
\gll May manok \textbf{kami} \textbf{i}. \\
    \textsc{ext.in}  chicken  1\textsc{p.excl.abs} \textsc{def.n} \\    
    \glt ‘WE have chicken(s).’ (Contrastive)
    \z
\z

If the RP in the predicate is longer than one word, then the enclitic pronoun intrudes \is{pronoun intrusion}inside the predicate \REF{ex:wehavefatchickens-a}.  The free pronoun does not intrude \REF{ex:wehavefatchickens-b}:
 
\ea
    \ea
    \label{ex:wehavefatchickens-a}
    May manok \textbf{kay} na tambek. (Or: May tambek \textbf{kay} na manok.) \\\smallskip
\gll May manok \textbf{kay} na tambek. \\
    \textsc{ext.in}  chicken  1\textsc{p.excl.abs} \textsc{lk} fat \\
    \glt ‘We have fat chicken(s).’ \\\smallskip
    *{ }May manok na tambek \textbf{kay}.
\ex 
    \label{ex:wehavefatchickens-b}
    May manok na tambek \textbf{kami} \textbf{i}. (Or: May tambek na manok \textbf{kami} \textbf{i}.) \\\smallskip
\gll May manok na tambek \textbf{kami} \textbf{i}. \\
    \textsc{ext.in}  chicken  \textsc{lk} fat 1\textsc{p.excl.abs} \textsc{def.n} \\    
    \glt ‘WE have fat chicken(s).’ \\\smallskip
    *{ }May manok \textbf{kami} na tambek.
    \z
\z

With indefinite given existential possessive clauses, the enclitic or free pronoun that refers to the possessor can occur after the two existential words \textit{may anen} \REF{ex:wehavesomefatchickens}. Speakers may make the possessor more prominent by placing the free pronoun, but not the enclitic, clause finally \REF{ex:wehavesomefatchickens}. 

\largerpage
\ea
\label{ex:wehavesomefatchickens}
May anen \textbf{kay} manok na tambek. (Or: May anen \textbf{kay} tambek na manok.) \\\smallskip \gll May anen \textbf{kay} manok na tambek. \\
\textsc{ext.in} \textsc{ext.g} chicken  \textsc{lk} fat \\
\glt ‘We have a/some fat chicken(s).’
\z
\ea
\label{ex:WEhavesomefatchickens}
May anen manok na tambek \textbf{kami} \textbf{i}. (Or: May anen tambek na manok \textbf{kami} \textbf{i}.) \\\smallskip \gll May anen manok na tambek \textbf{kami} \textbf{i}. \\
\textsc{ext.in} \textsc{ext.g} chicken  \textsc{lk} fat 1\textsc{p.excl.abs} \textsc{def.n} \\
\glt `\textsc{we} have a/some fat chicken(s).' \\\smallskip
*{ }May anen manok na tambek \textbf{kay}. \\
\z

With the definite given possessive existential clause, the possessor, whether pronoun or RP, can occur after the existential word \REF{ex:shovel-1}-\REF{ex:shovel-2}, but not clause initially.

\ea
\label{ex:shovel-1}
Anen  \textbf{ki}  \textbf{manong}  \textbf{ko}  pala  no  ya.  \\\smallskip \gll Anen  \textbf{ki}  \textbf{manong}  \textbf{ko}  pala  no  ya.  \\
\textsc{ext.g}  \textsc{obl.p}  older.brother  1\textsc{s.gen}  shovel  2\textsc{s.gen}  \textsc{def.f} \\
\glt ‘Your shovel is with my older brother\textsc{.’} \\\smallskip
*{ }Ki manong ko anen pala no an. \\
\z
\ea
\label{ex:shovel-2}
Anen  \textbf{ki}  \textbf{kami}  pala  no  ya.  \\\smallskip \gll Anen  \textbf{ki}  \textbf{kami}  pala  no  ya.  \\
\textsc{ext.g}  \textsc{obl.p}  1\textsc{p.exc}  shovel  2\textsc{s.gen}  \textsc{def.f} \\
\glt ‘Your shovel is with us.’ \\\smallskip
*{ }Ki kami anen pala no an.
\z

However, the argument may be fronted as in \REF{ex:shovel-3}.

\ea
\label{ex:shovel-3}
\textbf{Pala no ya}  anen  ki manong ko. \\\smallskip
\gll \textbf{Pala} \textbf{no} \textbf{ya}  anen  ki manong ko. \\
shovel 2\textsc{s.gen}  \textsc{def.f} \textsc{ext.g}  \textsc{obl.p}  older.brother  1\textsc{s.gen} \\
\glt `\textsc{your shovel} is with my older brother.'
\is{constituent orders in existential possessive clauses|)}
\z

\subsubsection{Semantic relations expressed by possessive existential clauses}
\is{semantic relations in possessive existential clauses|(}

An existential clause with an absolutive possessor can indicate possession (example \ref{ex:possession} and those presented previously), \isi{part-whole relationships} \REF{ex:part-whole-1}-\REF{ex:part-whole-2}, \isi{human relationships} \REF{ex:humanrelationships}, experiences \REF{ex:experiences-1}-\REF{ex:experiences-2}, characteristics or descriptions of temporary things or unexpected things \REF{ex:characteristics-1}-\REF{ex:characteristics-2}, actions \REF{ex:actions-1}-\REF{ex:actions-2}, contents \REF{ex:contents-1}-\REF{ex:contents-1}, and perhaps others. The following examples illustrate these usages:

\ea
\label{ex:possession}
Possession \\
Dey,  may  tirador  a  di. \\\smallskip \gll Dey,  may  tirador  a  di. \\
friend  \textsc{ext.in}  slingshot  1\textsc{s.abs}  \textsc{d1loc} \\
\glt ‘Friend, I have a slingshot here.’ [MEWN-T-02 3.1]
\z


\ea
\label{ex:part-whole-1}
Part-whole (usually temporary like leaves or fruit, unexpected, or \is{part-whole relationships}metaphorical) \\
May  bunga  ame  i  na  saging. \\\smallskip \gll May  bunga  ame  i  na  saging. \\
\textsc{ext.in}   fruit  1\textsc{p.excl.gen}  \textsc{def.n}  \textsc{lk}  banana \\
\glt ‘Our banana plant has fruit.’
\z
\ea
…may  benget  ubakan. \\\smallskip \gll …may  benget  ubakan. \\
\textsc{ext.in}  beard  goatfish \\
\glt ‘The goatfish has a beard.’ [JCWO-L-28 9.1]
\z
\ea
…may  utok  kay  man \\\smallskip \gll … may  utok  kay  man \\
{} \textsc{ext.in}   brain  1\textsc{p.excl.abs}  too \\
\glt ‘... we are smart too...’ (lit. we have a brain also. This is an idiomatic expression.) [CBWE-C-05 3.4]
\z
\ea
\label{ex:part-whole-2}
Sikad  baked  na  kaoy  may  sanga  na  sampuļo  daw  darwa. Kada  baked  na  sanga  may  appat  pa  gid  na  gamay  na  sanga. Kada  gamay  na  sanga  may  pitto  na  buļak. (Sabat:  taon,  buļan,  adlaw,  duminggo). \\\smallskip \gll Sikad  baked  na  kaoy  may  sanga  na  sampuļo  daw  darwa. Kada  baked  na  sanga  may  appat  pa  gid  na  gamay  na  sanga. Kada  gamay  na  sanga  may  pitto  na  buļak. (Sabat:  taon,  buļan,  adlaw,  duminggo). \\
very  big  \textsc{lk}  tree  \textsc{ext.in}  branch  \textsc{lk}  ten  and  two each  big  \textsc{lk}  branch  \textsc{ext.in}  four  \textsc{inc}  \textsc{int}  \textsc{lk}  small  \textsc{lk}  branch
each  small  \textsc{lk}  branch  \textsc{ext.in}  seven  \textsc{lk} flower
answer  year  month  day  week \\
\glt ‘A very big tree has ten and two branches. Each big branch has four small branches. Each small branch has seven flowers. Answer: year, month, week, day.’  (This is a riddle.) [SFWR-L-05 49.3]
\z
 
\ea
\label{ex:humanrelationships}
Human relationships\is{human relationships} \\
May  sawa  aren. \\\smallskip \gll May  sawa  aren. \\
\textsc{ext.in}  spouse  1\textsc{s.abs}  \\
\glt ‘I already have  a spouse.’ [ACWN-T-01 2.2]
\z
\ea
\label{ex:experiences-1}
Experience, usually negative such as sickness or other problem \\
May  swaļem  yaken  i. \\\smallskip \gll May  swaļem  yaken  i. \\
\textsc{ext.in}  chicken.pox  1\textsc{s.abs}  \textsc{def.n} \\
\glt‘I have chicken pox.’    
\z
\ea
Yaken  may  inagian  na  dili  ko  gid  malipatan. \\\smallskip \gll Yaken  may  <in>agi-an  na  dili  ko  gid  ma-lipat-an. \\
1\textsc{s.abs}  \textsc{ext.in} <\textsc{nr.res}>pass-\textsc{apl}  \textsc{lk}  \textsc{neg.ir}  1\textsc{s.erg}  \textsc{int}  \textsc{a.hap.ir}-forget-\textsc{apl} \\
\glt `As for me, I have an experience that I really can’t forget.’  [EMWN-T-05 2.1]
\z
\ea
Yaken  may  prublima  a  daw  oras  ta  kwarisma  a  waig. \\\smallskip \gll Yaken  may  prublima  a  daw  oras  ta  kwarisma  a  waig. \\
1\textsc{s.abs}  \textsc{ext.in}  problem  1\textsc{s.abs}  if/when  time/hour  \textsc{nabs} dry.season  \textsc{inj}  water. \\
\glt ‘As for me, I have a problem during dry season: water.’ [CNWE-L-01 2.4]
\z
\ea
… may  silot  ka  na  paabuton. \\\smallskip \gll … may  silot  ka  na  pa-abot-en. \\
{} \textsc{ext.in}  punish  2\textsc{s.abs}  \textsc{lk}  \textsc{caus}-arrive-\textsc{t.ir} \\
\glt ‘.. you will have a punishment coming.’ [MEWN-T-03 2.5]
\z
\ea
May  brown out  kay  gina.   \\\smallskip \gll May  brown out  kay  gina.   \\
\textsc{ext.in}  brown out  1\textsc{p.excl.abs}  earlier \\
\glt ‘We had a brown out earlier.’
\z
\ea
\label{ex:experiences-2}
Kiten  may  kabui  na  uļa  katapusan. \\\smallskip \gll Kiten  may  ka-bui  na  uļa  ka-tapos-an. \\
1\textsc{p.incl.abs}  \textsc{ext.in}  \textsc{nr}-live  \textsc{lk}  \textsc{neg.r}  \textsc{nr}-finish-\textsc{nr} \\
\glt ‘\textsc{we} have life without end.’ [EMWO-L-10 12.1]
\z
\ea
\label{ex:characteristics-1}
Characteristic or description \\
… Pedro  may  gayya. \\\smallskip \gll … Pedro  may  gayya. \\
  {}  Pedro  \textsc{ext.in}  shame \\
\glt ‘\textsc{pedro} has shame.’ [LGON-L-01]
\z
\ea
… may  paglaem  kay  pa \\\smallskip \gll … may  pag-laem  kay  pa \\
{} \textsc{ext.in}  {nr.act}-hope  1\textsc{p.excl.abs}  \textsc{inc} \\
\glt ‘… we have hope still.’ [CBWN-C-11 5.28]
\z
\ea
… may  lii  kanen  i  naan  apaw  ta    iya  na  mata. \\\smallskip \gll … may  lii  kanen  i  naan  apaw  ta    iya  na  mata. \\
{} \textsc{ext.in}  birthmark  3\textsc{s.abs}  \textsc{def.n}  \textsc{spat.def} above  \textsc{nabs}  3\textsc{s.gen}  \textsc{lk}  eye \\
\glt ‘He has a birthmark over his eye.' [DBWN-T-33 2.20]
\z
\ea
… may  kanļaman  kay  man… \\\smallskip \gll … may  kanļaman  kay  man… \\
{} \textsc{ext.in}  knowledge/wisdom  1\textsc{p.excl.abs}  too \\
\glt ‘…we have knowledge too…’ [PBWL-C-05  5.2]
\z
\ea
Iya  na  takong  may  buksoļ  daw  may  tellek  man. \\\smallskip \gll Iya  na  takong  may  buksoļ  daw  may  tellek  man. \\
3\textsc{s.gen}  \textsc{lk}  forehead  \textsc{ext.in}  bump  and  \textsc{ext.in}  sticker  too \\
\glt ‘Its forehead has a bump and stickers too.' (This is about the fish \textit{bantoļ} ‘stone fish.’) [AWE-T-10 9.3]
\z
\ea
\label{ex:characteristics-2}
… may  timpo  ka…  \\\smallskip \gll … may  timpo  ka…  \\
{} \textsc{ext.in}  time  2\textsc{s.abs} \\
\glt ‘…you have time.’ [PMWL-T-07 2.2] 
\z
\ea
\label{ex:actions-1}
Action/Activity \\
…tak  adlaw  na  Sabado  may  ubra  kanen  an. \\\smallskip \gll … tak  adlaw  na  Sabado  may  ubra  kanen  an. \\
 {}  because  day/sun  \textsc{lk}  Saturday  \textsc{ext.in}  work  3\textsc{s.abs}  \textsc{def,m} \\
\glt ‘…because Saturday he has work.’ [RMWN-L-01 2.2]
\z
\ea
\label{ex:actions-2}
… kada  adlaw-adlaw  may  exercise  kay. \\\smallskip
\gll … kada  adlaw\sim{}-adlaw  may  exercise  kay. \\
  {}  every  \textsc{red}\sim{}sun/day  \textsc{ext.in}  exercise  1\textsc{p.excl.abs} \\
\glt ‘…every day we have an exercise.’ [PBWL-T-09 7.11]
\z
\ea
\label{ex:contents-1}
Contents \\
May  sidda  kaldiro  an. \\\smallskip \gll May  sidda  kaldiro  an. \\
\textsc{ext.in}  fish  cooking.pot  \textsc{def.m} \\
\glt ‘The cooking pot has fish (inside).’
\z
\ea
\label{ex:contents-2}
… yo  na  lugar  may  ittaw  maat  o  malain  na  ispirito. \\\smallskip \gll … yo  na  lugar  may  ittaw  maat  o  ma-lain  na  ispirito. \\
 {}  \textsc{d4adj}  \textsc{lk}  place  \textsc{ext.in}  person  taboo  or  \textsc{adj}-evil  \textsc{lk}  spirit \\
\glt ‘… that place has taboo people or evil spirits.’ [VAOE-J-04 1.2]
\is{semantic relations in possessive existential clauses|)}
\is{possessive clauses!existential|)}
\is{predicative!possession|)} \is{possessive clauses|)}
\z

\section{Manner clauses}
\label{bkm:Ref444446090}\label{sec:mannerclauses} \is{manner clauses|(}

Manner clauses have the same Predicate+Argument structure common to other non-verbal predications, except that the argument is a clause nominalized with the action nominalizer \textit{pag-}, or a bare form (\ref{bkm:Ref246298440}-\ref{bkm:Ref429143130}). The predicate is an adverb that modifies the nominalized clause. For some adverbs, such as \textit{dasig} ‘quickly’ and \textit{tudo} `intense/all out effort', it is possible to invert the order of predicate and argument (\ref{bkm:Ref329886574}-\ref{bkm:Ref429143130}).

\ea
\label{bkm:Ref246298440}
Dasig  pagdļagan  din  an. \\\smallskip
Pred.\hspace{.3cm}Argument \\
\gll Dasig  (pag)-dļagan  din  an. \\
fast  \textsc{nr.act}-run  3\textsc{s.gen}  \textsc{def.m} \\
\glt ‘S/he is running fast.’ (lit. ‘His/Her running is fast.’) [EFWN-T-10 2.6]
\z
\ea
Dasig  iya  na  pagdļagan. \\\smallskip
Pred.\hspace{.3cm}Argument \\
\gll Dasig  iya  na  (pag)-dļagan. \\
fast  \textsc{3sgen}  \textsc{lk}  \textsc{nr.act-}run \\
\glt ‘S/he is running fast.’ (lit. ‘His/her running is fast.’)
\z
\ea
Dasig  (pag)panaw  din  ya. \\\smallskip
Pred\hspace{.3cm}Argument \\
\gll Dasig  (pag)-panaw  din  ya. \\
fast  \textsc{nr.act-}go/walk  \textsc{3sgen}  \textsc{def.f} \\
\glt ‘S/he is walking fast.' (lit. ‘His/Her walking is fast.’)
\z
\ea
Dasig  iya  na  (pag)panaw. \\\smallskip
Pred\hspace{.3cm}Argument \\
\gll Dasig  iya  na  (pag)-panaw. \\
fast  \textsc{3sgen}  \textsc{lk}  \textsc{nr.act-}go/walk \\
\glt  ‘S/he is walking fast.' (lit. ‘His/Her walking is fast.’) \\
\z

When the adverb \textit{sikad} ‘very’ is in the normal predicate position, it may occur by itself without an adjective, in which case it expresses intensity of action (\ref{bkm:Ref329934231}-\ref{bkm:Ref329934234}). However, when postposed it must occur with an adjective such as \textit{dasig} `fast' as in \REF{bkm:Ref329886574} or \textit{tudo} ‘intense/all out effort’ as in \REF{bkm:Ref429143130}.

\ea
\label{bkm:Ref329934231}
Sikad  iya  na  pag-agaļ. \\\smallskip \gll Sikad  (tudo)  iya  na  (pag)-{}-agaļ. \\
very  intense  \textsc{3}\textsc{s.gen}  \textsc{lk}  \textsc{nr.act}-cry \\
\glt ‘S/he is crying very intensely.’ (‘His/Her crying is very intense.’) 
\z
\ea
\label{bkm:Ref329934234}
Sikad  pag-agaļ  din  an. \\\smallskip \gll Sikad  (tudo)  (pag)-{}-agaļ  din  an. \\
very  intense  \textsc{nr.act}-cry  \textsc{3}\textsc{s.gen}  \textsc{def.m} \\
\glt ‘S/he is crying very intensely.’ (‘His/her crying is very intense.’).
\z
\ea
\label{bkm:Ref329886574}
Iya  na  pagdļagan  sikad  dasig. \\\smallskip \gll Iya  na  (pag)-dļagan  sikad  dasig. \\
\textsc{3sgen}  \textsc{lk}  \textsc{nr.act-}run  very  fast \\
\glt ‘Her/his \textsc{running} is fast.’ (Contrastive)
\z

\ea
\label{bkm:Ref429143130}
Pag-agaļ  din  an  sikad  tudo. \\\smallskip \gll (Pag)-{}-agaļ  din  (an)  sikad  tudo. \\
\textsc{nr.act}-cry  3\textsc{s.gen}  \textsc{def.m}  very  intense \\
\glt ‘S/he is \textsc{crying} very intensely.’ (‘Her/his crying is very intense.’) \\\smallskip
*{ }Iya na (pag)dļagan sikad. \\
*{ }Iya na (pag){}-agal sikad.
\z

\textit{Sikad} also functions as a canonical clause level adverb, in which case it expresses the idea of `habitually', and is followed by a clause with an Inflected Verb in irrealis modality, rather than a nominalization:  
\ea
Sikad  kanen  manaw. \\\smallskip \gll Sikad  kanen  m-panaw. \\
very  3\textsc{s.abs}  \textsc{i.v.ir}-go/walk \\
\glt ‘S/he habitually goes/walks.’ \\
\z

Some adverbs, including \textit{sigi} ‘continually’, \textit{diritso} `straight away', \textit{inay-inay} `slowly' and \textit{sali(ta)} `persistently', can only occur before the argument. Following these ``durational manner adverbs" (see \chapref{chap:modification}, \sectref{sec:non-canonicaladjunctadverbs}), the verb root in the argument must be in the bare form (\ref{bkm:Ref329934414}-\ref{ex:straightaway}):

\ea
\label{bkm:Ref329934414}
Sigi  panaw  din  an. \\\smallskip \gll Sigi  panaw  din  an. \\
continually  go/walk  \textsc{3}\textsc{s.gen}  \textsc{def.m} \\
\glt `S/he keeps going.’ (‘His/her going is continuous.’)
\z
\ea
Sigi  iya  na  panaw.  ( ... *manaw, ...*pagpanaw, etc.) \\\smallskip \gll Sigi  iya  na  panaw. \\
continually  \textsc{3}\textsc{s.gen}  \textsc{lk}  go/walk \\
\glt ‘S/he keeps going.’ (‘His/her going is continuous.’)
\z
\ea
\label{ex:straightaway}
Diritso  kanen  panaw. ( ... *manaw, ...*pagpanaw, etc.)  \\\smallskip \gll Diritso  kanen  panaw. \\
straightaway  \textsc{3}\textsc{s.abs}  go/walk \\
\glt ‘S/he straightaway goes.' (`Her/his going is straightaway.')
\is{manner clauses|)}
\z

\section{Comparative and superlative constructions} 
\label{sec:comparativesuperlativeclauses} \is{comparative constructions|(}\is{superlative constructions|(}

Comparative and superlative constructions are built on the non-verbal predicate template. They consist of a non-verbal predicate plus its argument followed by a standard usually expressed in an oblique case. A marker of comparison, \textit{kaysa} or \textit{kis-a}, ‘than’, may precede the standard, or the Marker may be a stem-forming morphological process that derives a comparative or superlative adjective. The basic comparative/superlative construction can be diagrammed as follows:

\ea
   Predicate Argument (Marker) Standard
\z

The predicate in this construction is a predicate modifier (as described in \sectref{sec:predicatemodifiers}), either a plain, comparative or superlative adjective. It may also be a nominalized clause.  In this section we describe comparative and superlative constructions based on predicate modifiers first. Further below we discuss comparative and superlative constructions based on nominalized clauses. Example \REF{bkm:Ref107820669} is a simple comparative construct based on the predicate modifier \textit{layog} ‘be tall’:

\ea
\label{bkm:Ref107820669} 
Layog    kanen  kaysa  ki  yaken. \\
Layog    kanen  kis-a  ki  yaken. \\
Layog    kanen  ki  yaken. \\\smallskip
Pred.\hspace{.2cm}Arg.\hspace{.4cm}Mkr.\hspace{1.3cm}  Standard \\
\gll Layog    kanen  (kaysa/kis-a)  ki  yaken. \\
tall    3\textsc{s.abs}  than  \textsc{obl.p}  1s \\
\glt ‘S/he is taller than me.’
\z

Consistent with the pattern for non-verbal constructions in general, the predicate modifier may appear in any of the orders described above in \sectref{sec:introduction-5}, including Argument+Predicate and \isi{pronoun intrusion}. The following are examples of various orders of elements in comparative constructions from the corpus:

\ea
Iya  na  labbot  bakod-bakod  pa  ta  iya  na  lawa… \\\smallskip
Argument\hspace{1.2cm}predicate\hspace{1.6cm}Standard \\
\gll Iya  na  labbot  bakod\sim{}-bakod  pa  ta  iya  na  lawa… \\
3\textsc{s.gen}  \textsc{lk}  bottom  \textsc{red}\sim{}big  \textsc{emph}  \textsc{nabs}  3\textsc{s.gen}  \textsc{lk}  body \\
\glt ‘Its bottom is somewhat bigger than its body….’ (This is a description of a coconut crab. Lit: ‘Its bottom is a little bit big to its body.’) [DBWE-T-27 2.3]
\z
\ea
Ittaw  na  gaistar  ta  baļay  ko  i  mas  pa  bakod  ki  yaken. \\\smallskip
Argument\hspace{5.3cm}predicate \\
\gll Ittaw  na  ga-istar  ta  baļay  ko  i  mas  pa  bakod {\quad} \\
person  \textsc{lk}  \textsc{i.r}-live  \textsc{nabs}  house  1\textsc{s.gen}  \textsc{def.n}  more  \textsc{emph}  big \\
Standard \\
\gll ki  yaken. \\
\textsc{obl.p}  1s \\
\glt ‘The person who lives in my house is bigger than me.’ [CBWN-C-107.14]
\z

In \REF{bkm:Ref107822397} the oblique case Standard intrudes between the predicate and the argument:

\ea
\label{bkm:Ref107822397}
Datas  pa  ki  kanen  ittaw  ya na bakod. \\\smallskip
Predicate\hspace{.5cm}Standard\hspace{.55cm}Argument \\
\gll Datas  pa  ki  kanen  ittaw  ya na bakod. \\
high  \textsc{emph}  \textsc{obl.p}  3s  person  \textsc{def.f} \textsc{lk} big \\
\glt ‘The person who was big was taller than him.’ (lit. ‘The big person was tall to him’).
\z

Consistent with the pattern for modifiers in general, when the modifier is completely or partially reduplicated\is{reduplication}, the extent of the quality expressed is attenuated, for example, ‘somewhat/a little bit tall’. 

\ea
Layog-layog  kanen  ki  yaken. \\\smallskip \gll Layog\sim{}-layog  kanen  (kaysa/kis-a)  ki  yaken. \\
\textsc{red}\sim{}tall  3\textsc{s.abs}  than  \textsc{obl.p}  1s \\
\glt ‘S/he is somewhat taller than me.’ \\
\z

Again, consistent with the pattern for modifiers in general, downtoning\is{downtoning} or intensifying adverbial elements may also modify the extent of the comparison (\ref{bkm:Ref107407647}-\ref{bkm:Ref107821212}):

\ea
\label{bkm:Ref107407647}
\textbf{Layog}  kanen  \textbf{sise}  ki  yaken. \\\smallskip \gll \textbf{Layog}  kanen  \textbf{sise}  (kaysa/kis-a)  ki  yaken. \\
tall  3\textsc{s.abs}  small  than  \textsc{obl.p}  1s \\
\glt ‘S/he is a little taller than me.’
\z
\ea
Layog-layog kanen \textbf{sise} ki yaken. \\\smallskip \gll Layog\sim{}-layog kanen \textbf{sise} (kaysa/kis-a) ki yaken. \\
\textsc{red}\sim{}tall  3\textsc{s.abs}  small  than  \textsc{obl.p}  1s \\
\glt ‘S/he is a very little bit taller than me.’
\z
\ea
\label{bkm:Ref107821212}
\textbf{Sikad}  \textbf{pa}  \textbf{gid}  layog  kanen  ki  yaken \\\smallskip \gll \textbf{Sikad}  \textbf{(pa)}  \textbf{(gid)}  layog  kanen  (kaysa/kis-a)  ki  yaken \\
very  \textsc{emph}  \textsc{int}  tall  3\textsc{s.abs}  than  \textsc{obl.p}  1s \\
\glt ‘S/he is (really) much taller than me.'
\z
\ea
Pandan  i  a  lipo    kaysa  ta  buli. \\\smallskip
 Argument\hspace{1.3cm}Pred.  Mkr.\hspace{.3cm}Standard \\
\gll Pandan  i  a  lipo    kaysa  ta  buli. \\
pandan  \textsc{def.n}  \textsc{ctr}  short    than  \textsc{nabs}  buri \\
\glt ‘Pandan is shorter than buri.'\footnote{\textit{Pandan} and \textit{buri} are types of palms.} [EFWE-T-04 19.1]
\z

The second comparative construction includes the word \textit{mas} ‘more’ (from Spanish) before the predicate. The optional intensifier \textit{gid} and emphasis particle \textit{pa} can also occur after \textit{mas}. The comparee is absolutive and the standard is normally in the oblique case. The markers of comparison\is{marker of comparison} \textit{kaysa} or \textit{kis-a} ‘than’ may occur before the oblique.

\ea
Mas  pa  gid  layog  kanen    ki  yaken. \\\smallskip
Predicate\hspace{2.2cm}Arg.\hspace{.3cm}Mkr.\hspace{1.2cm}  Standard \\
\gll Mas  (pa)  (gid)  layog  kanen    (kaysa/kis-a)  ki  yaken. \\
more  \textsc{emph}  \textsc{int}  tall  3\textsc{s.abs}    than  \textsc{obl.p}  1s \\
\glt ‘S/he is (really) (much) taller than me’
\z

Because of semantic incompatibility, the intensifying adverbials \textit{mas} and \textit{sikad} cannot occur with a reduplicated adjective.

\ea
* Mas laglayog kanen (kaysa/kis-a) ki yaken. \\
* Sikad laglayog kanen (kaysa/kis-a) ki yaken.
\z

The word \textit{mas} ‘more’ can also be used with an adjective to function as a complement taking predicate.

\ea 
Piro  \textbf{mas}  \textbf{dayad}  daw  legeman  buli  para  dayad  lagen. \\\smallskip
\gll Piro  \textbf{mas}  \textbf{dayad}  daw  \emptyset{}-legem-an  buli  para  dayad  luag-en. \\
but  more  good  if/when  \textsc{t.ir}-dye-\textsc{apl}  buri  \textsc{purp}  good  look-\textsc{t.ir} \\
\glt ‘But (it is) better if the buri leaves are dyed in order that it looks good.’ [BCWE-T-09 2.11]
\z

The following illustrate additional comparative constructions from the corpus. In these complex examples, the comparative constructions are bolded:

\ea
Paibitan  nay  ta  timbang  mama  na  duma  nay daw  muoy  piro  mama  an  galeddang  daw  nadaļa  din  man kami  tak  \textbf{mama}  \textbf{i}  \textbf{bakod}  \textbf{kis-a}  \textbf{ki}  \textbf{kami}. \\\smallskip \gll Pa-ibit-an  nay  ta  timbang  mama  na  duma  nay daw  m-luoy  piro  mama  an  ga-leddang  daw  na-daļa  din  man kami  tak  \textbf{mama}  \textbf{i}  \textbf{bakod}  \textbf{kis-a}  \textbf{ki}  \textbf{kami}. \\
\textsc{t.r}-hold.on-\textsc{apl}  1\textsc{p.excl.erg}  \textsc{nabs}  balance  man  \textsc{lk}  companion  1\textsc{p.excl.gen}
and  \textsc{i.v.ir}-swim  but  man  \textsc{def.m}  \textsc{i.r}-sink  and  \textsc{a.hap.r}-take/carry  3\textsc{s.erg}  too
1\textsc{p.excl.abs}  because  man  \textsc{def.n}  big  than  \textsc{obl.p}  1p \\
\glt ‘We held on to both sides of the man that was our companion and swam, but the man was sinking and took us also (down with him) because \textbf{the} \textbf{man} \textbf{was} \textbf{bigger} \textbf{than} \textbf{us}.’ [CBWN-C-11 4.8]
\z

\ea
Pawikan  i  a  \textbf{mas  bakod  kis-a  ta  bubuo}. \\\smallskip \gll Pawikan  i  a  \textbf{mas}  \textbf{bakod}  \textbf{kis-a}  \textbf{ta}  \textbf{bubuo}. \\
sea.turtle  \textsc{def.n}  \textsc{ctr}  more  big  than  \textsc{nabs}  tortoise \\
\glt ‘As for the sea turtle, (it is) \textbf{bigger than the tortoise}.’ [YBWE-T-05 2.1]
\z

\newpage
\ea
Dayad  gid  man  daw  kabistida  ki  tak  suwaļ  i  a pang-adlaw-adlaw  ta  nang  nasuot  suwaļ  an  a. \textbf{Mas}  \textbf{dayad}  \textbf{gid}  \textbf{bistida}  tak  yon  baļay  ta  Ginuo. \\\smallskip \gll Dayad  gid  man  daw  ka-bistida  ki  tak  suwaļ  i  a pang-{}-adlaw\sim{}-adlaw  ta  nang  na-suot  suwaļ  an  a. \textbf{Mas}  \textbf{dayad}  \textbf{gid}  \textbf{bistida}  tak  yon  baļay  ta  Ginuo. \\
good  \textsc{int}  too  if/when  \textsc{i.hap}-dress  1\textsc{p.incl.abs}  because  pants  \textsc{def.n}  \textsc{ctr}
\textsc{inst}-\textsc{red}-sun/day  1\textsc{p.incl.erg}  only/just  \textsc{a.hap.r}-wear  pants  \textsc{def.n}  \textsc{ctr}
more  good  \textsc{int}  dress  because  \textsc{d}3\textsc{abs}  house  \textsc{nabs}  Lord \\
\glt ‘(It is) good if we wear a dress because as for the pants, daily we just wear pants. \textbf{Dresses} \textbf{are} \textbf{really} \textbf{better} because that is the Lord’s house.’ [ETOE-C-04 3.8]
\z
\ea
Na  kaluoy  ta  Dios  na  grasya  ta  Dios  ki  kami  na  \textbf{mas} \textbf{subļa  pa}  \textbf{iran}  \textbf{na}  \textbf{mga}  \textbf{pangabuian}  \textbf{ki}  \textbf{kami}. \\\smallskip \gll Na  ka-luoy  ta  Dios  na  grasya  ta  Dios  ki  kami  na  \textbf{mas} \textbf{subļa}  \textbf{pa}  \textbf{iran}  \textbf{na}  \textbf{mga}  \textbf{pangabui-an}  \textbf{ki}  \textbf{kami}. \\
\textsc{lk}  \textsc{nr}-mercy  \textsc{nabs}  God  \textsc{lk}  grace  \textsc{nabs}  God  \textsc{obl.p}  1\textsc{p.exc}  \textsc{lk}  more too.much  \textsc{emph}  3\textsc{p.gen}  \textsc{lk}  \textsc{pl}  livelihood-\textsc{nr}  \textsc{obl.p}  1\textsc{p.exc} \\
\glt ‘When (it is) the mercy of God that God’s grace is with us, \textbf{then} \textbf{much} \textbf{more} \textbf{overwhelming} \textbf{will} \textbf{be} \textbf{their} \textbf{(the} \textbf{children’s)} \textbf{livelihood} \textbf{than} \textbf{ours}.’ [LTOE-C-01 2.3]
\z

When comparing the \isi{modification} of two events, the word \textit{mas} occurs at the beginning before the modifier of the clause that is being compared. Also \textit{kis-a/kaysa} ‘than’ occurs before the standard, which is expressed as an oblique RP. There are two constructions for this that are similar to the constructions for modifying a clause with an adverb.

The first construction must have an irrealis verb in the clause that is being compared \REF{ex:fasterthanme}. The ungrammatical example below shows that a verb inflected in realis mode (\textit{gadļagan}) may not occur in this construction.

\ea
\label{ex:fasterthanme}
Mas  dasig  kanen  \textbf{mļagan}  kaysa  ki  yaken. \\\smallskip
Pred.\hspace{1cm}Arg.\hspace{.4cm}Pred.\hspace{.9cm}Mkr.\hspace{.3cm}Standard \\
\gll Mas  dasig  kanen  \textbf{m-dļagan}  kaysa  ki  yaken. \\
more  fast  3\textsc{s.abs}  \textsc{i.v.ir}-run  than  \textsc{obl.p}  1s \\
\glt ‘S/he \textbf{runs} faster than me.’ \\\smallskip
* Mas dasig kanen gadļagan kis-a/kaysa ki yaken.
\z

The second comparative construction comparing events involves a nominalized clause. This construction is completely parallel to comparative and superlative constructions involving non-verbal predicates.

\ea
Mas  dasig  pagdļagan  din  an  kaysa  ki  yaken. \\\smallskip
Predicate\hspace{.4cm}Arg.\hspace{3.3cm}Mkr.\hspace{.2cm}Standard \\
\gll Mas  dasig  pag-dļagan  din  an  kaysa  ki  yaken. \\
more  fast  \textsc{nr.act}-run  3\textsc{s.erg}  \textsc{def.m}  than  \textsc{obl.p}  1s \\
\glt ‘S/he runs faster than me.’ (lit. ‘Her running is faster than mine.’)
\z

The following is an example of a comparative construction involving events from the text corpus.

\ea
Piro,  gineļet  nang  gid  na  uļo  ta  sidda  papili  nang  daen na  uļas  tak  sikad  gid  en  iran  na  kakulba  daw \textbf{mas}  \textbf{dali}  \textbf{tuugon}  \textbf{uļo  an}  a  kaysa  ta  ikog. \\\smallskip \gll Piro,  g<in>eļet  nang  gid  na  uļo  ta  sidda  pa-pili  nang  daen na  uļas  tak  sikad  gid  en  iran  na  ka-kulba  daw \textbf{mas}  \textbf{dali}  \textbf{tuog-en}  \textbf{uļo}  \textbf{an}  a  kaysa  ta  ikog. \\
but  <\textsc{nr.res}>cut  only/just  \textsc{int}  \textsc{lk}  head  \textsc{nabs}  fish  \textsc{t.r}-choose  only/just  3\textsc{p.erg}
\textsc{lk}  share  because  very  \textsc{int}  \textsc{cm}  3\textsc{p.gen}  \textsc{lk}  \textsc{nr}-fear  and
more  easy  put.through-\textsc{t.ir}  head  \textsc{def.m}  \textsc{ctr}  than  \textsc{nabs}  tail \\
\glt ‘But the slices of the head of the fish they chose as (their) share because their fear was really great and \textbf{(it is)} \textbf{easier} \textbf{to} \textbf{put} \textbf{(a string)} \textbf{through} \textbf{the} \textbf{head} than the tail.’ [JPWN-L-01 5.4]
\z

The superlative is formed in two ways. The first way is with the complex prefix \textit{pinaka}{}- on an adjective root. The thing being compared is in the absolutive case and usually fronted in the clause. The standard of comparison is again in the oblique case. In this construction, no syntactic marker of comparison (such as \textit{kis-a/kaysa}) occurs. Rather, the prefix constitutes the marker of the superlative construction.

\ea
Kanen  gid  pinakalayog  ki  kami. \\\smallskip
Argument  Mkr-Predicate  Standard \\
\gll Kanen  gid  pinaka-layog  ki  kami. \\
3\textsc{s.abs}  \textsc{int}  \textsc{superl}-tall  \textsc{obl.p}  1\textsc{p.excl} \\
\glt ‘\textsc{s/he} is really the tallest of us.’
\z

The second superlative construction includes an adjective with the nominalizing affixes \textit{ka-…-an} (See \chapref{chap:modification} \sectref{sec:ka-an}). With adjective roots this can have a superlative meaning.

\ea
Kanen  gid  kalayugan  ta  mag-utod. \\\smallskip
Argument  Mkr-Predicate-Mkr  Standard \\
\gll Kanen  gid  ka-luyog-an  ta  mag-{}-utod. \\
3\textsc{s.abs}  \textsc{int}  \textsc{nr}-tall-\textsc{nr}  \textsc{nabs}  \textsc{rel}-sibling \\
\glt ‘\textsc{s/he} is really the tallest of (his/her) siblings.’
\z

\ea
Dondonay  \textbf{kabakeran}  na  \textbf{puļo}. \\\smallskip \gll Dondonay  \textbf{ka-baked-an}  na  \textbf{puļo}. \\
Dondonay  \textsc{nr}-big-\textsc{nr}  \textsc{lk}  island \\
\glt ‘\textsc{Dondonay} is the biggest island.’ [LMWO-L-01 31.1]
\z

\ea
Pitto  na  mag-utod,  \textbf{kamangnguran}  ya  lain man  iya  na  apilido.  Duminggo  Gloria. \\\smallskip \gll Pitto  na  mag-{}-utod,  \textbf{ka-mangngud-an}  ya  lain man  iya  na  apilido.  Duminggo  Gloria. \\
seven  \textsc{lk}  \textsc{rel}-sibling  \textsc{nr}-younger.sibling-\textsc{nr}  \textsc{der.f}  different
too  3\textsc{s.gen}  \textsc{lk}  last.name  week/Sunday  Gloria \\
\glt ‘Seven siblings, the youngest has a different last name. Sunday Glory (or Easter day).’ (This is a riddle with the answer.) [MRWR-T-01 18.1] 
\is{superlative constructions|)} \is{comparative constructions|)}
\z

