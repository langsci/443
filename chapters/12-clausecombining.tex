\chapter{Clause combining}
\label{chap:clausecombining}
\section{Introduction}
\label{introduction-12}

To this point we have dealt almost exclusively with the grammar of individual words, phrases, and clauses. The typical communicative function of a clause is to express a rather simple \isi{discourse scene} that involves a referent or referents and some property, activity, or situation involving those referents. However, many ideas that speakers wish to communicate are more complex than what may be expressed in a simple scene. Therefore, in addition to individual clauses, every language includes structured habit patterns that are useful for combining simple scenes into more elaborate conceptual representations (i.e. ideas). Such structured combinations of conceptual scenes is often referred to as \textit{clause combining}.\is{clause combining}

In this chapter we will discuss several construction types in Kagayanen that involve clause combining. Most of these construction types involve one \textit{independent}\is{independent clauses} \is{clauses!independent}clause and one or more \textit{dependent}\is{dependent clauses}\is{clauses!dependent} clauses. We define an independent clause as one that is fully inflected\is{inflection} and capable of being integrated into discourse on its own. A dependent clause is one that depends on some other clause for at least part of its referential, temporal, or modal “grounding”. Such grounding information is also known as \textit{inflectional information} (see \chapref{chap:verbstructure}, \sectref{sec:verbinflection} for further discussion)\is{inflectional information} For example, in the following English clause combination, clause \REF{bkm:Ref460480260} is dependent on clause \REF{bkm:Ref460395811} because the subject, tense, and modality of clause \REF{bkm:Ref460480260} are only understood via the subject, tense, and modality of clause \REF{bkm:Ref460395811}:

\ea
    \ea
    \label{bkm:Ref460395811}
    She came in, \\
    \ex
    \label{bkm:Ref460480260}
    locking the door behind her.
    \z
\z


Clause \REF{bkm:Ref460395811} is grounded in time and modality by the past tense declarative form of the verb, and in participants by the subject pronoun \textit{she}. Therefore, it may be used in discourse on its own and can be considered an \textit{independent} clause. On the other hand,  the verb \textit{locking} has no subject or tense information, so clause \REF{bkm:Ref460480260} may not naturally be used in discourse on its own; it \textit{depends} on clause \REF{bkm:Ref460395811} for this important information. Therefore clause \REF{bkm:Ref460480260} is a \textit{dependent} \is{dependent clauses}clause.

In Kagayanen, \isi{dependent clauses} in clause combining constructions are of three structural types:

\begin{enumerate}
\item Nominalizations (\sectref{bkm:Ref474473730})\is{dependent clauses!nominalizations}
\item  Subjunctive clauses (\sectref{bkm:Ref477523359})\is{dependent clauses!subjunctive3}
\item Fully inflected clauses (\sectref{bkm:Ref474473773})\is{dependent clauses!finite}
\end{enumerate}

Each of these structural types may be used to fulfill a number of syntactic, semantic, and pragmatic functions in discourse. The functions that dependent clauses tend to fulfill are discussed within each of the following three subsections. This chapter concludes with a section describing \isi{dependent clauses} used as modifiers within Referring Phrases (i.e, \isi{relative clauses}, \sectref{bkm:Ref460483291}) and a section describing clause coordination (\sectref{sec:coordinateclauses})\is{coordinate clauses}.

\section{Nominalizations as dependent clauses}
\label{bkm:Ref474473730} \label{sec:nominalizationsasdependentclauses}

\is{nominalization|(}\is{dependent clauses!nominalization|(}A nominalization is a construction that describes an action, situation, or state, but has some structural characteristics of a noun. For example, \REF{bkm:Ref460480260} \textit{locking the door behind her}, can be called a nominalized clause because it is not inflected for some important verbal categories such as tense, modality, and subject reference. It also has some properties of nouns-it can take a genitive (possessor) argument: \textit{her locking the door}, and it can function as the subject of a sentence: \textit{Locking the door is a good idea}.

In Kagayanen, there are two types of nominalized clauses that may function as dependent clauses in clause combining constructions. These we will refer to as \textit{pag}{}- clauses\is{pag- clauses@\textit{pag}- clauses} and bare form\is{bare form clauses} clauses. There are three main reasons why we call these nominalizations.

\begin{enumerate}
    \item  They do not express grammatical transitivity (as do fully inflected verbs, see \chapref{chap:verbstructure}, \sectref{sec:grammaticaltransitivity}). Any overt argument is expressed in the genitive case (as though it is the possessor of a noun), regardless of the semantic transitivity of the scene.
    \item They do not express modality (as do fully inflected verbs, see \chapref{chap:verbstructure}, \sectref{sec:modality}). The semantic modality of the nominalized clause is only understood in relation to the modality of the related independent clause.
    \item They can function as heads of referential phrases (or “Noun Phrases”) within other construction types.
\end{enumerate}

As described in \chapref{chap:referringexpressions}, \sectref{sec:pag}, the prefix \textit{pag-} can be used productively to form an \textit{action nominalization}\is{action nominalization}, that is, a noun that refers to an action associated with the normally verbal root \citep[224]{payne1997}. In some traditions these are called “verbal nouns” or “gerunds.” It is often difficult to distinguish a purely nominal usage of these forms from their usage in dependent clauses. For example, if someone says \textit{I like listening to Bach}, the speaker is not referring to any particular instance of listening to Bach, but rather just a general activity. In this case, \textit{listening} would be an action nominalization-a “gerund” in traditional English grammar. On the other hand, if someone says \textit{I finished listening to Bach}, then the person is describing the ending phase of a particular event. In this case we would say that \textit{(I was) listening to Bach} is a clause that is dependent on the main verb \textit{finish}-the verb \textit{listening} in this case would be a “present participle” according to traditional English grammar. Note that the construction \textit{listening to Bach} is the same in both instances; the distinction between them is purely functional. Gerunds and present participles are not different things; they are just different functions for one verbal form in English.

Something similar occurs in Kagayanen with respect to verb roots that carry the nominalizing prefix \textit{pag}{}-; roots prefixed with \textit{pag}- serve a number of syntactic and discourse functions. Consistent with the communication-first perspective taken throughout this grammar, we discuss nominalizations with \textit{pag}- according to their various functions.

Some nominalizations formed with \textit{pag}- are clearly lexicalized as nouns because their meanings are not consistent with the productive action nominalization pattern evoked by \textit{pag}-. Examples of these include \textit{pag-ampangen} ‘game’ (from \textit{pag}+‘play’+\textsc{t.ir}), \textit{paggwa} ‘show’ (from \textit{pag}+‘out’), or \textit{pagdļeen} ‘government’/‘administration’ from (\textit{pag}+‘carry’+\textsc{t.ir}). These do not exactly mean the expected ‘act of playing', ‘act of going out', or ‘act of carrying' respectively. The form \textit{pagkaan} (\textit{pag}+\textit{kaan} ‘eat’) is a good example of a form that is both a productive action nominalization referring to the act of eating and as a lexicalized noun meaning ‘food’. In this section, we will be liberal in our identification of dependent clauses headed by action nominalizations, eliminating those that are clearly lexicalized as nouns.  All others,  will be considered dependent clauses.

We will informally describe nominalizations with \textit{pag}{}- in terms of their structure as “\textit{pag}{}- verbs” and clause-like structures headed by \textit{pag}{}- verbs as “\textit{pag}{}- clauses”. Similarly, we will describe nominalized clauses based on bare-form verbs (i.e. verbs with no inflectional affixes) as “bare-form clauses”. For purposes of exposition, we identify three syntactic functions for nominalizations in clause combining: \textit{direct complement clauses}\is{direct complement clauses}, \textit{oblique complement clauses}\is{oblique complement clauses}, and \textit{adverbial clauses}\is{adverbial clauses}, as described in the following subsections.

\textit{Pag}{}-clauses appear to be more common in planned discourse than bare-form clauses. This observation leads to the hypothesis that bare-form clauses may simply be conversational reductions of \textit{pag}{}- clauses (dropping of \textit{pag}{}-). While this may be the case in some instances, there are many situations in which only a \textit{pag-} clause is acceptable. For this reason, we are treating the two types of nominalized clauses separately and provide examples of both wherever they can be found in the corpus. Interestingly, \textit{pag}{}- clauses never function as modifiers (relative clauses) within RPs, though bare-form clauses may (see \sectref{bkm:Ref115159779}).

\subsection{Nominalizations as direct complement clauses}
\label{sec:nominalizationsasdirectcomplementclauses}
\is{direct complement clauses|(}
\is{complement clauses!direct|(}
The term “complement” evokes “completion”. It is based on the insight that sometimes the head of a construction alone does not constitute a complete expression of the idea that the speaker intends. Something else is needed to complete the idea. “Complement” in this sense contrasts with “modifier” in that modifiers add additional, possibly important, information about the head, but are not necessary for the structure to be fully integrated into discourse and understood by the intended audience (see, e.g., \citealt[219ff]{huddleston2002} for a discussion of this usage of the term “complement”). For example, a preposition is the syntactic head of a Prepositional Phrase, but a preposition needs a following RE (Referring Expression) in order to be a complete phrase. Therefore we say that the RE in a prepositional phrase is the complement of the preposition.

We define a \textit{direct complement clause}\is{direct complement clauses} in Kagayanen as a dependent clauses that is an argument (absolutive or ergative) of another verb:

\ea
\label{bkm:Ref474213030}
Paumpisaan  din  en  \textbf{pagtanem}  \textbf{ta}  \textbf{kamuti}. \smallskip\\
\gll Pa-umpisa-an  din  en  \textbf{pag-tanem}  \textbf{ta}  \textbf{kamuti}. \\
\textsc{t.r}-start-\textsc{apl}  3\textsc{s.erg}  \textsc{cm}  \textsc{nr.act}-plant  \textsc{nabs}  cassava \\
\glt ‘S/he started \textbf{planting} \textbf{cassava}.’
\z

In example \REF{bkm:Ref474213030}, the nominalized clause meaning ‘planting cassava’ is the absolutive argument of the main clause ‘s/he started . . .’ This is because the main verb is grammatically transitive (indicated by the prefix \textit{pa}- and applicative suffix -\textit{an}), the Actor is ergative, and there is no other argument that could possibly be the absolutive. Direct complement clauses are often absolutive arguments of transitive matrix clauses, as in \REF{bkm:Ref474213030}. Examples \REF{bkm:Ref114814303} through \REF{bkm:Ref114832637} are from the corpus. In these examples, the complement is bolded in Kagayanen and in the English free translations:

\ea
\label{bkm:Ref114814303}
 Daw  asuron  en  man  isab  aged  na  maimo  en  man  na beggas  asta  nang  na  matapos  \textbf{pag-asod}  \textbf{ta}  \textbf{batad}. \smallskip\\
 \gll Daw  asod-en  en  man  isab  aged  na  ma-imo  en  man  na beggas  asta  nang  na  ma-tapos  \textbf{pag-asod}  \textbf{ta}  \textbf{batad}. \\
and  pound-\textsc{t.ir}  \textsc{cm}  also  again  so.that  \textsc{lk}  \textsc{a.hap.ir}-make  \textsc{cm}  also  \textsc{lk}\\
uncooked.rice  until  only  \textsc{lk}  \textsc{a.hap.ir}-finish  \textsc{nr.act}-pound  \textsc{nabs}  sorghum \\
\glt ‘And pound (it) again so that it will become grain (ready for cooking) until (you) just finish \textbf{pounding} \textbf{sorghum}.’ [YBWE-T-04 2.12]
\z

\ea
Dasigen  ta  pa  gid  \textbf{pagpanaw}  aged  makalambot ki  dayon  ta  bukid  ya. \smallskip\\
\gll Dasig-en  ta  pa  gid  \textbf{pag-panaw}  aged  maka-lambot ki  dayon  ta  bukid  ya. \\
fast-\textsc{t.ir}  1\textsc{p.incl.erg}  \textsc{inc}  \textsc{int}  \textsc{nr.act}-go/walk  so.that  \textsc{i.hap.ir}-reach 1\textsc{p.incl.abs}  immediately  \textsc{nabs}  mountain  \textsc{def.f} \\
\glt `Let’s speed up still more \textbf{walking} so that we will be able to reach immediately the mountain.’ [CBWN-C-16 3.3]
\z

\ea
Kinangļan  bantayan  ta  \textbf{iran}  \textbf{na}  \textbf{pag-uli}. \smallskip\\
\gll Kinangļan  \emptyset{}-bantay-an  ta  \textbf{iran}  \textbf{na}  \textbf{pag-uli}. \\
need  \textsc{t.ir}-watch/guard-\textsc{apl}  1\textsc{p.incl.erg}  3\textsc{p.gen}  \textsc{lk}  \textsc{nr.act}-return.home \\
\glt ‘It is necessary we watch for \textbf{their} \textbf{coming} \textbf{home}.’ [CBWN-C-18 9.1]
\z
\ea
Mļaman  no  na  kita  ko  \textbf{paglarga}  \textbf{no} … \smallskip\\
\gll Ma-aļam-an  no  na  ...-kita  ko  \textbf{pag-larga}  \textbf{no} … \\
\textsc{a.hap.ir}-know-\textsc{apl}  2\textsc{s.erg}  \textsc{lk}  \textsc{a.hap.r}-see  1\textsc{s.erg}  \textsc{nr.act}-depart  2\textsc{s.gen} \\
\glt ‘You know that I saw \textbf{your} \textbf{departing} …’ [BCWL-C-01 2.4]
\z
\ea
… dili  ko  malipatan  a  \textbf{pagbasa}  \textbf{ta}  \textbf{ake}  \textbf{na}  \textbf{Biblia}. \smallskip\\
\gll … dili  ko  ma-lipat-an  a  \textbf{pag-basa}  \textbf{ta}  \textbf{ake}  \textbf{na}  \textbf{Biblia}. \\
{}  \textsc{neg.ir}  1\textsc{s.erg}  \textsc{a.hap.ir}-forget-\textsc{apl}  \textsc{inj}  \textsc{nr.act}-read  \textsc{nabs}  1\textsc{s.gen}  \textsc{lk}  Bible \\
\glt ‘… I will not forget \textbf{reading} \textbf{my} \textbf{Bible}.’ [JCOE-C-04 15.1]
\z

\ea
Man-o  paran  tak  dili  ta  isalyo \textbf{ate}  \textbf{na}  \textbf{pagtanem}  \textbf{ta}  \textbf{kaoy}  \smallskip\\
\gll Man-o  paran  tak  dili  ta  i-salyo \textbf{ate}  \textbf{na}  \textbf{pag-tanem}  \textbf{ta}  \textbf{kaoy}  \\
why  perhaps  because  \textsc{neg.ir}  1\textsc{p.incl.erg}  \textsc{t.deon}-change
1\textsc{p.incl.gen}  \textsc{lk}  \textsc{nr.act}-plant  \textsc{nabs}  tree/wood \\
\glt ‘Why perhaps should we not change \textbf{our} \textbf{planting} \textbf{to} \textbf{trees}.’ [ROOB-T-01 11.9]
\z
\ea
Natapos ko gid man ake na pag-iskwila ta high school naan St. Andrew's High School, Anini-y, Antique. \\
\gll Na-tapos	ko	gid	man	ake	na	pag-iskwila	ta	high school naan	St. Andrew's High School,	Anini-y,	Antique. \\
\textsc{a.hap.r}-finish	1\textsc{s.erg}	\textsc{int}	also	1\textsc{s.gen}	\textsc{lk}	\textsc{nr.act}-school	NABS	high school \textsc{spat.def}	St. Andrew's High School,	Anini-y,	Antique\\
\glt ‘I had really finished my schooling of high school at St. Andrew’s High School Anini-y Antique.’ (JBON-J-01 2.9)\\
\gll Gani patapos ko sulat lesson plan ko …\\
Gani	pa-tapos	ko	sulat	lesson plan	ko …\\
so	\textsc{t.r}-finish	1\textsc{s.erg}	write	lesson plan	1\textsc{s.gen}\\
\glt ‘So I finished writing my lesson plan …’ (JCWN-L-31 2.4)
\z

The English free translations of \REF{ex:thehitting} and \REF{ex:smashedfine} may make it seem that the bolded portions are simple verbal nouns-they have no expressed arguments or other overt trappings of clauses. Nevertheless, they are are complement clauses by our definition because they refer to specific events in the scene being depicted by the speaker, rather than general activities of hitting \REF{ex:thehitting} and pounding \REF{ex:smashedfine}.

\ea
\label{ex:thehitting}
Nakita  ko  \textbf{pag-igo}  \textbf{ya}. \smallskip\\
\gll Na-kita  ko  \textbf{pag-igo}  \textbf{ya}. \\
\textsc{a.hap.r}-see  1\textsc{s.erg}  \textsc{nr.act}-hit  \textsc{def.f} \\
\glt ‘I saw \textbf{the} \textbf{hitting}.’ (This text is about spearing a wild pig. The speaker asserts in this sentence that he saw the spear hit the pig.) [RCON-L-01 4.1]
\z
\ea
\label{ex:smashedfine}
Pagiran  ko  \textbf{pag-asod}  asta  na  nakita  ko na  naleg-as  en. \smallskip\\
\gll Pa-gid-an  ko  \textbf{pag-asod}  asta  na  na-kita  ko na  na-leg-as  en. \\
\textsc{t.r}-\textsc{int}-\textsc{ap}  1\textsc{s.erg}  \textsc{nr.act}-pound  until  \textsc{lk}  \textsc{a.hap.r}-see  1\textsc{s.erg}
\textsc{lk}  \textsc{a.hap.r}-smash  \textsc{cm} \\
\glt ‘I intensified \textbf{pounding} until I saw that (it) had been smashed fine.’ (This is about pounding rice.) [JCWE-T-13 2.7]
\z

Example \REF{bkm:Ref114832637} includes three nominalizations, two with \textit{pag}- (\textit{pag-ambaļ} `speaking' and \textit{pag-suļat} `writing') and one bare form (\textit{basa} ‘reading’). In such cases, \textit{pag}- is much more likely to be omitted, as it is in the third member of the sequence:

\ea
\label{bkm:Ref114832637}
Paistudyuan  danen  \textbf{pag-ambaļ}.  Paistudyuan  danen  \textbf{pagsuļat}, \textbf{basa} kag tanan-tanan  en  danen. \smallskip\\
\gll Pa-istudyo-an  danen  \textbf{pag-ambaļ}.  Pa-istudyo-an  danen  \textbf{pag-suļat}, \textbf{basa}  kag\footnotemark{}  tanan\sim{}-tanan  en  danen. \\
\textsc{t.r}-astudy-\textsc{apl}  3\textsc{p.erg}  \textsc{nr.act}-say  \textsc{t.r}-study-\textsc{apl}  3\textsc{p.erg}  \textsc{nr.act}-write
read  and  \textsc{red}\sim{}all  \textsc{cm}  3\textsc{p.gen} \\
\footnotetext{This word is code switching from \isi{Hiligaynon}.}
\glt ‘They studied \textbf{speaking}. They studied \textbf{writing}, \textbf{reading}, and absolutely everything is (what) they (studied).’ [JCOE-T-06 7.4-5]
\z

\textit{Pag}{}- clauses may also function as absolutive arguments of intransitive matrix clauses, as in examples \REF{bkm:Ref474479194} through \REF{bkm:Ref114812177} from the corpus:

\ea
\label{bkm:Ref474479194}
Siguro,  baked  gid  \textbf{imo}  \textbf{na}  \textbf{pagtingaļa}   na  man-o  tak nļaman  ko  imo  na  ngaran. \smallskip\\
\gll Siguro,  baked  gid  \textbf{imo}  \textbf{na}  \textbf{pag-tingaļa}   na  man-o  tak na-aļaman  ko  imo  na  ngaran. \\
perhaps  big  \textsc{int}  2\textsc{s.gen}  \textsc{lk}  \textsc{nr.act}-wonder  \textsc{lk}  why  because
\textsc{a.hap.r}-know-\textsc{apl}  1\textsc{s.erg}  2\textsc{s.gen}  \textsc{lk}  name \\
\glt ‘Perhaps, \textbf{your} \textbf{wondering} is really great why I know your name.’ [EMWL-T-04 5.2]
\z
\ea
Nabugtu  en  \textbf{ate}  \textbf{na}  \textbf{pagpari}. \smallskip\\
\gll Na-bugtu  en  \textbf{ate}  \textbf{na}  \textbf{pag-pari}. \\
\textsc{a.hap.r}-break  \textsc{cm}  1\textsc{p.incl.gen}  \textsc{lk}  \textsc{nr.act}-friend \\
\glt ‘\textbf{Our} \textbf{being} \textbf{friends} has been broken now.’ [RBWN-T-02 5.6]
\z
\ea
Na  natapos  en  \textbf{ake}  \textbf{na}  \textbf{pag-indyiksyon}  daw  education campaign, … \smallskip\\
\gll Na  na-tapos  en  \textbf{ake}  \textbf{na}  \textbf{pag-indyiksyon}  daw  education campaign, … \\
\textsc{lk}  \textsc{a.hap.r}-finish  \textsc{cm}  1\textsc{s.gen}  \textsc{lk}  \textsc{nr.act}-injection  and  education campaign \\
\glt ‘When \textbf{my} \textbf{giving} \textbf{injections} and education campaign was finished, …’ [JCWN-T-21 14.1]
\z
\ea
… na  magbiskeg  pa  gid  \textbf{inyo}  \textbf{na}  \textbf{pagsalig}  \textbf{ta}  \textbf{Dyos}. \smallskip\\
\gll … na  mag-biskeg  pa  gid  \textbf{inyo}  \textbf{na}  \textbf{pag-salig}  \textbf{ta}  \textbf{Dyos}. \\
{} \textsc{lk}  \textsc{i.ir}-strong  \textsc{inc}  \textsc{int}  2\textsc{p.gen}  \textsc{lk}  \textsc{nr.act}-trust  \textsc{nabs}  God \\
\glt ‘… that \textbf{your} \textbf{trusting} \textbf{God} will become really stronger. [BCWL-C-03 8.7]
\z
\ea
Pamikawan  bata  an  aged  magdayad  \textbf{pagdako}  \textbf{din}  \textbf{an}, ... \smallskip\\
\gll Pa-mikaw-an  bata  an  aged  mag-dayad  \textbf{pag-dako}  \textbf{din}  \textbf{an}, ... \\
\textsc{t.r}-food.sacrifice-\textsc{apl}  child  \textsc{def.m}  so.that  \textsc{i.ir}-good  \textsc{nr.act}-large  3\textsc{s.gen}  \textsc{def.m} \\
\glt ‘A child has a food offering done (for him/her) so that \textbf{his/her} \textbf{growing} \textbf{up} will be good, …’ [JCWE-T-16 2.4]
\z
\ea
\label{bkm:Ref114812177}
Madayon  gid  \textbf{iran}  \textbf{na}  \textbf{pagsawaay}. \smallskip\\
\gll Ma-dayon  gid  \textbf{iran}  \textbf{na}  \textbf{pag-sawa-ay}. \\
\textsc{a.hap.ir}-continue  \textsc{int}  3\textsc{p.gen}  \textsc{lk}  \textsc{nr.act}-spouse-\textsc{rec} \\
\glt ‘\textbf{Their} \textbf{being} \textbf{married} will continue on.’ (In the context the parents of the boy and girl make an agreement that their children will be married and the children can’t do anything about it even though they have not courted each other.) [HCWE-J-02 4.3]
\z

Examples of \textit{pag}{}- clauses functioning in the ergative role are rare, however they do occur occasionally in the corpus, as in examples \REF{bkm:Ref114812284} and \REF{bkm:Ref114812335}:

\ea
\label{bkm:Ref114812284}
Bellayan  ka  taan  \textbf{ta}  \textbf{pagbasa}  \textbf{ta}  \textbf{ake}  \textbf{na}  \textbf{suļat}. \smallskip\\
\gll Bellay-an  ka  taan  \textbf{ta}  \textbf{pag-basa}  \textbf{ta}  \textbf{ake}  \textbf{na}  \textbf{suļat}. \\
tire-\textsc{apl}  2\textsc{s.abs}  maybe  \textsc{nabs}  \textsc{nr.act}-read  \textsc{nabs}  1\textsc{s.gen} \textsc{lk}  write/letter \\
\glt ‘\textbf{Reading} \textbf{my} \textbf{letter} maybe will tire you.’ [BCWL-C-01 3.34]
\z
\ea
\label{bkm:Ref114812335}
Gani  a  mga  ittaw  naan  ta  Bario  nakulian \textbf{ta}  \textbf{pagnubig}  kumo  waig  naan  pa  kamangen ta  Barangay  Wahig. \smallskip\\
\gll Gani  a  mga  ittaw  naan  ta  Bario  na-kuli-an \textbf{ta}  \textbf{pag-nubig}  kumo  waig  naan  pa  kamang-en ta  Barangay  Wahig. \\
so/therefore  \textsc{inj}  \textsc{pl}  people  \textsc{spat.def}  \textsc{nabs}  Bario  \textsc{a.hap.r}-difficult/slow-\textsc{apl}
\textsc{nabs}  \textsc{nr.act-}haul.water  because  water  \textsc{spat.def}  \textsc{inc}  get-\textsc{t.ir} \textsc{nabs}  community  Wahig \\
\glt `So \textbf{hauling} \textbf{water} difficults/slows the people in Bario because the water is even gotten from the community of Wahig.’ [VPWE-T-01 2.7]
\z

Bare-form verbs may also function as nominalized \textit{direct complement clauses}\is{direct complement clauses} (see example \ref{bkm:Ref114832637} above in which the bare-form \textit{basa} ‘read’ is coordinated with a \textit{pag}{}- clause). Note that the verb \textit{tanem} in example \REF{bkm:Ref114659732} has no inflection. Native speakers report no semantic differences between these and corresponding constructions with \textit{pag-}verbs in the complements (c.f. example \ref{bkm:Ref474213030} above):

\ea
\label{bkm:Ref114659732}
Paumpisaan  din  en  \textbf{tanem}  \textbf{ta}  \textbf{kamuti}. \smallskip\\
\gll Pa-umpisa-an  din  en  \textbf{tanem}  \textbf{ta}  \textbf{kamuti}. \\
\textsc{t.r}-start-\textsc{apl}  3\textsc{s.erg}  \textsc{cm}  plant  \textsc{nabs}  cassava \\
\glt ‘S/he started \textbf{planting} \textbf{cassava}.’
\z

The following are examples of bare-form direct complement clauses from the corpus:

\ea
Naan  nay  dya  namatii  a  \textbf{kanta}  na,  ``Aliluya,  aliluya.” \smallskip\\
\gll Naan  nay  dya  na-mati-i  a  \textbf{kanta}  na,  ‘Aliluya,  aliluya.” \\
\textsc{spat.def}   1\textsc{p.excl.erg}  \textsc{d}4\textsc{loc}  \textsc{a.hap.r}-hear-\textsc{xc.apl}  \textsc{inj}  sing    \textsc{lk}  hallelujah  hallelujah \\
\glt ‘There we hear singing, “Hallelujah, hallelujah.” (They heard \textbf{singing} coming from the top of the mast of their boat they were riding.) [VAWN-T-18 5.9]
\z
\ea
Gani  patapos  ko  \textbf{suļat}  lesson plan  ko  daw  manuga … \smallskip\\
\gll Gani  pa-tapos  ko  \textbf{suļat}  lesson plan  ko  daw  m-tanuga … \\
so  \textsc{t.r}-finish  1\textsc{s.erg}  write  lesson plan  1\textsc{s.gen}  and  \textsc{i.v.ir}-sleep \\
\glt ‘So, I finished \textbf{writing} my lesson plan and went to sleep …. [JCWN-L-31 2.4]
\is{complement clauses!direct|)}
\is{direct complement clauses|)}
\z

\subsection{Nominalizations as oblique complement clauses}
\label{sec:nominalizationsasobliquecomplementclauses}
\is{oblique complement clauses|(}
\is{complement clauses!oblique|(}
In many cases, nominalized clauses are not absolutive or ergative arguments of their matrix as defined in the previous section, yet appear to be required to complete the idea expressed by the verb. Neither are they adverbial clauses because they are not optional sentence adjuncts (see \sectref{bkm:Ref115861741}). For these reasons, it makes sense to call them complement clauses – they “complete” the meaning of the main clause. However, they are not direct complements because there is no evidence that they are core arguments of the matrix verb. Therefore, we call such complements “oblique complements” or “OCs”. Example \REF{bkm:Ref462325755} illustrates the same idea as example \REF{bkm:Ref474213030}, but in a detransitive, or “Actor voice”, construction. Here the verb is grammatically intransitive, the Actor is absolutive, and a clause headed by a nominalized verb is preceded by the non-absolutive particle \textit{ta}:

\ea
\label{bkm:Ref462325755}
Gaumpisa  en  kanen  an    ta  \textbf{(pag)}\textbf{tanem}  \textbf{ta}  \textbf{kamuti}. \smallskip\\
\gll Ga-umpisa  en  kanen  an    ta  \textbf{(pag)-tanem}  \textbf{ta}  \textbf{kamuti}. \\
\textsc{i.r}-start  \textsc{cm}  3\textsc{s.abs}  \textsc{def.m}  \textsc{nabs}  (\textsc{nr.act}-)plant  \textsc{nabs}  cassava \\
\glt ‘S/he started \textbf{to plant cassava}.’
\z

Again, in example \REF{bkm:Ref462325755} there is little if any reported difference in meaning whether the verb in the OC is in the bare form \textit{tanem} or the \textit{pag}- form \textit{pagtanem}. The transitive, or ``patient voice" construction (given earlier in \ref{bkm:Ref474213030} and recapitulated in \ref{ex:startedplantingcassava-2} for convenience) is nearly synonymous:

\ea
\label{ex:startedplantingcassava-2}
Paumpisaan  din  en  \textbf{pagtanem}  \textbf{ta}  \textbf{kamuti}. \smallskip\\
\gll Pa-umpisa-an  din  en  \textbf{pag-tanem}  \textbf{ta}  \textbf{kamuti}. \\
\textsc{t.r}-start-\textsc{apl}  3\textsc{s.erg}  \textsc{cm}  \textsc{nr.act}-plant  \textsc{nabs}  cassava \\
\glt ‘S/he started \textbf{planting} \textbf{cassava}.’
\z

The difference in meaning between \REF{bkm:Ref462325755} and \REF{ex:startedplantingcassava-2} is difficult to capture-they are approximately as synonymous as the English translations. In \REF{bkm:Ref462325755} the fact that s/he has started doing something is in perspective, as though we were wondering when the Actor was going to get up and do something, with little expectation of what s/he would do. In \REF{ex:startedplantingcassava-2}, on the other hand, the act of planting cassava is in perspective, as though we were expecting the Actor to plant cassava.

Examples \REF{bkm:Ref114821261}--\REF{bkm:Ref114821264} illustrate \textit{pag}{}-clauses as oblique complement clauses from the corpus. In each of these examples, the matrix verb by itself is semantically transitive\is{semantic transitivity}\is{transitivity!semantic}. For example, \textit{tagad} ‘waiting’ \REF{bkm:Ref114821261} always implies waiting for something. The same is true for all of the examples in \REF{bkm:Ref114821261} through \REF{bkm:Ref114821264}; the main (or “matrix”) clause evokes a scene that involves an Undergoer which may be expressed in an oblique nominalized clause (bolded). Therefore the oblique clauses in these examples can be considered complements:

\ea
\label{bkm:Ref114821261}
Manang,  magtagad  kay  \textbf{ta imo na pagbalik}. \smallskip\\
\gll Manang,  mag-tagad  kay  \textbf{ta}  \textbf{imo}  \textbf{na}  \textbf{pag-balik}. \\
older.sister  \textsc{i.ir}-wait  1\textsc{p.excl.abs}  \textsc{nabs}  2\textsc{s.gen}  \textsc{lk}  \textsc{nr.act}-return \\
\glt ‘Older sister, we will wait \textbf{for your return}.’ [SBWL-C-02 6.1]
\z
\ea
Mga  sampuļo  daw  annem  na  bļangay  daw  batil na  gadengngan  \textbf{ta}  \textbf{pag-alin  ta  Cagayancillo}. \smallskip\\
\gll Mga  sampuļo  daw  annem  na  bļangay  daw  batil na  ga-dengngan  \textbf{ta}  \textbf{pag-alin}   \textbf{ta}  \textbf{Cagayancillo}. \\
\textsc{pl}  ten  and  six  \textsc{lk}  2.masted.boat  and  1.masted.boat
\textsc{lk}  \textsc{i.r}-do.at.same.time  \textsc{nabs}  \textsc{nr.act}-from  \textsc{nabs}  Cagayancillo \\
\glt `About sixteen two-masted boats and one-masted boats were going at the same time together \textbf{leaving Cagayancillo}.’ [VAWN-T-18 2.7]
\z
\ea
Gadiritso  kay  gid  \textbf{ta  paglarga}. \smallskip\\
\gll Ga-diritso  kay  gid  \textbf{ta}  \textbf{pag-larga}. \\
\textsc{i.r}-straight  1\textsc{p.excl.abs}  \textsc{int}  \textsc{nabs}  \textsc{nr.act}-depart \\
\glt ‘We right away were \textbf{departing}.’ [VAWN-T-18 3.6]
\z
\ea
Tudluan  \textbf{ta}  \textbf{pagpasalamat  ta  Ginuo ta  mga betang  na  paatag  din  ki  kiten}. \smallskip\\
\gll \emptyset{}-Tudlo-an  \textbf{ta}  \textbf{pag-pa-salamat}  \textbf{ta}  \textbf{Ginuo}  \textbf{ta}  \textbf{mga} \textbf{betang}  \textbf{na}  \textbf{pa-atag}  \textbf{din}  \textbf{ki}  \textbf{kiten}. \\
\textsc{t.ir}-teach-\textsc{apl}  \textsc{nabs}  \textsc{nr.act}-\textsc{caus}-thanks  \textsc{nabs}  Lord  \textsc{nabs}  \textsc{pl}
things  \textsc{lk}  \textsc{t.r}-give  3\textsc{s.erg}  \textsc{obl.p}  1\textsc{p.incl} \\
\glt `Teach (them) \textbf{to give thanks to the Lord for the things he gave to us}.’ [ETOP-C-10 1.2]
\z
\ea
… nabatyagan  din  kon  na  naluyaan  yi  na  sidda daw  gauntat  \textbf{ta}  \textbf{iya}  \textbf{na}  \textbf{pagluoy}. \smallskip\\
\gll … na-batyag-an  din  kon  na  na-luya-an  yi  na  sidda daw  ga-untat  \textbf{ta}  \textbf{iya}  \textbf{na}  \textbf{pag-luoy}. \\
{}  \textsc{a.hap.r}-feel-\textsc{apl}  3\textsc{s.erg}  \textsc{hsy}  \textsc{lk}  \textsc{a.hap.r}-weak-\textsc{apl}  \textsc{d}1\textsc{adj}  \textsc{lk}  fish
and  \textsc{i.r}-stop  \textsc{nabs}  3\textsc{s.gen}  \textsc{lk}  \textsc{nr.act}-swim \\
\glt `… he felt that this fish had become weak and stopped \textbf{his swimming}.’ (This is a text about a man who was swallowed by a whale or big fish, and was still alive inside the stomach when he felt the whale stop swimming because it had been beached in a river.) [CBWN-C-21 4.6]
\z
\ea
\label{bkm:Ref114821264}
Pagtapos  ko  ta  iskwila  ta  haiskol  gapadayon a  man  \textbf{ta  pag-iskwila  naan  unti Puerto Princesa City  ta  College} … \smallskip\\
\gll Pag-tapos  ko  ta  iskwila  ta  haiskol  ga-pa-dayon a  man  \textbf{ta}  \textbf{pag-iskwila}  \textbf{naan}  \textbf{unti} \textbf{Puerto} \textbf{Princesa} \textbf{City}  \textbf{ta}  \textbf{College} … \\
\textsc{nr.act}-finish  1\textsc{s.erg}  \textsc{nabs}  school  \textsc{nabs}  high.school  \textsc{i.r}-\textsc{caus}-continue
1\textsc{s.abs}  also  \textsc{nabs}  \textsc{nr.act}-school  \textsc{spat.def}  \textsc{d}1\textsc{loc.pr}
Puerto Princesa City  \textsc{nabs}  College \\
\glt `When I finished high school education I continued \textbf{education also here in Puerto Princesa City College}.’ [DDWN-C-01 7.5]
\z

Example \REF{bkm:Ref477269206} illustrates a construct which contains a \textit{pag}{}- complement clause embedded within another \textit{pag}{}- complement clause. The matrix verb \textit{atagan} ‘to give to’ takes an oblique complement \textit{ta pagtamed ...} ‘focus attention on ...’. This oblique complement in turn takes a direct complement \textit{imo na pag-iskwila} ‘your schooling’/`education’:

\ea
\label{bkm:Ref477269206}
Ake  nang  na  malaygay  ki  kaon  na  atagan  no gid  \textbf{ta pagtamed imo na pag-iskwila}. \smallskip\\
\gll Ake  nang  na  ma-laygay  ki  kaon  na  \emptyset{}-atag-an  no gid  \textbf{ta}  \textbf{pag-tamed}  \textbf{imo}  \textbf{na}  \textbf{pag-iskwila}. \\
1\textsc{s.gen}  only  \textsc{lk}  \textsc{a.hap.ir}-advise  \textsc{obl.p}  2s  \textsc{lk}  \textsc{t.ir}-give-\textsc{apl}  2\textsc{s.erg}
\textsc{int}  \textsc{nabs}  \textsc{nr.act}-focus.attention  2\textsc{s.gen}  \textsc{lk}  \textsc{nr.act}-school \\
\glt `What I can advise you only is that \textbf{you really focus (your) attention on your schooling}.’ [YBWL-T-02 2.3]
\z

\largerpage
Example \REF{ex:persevere} illustrates the Class VIII verb \textit{pursigir} ‘persevere’. Class VIII is the class of verbs that requires the applicative suffix -\textit{an} when it appears in a  grammatically transitive form (see \chapref{chap:stemformingprocesses}, \sectref{sec:applicative-an} and \chapref{chap:verbclasses-1}, \sectref{sec:volitionaltransitiveroots}). This verb may occur with an oblique complement when grammatically intransitive \REF{bkm:Ref114840584}. Several complement-taking verbs fall into this grammatical class, as exemplified in \REF{bkm:Ref114842025} and \REF{bkm:Ref114842515}:

\ea
\label{ex:persevere}
    \ea[]{
    \label{bkm:Ref114840584}
    Yaken  gapursigir  gid  \textbf{ta  pagsuļat  ki  kyo}  aged mļaman  ko  man  inyo  na  kaimtangan  dyan. \\\smallskip
\gll Yaken  ga-pursigir  gid  \textbf{ta}  \textbf{pag-suļat}  \textbf{ki}  \textbf{kyo}  aged ma-aļam-an  ko  man  inyo  na  kaimtangan  dyan. \\
    1\textsc{s.abs}  \textsc{i.r}-persevere  \textsc{int}  \textsc{nabs}  \textsc{nr.act}-write  \textsc{obl.p}  2p  so.that \textsc{a.hap.ir}-know-\textsc{apl}  1\textsc{s.erg}  also  2\textsc{p.gen}  \textsc{lk}  situation/condition  \textsc{d}2\textsc{loc} \\
    \glt ‘As for me (I) really persevered in \textbf{writing to you} so that I also will know your   situation there.’ [PBWL-T-09 2.1]
    }
    \ex[*]{
    \label{bkm:Ref114840584b}
    Papursigir ko pagsuļat ki kaon.
    }
    \ex[]{
    \label{bkm:Ref114840584c}
    Papursigiran ko pagsuļat ki kaon.
    }
    \z
\z
\ea
\label{bkm:Ref114842025}
    \ea[]{
    Kinangļan  bantayan  ta  \textbf{iran}  \textbf{na}  \textbf{pag-uli}. \\\smallskip
\gll Kinangļan  \emptyset{}-bantay-an  ta  \textbf{iran}  \textbf{na}  \textbf{pag-uli}. \\
    need  \textsc{t.ir}watch/guard-\textsc{apl}  1\textsc{p.incl.erg}  3\textsc{p.gen}  \textsc{lk}  \textsc{nr.act}-go.home \\
    \glt ‘It is necessary we watch for \textbf{their} \textbf{coming} \textbf{home}.’ [CBWN-C-18 9.1]
    }
    \ex[]{
    Magbantay ki ta iran na pag-uli.
    }
    \ex[*]{
    Bantayen ta iran na pag-uli.
    }
    \z
\z

\ea
\label{bkm:Ref114842515}
    \ea[]{
    Manang,  magtagad  kay  \textbf{ta}  \textbf{imo}  \textbf{na}  \textbf{pagbalik}. \\\smallskip
\gll Manang,  mag-tagad  kay  \textbf{ta}  \textbf{imo}  \textbf{na}  \textbf{pag-balik}. \\
older.sister  \textsc{i.ir}-wait  1\textsc{p.excl.abs}  \textsc{nabs}  2\textsc{s.gen}  \textsc{lk}  \textsc{nr.act}-return \\
    \glt ‘Older sister, we will wait for your return.’ [SBWL-C-02 6.1] \\
    }
    \ex[*]{
    Tagaren nay imo na pagbalik.
    }
    \ex[]{
    Tagaran nay imo na pagbalik.
    }
    \z
\z

The following are some additional examples of oblique complements from the corpus. Matrix verbs that take oblique complements include verbs that describe the manner, intensity, or Actor’s involvement with the action described in the complement (\ref{bkm:Ref115163609}-\ref{bkm:Ref115163660}):

\ea
\label{bkm:Ref115163609}
Dili  ka  gid  malipat  \textbf{ta}  \textbf{pagsalig}  \textbf{ki}  \textbf{kanen}. \smallskip\\
\gll Dili  ka  gid  ma-lipat  \textbf{ta}  \textbf{pag-salig}  \textbf{ki}  \textbf{kanen}\footnotemark{}. \\
\textsc{neg.ir}  2\textsc{s.abs}  \textsc{int}  \textsc{a.hap.ir}-forget  \textsc{nabs}  \textsc{nr.act}-trust  \textsc{obl.p}  3s \\
\footnotetext{There are two verb roots spelled \textit{salig} in Kagayanen. In this example, stress falls on the first syllable ([sálig]), and the meaning is `to trust.' In the other, stress falls on the second syllable ([salíg]), and means `to think wrongly'.}
\glt ‘Do not forget really to \textbf{trust him}.’ [VBWL-T-08 2.7]
\z
\ea
Daw  may  isya  pa  duti  na  gaduwa-duwa  pa  \textbf{ta} \textbf{pagbaton}  ki  Jesu Kristo  na  iya  na  manunubos  pirsunal … \smallskip\\
\gll Daw  may  isya  pa  duti  na  ga-duwa-duwa  pa  \textbf{ta} \textbf{pag-baton}  ki  Jesu Kristo  na  iya  na  ma-ng-tubos  pirsunal … \\
if/when  \textsc{ext.in}  one  \textsc{inc}  \textsc{d}1\textsc{loc.pr}  \textsc{lk}  \textsc{i.r}-waver/doubt  \textsc{inc}  \textsc{nabs}
\textsc{nr.act}-receive  \textsc{obl.p}  Jesus Christ  \textsc{lk}  3\textsc{s.gen}  \textsc{lk}  \textsc{a.hap.ir-pl}-redeem  personal \\
\glt `If there is one here who is still wavering/doubting \textbf{of} \textbf{receiving} Jesus Christ his personal redeeemer …' [TTOB-L-03 7.14]
\z

\ea
    \ea[]{
    \textit{dugang} ‘to add or to increase VERBing’: \\
    Gadugang  kay  \textbf{ta}  \textbf{pagpursigir}  ta ame  na  pag-ubra. \\\smallskip
\gll Ga-dugang  kay  \textbf{ta}  \textbf{pag-pursigir}  ta ame  na  pag-ubra. \\
    \textsc{i.r}-add/increase  1\textsc{p.excl.abs}  \textsc{nabs}  \textsc{nr.act}-persevere  \textsc{nabs}
    1\textsc{p.excl.gen}  \textsc{lk}  \textsc{nr.act}-work/do \\
    \glt `We increased \textbf{persevering} in our work.’
    }
    \ex[*]{
    Padugang nay pagprusigir ta ame na ubra.
    }
    \ex[]{
    Padugangan nay pagprusigir ta ame na ubra.
    }
    \z
\z

The verb \textit{lipat} ‘to forget’ takes an obligatory complement clause that is either an irrealis clause preceded by the linker/complementizer \textit{na} \REF{ex:toforgetverbing}, a bare-form clause, or a \textit{pag}{}- clause. All of the examples in \REF{bkm:Ref115163660} have the same basic meaning, but with different nuances that are difficult to capture in the English translations.
\ea
\label{bkm:Ref115163660}
    \ea{
    \label{ex:toforgetverbing}
    Grammatically transitive frame, irrealis complement: \\
    Nalipatan  ko  na  patayen  apoy  an.   \\\smallskip
\gll Na-lipat-an  ko  na  patay-en  apoy  an. \\
    \textsc{a.hap.r}-forge-\textsc{apl}  1\textsc{s.erg}  \textsc{lk}  die-\textsc{t.ir}  fire  \textsc{def.m} \\
    \glt ‘I forgot to extinguish the fire.’
    }
    \ex{
    \label{ex:pagpataytaapoyan}
    Grammatically transitive frame, \textit{pag}- complement: \\
    Nalipatan ko pagpatay ta apoy an.
    }
    \ex{
    \label{ex:patayenapoyan}
    Grammatically intransitive frame, irrealis complement: \\
    Nalipat a na patayen apoy an.
    }
    \ex{
    \label{ex:pagpataytaapoy}
    Grammatically intransitive frame, \textit{pag}- complement: \\
    Nalipat a ta pagpatay ta apoy.
    }
    \z
\z

Since most lexical roots are not strictly categorized as nominal or verbal, at times it is not clear as to whether a particular uninflected form is a “bare verb” or just a root being used nominally. For example, in \REF{bkm:Ref117751045} the root \textit{tabang} ‘help’ is used twice, first as a nominal, and second as an inflected verb. Is the first instance a bare-form nominalized clause meaning “God helping us”, or just a noun? In this section we have made the general decision that if a bare form has one or more overt arguments, we can call it a nominalized clause. Otherwise, it is just a root being used as a noun. In \REF{bkm:Ref117751045} the first instance of \textit{tabang} has no arguments (\textit{naan ta Dios} `from God' is clearly oblique because of the spatial demonstrative \textit{naan}), therefore this is not a complement clause. There may be situations where this determination remains unclear, or where we have deviated from this general guideline.

\ea
\label{bkm:Ref117751045}
Gagaļ  a  na  gangayo  ta  \textbf{tabang}  naan  ta  Dios na  ambaļ  ko,  ``Dios  ko  tabangan  no  kami  i." \smallskip\\
\gll Ga-gaļ  a  na  ga-ng-ngayo  ta  \textbf{tabang}  naan  ta  Dios na  ambaļ  ko,  ``Dios  ko  \emptyset{}-tabang-an  no  kami  i." \\
\textsc{i.r}-cry  1\textsc{s.abs}  \textsc{lk}  \textsc{i.r-pl}-request  \textsc{nabs}  help  \textsc{spat.def}  \textsc{nabs}  God
\textsc{lk}   say  1\textsc{s.erg}  God  1\textsc{s.gen}  \textsc{t.ir}-help-\textsc{apl}  2\textsc{s.erg}  2\textsc{p.abs}  \textsc{def.n} \\
\glt `I was crying requesting \textbf{help} from God when I said, “My God help us.” [CBWN-C-11 4.14]
\is{complement clauses!oblique|)}
\is{oblique complement clauses|)}
\z

\subsection{Nominalizations as adverbial clauses}
\label{bkm:Ref115861741} \label{sec:nominalizationsasadverbialclauses}
\is{adverbial clauses|(}
\textit{Adjuncts}\is{adjuncts} are clause constituents that are not part of the argument structure of the clause. They contrast with complements in that they do not fill out or “complete” the idea expressed; they merely add additional information. Adjuncts do not affect nor are they affected by the grammatical transitivity of the main verb. Some linguists (e.g., \citealt{huddleston2002, payne2011}) argue that the difference between complement and adjunct is more a continuum than an absolute distinction. This does seem to be the case in Kagayanen, though a full study of the properties of complements and adjuncts must await future research. Other theoretical linguists would say that verbs don’t “govern”, “select”, or “license” their adjuncts, as they do their complements; any verb can in principle take any adjunct, though typically certain semantic classes of verbs are compatible with only certain semantic types of complements. This also seems to be the case in Kagayanen.

When a clause fills an adjunct role within another clause, the adjunct is sometimes called an \textit{adverbial clause}\is{adverbial clauses} This is because \isi{adjunct clauses} tend to express the same kinds of information that adverbs do, such as time (sequence, simultaneity), manner, reason, purpose, and condition. In Kagayanen, adjunct clauses may be subjunctive (\sectref{bkm:Ref117759078}), fully finite (\sectref{bkm:Ref460483264}), or nominalizations formed with \textit{pag}{}-. In this section we will illustrate and describe some \textit{pag}{}- clauses functioning as adjuncts to other clauses. While some of the constructions described in this section may seem like complement clauses from the perspective of the English translations, they are all non-obligatory, and so must be classified as adjuncts according to the grammar of Kagayanen.

The following examples illustrate a selection of \textit{pag}{}- clauses serving adjunct functions that occur in the corpus. These are not complements because the main clause (e.g., \textit{yaken en balik} ‘I returned’ in example \ref{bkm:Ref115164523}) is a complete assertion without it:

\ea
\label{bkm:Ref115164523}
... yaken  en  balik  \textbf{ta}  \textbf{pag-iskwila}  nyaan  ta  Puerto. \smallskip\\
\gll ... yaken  en  ...-balik  \textbf{ta}  \textbf{pag-iskwila}  nyaan  ta  Puerto. \\
{}   1\textsc{s.abs}  \textsc{cm}  \textsc{i.r}-return  \textsc{nabs}  \textsc{nr.act}-school  \textsc{spat.def}  \textsc{nabs}  Puerto  \\
\glt ‘… I returned \textbf{to} \textbf{studying} in Puerto.’ [JCOEC-04 13.2]
\z
\ea
Gakuyog  a  ki  danen  \textbf{ta}  \textbf{paglagaw}. \smallskip\\
\gll Ga-kuyog  a  ki  danen  \textbf{ta}  \textbf{pag-lagaw}. \\
\textsc{i.r}-go.with  1\textsc{s.abs}  \textsc{obl.p}  3p  \textsc{nabs}  \textsc{nr.act}-walk.around \\
\glt ‘I went with them \textbf{walking} \textbf{around}.’
\z

Example \REF{bkm:Ref115164893} illustrates two \textit{pag-} clauses. The first is an adverbial clause modifying the main clause \textit{gapasalamat a} ... ‘I give thanks’, while the second seems to be an adverbial clause modifying \textit{pagtapna} ‘instructing’:

\ea
\label{bkm:Ref115164893}
Una  ta  tanan  bilang  gapasalamat  a  ta  mga  maistra ta  high  school  \textbf{ta}  \textbf{pagtapna}  \textbf{ta}  \textbf{pag-ubraen} \textbf{ta}  \textbf{stage}  na  wi  anduni  naiwasan ta  ate  na  prugrama. \smallskip\\
\gll Una  ta  tanan  bilang  ga-pa-salamat  a  ta  mga  maistra ta  high  school  \textbf{ta}  \textbf{pag-tapna}  \textbf{ta}  \textbf{pag-ubra-en} \textbf{ta}  \textbf{stage}  na  wi  anduni  na-iwas-an ta  ate  na  prugrama. \\
first  \textsc{nabs}  all  as/in.the.role.of  \textsc{i.r}-\textsc{caus}-thanks  1\textsc{s.abs} \textsc{nabs}  \textsc{pl}  teacher
\textsc{nabs}  high  school  \textsc{nabs}  \textsc{nr.act}-instruct  \textsc{nabs}  \textsc{nr.act}-work-\textsc{t.ir}
\textsc{nabs}  stage  \textsc{lk}  \textsc{emph.d}1\textsc{abs}  now/today  \textsc{a.hap.r}-hold/present-\textsc{apl}
\textsc{nabs}  1\textsc{p.incl.gen}  \textsc{lk}  program \\
\glt `First of all it is like I give thanks to the high school teachers \textbf{for} \textbf{instructing} (us) \textbf{in} \textbf{making} \textbf{the} \textbf{stage} which this now is where we held our program.’ [FBOE-C-01 2.1]
\z
\ea
Yaken  anduni  nagatabang  ki  Maam  \textbf{ta}  \textbf{pag-ubra} ta  ate  na  ambaļ  Kagayanen. \smallskip\\
\gll Yaken  anduni  naga-tabang\footnotemark{}  ki  Maam  \textbf{ta}  \textbf{pag-ubra} ta  ate  na  ambaļ  Kagayanen. \\
1\textsc{s.abs}  now/today  \textsc{i.r}-help  \textsc{obl.p}  Ma’am  \textsc{nabs}  \textsc{nr.act}-work/do
\textsc{nabs}  1\textsc{p.incl.gen}  \textsc{lk}  say  Kagayanen \\
\footnotetext{The use of the prefix \textit{naga}{}- is code switching from \isi{Hiligaynon}.}
\glt `As for me now, (I) am helping Ma’am \textbf{work} on our language Kagayanen.’ [JCWB-T-12 11.2]
\z
\ea
Siguro  sake  ka  taan  \textbf{ta}  \textbf{pag-ubra}. \smallskip\\
\gll Siguro  sake  ka  taan  \textbf{ta}  \textbf{pag-ubra}. \\
perhaps  busy  2\textsc{s.abs}  maybe  \textsc{nabs}  \textsc{nr.act}-work \\
\glt ‘Perhaps you maybe are busy \textbf{working}.’ [PBWL-C-03 2.3]
\z
\ea
Sadya  kaw  taan  \textbf{ta}  \textbf{pagtranslate}   \textbf{ta}  \textbf{libro}  \textbf{na}  \textbf{Genesis}. \smallskip\\
\gll Sadya  kaw  taan  \textbf{ta}  \textbf{pag-translate}   \textbf{ta}  \textbf{libro}  \textbf{na}  \textbf{Genesis}. \\
happy  2\textsc{p.abs}  maybe  \textsc{nabs}  \textsc{nr.act}-translate  \textsc{nabs}  book  \textsc{lk}  Genesis \\
\glt ‘You are maybe happy \textbf{translating} \textbf{the} \textbf{book} \textbf{of} \textbf{Genesis}.’ [PBWL-C-03 2.4]
\z
\ea
Dili  kaw  magtamad  tak  daw  tamad  ki \textbf{ta}  \textbf{pag-ubra},  uļa  ki  kan-enen. \smallskip\\
\gll Dili  kaw  mag-tamad  tak  daw  tamad  ki \textbf{ta}  \textbf{pag-ubra},  uļa  ki  kan-en-en. \\
\textsc{neg.ir}  2\textsc{p.abs}  \textsc{i.ir}-lazy  because  if/when  lazy  1\textsc{p.incl.abs}
\textsc{nabs}  \textsc{nr.act}-work/do  \textsc{neg.r}  1\textsc{p.incl.abs}  cooked.rice-\textsc{t.ir} \\
\glt ‘Do not be lazy because if we are lazy \textbf{in} \textbf{working}, we will have nothing to eat’ [RZWE-J-01 15.3]
\z
\ea
Ta  pitsa  1974,  yaken  nakakuyog  \textbf{ta}  \textbf{pagpangawil} ta  pailing  na  matambakoļ. \smallskip\\
\gll Ta  pitsa  1974,  yaken  naka-kuyog  \textbf{ta}  \textbf{pag-pang-kawil} ta  pa-iling  na  matambakoļ. \\
\textsc{nabs}  date  1974  1\textsc{s.abs}  \textsc{i.hap.r}-go.with  \textsc{nabs}  \textsc{nr.act}-\textsc{pl}-fishhook
\textsc{nabs}  \textsc{t.r}-say  \textsc{lk}  skipjack.tuna \\
\glt ‘On the date of 1974, as for me (I) went with (others) \textbf{hook} \textbf{and} \textbf{line} \textbf{fishing} for what is called skipjack tuna.’ [MCWN-L-01 2.2]
\z
\ea
… kanen  i  nagapakuntis  \textbf{ta}  \textbf{pagsuļat  ta  Kagayanen}. \smallskip\\
\gll … kanen  i  naga-pa-kuntis  \textbf{ta}  \textbf{pag-suļat}  \textbf{ta}  \textbf{Kagayanen}. \\
{} 3\textsc{s.abs}  \textsc{def.n}  \textsc{i.r}-\textsc{caus}-contest  \textsc{nabs}  \textsc{nr.act}-write  \textsc{nabs}  Kagayanen \\
\glt ‘… as for her (she) held a contest \textbf{for} \textbf{writing} \textbf{Kagayanen}.’ [SOWN-L-01 1.29]
\z
\ea
Maļaman  ta  surano  daw  may  sabid isya  na  ittaw  paagi  ta  luy-a  na  pagamit  din  \textbf{ta} \textbf{pagteyep}  tak  iya  na  pasuppa. \smallskip\\
\gll Ma-aļam-an  ta  surano  daw  may  sabid isya  na  ittaw  paagi  ta  luy-a  na  pa-gamit  din  \textbf{ta} \textbf{pag-teyep}  tak  iya  na  pa-suppa. \\
\textsc{a.hap.ir}-know-\textsc{apl}  \textsc{nabs}  shaman  if/when  \textsc{ext.in}  sickness.from.spirit
one  \textsc{lk}  person  by.means  \textsc{nabs}  ginger  \textsc{lk}  \textsc{t.r}-use  3\textsc{s.erg}  \textsc{nabs}
\textsc{nr.act}-blow.on  because  3\textsc{s.gen}  \textsc{lk}  \textsc{t.r}-chew \\
\glt `The shaman knows if a person has a sickness from a spirit by means of ginger that he uses \textbf{in} \textbf{blowing} \textbf{on} (the person) because (it is what) he chews on.’ [CBWE-T-07 3.9]
\z
\ea
Anen  dyan  na  adlaw  na  magaluto  ka  ta  lana  daw apog  para  gamiten  \textbf{ta}  \textbf{pag-ubra}  \textbf{na}  \textbf{imuon} \textbf{na}  \textbf{siminto} … \smallskip\\
\gll Anen  dyan  na  adlaw  na  maga-luto  ka  ta  lana  daw apog  para  gamiten  \textbf{ta}  \textbf{pag-ubra}  \textbf{na}  \textbf{imuon} \textbf{na}  \textbf{siminto} … \\
\textsc{ext.g}  \textsc{d}2\textsc{loc}  \textsc{lk}  day/sun  \textsc{lk}  \textsc{i.ir}-cook  2\textsc{s.abs}  \textsc{nabs}  coconut.oil  and lime  for  use-\textsc{t.ir}  \textsc{nabs}  \textsc{nr.act}-work/do  \textsc{lk}  make/do-\textsc{t.ir}
\textsc{lk}  cement \\
\glt `There were days that you would cook coconut oil and lime in order to use \textbf{for the work of} making cement. ‘ [MEWN-T-03 2.4]
\z
\ea
… bata  pa  yan  na  uļo  na  anda  pa  \textbf{ta}  \textbf{pagbaton} \textbf{ta}  \textbf{mga}  \textbf{tudlo}. \smallskip\\
\gll … bata  pa  yan  na  uļo  na  anda  pa  \textbf{ta}  \textbf{pag-baton} \textbf{ta}  \textbf{mga}  \textbf{tudlo}. \\
{} young  \textsc{inc}  \textsc{d2adj}  \textsc{lk}  head  \textsc{lk}  ready  \textsc{inc}  \textsc{nabs}  \textsc{nr.act}-receive
\textsc{nabs}  \textsc{pl}  teach \\
\glt ‘.. that head is still young ready still \textbf{to} \textbf{receive} things that are taught.’ [JCWB-T-12 4.4]
\z
\ea
Dili  ka  magmatamad  \textbf{ta}    \textbf{pag-asikaso}  \textbf{ta}  \textbf{sarili}, mga  gamit  no,  \textbf{pag-imes}  ta  mga  dapat  imesen … \smallskip\\
\gll Dili  ka  mag-ma-tamad  \textbf{ta}    \textbf{pag-asikaso}  \textbf{ta}  \textbf{sarili}\footnotemark{}, mga  gamit  no,  \textbf{pag-imes}  ta  mga  dapat  imes-en … \\
\textsc{neg.ir}  2\textsc{s.abs}  \textsc{i.ir}-\textsc{adj-}lazy    \textsc{nabs}  \textsc{nr.act}-take.care.of  \textsc{nabs}  self
\textsc{pl}  use  2\textsc{s.gen}  \textsc{nr.act}-prepare  \textsc{nabs}  \textsc{pl}  must  prepare-\textsc{t.ir} \\
\footnotetext{This word is code switching from \isi{Tagalog}. The Kagayanen word here would be \textit{kaugalingon}, which itself is a Bisayan loan word.}
\glt `Do not become lazy in \textbf{taking} \textbf{care} of yourself, your things, \textbf{preparing} what you must prepare …’ [JCWB-T-12 4.5]
\z
\ea
Lugar  en  na  naan  ta  batangan  ta  Mambeng Cave,  Maria  i nali  nang  na  pag-uļog  \textbf{ta}  \textbf{iya}  \textbf{na}  \textbf{pagpanimbang} na  dili  man  kon  pelles  angin  an. \smallskip\\
\gll Lugar  en  na  naan  ta  batang-an  ta  Mambeng Cave,  Maria  i nali  nang  na  pag-uļog  \textbf{ta}  \textbf{iya}  \textbf{na}  \textbf{pag-pang-timbang} na  dili  man  kon  pelles  angin  an. \\
then  \textsc{cm}  \textsc{lk}  \textsc{spat.def}  \textsc{nabs}  put-\textsc{apl}  \textsc{nabs}  Mambeng Cave  Maria  \textsc{def.n}
abruptly  only  \textsc{lk}  \textsc{nr.act}-fall  \textsc{nabs}  3\textsc{s.gen}  \textsc{lk}  \textsc{nr.act}-\textsc{pl}-balance
\textsc{lk}  \textsc{neg.ir}  also  \textsc{hsy}  strong.wind  wind  \textsc{def.m} \\
\glt `Then when we were in the location of Mambeng Cave, as for Maria (she) abruptly fell \textbf{while} \textbf{balancing} (the boat) when the winds were not strong.’ [EMWN-T-06 5.3]
\z
\ea
Naagian  ko,  naninaan  a  ta  sundang naan  ta  dagat  tak  gakuyog  \textbf{ta}  \textbf{pag-uļog  ta  pukot}. \smallskip\\
\gll Na-agi-an  ko,  na-nina-an  a  ta  sundang naan  ta  dagat  tak  ga-kuyog  \textbf{ta}  \textbf{pag-uļog}  \textbf{ta}  \textbf{pukot}. \\
\textsc{a.hap.r}-pass-\textsc{apl}  1\textsc{s.erg}  \textsc{a.hap.r}-wound-\textsc{apl}  1\textsc{s.abs}  \textsc{nabs}  machete
\textsc{spat.def}  \textsc{nabs}  sea  because  \textsc{i.r}-go.with  \textsc{nabs}  \textsc{nr.act}-fall  \textsc{nabs}  fishnet \\
\glt `(What) I experienced, I was wounded with a machete in the sea because I went with (others) \textbf{while} \textbf{lowering} \textbf{fishnets}.’ [ANWN-T-01 1.1]
\z

In example \REF{bkm:Ref474577376} the main verb \textit{buy-an} ‘let go of’ is in a transitive frame with a first person singular ergative argument and \textit{bata an} ‘the child’ as absolutive. The nominalized clause is an optional adjunct that is preceded by the marker \textit{ta}:

\ea
\label{bkm:Ref474577376}
Nabuy-an  ko  taan  bata  an  \textbf{ta}  \textbf{ame}  \textbf{na}  \textbf{pag-uļog}. \smallskip\\
\gll Na-buy-an  ko  taan  bata  an  \textbf{ta}  \textbf{ame}  \textbf{na}  \textbf{pag-uļog}. \\
\textsc{a.hap.r}-let.go.of  1\textsc{s.erg}  maybe  child  \textsc{def.m}  \textsc{nabs}  1\textsc{p.excl.gen}  \textsc{lk}  \textsc{nr.act}-fall \\
\glt ‘Maybe I let go of the child \textbf{as} \textbf{we} \textbf{fell}.’ [EDWN-T-05 2.13]
\z

The pre-nominal case marker \textit{ta} preceding the adjunct clause in \REF{bkm:Ref474577376} may be omitted, in which case the meaning is more general, allowing for a sequential as well as simultaneous interpretation:

\ea
Nabuy-an  ko  taan  bata  an  \textbf{pag-uļog  nay}. \smallskip\\
\gll Na-buy-an  ko  taan  bata  an  \textbf{pag-uļog}  \textbf{nay}. \\
\textsc{a.hap.r}-let.go.of  1\textsc{s.erg}  maybe  child  \textsc{def.m}  \textsc{nr.act}-fall  1\textsc{p.excl.gen} \\
\glt ‘I let go maybe of the child \textbf{when/after} \textbf{we} \textbf{fell}.’
\z

\hspace*{-.4pt}Sometimes the \textit{ta} marker cannot easily occur with nominalized adjunct clauses, depending on the context of the conversation. In example, in \REF{bkm:Ref474577631a} the adverbial \textit{pag}{}- clause expresses a simultaneous event. In this context the \textit{ta} preposition sounds very awkward \REF{bkm:Ref474577631b}:

\ea
    \ea[]{
    \label{bkm:Ref474577631a}
    Ta  sunod  na  adlaw  gakuyog  kay  man  \textbf{pagbalik}   tak  galinaw  pa. \\\smallskip
\gll Ta  sunod  na  adlaw  ga-kuyog  kay  man  \textbf{pag-balik}   tak  ga-linaw  pa. \\
    \textsc{nabs}  follow  \textsc{lk}  sun/day  \textsc{i.r}-go.with  1\textsc{p.excl.abs}  too  \textsc{nr.act}-return  because  \textsc{i.r}-calm  \textsc{inc} \\
    \glt `The next day we went with (them) \textbf{when} \textbf{returning} because (the weather was) becoming calm.’ [EDWN-T-05 2.5]
    }
    \ex[*]{
    \label{bkm:Ref474577631b}
    ?Ta sunod na adlaw gakuyog kay man \textbf{ta} pagbalik tak galinaw pa.
    }
    \z
\z

The reason this clause cannot easily be preceded by \textit{ta} (example \ref{bkm:Ref474577631b}) is because \textit{ta} would imply that going and returning were two distinct simultaneous events. This is about as awkward in Kagayanen as “We went while returning” is in English.

However, the verb \textit{kuyog} ‘go together’ can take a clause with \textit{ta pag-}VERB in other contexts. In \REF{bkm:Ref115166079}, the two events, ‘going with the jeep’ and ‘turning upside down’, are presented as distinct events that occurred at the same time, rather than two facets of one complex event.

\ea
\label{bkm:Ref115166079}
Kami  na  naan  ta  selled  ta  dyip  gakuyog man  \textbf{ta}  \textbf{pagbaliskad}.  \smallskip\\
\gll Kami  na  naan  ta  selled  ta  dyip  ga-kuyog man  \textbf{ta}  \textbf{pag-baliskad}.  \\
1\textsc{p.excl.abs}  \textsc{lk}  \textsc{spat.def}  \textsc{nabs}  inside  \textsc{nabs}  jeep  \textsc{i.r}-go.with
also  \textsc{nabs}  \textsc{nr.act}-upside.down/opposite \\
\glt `We who were inside the jeep went with (the jeep) \textbf{in} \textbf{turning} \textbf{upside} \textbf{down}.’ (In the context the jeep they were riding turned over.) [PMWN-T-02 2.11]
\z

The \textit{pag}{}- clause often occurs before the main clause specifying distinct but simultaneous events in a narrative, as in \REF{bkm:Ref478135574} and \REF{ex:acoconuttree}:

\ea
\label{bkm:Ref478135574}
\textbf{Pagbalik}  \textbf{nay}  gadaļa  yaken  i  isab. \smallskip\\
\gll \textbf{Pag-balik}  \textbf{nay}  ga-daļa  yaken  i  isab. \\
\textsc{nr.act}-return  1\textsc{p.excl.gen}  \textsc{i.r}-take/carry  1\textsc{s.abs}  \textsc{def.n}  again \\
\glt ‘\textbf{When we returned}, I again was carrying (some harvested corn).’ [DBON-C-08 2.3]
\z
\ea
\label{ex:acoconuttree}
\textbf{Ta}  \textbf{pagpanaw}  \textbf{ko}  may  nakita  a  na  tallo na  kabataan  na  gakatay  ta  niog. \smallskip\\
\gll \textbf{Ta}  \textbf{pag-panaw}  \textbf{ko}  may  na-kita  a  na  tallo na  ka-bata-an  na  ga-katay  ta  niog. \\
\textsc{nabs}  \textsc{nr.act}-walk/go  1\textsc{p.excl.gen}  \textsc{ext.in}  \textsc{a.hap.r}-see  1\textsc{s.abs}  \textsc{lk}  three
\textsc{lk}  \textsc{nr}-child-\textsc{nr}  \textsc{lk}  \textsc{i.r}-climb  \textsc{nabs}  coconut \\
\glt `While I was walking, I saw some three children climbing a coconut tree.’ [EFWN-T-11 15.4]
\z

When the adjunct \textit{pag}{}-clause expresses one part of the same event expressed in the main clause, the \textit{pag}{}-clause always follows the main clause. In example \REF{bkm:Ref115074170}, \textit{pagbanyos} `rubbing/massaging' is understood as the manner of using the coconut oil, rather than a separate event that occured at the same time or subsequent to the event of using coconut oil.

\ea
\label{bkm:Ref115074170}
Pagamit  din  lana  an  \textbf{ta}  \textbf{pagbanyos}. \smallskip\\
\gll Pa-gamit  din  lana  an  \textbf{ta}  \textbf{pag-banyos}. \\
\textsc{t.r}-use  3\textsc{s.erg}  coconut.oil  \textsc{def.m}  \textsc{nabs}  \textsc{nr.act}-rub/massage.on \\
\glt ‘S/he used the coconut oil \textbf{in} \textbf{rubbing} \textbf{(on} \textbf{the} \textbf{body)}.’
\z


As illustrated in the earlier examples in this section, When the two events are different, the \textit{pag}{}-clause may occur sentence initially or sentence finally, but it is more often found sentence initially. In example \REF{bkm:Ref474575856} the meaning is ‘during his running’, or ‘while he was running.’ Such sentence-initial \textit{pag}{}- or \textit{ta pag}{}- clauses tend to describe repeated information already known to the readers or obviously implied. They are used to resume the main line in a narrative after a speech event, digression or some background information. They are also used to slow down the narrative before exciting or climactic events.

\ea
\label{bkm:Ref474575856}
\textbf{Ta}  \textbf{iya}  \textbf{na}  \textbf{pagdļagan}  gabalikid  Pwikan  i  daw gasinggit,  “Umang  indi  ka  en?” \smallskip\\
\gll \textbf{Ta}  \textbf{iya}  \textbf{na}  \textbf{pag-dļagan}  ga-balikid  Pwikan  i  daw ga-singgit,  “Umang  indi  ka  en?” \\
\textsc{nabs}  3\textsc{s.gen}  \textsc{lk}  \textsc{nr.act}-run  \textsc{i.r}-look.back  sea.turtle  \textsc{def.n}  and
\textsc{i.r}-shout  Hermit.crab  where  2\textsc{s.abs}  \textsc{cm} \\
\glt `During his running Sea Turtle looked back and shouted, “Hermit Crab where are you?”‘ [DBWN-T-26 8.3]
\z

When a \textit{pag-} clause precedes the main clause, often the non-absolutive marker \textit{ta} is omitted. When \textit{ta} is present, the meaning tends to be simultaneous, translated as ‘during’ or ‘while’ \REF{bkm:Ref474575856}. When \textit{ta} is omitted the meaning tends to be more sequential, appropriately translated as ‘when’ or ‘after’ (\ref{bkm:Ref474575743}-\ref{bkm:Ref474575734}):

\ea
\label{bkm:Ref474575743}
\textbf{Pag-abot}  nay  ta  suba,  gatan-aw  kay  daw  indi miad  na  agian  nay. \smallskip\\
\gll \textbf{Pag-abot}  nay  ta  suba,  ga-tan-aw  kay  daw  indi miad  na  agi-an  nay. \\
\textsc{nr.act}-arrive  1\textsc{p.excl.gen}  \textsc{nabs}  river  \textsc{i.r}-look  1\textsc{p.excl.abs}  if/when  where
good  \textsc{lk}  pass-\textsc{apl}  1\textsc{p.excl.erg}. \\
\glt `When we arrived at the river, we looked for where would be a good way for us (to go).’ [BGON-L-01 1.7]
\z

\ea
\textbf{Pag-abot}  \textbf{ta}  \textbf{iran}  \textbf{na}  \textbf{baļay}  gabasuļay  en  danen  ya. \smallskip\\
\gll \textbf{Pag-abot}  \textbf{ta}  \textbf{iran}  \textbf{na}  \textbf{baļay}  ga-basoļ-ay  en  danen  ya. \\
\textsc{nr.act}-arrive  \textsc{nabs}  3\textsc{p.gen}  \textsc{lk}  house  \textsc{i.r}-scold-\textsc{rec}  \textsc{cm}  3\textsc{p.abs}  \textsc{def.f} \\
\glt ‘After arriving at their house, they scolded each other.’ [YBWN-T-01 2.20]
\z
\ea
\label{bkm:Ref474575734}
\textbf{Pagsangga}  \textbf{ta}  \textbf{baked}  \textbf{ya}  \textbf{na}  \textbf{manunggo}ļ,  aroy,  sikad  batyag din  na  sakit  naan  ta  takong  din  an. \smallskip\\
\gll \textbf{Pag-sangga}  \textbf{ta}  \textbf{baked}  \textbf{ya}  \textbf{na}  \textbf{manunggoļ},  aroy,  sikad  batyag din  na  sakit  naan  ta  takong  din  an. \\
\textsc{nr.act}-bump  \textsc{nabs}  big  \textsc{def.f}  \textsc{lk}  limestone  ouch  very  feel
3\textsc{s.erg}  \textsc{lk}  pain  \textsc{spat.def}  \textsc{nabs}  forehead  3\textsc{s.gen}  \textsc{def.m} \\
\glt `After banging into the big limestone formation, ouch, he really felt pain on his forehead.’ [JCON-L-07 4.5]
\z

An adverbial \textit{pag}{}-clause following the main clause may express a reason for the situation expressed in the main clause, with or without the introducer \textit{tenged} ‘because’:

\ea
May  mga  duma  na  mga  Kagayanen  unduni  na  nagamanggad daw  may  mga  darko  na  mga  pambot  \textbf{tenged}  \textbf{nang}  \textbf{ta} \textbf{pagtanem}  \textbf{ta}  \textbf{guso}. \smallskip\\
\gll May  mga  duma  na  mga  Kagayanen  unduni  na  naga-manggad daw  may  mga  darko  na  mga  pambot  \textbf{tenged}  \textbf{nang}  \textbf{ta} \textbf{pag-tanem}  \textbf{ta}  \textbf{guso}. \\
\textsc{ext.in}  \textsc{pl}  some  \textsc{lk}  \textsc{pl}  Kagayanen  now/today  \textsc{lk}  \textsc{i.r}-wealth and  \textsc{ext.in}  \textsc{pl}  big.\textsc{pl}  \textsc{lk}  \textsc{pl}  motorboat  because  only  \textsc{nabs} \textsc{nr.act}-plant  \textsc{nabs}  seaweed \\
\glt ‘There are some  Kagayanens now/today who are becoming rich and there are also large motorboats only \textbf{because of planting agar seaweed}.’ [VPWL-T-03 5.2]
\z

\ea
Gamanggaranen  danen  an  \textbf{ta}  \textbf{pagtanem}  \textbf{ta}  \textbf{guso}. \smallskip\\
\gll Ga-manggaranen  danen  an  \textbf{ta}  \textbf{pag-tanem}  \textbf{ta}  \textbf{guso}. \\
\textsc{i.r}-rich  3\textsc{p.abs}  \textsc{def.m}  \textsc{nabs}  \textsc{nr.act}-plant  \textsc{nabs}  seaweed. \\
\glt ‘They are becoming rich \textbf{(because) of planting agar seaweed}.’
\z

With the complex preposition \textit{para ta} a \textit{pag-}verb, may express purpose:

\ea
Isya  pa,  uļa  naan  ta  lugar  danen  parias  pandan  o  buli tak  yon  gamit  \textbf{para}  \textbf{ta}  \textbf{pag-ubra}  \textbf{ta}  \textbf{ikam}. \smallskip\\
\gll Isya  pa,  uļa  naan  ta  lugar  danen  parias  pandan  o  buli tak  yon  gamit  \textbf{para}  \textbf{ta}  \textbf{pag-ubra}  \textbf{ta}  \textbf{ikam}. \\
one  \textsc{inc}  \textsc{neg.r}  \textsc{spat.def}  \textsc{nabs}  place  3\textsc{p.gen}  similar  pandan  or  buri because  \textsc{d3abs}  use  for  \textsc{nabs}  \textsc{nr.act}-work/do  \textsc{nabs}  mat \\
\glt ‘Another thing, there is nothing in their place like \textit{pandan} or \textit{buli} because that is used \textbf{for} \textbf{making} \textbf{mats} [BCWE-T-09 2.3]
\z

With the complex preposition \textit{paagi ta}, a \textit{pag}{}-verb may express means:

\ea
Kanen  i  giling  ta  yi  na  puļo  \textbf{paagi}  \textbf{ta}  \textbf{pagsakay} \textbf{ta}  \textbf{lunday}  para  manglaya. \smallskip\\
\gll Kanen  i  ga-iling  ta  yi  na  puļo  \textbf{paagi}  \textbf{ta}  \textbf{pag-sakay} \textbf{ta}  \textbf{lunday}  para  ma-ng-laya. \\
3\textsc{s.abs}  \textsc{def.n}  \textsc{i.r}-go  \textsc{nabs}  \textsc{d1adj}  \textsc{lk}  island  by.means  \textsc{nabs}  \textsc{nr.act}-ride
\textsc{nabs}  outrigger.canoe  for  \textsc{a.hap.ir}-\textsc{pl}-cast.net \\
\glt `As for him (he) came to this island \textbf{by} \textbf{means} \textbf{of} \textbf{riding} \textbf{an} \textbf{outrigger} \textbf{canoe} in order to fish with cast nets.’ [VAWN-T-17 2.2]
\z

With the complex preposition \textit{parti ta}, a \textit{pag}{}-verb may express the notion of ‘about’:

\ea
Wi  nang  ake  na  isturya  na  dili  ko  malipatan \textbf{parti ta pagpangita}  \textbf{ta}  \textbf{ake}  \textbf{na}  \textbf{pangabui}. \smallskip\\
\gll Wi  nang  ake  na  isturya  na  dili  ko  ma-lipat-an \textbf{parti}  \textbf{ta}  \textbf{pag-pangita}  \textbf{ta}  \textbf{ake}  \textbf{na}  \textbf{pangabui}. \\
\textsc{emph.d}1\textsc{abs}  only  1\textsc{s.gen}  \textsc{lk}  story  \textsc{lk}  \textsc{neg.ir}  1\textsc{s.erg}  \textsc{a.hap.ir}-forget-\textsc{apl}
about  \textsc{nabs}  \textsc{nr.act}-search  \textsc{nabs}  1\textsc{s.gen}  \textsc{lk}  livelihood \\
\glt `This is my story which I can’t forget \textbf{about} \textbf{looking} \textbf{for} \textbf{my} \textbf{livelihood}.’ [MCWN-L-01 2.22]
\z

The complex preposition \textit{tenged ta} may also express this ‘aboutness’ relation, as well as reason.

\ea
Isturya  ko  i  \textbf{tenged}  \textbf{ta}  \textbf{pagbirthday}  \textbf{naan}  \textbf{kan-o}  \textbf{Bario}. \smallskip\\
\gll Isturya  ko  i  \textbf{tenged}  \textbf{ta}  \textbf{pag-birthday}  \textbf{naan}  \textbf{kan-o}  \textbf{Bario}. \\
story  1\textsc{s.gen}  \textsc{def.n}  about  \textsc{nabs}  \textsc{nr.act}-birthday  \textsc{spat.def}  previously  Bario \\
\glt ‘My story is \textbf{about} \textbf{having} \textbf{a} \textbf{birthday} \textbf{in} \textbf{Bario} \textbf{previously}.’ [DBON-C-09 1.1]
\is{adverbial clauses|)}
\is{dependent clauses!nominalization|)}
\is{nominalization|)}
\z
\section{Subjunctive clauses}
\label{bkm:Ref477523359} \label{sec:subjunctiveclauses}
\is{subjunctive clauses|(}
\is{dependent clauses!subjunctive|(}
Certain clause combinations involve a dependent clause that must be in irrealis modality. We will term the irrealis clause or clauses in such constructions \textit{subjunctive clauses}\is{}, since the irrealis marking is conditioned by the \textit{construction} and not necessarily by any inherent irreality of the event expressed (see \citealt{givon1994} for the commonality between irrealis modality and what have been called ``subjunctive" forms in many language traditions). In other words, subjunctive clauses may express real events, but they must be marked as irrealis. We consider subjunctive clauses to be semi-finite, since they do retain the transitive/intransitive distinction, and the dynamic/happenstantial distinction. The realis/irrealis distinction, however, is neutralized in favor of irrealis marking (see \chapref{chap:verbstructure}, \sectref{sec:verbinflection} for a discussion of transitivity and modality as inflectional categories in Kagayanen verbs). In this section we describe subjunctive clauses functioning as complements (\sectref{sec:subjunctivecomplementclauses}) and those functioning as adverbial clauses (\sectref{sec:subjunctiveadverbialclauses}).

The linker \textit{na} is the most common form that introduces subjunctive adverbial clauses. Two other introducers are fairly common. These are \textit{daw} ‘if/ when/ whether’, and \textit{aged} ‘purpose/result’. Less common introducers include \textit{para (na)} ‘purpose’, \textit{imbis (na)} ‘instead of’, and \textit{bag-o (na)} ‘before’. \textit{Daw} is an extremely common and useful clause introducer that precedes all kinds of dependent clauses, including nominalizations (\sectref{bkm:Ref115861741}), subjunctive adverbial clauses (this section) and fully finite adverbial clauses (\sectref{bkm:Ref460483264}). \textit{Daw} also serves as a common conjunction, joining RPs, predicates or clauses (\sectref{sec:daw}). We gloss \textit{daw} as ‘and’ when it functions as a conjunction, and ‘if/when’ when it functions as an adverbial clause introducer.
\subsection{Subjunctive complement clauses}
\label{sec:subjunctivecomplementclauses}
\is{subjunctive complement clauses|(}
\is{complement clauses!subjunctive|(}
Certain matrix verbs require subjunctive (irrealis) complements. These include the following:

\ea
\label{ex:listofctpstakingsubjunctivecomplements}
\begin{tabbing}
\hspace{2cm} \= \kill
\textit{adlek } \>  ‘be afraid to do X’ \\
\textit{anda  } \>  ‘be ready to do X’  \\
\textit{bawal } \>   ‘forbid someone to do X’  \\
\textit{dapat  } \>  ‘must do X’ \\
\textit{disidido } \>   ‘be keen/eager to do X’ \\
\textit{disisyon } \>  ‘decide to do X’ \\
\textit{gusto  } \>  ‘like/want to do X’ \\
\textit{istudyo } \>  ‘study X’ \\
\textit{kinangļan } \>   ‘be necessary to do X’ \\
\textit{leges  } \>  ‘coerce someone to do X’ \\
\textit{liag } \>  ‘like/want to do X’ \\
\textit{miyag  } \>  ‘agree/want to do X’ \\
\textit{padayon  } \>  ‘continue to do X’  \\
\textit{paliyog  } \>  ‘request to do X’  \\
\textit{pangabay } \>   ‘beg someone to do X’  \\
\textit{pasinayen } \>  ‘make someone used to doing X’  \\
\textit{plano } \>  ‘plan to do X’ \\
\textit{prusigir } \>  ‘persevere in doing X’  \\
\textit{pwirsa } \>   ‘force someone to do X’ \\
\textit{sagad } \>  ‘be skillful in doing X’ \\
\textit{sigurado } \>  ‘make sure that X’ \\
\textit{sugo  } \>  ‘order someone to do X’  \\
\textit{tabang } \>  ‘help someone to do X’ \\
\textit{tingwa } \>  ‘try hard to do X’ \\
\textit{tudlo  } \>  ‘teach X’ \\
\textit{tugot  } \>  ‘permitted/allowed to do X’
\end{tabbing}
\z

Subjunctive complement clauses normally follow the linker \textit{na} functioning as a complementizer (see examples and discussion below). However, they absolutely may not be preceded by the non-absolutive prenominal case marker \textit{ta}. In other words, they are not ergative or oblique complements. In the examples in \REF{bkm:Ref474418733}, a subjunctive clause follows the stative root \textit{adlek} ‘be afraid’, and describes the source of the speaker’s fear. Example \REF{bkm:Ref474418733a} is in the basic intransitive, irrealis form, while \REF{bkm:Ref474418733b} is in the marked necessarily volitional form. The difference in meaning expressed by these two prefixes is reflected in the free translations. Finally, \REF{bkm:Ref474418733c} illustrates that the verb following \textit{na} may not occur in the realis form, and thus is not fully finite:

\newpage
\ea
\label{bkm:Ref474418733}
    \ea
    \label{bkm:Ref474418733a}
    Adlek  a  \textbf{na}  \textbf{magluoy}  \textbf{ta}  \textbf{dagat}. \\\smallskip
\gll Adlek  a  \textbf{na}  \textbf{mag-luoy}  \textbf{ta}  \textbf{dagat}. \\
    fear  1\textsc{s.abs}  \textsc{lk}  \textsc{i.ir}-swim  \textsc{nabs}  sea \\
    \glt ‘I am afraid whenever swimming in the sea.’ (Any time.)
    \ex
    \label{bkm:Ref474418733b}
    Adlek  a  \textbf{na}  \textbf{muoy}  \textbf{ta}  \textbf{dagat}. \\\smallskip
\gll Adlek  a  \textbf{na}  \textbf{m-luoy}  \textbf{ta}  \textbf{dagat}. \\
    fear  1\textsc{s.abs}  \textsc{lk}  \textsc{i.v.ir}-swim  \textsc{nabs}  sea \\
    \glt ‘I am afraid to swim in the sea.’ (Right now.)
    \ex
    \label{bkm:Ref474418733c}
    *Adlek  a  \textbf{na}  \textbf{galuoy}  \textbf{ta}  \textbf{dagat}. \\\smallskip
\gll *Adlek  a  \textbf{na}  \textbf{ga-luoy}  \textbf{ta}  \textbf{dagat}. \\
    fear  1\textsc{s.abs  lk}  \textsc{i.r}-swim  \textsc{nabs}  sea \\
    \glt ('I'm afraid that I swim in the sea.')
    \z
\z

It is significant that these verbs do not take \textit{pag-} clause complements (see \sectref{bkm:Ref474473730} above). A few examples from the corpus follow:
\ea
Gani,  kiten  na  mga  Kagayanen  kinangļan  \textbf{na  magtabangay  ki daw  ino  ubra  para  dili  mabeg-atan  isya  an}. \smallskip\\
\gll Gani,  kiten  na  mga  Kagayanen  kinangļan  \textbf{na}  \textbf{mag-tabang-ay}  \textbf{ki} \textbf{daw}  \textbf{ino}  \textbf{ubra}  para  dili  ma-beg-at-an  isya  an. \\
so  1\textsc{p.incl.abs}  \textsc{lk}  \textsc{pl}  Kagayanen  necessary  \textsc{lk}  \textsc{i.ir}-help-\textsc{rec}  1\textsc{p.incl.abs}
if/when  what  work/do  for  \textsc{neg.ir}  \textsc{a.hap.r}-heavy-\textsc{apl}  one  \textsc{def.m} \\
\glt `So, we Kagayanens, it is necessary \textbf{that we help each other in whatever work} in order that one (of us) will not be burdened.’ [BCWL-T-12 4.1]
\z

\ea
… dapat  \textbf{na  isya-isya  kiten  mag-atag  kayaran  ta masigkaittaw  ta}. \smallskip\\
\gll … dapat  \textbf{na}  \textbf{isya\sim{}-isya}  \textbf{kiten}  \textbf{mag-atag}  \textbf{ka-ayad-an}  \textbf{ta} \textbf{masigka-ittaw}  \textbf{ta}. \\
{}    must  \textsc{lk}  \textsc{red}\sim{}one  1\textsc{p.incl.abs}  \textsc{i.ir}-give  \textsc{nr}-good/well-\textsc{nr}  \textsc{nabs}
fellow-person  1\textsc{p.incl.gen} \\
\glt `… it must be \textbf{that each one of us gives goodness to our fellow person}.’ [JCOB-L-02 10.4]
\z

\ea
Dapat  no  \textbf{demdemen  ni  ake  na  paglaygay  ki  kaon}. \smallskip\\
\gll Dapat  no  \textbf{demdem-en}  \textbf{ni}  \textbf{ake}  \textbf{na}  \textbf{paglaygay}  \textbf{ki}  \textbf{kaon}.\\
must  2\textsc{s.erg}  remember-\textsc{t.ir}  \textsc{d}1\textsc{abs}  1\textsc{s.gen}  \textsc{lk}  \textsc{nr.act}-advice  \textsc{obl.p}  2s\\
\glt ‘You must \textbf{remember this my advice to you}.’ [NEWL-T-03 3.6]
\z

\ea
Disidido  \textbf{kanen  na  melled  ta  ubra}. \smallskip\\
\gll Disidido  \textbf{kanen}  \textbf{na}  \textbf{m-selled}  \textbf{ta}  \textbf{ubra}. \\
keen  3\textsc{s.abs}  \textsc{lk}  \textsc{i.ir}-go.inside  \textsc{nabs}  work/do \\
\glt ‘S/he is keen \textbf{to get hired to work}.’
\z
\ea
… daw  anda  a  gid  en  \textbf{na  magpakasaļ  ki  kanen  ta  madali}. \smallskip\\
\gll … daw  anda  a  gid  en  \textbf{na}  \textbf{mag-pa-kasaļ}  \textbf{ki}  \textbf{kanen}  \textbf{ta}  \textbf{ma-dali}. \\
{} and  ready  1\textsc{s.abs}  \textsc{int}  \textsc{cm}  \textsc{lk}  \textsc{i.ir}-\textsc{caus}-wedding  \textsc{obl.p}  3s  \textsc{nabs}  \textsc{adj}-soon \\
\glt ‘… and I am ready \textbf{to get married to her soon}.’ [PBWN-C-12 17.1]
\z
\ea
Dili  matugot  ta  maaļ  na  Ginuo  \textbf{na  patayen  a}. \smallskip\\
\gll Dili  ma-tugot  ta  maaļ  na  Ginuo  \textbf{na}  \textbf{patay-en}  \textbf{a}. \\
\textsc{neg.ir}  \textsc{a.hap.ir}-allow  \textsc{nabs}  love  \textsc{lk}  Lord  \textsc{lk}  kill-\textsc{t.ir}  1\textsc{s.abs} \\
\glt ‘The Beloved Lord will not permit \textbf{me to be killed}.’ [BBON-C-06 4.4]
\z
\ea
Gatugot  kanen  \textbf{na  muyog  a  ki  kyo}. \smallskip\\
\gll Ga-tugot  kanen  \textbf{na}  \textbf{m-kuyog}  \textbf{a}  \textbf{ki}  \textbf{kyo}. \\
\textsc{i.r}-allow  3\textsc{s.abs}  \textsc{lk}  \textsc{i.v.ir}-come.with  1\textsc{s.abs}  \textsc{obl.p}  2p \\
\glt ‘S/he allowed me \textbf{to go with you}.’
\z
\ea
Nanay  i  daw  tatay  miyag  man  \textbf{na  muyog  a  ta  maistra  i}. \smallskip\\
\gll Nanay  i  daw  tatay  miyag  man  \textbf{na}  \textbf{m-kuyog}  \textbf{a}  \textbf{ta}  \textbf{maistra}  \textbf{i}. \\
mother  \textsc{def.n}  and  father   agree/want  also  \textsc{lk}  \textsc{i.v.ir}-go.with  1\textsc{s.abs}  \textsc{nabs}  teacher \textsc{def.n} \\
\glt ‘Mother and father also agreed that I go with the teacher.’ [DBWN-T-21 2.3]
\z
The ungrammatical example \REF{ex:nagangawil} illustrates that the matrix roots \textit{sagad} `be skillful' and \textit{aļam} `to understand/know' do not allow realis complements. This is true for all of the roots listed in \REF{ex:listofctpstakingsubjunctivecomplements}:

\ea
    \ea[]{
    Sagad  gid  kanen  \textbf{na  mangawil}. \\\smallskip
\gll Sagad  gid  kanen  \textbf{na  ma-ng-kawil}. \\
    skillful  \textsc{int}  3\textsc{s.abs}  \textsc{lk}  \textsc{a.hap.ir}-\textsc{pl}-fishhook \\
    \glt ‘S/he is really skillful \textbf{at hook and line fishing}.’
    }
    \ex[]{
    Naļam  gid  kanen  \textbf{na  mangawil}. \\\smallskip
\gll Na-aļam  gid  kanen  \textbf{na}  \textbf{ma-ng-kawil}. \\
    \textsc{a.hap.r}-know  \textsc{int}  3\textsc{s.abs}  \textsc{lk}  \textsc{a.hap.ir}-\textsc{pl}-fishhook \\
    \glt ‘S/he really knows how \textbf{to hook-and-line fish}.’
    }
    \ex[*]{
    \label{ex:nagangawil}
    Sagad/Naļam gid kanen \textbf{na} \textbf{gangawil}
    }
    \z
\z

There is sometimes evidence that subjunctive complements are absolutive arguments of a matrix verb. In example \REF{ex:notdriftoff} the root \textit{sigurado} ‘to be certain’ is grammatically transitive, occurring in a transitive form and expressing the Actor in the ergative case (see \chapref{chap:verbstructure}, \sectref{sec:grammaticaltransitivity}). Therefore the following subjunctive clause, \textit{na pambot i dili maanod}, is the absolutive:

\ea
\label{ex:notdriftoff}
Pasigurado ko gid \textbf{na pamboat i dili maanod}. \\
\gll pa-sigurado	ko	gid	\textbf{na}	\textbf{pambot}		\textbf{i}	\textbf{dili}	\textbf{ma-anod}. \\
\textsc{t.r}-certain	1\textsc{s.erg}	\textsc{int}	\textsc{lk}	motor.boat	\textsc{def.n}	\textsc{neg.ir}	\textsc{a.hap.ir}-drift \\
\glt ‘I made it certain that the motor boat will not drift off.’
\z

Example \REF{ex:iwasaiming} is a similar example from the corpus:

\ea
\label{ex:iwasaiming}
Pasiguro ko gid na migo ta pagbira ko. \\
\gll Pa-siguro	ko	gid	na	ma-igo	ta	pag-bira	ko. \\
\textsc{t.r}-sure	1\textsc{s.erg}	\textsc{int}	\textsc{lk}	\textsc{a.hap.ir}-hit	\textsc{nabs}	\textsc{nr.act}-aim	1\textsc{s.gen} \\
\glt ‘I really made it sure that it will hit when I was aiming.’ (The speaker was shooting at wild pig with a spear.) [RCON-L-01 3.15]
\z

Similarly, in example \REF{bkm:Ref474831142}, the verb \textit{tilawan} ‘to try’ is grammatically transitive, occurring in a transitive form and expressing the Actor in the ergative case. Therefore the following subjunctive clause, \textit{magpaļam . . .}, is the absolutive (the clause introduced by \textit{daw} ‘if/when’ is an adverbial clause as discussed in the following section):

\ea
\label{bkm:Ref474831142}
Tilawan  \textbf{no  magpaļam}  daw  miyag  kanen  an na  muyog   ka. \smallskip\\
\gll \emptyset{}-Tilaw-an  \textbf{no}  \textbf{mag-pa-aļam}  daw  m-iyag  kanen  an na  m-kuyog   ka. \\
\textsc{t.ir}-try-\textsc{apl}  2\textsc{s.erg}  \textsc{i.ir}-\textsc{caus}-know  if/when  \textsc{i.ir}-want/agree  3\textsc{s.abs}  \textsc{def.m} \textsc{lk}  \textsc{i.v.ir}-go.with  2\textsc{s.abs} \\
\glt `Try \textbf{to ask permission} whether s/he will agree/allow you to go with (them).’
\z

Example \REF{bkm:Ref474832038} illustrates another subjunctive clause filling the absolutive role in the main clause. The verb \textit{gusto} is in a transitive case frame with the Actor, \textit{mga sinakepan}, ‘the subjects’, marked as ergative with the non-absolutive prenominal marker \textit{ta}. This leaves the following subjunctive clause, introduced with \textit{na}, as the only reasonable candidate for absolutive status:

\ea
\label{bkm:Ref474832038}
\textbf{Gusto}  ta  mga  sinakepan  \textbf{na}  \textbf{malibri}  en  danen  ta pagdumaļa  na  mapintas. \smallskip\\
\gll \textbf{Gusto}  ta  mga  s<in>akep-an  \textbf{na}  \textbf{ma-libri}  en  danen  ta pag-dumaļa  na  ma-pintas. \\
want  \textsc{nabs}  \textsc{pl}  <\textsc{nr.res}>subjects-\textsc{apl}  \textsc{lk}  \textsc{a.hap.ir}-free  \textsc{cm}  3\textsc{p.abs}  \textsc{nabs}
\textsc{nr.act}-rule  \textsc{lk}  \textsc{adj}-cruel \\
\glt `The subjects wanted to be free from the cruel ruling.’ [JCWN-T-20 16.2]
\z

Manipulation verbs such as \textit{pwirsa} ‘force’, \textit{leges} ‘coerce’, \textit{sugo} ‘order’, and \textit{bawal} ‘forbid’ are followed by subjunctive clauses, though there is no evidence that the dependent clauses are syntactic arguments. Rather the participant manipulated (the “causee”) appears in the absolutive role, and the complement follows, preceded by \textit{na}:

\ea
Kaysan  \textbf{pwirsa}  kay  na  mutang  ta  duma  bag-o pautang  en  man  ki  danen. \smallskip\\
\gll Kaysan  \textbf{pwirsa}  kay  na  m-utang  ta  duma  bag-o pa-utang  en  man  ki  danen. \\
sometimes  force  1\textsc{p.excl.abs}  \textsc{lk}  \textsc{i.v.ir}-borrow.money  \textsc{nabs}  other  before
\textsc{caus}-borrow.money  \textsc{cm}  too  \textsc{obl.p}  3p \\
\glt `Sometimes we were \textbf{forced} to borrow money from others before lending to them too.’ [TTOB-L-03 6.1]
\z

\ea
Isya  na  adlaw  kanen  \textbf{pasugo}  ta  iya  na  nanay  \textbf{na}  \textbf{mandok} \textbf{ta}  \textbf{waig}. \smallskip\\
\gll Isya  na  adlaw  kanen  \textbf{pa-sugo}  ta  iya  na  nanay  \textbf{na}  \textbf{m-sandok} \textbf{ta}  \textbf{waig}. \\
one  \textsc{lk}  day/sun  3\textsc{s.abs}  \textsc{t.r}-order  \textsc{nabs}  3\textsc{s.gen}  \textsc{lk}  mother  \textsc{lk}  \textsc{i.v.r}-carry.water \textsc{nabs}  water \\
\glt `One day as for her, her mother \textbf{ordered} her to carry water.’ [VAWN-T-20 3.1]
\z
\ea
Piro  anduni  \textbf{pabawalan}  en  na  magdakep  ta  pawikan tak  kanen  i  makatabang  man  ta  mga  ittaw. \smallskip\\
\gll Piro  anduni  \textbf{pa-bawal-an}  en  na  mag-dakep  ta  pawikan tak  kanen  i  maka-tabang  man  ta  mga  ittaw. \\
but  now/today  \textsc{t.r}-forbid-\textsc{apl}  \textsc{cm}  \textsc{lk}  \textsc{i.ir}-catch  \textsc{nabs}  sea.turtle
because  3\textsc{s.abs}  \textsc{def.n}  \textsc{i.hap.ir}-help  too  \textsc{nabs}  \textsc{pl}  person \\
\glt `But now (the government) \textbf{forbids} (us) to catch a sea turtle because s/he can help people too.’ [YBWE-T-05 2.6]
\z

The verb \textit{ambaļ} ‘to say’ can be used in this construction as a manipulation verb:

\ea
\textbf{Pambaļan}  a  ta  mga  manakem  na  dili  a  kon  maggwa tak  dilikado  kon  ta  kasaļen  na  sigi  panaw. \smallskip\\
\gll \textbf{Pa-ambaļ-an}  a  ta  mga  manakem  na  dili  a  kon  mag-gwa tak  dilikado  kon  ta  kasaļ-en  na  sigi  panaw. \\
\textsc{t.r}-say-\textsc{apl}  1\textsc{s.abs}  \textsc{nabs}  \textsc{pl}  older  \textsc{lk}  \textsc{neg.ir}  1\textsc{s.abs}  \textsc{hsy}  \textsc{i.ir}-go.out
because  dangerous  \textsc{hsy}  \textsc{nabs}  wedding-\textsc{nr}  \textsc{lk}  continuously  go/walk \\
\glt `The older ones \textbf{told/warned} me that I (should) not go out they said because (it is) dangerous they said for one to be married to keep going somewhere.’ [VAWN-T-16 2.7]
\z

Example \REF{bkm:Ref474854233} illustrates a transitive subjunctive complement clause in which none of the arguments of the dependent clause are coreferential with anything in the matrix clause. Though it is very common for the two clauses in a main+sub\-junc\-tive clause construction to share a referent, this example illustrates that such coreference is not a matter of syntactic “control”:

\ea
\label{bkm:Ref474854233}
Miyag  a  man  yaken  i  \textbf{na}  \textbf{duon}  no  bai  i. \smallskip\\
\gll Miyag  a  man  yaken  i  \textbf{na}  \textbf{daļa-en}  no  bai  i. \\
agree/want  1\textsc{s.abs}  also  1\textsc{s.abs}  \textsc{def.n}  \textsc{lk}  carry/take-\textsc{t.ir}  2\textsc{s.erg}  woman  \textsc{def.n} \\
\glt ‘I myself want/agree also that you \textbf{take} the woman.’ [PBON-T-01 4.17]
\z

Sometimes a subjunctive clause introduced by \textit{daw} fills an absolutive function within a matrix clause, thus qualifying as a complement clause. Example \REF{bkm:Ref115872660} illustrates the stative root \textit{sigurado} ‘be certain’ followed by a subjunctive complement clause:

\ea
\label{bkm:Ref115872660}
Dili  ko  sigurado  \textbf{daw}  \textbf{muyog}  a  kisyem. \smallskip\\
\gll Dili  ko  sigurado  \textbf{daw}  \textbf{m-kuyog}  a  kisyem. \\
\textsc{neg.ir}  1\textsc{s.erg}  sure  if/when  \textsc{i.v.ir}-go.with  1\textsc{s.abs}  tomorrow \\
\glt ‘I am not sure \textbf{whether} \textbf{I} \textbf{will} \textbf{come} \textbf{with} (others) tomorrow.’
\z

In examples \REF{bkm:Ref115094853} and \REF{bkm:Ref115873391} the actors are expressed in the ergative role, and the main clause verbs are marked as transitive, thus leaving the clause introduced by \textit{daw} as the absolutive argument. The \textit{daw} clauses in these examples are also required to complete the meaning of the matrix predicate.

\ea
\label{bkm:Ref115094853}
... daw  painsaan  nay  \textbf{daw}  \textbf{magdayon}  kay  pa. \smallskip\\
\gll ... daw  pa-insa-an  nay  \textbf{daw}  \textbf{mag-dayon}  kay  pa. \\
{} and  \textsc{t.r}-ask-\textsc{apl}  1\textsc{p.excl.erg}  if/when  \textsc{i.ir}-continue  1\textsc{p.excl.abs}  \textsc{inc} \\
\glt '... and we asked (him) \textbf{whether} \textbf{we} \textbf{will} \textbf{still} \textbf{continue} on (the planned trip).’ [RMWN-L-01 2.1]
\z

Examples \REF{bkm:Ref115873391} and \REF{bkm:Ref115873572} illustrate similar constuctions with \textit{daw} clauses as complements:

\ea
\label{bkm:Ref115873391}
Unso  a  nang  tunuga  aged  mļaman  ko  \textbf{daw} \textbf{matuod}  gid  sugid  nyo  an  na  maļbaļ  danen  mari. \smallskip\\
\gll Unso  a  nang  tunuga  aged  ma-aļam-an  ko  \textbf{daw} \textbf{matuod}  gid  sugid  nyo  an  na  maļbaļ  danen  mari. \\
\textsc{d}4\textsc{loc.pr}  1\textsc{s.abs}  only  sleep  so.that  \textsc{a.hap.ir}-know-\textsc{apl}  1\textsc{s.erg}  if/when
true  \textsc{int}  tell  2\textsc{s.erg}  \textsc{def.m}  \textsc{lk}  witch  3\textsc{p.abs}  godmother \\
\glt `I will sleep there so-that I will know \textbf{whether} \textbf{it} \textbf{is} \textbf{really} \textbf{true} what you told me that my godmother and her companions are witches.’ [CBWN-C-13 4.4]
\z
\ea
\label{bkm:Ref115873572}
Tapos  sin-ad,  tagaran  \textbf{daw}  \textbf{masikaļ  en  sinin-ad  an} daw  ukayen \smallskip\\
\gll Tapos  sin-ad,  \emptyset{}-tagad-an  \textbf{daw}  \textbf{ma-sikaļ}  \textbf{en}  \textbf{s<in>in-ad}  \textbf{an} daw  ukay-en \\
then  cook.grain  \textsc{t.ir}-wait-\textsc{apl}  if/when  \textsc{a.hap.ir}-boil  \textsc{cm}  <\textsc{nr.res}>cook.grain  \textsc{def.m}
and  stir-\textsc{t.ir} \\
\glt `Then cooking rice, wait until \textbf{whenever} \textbf{the} \textbf{cooked} \textbf{rice} \textbf{boils} and stir it.’ [DBOP-J-17 1.6]
\is{complement clauses!subjunctive|)}
\is{subjunctive complement clauses|)}
\z

\subsection{Subjunctive adverbial clauses}
\is{subjunctive adverbial clauses|(}
\is{adverbial clauses!subjunctive|(}
\label{bkm:Ref117759078} \label{sec:subjunctiveadverbialclauses}
Subjunctive clauses may also function as optional adverbial clauses. For example, in \REF{bkm:Ref460685260} the irrealis clause is not needed to complete the idea expressed in the main clause, and there is no evidence that it may be an argument of the intransitive main verb. It also expresses the same kinds of notions that adverbs express. In \REF{bkm:Ref460685260}, the subjunctive clause may be interpreted as simultaneous with the event expressed in the main clause, as a separate event occurring in sequence, or as an expression of purpose:


\ea
\label{bkm:Ref460685260}
Gapanaw  kanen  an  na  muli  ta  iran  na  baļay. \smallskip\\{}
[\hspace{1.1cm}CL\textsc{\textsubscript{infl}\hspace{1.1cm}}]\hspace{.4cm}[na][\hspace{2.4cm}CL\textsc{\textsubscript{irr}}\hspace{2.4cm}] \\
\gll Ga-panaw  kanen  an  na  m-uli  ta  iran  na  baļay. \\
\textsc{i.r}-go/walk  3\textsc{s.abs}  \textsc{def.m}  \textsc{lk}  \textsc{i.v.ir}-go.home  \textsc{nabs}  3\textsc{p.gen}  \textsc{lk}  house \\
\glt
a. ‘S/he left \textit{going home to their house}.’ Simultanaity \\
b. ‘S/he left \textit{and then went home to their house}.’ Sequence \\
c. ‘S/he left \textit{to go home to their house.’} Purpose
\z

Example \REF{bkm:Ref474392546} illustrates a similar construction with two subjunctive clauses following the main clause. Technically, there are nine possible semantic interpretations of this construction, though some may be highly unlikely. In context, however, the range of possible interpretations is more constrained, as will be seen in the corpus examples that follow. In the following examples, the subjunctive Clauses are bolded in Kagayanen and the English free translations:

\ea
\label{bkm:Ref474392546}
Gadali  kanen  an  \textbf{na}  \textbf{manaw}  \textbf{na}  \textbf{muli}. \smallskip\\
\gll Ga-dali  kanen  an  \textbf{na}  \textbf{m-panaw}  \textbf{na}  \textbf{m-uli}. \\
\textsc{i.r}-hurry  3\textsc{s.abs}  \textsc{def.m}  \textsc{lk}  \textsc{i.v.ir}-go.walk  \textsc{lk}  \textsc{i.v.ir}-go.home \\
\glt
a. ‘S/he hurries \textit{leaving/walking going home}.’ All simultaneous \\
b. ‘S/he hurried, \textit{then left/walked and then went home}.’ All sequential \\
c. ‘S/he hurries \textit{leaving/walking and then went home}.’ Simultaneous + sequential \\
d. ‘S/he hurries \textit{leaving/walking to go home}.’ Simultaneous + purpose \\
e. ‘S/he is hurrying \textit{to leave/walk and then go home}.’ Purpose + sequential \\
etc.
\z

Although some such examples may seem to be complement clauses from the perspective of the English translations, in fact these follow the same structural template as the simultaneous, sequential and purpose subjunctive clauses (given in \ref{bkm:Ref460685260}), and they are not required to complete the meaning of the main verb.

The linker \textit{na} preceding a subjunctive adverbial clause is always “optional” from a purely syntactic point of view. However, there is a tendency for \textit{na} to occur when the dependent clause expresses sequence or purpose, and be absent when the two clauses describe simultaneous events, or two facets of the same event. The following example from the corpus contains \textit{na} and is therefore more likely to express sequence or purpose than simultaneous events:

\ea
… muyog  a  kon  ki  kanen  \textbf{na}  \textbf{muli}  ta  iran  na  lugar. \smallskip\\
\gll … m-kuyog  a  kon  ki  kanen  \textbf{na}  \textbf{m-uli}  ta  iran  na  lugar. \\
{} \textsc{i.v.ir}-go.with  1\textsc{s.abs}  \textsc{hsy}  \textsc{obl.p}  3s  \textsc{lk}  \textsc{i.v.ir}-go.home  \textsc{nabs}  3\textsc{p.gen}  \textsc{lk}  place \\
\glt ‘… I will go with her and \textbf{go} \textbf{home} \textbf{to} \textbf{their} \textbf{place}.’ or ‘… I will go with her \textbf{in} \textbf{order}  \textbf{to} \textbf{go} \textbf{home} \textbf{to} \textbf{their} \textbf{place}.’ [DBWN-T-21 2.1]
\z

Examples \REF{bkm:Ref474394726}{}-\REF{bkm:Ref474420292} illustrate constructions in which the subjunctive clause expresses a purpose. The sequential or even simultaneous interpretations may be technically possible, but in the context of the whole discourse, it is clear in each case that the speaker intended purpose:

\ea
\label{bkm:Ref474394726}
Dayon  en  garay  na  paumaw  mga  maligno  ta  bisan  indi nang  na  lugar  \textbf{na}  \textbf{marani}  tak  maan  en  ta  pagkaan an  \textbf{na}  \textbf{ayaren}  en  masakit  ta  ittaw  na  pangaranan. \smallskip\\
\gll Dayon  en  ga-aray  na  pa-umaw  mga  ma-ligno  ta  bisan  indi nang  na  lugar  \textbf{na}  \textbf{m-pa-dani}  tak  m-kaan  en  ta  pagkaan an  \textbf{na}  \textbf{ayad-en}  en  masakit  ta  ittaw  na  pa-ngaran-an. \\
right.away  \textsc{cm}  \textsc{i.r}-chant  \textsc{lk}  \textsc{t.r}-call  \textsc{pl}  \textsc{adj}-evil  \textsc{nabs}  even  where
only  \textsc{lk}  place  \textsc{lk}  \textsc{i.v.ir}-\textsc{caus}-close  because  \textsc{i.v.ir}-eat  \textsc{cm}  \textsc{nabs}  food
\textsc{def.m}  \textsc{lk}  heal-\textsc{t.ir}  \textsc{cm}  sickness  \textsc{nabs}  person  \textsc{lk}  \textsc{t.r-}name-\textsc{apl} \\
\glt `Right away (he) chants to call evil spirits in whatever place \textbf{to} \textbf{come} \textbf{close} because (they) will eat food \textbf{to} \textbf{heal} the sickness of the person who was named.’ [JCWE-T-17 3.5]
\z

\ea
Panggat  kay  ta  ame  na  auntie  \textbf{na}  \textbf{miling} kay  kon  ta  Manila  Zoo. \smallskip\\
\gll Pa-nggat  kay  ta  ame  na  auntie  \textbf{na}  \textbf{m-iling} kay  kon  ta  Manila  Zoo. \\
\textsc{t.r}-invite.to.accompany  1\textsc{p.excl.abs}  \textsc{nabs}  1\textsc{p.excl.gen}  \textsc{lk}  aunt  \textsc{lk}  \textsc{i.v.ir}-go
1\textsc{p.excl.gen}  \textsc{hsy}  \textsc{nabs}  Manila  Zoo \\
\glt `Our aunt invited us to go, it was said, to the Manila Zoo.’ [VAWN-T-15 7.4]
\z
\ea
Patuboļ  nay  bļangay  ya  en  \textbf{na}  \textbf{melled}  ta suba  ya. \smallskip\\
\gll Pa-tuboļ  nay  bļangay  ya  en  \textbf{na}  \textbf{m-selled}  ta suba  ya. \\
\textsc{t.r}-position  1\textsc{p.excl.erg}  2.masted.boat  \textsc{def.f}  \textsc{cm}  \textsc{lk}  \textsc{i.v.ir}-inside  \textsc{nabs}
river  \textsc{def.f} \\
\glt ‘We positioned the 2 mast boat \textbf{to} \textbf{go} \textbf{into} the river (from the ocean).’ [PCON-C-01 3.2]
\z

\ea
Listo  kon  danen  tanan  parani  \textbf{na}  \textbf{magyapon}. \smallskip\\
\gll Listo  kon  danen  tanan  ...-pa-dani  \textbf{na}  \textbf{mag-yapon}. \\
promptly  \textsc{hsy}  3\textsc{p.abs}  all  \textsc{t.r}-\textsc{caus}-close  \textsc{lk}  \textsc{i.ir}-supper \\
\glt ‘Promptly all of them came near \textbf{to} \textbf{eat} supper.’ [CBWN-C-13 6.4]
\z
\ea
Gapanaw  kon  en  bai  ya  \textbf{na}  \textbf{magmugon}. \smallskip\\
\gll Ga-panaw  kon  en  bai  ya  \textbf{na}  \textbf{mag-mugon}. \\
\textsc{i.r}-go/walk  \textsc{hsy}  \textsc{cm}  woman  \textsc{def.f}  \textsc{lk}  \textsc{i.ir}-day.work \\
\glt ‘The woman left \textbf{to} \textbf{do} \textbf{day} \textbf{work}.’ [MBON-C-01 2.6]
\z
\ea
... manaw  a  \textbf{na}  \textbf{mamugon}  ta  pagkaan  ta.” \smallskip\\
\gll ... m-panaw  a  \textbf{na}  \textbf{ma-ng-mugon}  ta  pagkaan  ta.” \\
{}   \textsc{i.v.ir}-go/walk  1\textsc{s.abs}  \textsc{lk}  \textsc{a.hap.ir}-\textsc{pl}-day.work  \textsc{nabs}  food  1\textsc{p.incl.gen} \\
\glt ‘... I will go \textbf{to} \textbf{do} \textbf{day} \textbf{work} for our food.’ [MBON-C-01 2.4]
\z
\ea
\label{bkm:Ref474420292}
Miling  ki  naan  ta  bukid  \textbf{na}  \textbf{mangita} ta  lungag. \smallskip\\
\gll M-iling  ki  naan  ta  bukid  \textbf{na}  \textbf{ma-ng-ngita} ta  lungag. \\
\textsc{i.v.ir}-go  1\textsc{p.incl.abs}  \textsc{spat.def}  \textsc{nabs}  mountain  \textsc{lk}  \textsc{a.hap.ir}-\textsc{pl}-search
\textsc{nabs}  hole \\
\glt `Let’s go to the mountain \textbf{to} \textbf{search} for the hole.’ [PBWN-C-12 4.4]
\z

Example \REF{bkm:Ref474396378} is an extended excerpt from a text titled “Monkey and Turtle” that clearly shows a subjunctive construction expressing a purpose function. In this excerpt, there are four subjunctive adverbial clauses, including three repetitions of the event of searching (\textit{mangita}). The first two events of searching are adverbial clauses introduced by the explicit marker of purpose \textit{aged}. The last instance is introduced with \textit{na} alone. This shows that the \textit{na} construction is a kind of “shortcut” for expressing the same semantic relation as a purpose adverbial clause:

\ea
\label{bkm:Ref474396378}
Ta  isya  na  adlaw,  amo  i  panggat  bubuo \textbf{na}  \textbf{magpanaw}  kon  danen  \textbf{aged}  \textbf{mangita}  ta  iran  na pagkaan.  Ambaļ  ta  bubuo,  “Dey,  manaw  ki  naan ta  bukid  \textbf{aged}  \textbf{mangita}  ki  ta  ate na  kan-enen.”  Sabat  man  ta  amo,  “Mos,  manaw  ki \textbf{na}  \textbf{mangita}  ta  ate  na  pagkaan.”  Gapanaw  en darwa  i  na  mag-arey. \smallskip\\
\gll Ta  isya  na  adlaw,  amo  i  pa-anggat  bubuo \textbf{na}  \textbf{mag-panaw}  kon  danen  \textbf{aged}  \textbf{ma-ng-ngita}  ta  iran  na pagkaan.  Ambaļ  ta  bubuo,  “Dey,  m-panaw  ki  naan ta  bukid  \textbf{aged}  \textbf{ma-ng-ngita}  ki  ta  ate na  kan-en-en.”  Sabat  man  ta  amo,  “Mos,  m-panaw  ki \textbf{na}  \textbf{ma-ng-ngita}  ta  ate  na  pagkaan.”  Ga-panaw  en darwa  i  na  mag-arey. \\
\textsc{nabs}  one  \textsc{lk}  day/sun  monkey  \textsc{def.n}  \textsc{t.r}-invite.to.go.with  tortoise
\textsc{lk}  \textsc{i.v.ir}-go/walk  \textsc{hsy}  3\textsc{p.abs}  for  \textsc{a.hap.ir}-\textsc{pl}-search  \textsc{nabs}  3\textsc{p.gen}  \textsc{lk}
food  say  \textsc{nabs}  tortoise  friend  \textsc{i.v.ir}-go/walk  1\textsc{p.incl.abs}  \textsc{spat.def}
\textsc{nabs}  mountain  for  \textsc{a.hap.ir}-\textsc{pl}-search  1\textsc{p.incl.abs}  \textsc{nabs}  1\textsc{p.incl.gen}
\textsc{lk}  cooked.rice-\textsc{t.ir}  reply  also  \textsc{nabs}  monkey  let’s.go  \textsc{i.v.ir}-go/walk  1\textsc{p.incl.abs} \textsc{lk}  \textsc{a.hap.ir}-\textsc{pl}-search  \textsc{nabs}  1\textsc{p.incl.gen}  \textsc{lk}  food  \textsc{i.r}-go/walk  \textsc{cm}
two  \textsc{def.n}  \textsc{lk}  \textsc{rel}-friend \\
\glt `On one day, as for the monkey the tortoise invited (him) to go \textbf{walking} \textbf{in} \textbf{order} \textbf{to} \textbf{search} for their food. The tortoise said, “Friend, let’s go to the mountain \textbf{in} \textbf{order} \textbf{to} \textbf{search} for something we will eat.” The monkey replied, “Let’s go, let’s go \textbf{to} \textbf{search} for our food.” The two friends left.’ [CBWN-C-16 2.1-5]
\z

The subjunctive clauses in examples \REF{bkm:Ref474420325}{}-\REF{bkm:Ref474489008} could be understood either as purpose or simply as sequential events. The contexts in these cases do not require one interpretation over the other. This is probably the most common usage of subjunctive dependent clauses:

\ea
\label{bkm:Ref474420325}
Gadļagan  kanen  \textbf{na}  \textbf{mil-ing}  naan  ta  waig … \smallskip\\
\gll Ga-dļagan  kanen  \textbf{na}  \textbf{m-sil-ing}  naan  ta  waig … \\
\textsc{i.r}-run  3\textsc{s.abs}  \textsc{lk}  \textsc{i.v.ir}-peer  \textsc{spat.def}  \textsc{nabs}  water \\
\glt ‘She ran \textbf{to} \textbf{peer/and} \textbf{peered} into the water (probably a pool or mud puddle) …’ [CBWN-C-17 3.3]
\z

\newpage
\ea
Pagbati  ta  amay  ta  bai  na  yaan  Roxas  en gistar  na  uļa  en  sawa  bata  din  an,  dayon  din  kamangen \textbf{na}  \textbf{daļaen}  ta  Roxas. \smallskip\\
\gll Pag-bati  ta  amay  ta  bai  na  yaan  Roxas  en ga-istar  na  uļa  en  sawa  bata  din  an,  dayon  din  kamang-en \textbf{na}  \textbf{daļa-en}  ta  Roxas. \\
\textsc{nr.act}-hear  \textsc{nabs}  father  \textsc{nabs}  woman  \textsc{lk}  \textsc{spat.def}  Roxas  \textsc{cm}
\textsc{i.r}-live  \textsc{lk}  \textsc{neg.r}  \textsc{cm}  spouse  child  3\textsc{s.gen}  \textsc{def.m}  right.away  3\textsc{s.erg}  get-\textsc{t.ir}
\textsc{lk}  carry/bring-\textsc{t.ir}  \textsc{nabs}  Roxas \\
\glt ‘When the father of the woman who was living in Roxas heard that his child no longer had a spouse, right away he got (her) \textbf{to} \textbf{take/taking/and} \textbf{took} (her) to Roxas.’ [JCWN-T-26 18.1]
\z
\ea
Manaw  taan  kani  danen  an  \textbf{na}  \textbf{magnakaw}. \smallskip\\
\gll M-panaw  taan  kani  danen  an  \textbf{na}  \textbf{mag-nakaw}. \\
\textsc{i.v.ir}-go/walk  maybe  later  3\textsc{p.abs}  \textsc{def.m}  \textsc{lk}  \textsc{i.ir}-steal \\
\glt ‘They maybe will go later \textbf{to steal/and steal}.’ [CBWN-C-18 7.19]
\z
\ea
Mangngod  din  i  ubra  din  kada  adlaw  giling naan  ta  bukid  \textbf{na}  \textbf{magdakep}  kanen  ta  mga  tļunon  na  baboy. \smallskip\\
\gll Mangngod  din  i  ubra  din  kada  adlaw  ga-iling naan  ta  bukid  \textbf{na}  \textbf{mag-dakep}  kanen  ta  mga  tļunon  na  baboy. \\
younger.sibling  3\textsc{s.gen}  \textsc{def.n}  work/do  3\textsc{s.gen}  every  day/sun  \textsc{i.r}-go
\textsc{spat.def}  \textsc{nabs}  mountain  \textsc{lk}  \textsc{i.ir}-catch  3\textsc{s.abs}  \textsc{nabs}  \textsc{pl}  wild.pig  \textsc{lk}  pig \\
\glt `As for his younger sibling, his work every day was going to the mountain \textbf{to} \textbf{catch/and} \textbf{catch} wild pigs.’ [CBWN-C-22 3.2]
\z
\ea
Mama  ko  ya  giling  ta  Iloilo  \textbf{na}  \textbf{malit}  ta  buļong  ko. \smallskip\\
\gll Mama  ko  ya  ga-iling  ta  Iloilo  \textbf{na}  \textbf{m-palit}  ta  buļong  ko. \\
mother  1\textsc{s.gen}  \textsc{def.f}  \textsc{i.r}-go  \textsc{nabs}  Iloilo  \textsc{lk}  \textsc{i.v.ir}-buy  \textsc{nabs}  medicine  1\textsc{s.gen} \\
\glt ‘My mother (she) went to Iloilo \textbf{to} \textbf{buy/and} \textbf{bought} my medicine.’ [JCOE-C-05 43.1]
\z

\newpage

\ea
Ta  anyo  na  1974  ta  buļan  ta  Hunio,  yaken  gakuyog  ta ake  na  sawa  \textbf{na}  \textbf{magbligya}  ta  bakaw  ta  Iloilo. \smallskip\\
\gll Ta  anyo  na  1974  ta  buļan  ta  Hunio,  yaken  ga-kuyog  ta ake  na  sawa  \textbf{na}  \textbf{mag-bligya}  ta  bakaw  ta  Iloilo. \\
\textsc{nabs}  year  \textsc{lk}  1974  \textsc{nabs}  month  \textsc{nabs}  June  1\textsc{s.abs}  \textsc{i.r}-go.with  \textsc{nabs}
1\textsc{s.gen}  \textsc{lk}  spouse  \textsc{lk}  \textsc{i.ir}-sell  \textsc{nabs}  mangrove  \textsc{nabs}  Iloilo \\
\glt `In the year 1974 in the month of June, as for me (I) went with my spouse \textbf{to}
\textbf{sell/selling/and} \textbf{sold} mangrove wood in Iloilo.’ [CTWN-L-01 2.2]
\z
\ea
Bisan  sikad  pa  dangga  adlaw  an  o  maabutan  ta  uran sigi  gid  panaw  \textbf{na}  \textbf{malambot}  mga  lugar  na  dapat gid  litunan. \smallskip\\
\gll Bisan  sikad  pa  dangga  adlaw  an  o  ma-abot-an  ta  uran sigi  gid  panaw  \textbf{na}  \textbf{ma-lambot}  mga  lugar  na  dapat gid  litunan. \\
even  very  \textsc{inc}  hot  day/sun  \textsc{def.m}  or  \textsc{a.hap.ir}-arrive-\textsc{apl}  \textsc{nabs}  rain continue  \textsc{int}  walk/go  \textsc{lk}  \textsc{a.hap.ir}-reach  \textsc{pl}  place  \textsc{lk}  must
\textsc{int}  present.to.ancestors \\
\glt `Even when the sun is very hot or rain comes upon (them) (they) keep walking \textbf{to} \textbf{reach/reaching/and} \textbf{reach} the places where they must present the child to the ancestors.’ [JCWE-T-15 5.7]
\z
\ea
Pagbwi  ta  suso,  dayon  bangon  sawa  ko  \textbf{na}  \textbf{mwa}. \smallskip\\
\gll Pag-bwi  ta  suso,  dayon  bangon  sawa  ko  \textbf{na}  \textbf{m-gwa}. \\
{nr.act}-let.go  \textsc{nabs}  breast  right.away  get.up  spouse  1\textsc{s.gen}  \textsc{lk}  \textsc{i.v.ir}-out \\
\glt ‘When (the baby) finished nursing (lit. let go of the breast), right away my wife got up \textbf{to} \textbf{come} \textbf{out/and} \textbf{came} \textbf{out}.’ [JCWN-L-31 18.10]
\z
\ea
Tama  na  mga  ittaw  gailing  di  \textbf{na}  \textbf{magtan-aw}  nang ta  ame  na  lugar. \smallskip\\
\gll Tama  na  mga  ittaw  ga-iling  di  \textbf{na}  \textbf{mag-tan-aw}  nang ta  ame  na  lugar. \\
many  \textsc{lk}  \textsc{pl}  people  \textsc{i.r}-go  \textsc{d}1\textsc{loc}  \textsc{lk}  \textsc{i.ir}-look  only
\textsc{nabs}  1\textsc{p.excl.gen}  \textsc{lk}  place \\
\glt `Many people come here \textbf{just} \textbf{to} \textbf{look/and} \textbf{just} \textbf{look} at  our place.’ [DBWL-T-19 9.4]
\z

\newpage

\ea
Apos a  bubod,  yaken  gasakay  ta  lunday \textbf{na}  \textbf{mangawil}  aren. \smallskip\\
\gll Apos\footnotemark{}  a  bubod,  yaken  ga-sakay  ta  lunday \textbf{na}  \textbf{ma-ng-kawil}  aren. \\
after  \textsc{inj}  pour.out  1\textsc{s.abs}  \textsc{i.r}-rude  \textsc{nabs}  outrigger.canoe
\textsc{lk}  \textsc{a.hap.ir}-\textsc{pl}-fishhook  1\textsc{s.abs} \\
\footnotetext{\textit{Apos} is an alternate form of \textit{tapos}, ‘finish’/‘after’.}
\glt `After pouring out (cassava powder to make the worms come out to use as bait), as for me, (I) rode the outrigger canoe \textbf{so} \textbf{that} \textbf{I} \textbf{will} \textbf{hook} \textbf{and} \textbf{line} \textbf{fish/and} \textbf{I} \textbf{fished} \textbf{with} \textbf{hook} \textbf{and} \textbf{line}.’ [RPWN-T-01 1.3]
\z
\ea
Isya  na  adlaw,  tallo  na  gaarey  giling  ta  suba \textbf{na}  \textbf{maglangoy}. \smallskip\\
\gll Isya  na  adlaw,  tallo  na  ga-arey  ga-iling  ta  suba \textbf{na}  \textbf{mag-langoy}. \\
one  \textsc{lk}  day/sun  three  \textsc{lk}  \textsc{i.r}-friend  \textsc{i.r}-go  \textsc{nabs}  river
\textsc{lk}  \textsc{i.ir}-bathe \\
\glt `One day, three friends went to the river \textbf{to bathe/and bathed}.’ [PMWN-T-01  2.1]
\z
\ea
... amo  i  sagad  mukay  ta  buok  \textbf{na}  \textbf{makamang} ta  kuto … \smallskip\\
\gll ... amo  i  sagad  m-sukay  ta  buok  \textbf{na}  \textbf{ma-kamang} ta  kuto … \\
... monkey  \textsc{def.n}  skillful  \textsc{i.v.ir}-delouse  \textsc{nabs}  hair  \textsc{lk}  \textsc{a.hap.ir}-get
\textsc{nabs}  lice/ticks \\
\glt `… the monkey is skillful at checking the hair/scalp for lice/ticks \textbf{to} \textbf{remove/removing} lice/ticks …’ [NEWE-T-01 2.7]
\z
\ea
Gaubay-ubay  kay  ta  mga  baybay  daw  pangpang \textbf{na}  \textbf{miling}  ta  Cebu. \smallskip\\
\gll Ga-ubay\sim{}-ubay  kay  ta  mga  baybay  daw  pangpang \textbf{na}  \textbf{m-iling}  ta  Cebu. \\
\textsc{i.r}-\textsc{red}-beside  1\textsc{p.excl.abs}  \textsc{nabs}  \textsc{pl}  beach  and    clift
\textsc{lk}  \textsc{i.v.ir}-go  \textsc{nabs}  Cebu \\
\glt `We went along the beaches and cliffs \textbf{to} \textbf{go/going} to Cebu.’  [DBWN-T-23 2.5]
\z
\ea
Kaon  nang  pagustuon  daw  miling  ka  ta  Manila \textbf{na}  \textbf{magtudlo}  ki  kanen  ta  ambaļ  ta. \smallskip\\
\gll Kaon  nang  pa-gusto-en  daw  m-iling  ka  ta  Manila \textbf{na}  \textbf{mag-tudlo}  ki  kanen  ta  ambaļ  ta. \\
2\textsc{s.abs}  only  \textsc{caus}-want/like-\textsc{t.ir}  if/when  \textsc{i.v.ir}-go  2\textsc{s.abs}  \textsc{nabs}  Manila
\textsc{lk}  \textsc{i.ir}-teach  \textsc{obl.p}  3s  \textsc{nabs}  say  1\textsc{p.incl.gen} \\
\glt `You are the one to do what you want if you will go to Manila \textbf{to} \textbf{teach/teaching/and} \textbf{teach} her our language.’ [RCON-L-03 4.4]
\z
\ea
\label{bkm:Ref474489008}
Ta  di  gapanaw  kay  eman  \textbf{na}  \textbf{magbantay} en  ta  sakayan? \smallskip\\
\gll Ta  di  ga-panaw  kay  eman  \textbf{na}  \textbf{mag-bantay} en  ta  sakay-an? \\
so  \textsc{inj}  \textsc{i.r}-go/walk  1\textsc{p.excl.abs}  again.as.before  \textsc{lk}  \textsc{i.ir}-watch/guard
\textsc{cm}  \textsc{nabs}  ride-\textsc{nr} \\
\glt `So what else, did we leave again \textbf{to} \textbf{watch/and} \textbf{watched} for a vehicle?’ [BGON-L-01 1.13]
\z

In example \REF{bkm:Ref474395434}, the sequential interpretation is most likely, since the child does not meet her mother \textit{in order to} grab whatever she carries. The grabbing is simply an event that occurs after (or possibly simultaneous with) the event of meeting:

\ea
\label{bkm:Ref474395434}
... tak  kada  uli  din  bata  an  pirmi  gasugat \textbf{na}  \textbf{mag-agaw}  daw  ino  nadaļa  din. \smallskip\\
\gll ... tak  kada  uli  din  bata  an  pirmi  ga-sugat \textbf{na}  \textbf{mag-agaw}  daw  ino  na-daļa  din. \\
{}   because  every  go.home  3\textsc{s.gen}  child  \textsc{def.m}  always  \textsc{i.r}-meet
\textsc{lk}  \textsc{i.ir}-grab  if/when  what  \textsc{a.hap.r}-carry  3\textsc{s.erg} \\
\glt `… because every (time) she went home her child always met (her) \textbf{and} \textbf{grabbed} whatever she carried.’ [PBWN-C-12 8.1]
\z

On the other hand, example \REF{bkm:Ref117771448} illustrates a construct in which a subjunctive adverbial clause clearly expresses the purpose for the activity in the main clause:

\ea
\label{bkm:Ref117771448}
Gangita  kay  man  ta  babaw  \textbf{na}  \textbf{makalakkat}  kay ta  bato. \smallskip\\
\gll Ga-ngita  kay  man  ta  babaw  \textbf{na}  \textbf{maka-lakkat}  kay ta  bato. \\
\textsc{i.r}-search  1\textsc{p.excl.abs}  also  \textsc{nabs}  shallow  \textsc{lk}  \textsc{i.hap.ir}-step.on  1\textsc{p.excl.abs}
\textsc{nabs}  stones \\
\glt `We were looking for a shallow [part of the river/stream) \textbf{so} \textbf{that} \textbf{we} \textbf{can} \textbf{step} \textbf{on} stones (to cross over).’ [BGON-L-01 1.13]
\z
Examples \REF{bkm:Ref474396926}{}-\REF{bkm:Ref474488975} illustrate subjunctive constructions without the \textit{na} linker between the two clauses. Again, \textit{na} is more likely to be omitted when the speaker intends to express simultaneous events, though the purpose and sequential interpretations are not excluded:

\ea
\label{bkm:Ref474396926}
Malin  kay  mga  ala  una  \textbf{miling}  ta  Puerto. \smallskip\\
\gll M-alin  kay  mga  ala  una  \textbf{m-iling}  ta  Puerto. \\
\textsc{i.v.ir}-from  1\textsc{p.excl.abs}  \textsc{pl}  o’clock  one  \textsc{i.v.ir}-go  \textsc{nabs}  Puerto \\
\glt ‘We will leave at one o’clock \textbf{going} to Puerto.’ [RCON-L-03 10.4]
\z
\ea
Lugar  na  manaw  kay  inta  \textbf{megbeng}  en, gasinggit  man  bai  i,  bata  din. \smallskip\\
\gll Lugar  na  m-panaw  kay  inta  \textbf{m-tegbeng}  en, ga-singgit  man  bai  i,  bata  din. \\
place  \textsc{lk}  \textsc{i.v.ir}-go/walk  1\textsc{p.excl.abs}  \textsc{opt}  \textsc{i.v.ir}-go.downhill  \textsc{cm}
\textsc{i.r}-shout  also  woman  \textsc{def.n}  child  3\textsc{s.gen} \\
\glt `After some time when we were about to leave \textbf{going} \textbf{down} \textbf{hill,} the woman, his child, shouted.’ [BGON-L-01 15.1]
\z
\ea
Naan  ta  baļay  may  isya  na  ittaw  na  gatagad  kami na  muyog  kay  kon  \textbf{miling}  ta  Pasil. \smallskip\\
\gll Naan  ta  baļay  may  isya  na  ittaw  na  ga-tagad  kami na  m-kuyog  kay  kon  \textbf{m-iling}  ta  Pasil. \\
\textsc{spat.def}  \textsc{nabs}  house  \textsc{ext.in}  one  \textsc{lk}  person  \textsc{lk}  \textsc{i.r}-wait  1\textsc{p.exc}
\textsc{lk}  \textsc{i.v.ir}-go.with  1\textsc{p.excl.abs}  \textsc{hsy}  \textsc{i.v.ir}-go  \textsc{nabs}  Pasil \\
\glt `At the house there was one person who was waiting for us to go/going with (him/her), it was said, \textbf{going} to Pasil.’ [DBWN-T-23 5.1]
\z
\ea
Dayon  kanen  i  tindeg  na  malik  ta  iran  na  sakayan daw  kamangen  iya  na  sawa  naan  ta  luwang.  Dayon  din usong  daw  dļaen  \textbf{makas}  naan  ta iran	na	baļay. \smallskip\\
\gll Dayon  kanen  i  tindeg  na  m-balik  ta  iran  na  sakay-an daw  kamang-en  iya  na  sawa  naan  ta  luwang.  Dayon  din usong  daw  daļa-en  \textbf{m-takas}  naan  ta iran	na	baļay. \\
right.away  3\textsc{s.abs}  \textsc{def.n}  stand  \textsc{lk}  \textsc{i.v.ir}-return  \textsc{nabs}  3\textsc{s.gen}  \textsc{lk}  ride-\textsc{nr}
and  get-\textsc{t.ir}  3\textsc{s.gen}  \textsc{lk}  spouse  \textsc{spat.def}  \textsc{nabs}  wide  right.away  3\textsc{s.erg}
carry.on.shoulder  and  carry/take-\textsc{t.ir}  \textsc{i.v.ir}-go.uphill  \textsc{spat.def}  \textsc{nabs}
3\textsc{s.gen}  \textsc{lk}  house \\
\glt `Right away he stood up to go/going back to their boat and to get/getting his spouse in the wide part (of the vehicle). Right away he carried (her) on his shoulders and took (her) \textbf{going} uphill to their house.’ [DBWN-T-33 2.35-36]
\z
\ea
\label{bkm:Ref474488975}
Diritso  ko  pas-an  baboy  ya  \textbf{dļaen}   ta  baļay. \smallskip\\
\gll Diritso  ko  ...-pas-an  baboy  ya  \textbf{dļa-en}   ta  baļay. \\
straight  1\textsc{s.erg}  \textsc{t.r}-carry.on.shoulder  pig  \textsc{def.f}  carry/take-\textsc{t.ir}  \textsc{nabs}  house \\
‘Straight away I carried the pig on my shoulder \textbf{taking} (it) to the house.’ [RCON-L-01 9.5]
\z


Examples \REF{bkm:Ref115094786}{}-\REF{bkm:Ref115095578} illustrate \textit{daw} used as an introducer of subjunctive clauses. Many of these may be considered to be \textit{embedded questions}\is{embedded questions}:

\ea
\label{bkm:Ref115094786}
Galangkag  a  \textbf{daw}  \textbf{muli}  kaw  di. \smallskip\\
\gll Ga-langkag  a  \textbf{daw}  \textbf{m-uli}  kaw  di. \\
\textsc{i.r}-expect  1\textsc{s.abs}  if/when  \textsc{i.v.ir}-go.home  2\textsc{p.abs}  \textsc{d}1\textsc{loc} \\
\glt ‘I am expectantly waiting \textbf{whether/if/when} \textbf{you} \textbf{will} \textbf{come} \textbf{home}.’
\z
\ea
Gaduwa-duwa  a  \textbf{daw}  \textbf{mabang}  kanen  an  ki  yaken. \smallskip\\
\gll Ga-duwa-duwa  a  \textbf{daw}  \textbf{m-tabang}  kanen  an  ki  yaken. \\
\textsc{i.r}-waver/doubt  1\textsc{s.abs}  if/when  \textsc{i.v.ir}-help  3\textsc{s.abs}  \textsc{def.m}  \textsc{obl.p}  1s \\
\glt ‘I am wavering \textbf{whether/if/when} \textbf{s/he} \textbf{will} \textbf{help} me.’
\z
\ea
Ambaļ  din,  “Inday,  kaon,  \textbf{daw}  \textbf{miyag}  \textbf{ka}?” \smallskip\\
\gll Ambaļ  din,  “Inday,  kaon,  \textbf{daw}  \textbf{miyag}  \textbf{ka}?” \\
say  3\textsc{s.erg}  don’t.know  2\textsc{s.abs}  if/when  want/agree  2\textsc{s.abs} \\
\glt `He said, “I don’t know, (about) you, \textbf{whether/if you agree}"’ [RCON-L-03 2.15]
\z
\ea
Malipayen  a  \textbf{daw}  \textbf{makita}  \textbf{ko}  ake  na  mga  kabataan na  gapalanggaanay  daw  gataod … \smallskip\\
\gll Ma-lipay-en  a  \textbf{daw}  \textbf{ma-kita}  \textbf{ko}  ake  na  mga  ka-bata-an na  ga-palangga-anay  daw  ga-taod … \\
\textsc{adj}-joy-\textsc{adj}  1\textsc{s.abs}  if/when  \textsc{a.hap.ir}-see  1\textsc{s.abs}  1\textsc{s.gen}  \textsc{lk}  \textsc{pl}  \textsc{nr}-child-\textsc{nr}
\textsc{lk}  \textsc{i.r}-love-\textsc{rec}  and  \textsc{i.r}-respect \\
\glt `I am joyful \textbf{whenever} \textbf{I} \textbf{see} my children loving each other and giving respect …’ [CBWE-C-05 5.7]
\z

It is common for \textit{daw} to introduce adverbial clauses that express a condition under which the main clause holds true:

\ea
Baked  gid  ake  na  pagpasalamat  \textbf{daw}  \textbf{mabaton} no  ake  na  suļat  na  uļa  man  dipirinsya. \smallskip\\
\gll Baked  gid  ake  na  pag-pa-salamat  \textbf{daw}  \textbf{ma-baton} no  ake  na  suļat  na  uļa  man  dipirinsya. \\
big  \textsc{int}  1\textsc{s.gen}  \textsc{lk}  \textsc{nr.act}-\textsc{caus}-thannks  if/when  \textsc{a.hap.ir}-receive
2\textsc{s.erg}  1\textsc{s.gen}  \textsc{lk}  letter  \textsc{lk}  \textsc{neg.r}  also  damage \\
\glt `My giving thanks will be really big \textbf{when/if} \textbf{you} \textbf{will} \textbf{receive} my letter without any damage.’ [NEWL-T-04 5.2]
\z
\ea
\textbf{Daw}  bagnes  en  isya  na  nanay  kinangļan  kanen  \textbf{magkaan} ta  gulay  daw  \textbf{mag-inem}  kanen  ta  bitamina  \textbf{agod}  \textbf{daw} \textbf{mwa}  bata  an  biskeg  iya  na  lawa  daw  madyo  ta  masakit. \smallskip\\
\gll \textbf{Daw}  bagnes  en  isya  na  nanay  kinangļan  kanen  \textbf{mag-kaan} ta  gulay  daw  \textbf{mag-inem}  kanen  ta  bitamina  \textbf{agod}  \textbf{daw} \textbf{m-gwa}  bata  an  biskeg  iya  na  lawa  daw  madyo  ta  masakit. \\
if/when  pregnant  \textsc{cm}  one  \textsc{lk}  mother  need  3\textsc{s.abs}  \textsc{i.ir}-eat
\textsc{nabs} vegetables  and  \textsc{i.ir}-drink  3\textsc{s.abs}  \textsc{nabs}  vitamins  so.that  if/when
\textsc{i.v.ir}-go.out  child  \textsc{def.m}  strong  3\textsc{s.gen}  \textsc{lk}  body  and  far  \textsc{nabs} sickness \\
\glt `When a mother is pregnant she should eat vegetables and take (lit. drink) vitamins so that when the baby comes out, his/her body is strong and (s/he) is far from sickness.’ [LBOP-C-03 11.3]
\z

\textit{Daw} clauses expressing conditions may follow the main clause:

\ea
Magamit  ka  ta  tinidor  daw  kutsara  \textbf{daw}  \textbf{maan}. \smallskip\\
\gll Ma-gamit  ka  ta  tinidor  daw  kutsara  \textbf{daw}  \textbf{m-kaan}. \\
\textsc{a.hap.ir}-use  1\textsc{s.abs}  \textsc{nabs}  fork  and  spoon  if/when  \textsc{i.v.ir}-eat \\
\glt ‘You can use a fork and spoon \textbf{when} \textbf{eating}.’ [ETOP-C-10 2.11]
\z

\textit{Daw} may also follow \textit{basi} ‘perhaps’:

\ea
\label{bkm:Ref117832478}
Daw  kilem,  dili  katunuga  tak  \textbf{basi}  \textbf{daw} \textbf{matuuron}  ta  iran  na  silingan  na  patayen  kanen  an. \smallskip\\
\gll Daw  kilem,  dili  ka-tunuga  tak  \textbf{basi}  \textbf{daw} \textbf{matuod-en}  ta  iran  na  silingan  na  patay-en  kanen  an. \\
if/when  night  \textsc{neg.ir}  \textsc{i.exm}-sleep  because  perhaps  if/when
true-\textsc{t.ir}  \textsc{nabs}  3\textsc{p.gen}  \textsc{lk}  neighbor  \textsc{lk}  kill-\textsc{t.ir}  3\textsc{s.abs}  \textsc{def.m} \\
\glt `When it is night, (he) is not able to sleep because \textbf{perhaps} their neighbor will make it true (what he said previously) that he will kill him.’ [CBWN-C-12 4.4]
\z

Since the existential particles \textit{may} and \textit{anen}, do not inflect, they have no specifically irrealis forms (see \chapref{chap:non-verbalclauses}, \sectref{sec:existentialconstructions}). Therefore, the bare forms appear in subjunctive clauses (\ref{bkm:Ref115098917} through \ref{bkm:Ref115098919}):

\ea
\label{bkm:Ref115098917}
Unso  ki,  bistaan  ta  dya  \textbf{daw} \textbf{may}  \textbf{anen}  \textbf{dya}. \smallskip\\
\gll Unso  ki,  \emptyset{}-bista-an  ta  dya  \textbf{daw} \textbf{may}  \textbf{anen}  \textbf{dya}. \\
\textsc{d}4\textsc{loc.pr}  1\textsc{p.incl.abs}  \textsc{t.ir}find.out-\textsc{apl}  1\textsc{p.incl.erg}  \textsc{d}4\textsc{loc}  if/when
\textsc{ext.in}  \textsc{ext.g}  \textsc{d}4\textsc{loc} \\
\glt ‘Let’s (go/be) there, let’s find out there \textbf{whether} \textbf{there} \textbf{is} \textbf{any} \textbf{there}.’ (The monkey and tortoise are looking for some food.) [CBWN-C-16 3.8]
\z

\ea
… mabantaw  ta  baļay  an  \textbf{daw}  \textbf{may}  \textbf{ittaw}. \smallskip\\
\gll … ma-bantaw  ta  baļay  an  \textbf{daw}  \textbf{may}  \textbf{ittaw}. \\
{} \textsc{a.hap.ir}-look.far.off  \textsc{nabs}  house  \textsc{def.m}  if/when  \textsc{ext.in}  person \\
\glt ‘... let’s look far off at the house \textbf{whether} \textbf{there} \textbf{are} \textbf{people}.’ [CBWN-C-18 7.5]
\z
\ea
\label{bkm:Ref115098919}
Pagsekeb  danen  uļa  danen  natan-aw  gantangan  ya \textbf{daw}  \textbf{may}  \textbf{gapilit}  \textbf{na}  \textbf{bļawan}. \smallskip\\
\gll Pag-sekeb  danen  uļa  danen  na-tan-aw  gantangan  ya \textbf{daw}  \textbf{may}  \textbf{ga-pilit}  \textbf{na}  \textbf{bļawan}. \\
\textsc{nr.act}-measure  3\textsc{p.gen}  \textsc{neg.r}  3\textsc{p.erg}  \textsc{a.hap.r}-look.at  3.quart.measurer  \textsc{def.f}
if/when  \textsc{ext.in}  \textsc{i.r}-stick  \textsc{lk}  gold \\
\glt `When they were measuring, they had not looked at the 3 quart measurer \textbf{whether} \textbf{there} \textbf{was} \textbf{gold} \textbf{stuck} \textbf{(in} \textbf{it)}.’ [CBWN-C-22 7.1]
\z
There are also no specifically irrealis forms for other types of non-verbal predicates \REF{bkm:Ref115095578}:

\ea
\label{bkm:Ref115095578}
Dayon  insa  \textbf{daw}  \textbf{mga}  \textbf{kaļa}  \textbf{ko}  \textbf{kon}  \textbf{danen}  \textbf{i}. \smallskip\\
\gll Dayon  insa  \textbf{daw}  \textbf{mga}  \textbf{kaļa}  \textbf{ko}  \textbf{kon}  \textbf{danen}  \textbf{i}. \\
right.away  ask  if/when  \textsc{pl}  know  1\textsc{s.gen}  \textsc{hsy}  3\textsc{p.abs}  \textsc{def.n} \\
\glt ‘Right away (she) asked (me) \textbf{whether} \textbf{they} \textbf{were} \textbf{ones} \textbf{I} \textbf{knew}.’
\footnote{The root \textit{kaļa} ‘know’ often occurs with verbal inflection, but in this example it is functioning as a nominal (see \chapref{chap:referringexpressions}, \sectref{sec:zerodeverbalization} on “zero deverbalization”), as evidenced by the presence of the plural marker \textit{mga} A literal translation of this phrase would be ‘my known ones’.} [CBWN-C-23 3.11]
\z

See also example \REF{bkm:Ref117832478} above that contains the clause \textit{daw kilem} ‘when (it was) night’.


The introducer \textit{aged} (also pronounced and spelled \textit{agod} by some speakers) can optionally be followed by the linker \textit{na}. It seems \textit{aged na} is used when the subjunctive clause expresses purpose and \textit{aged} alone is used for result, but a full discourse study is needed to confirm or disconfirm this hypothesis (\ref{bkm:Ref115099737} through \ref{bkm:Ref115100547}):

\ea
\label{bkm:Ref115099737}
Mamang  ki  anduni  ta  uling  aged  na  legeran ta  lawa  i  ta  Indangan  \textbf{aged}  \textbf{kanen}  \textbf{i}  \textbf{magmitem}. \smallskip\\

\gll M-kamang  ki  anduni ta  uling   \\
\textsc{i.v.ir}-get  1\textsc{p.incl.abs}  now/today \textsc{nabs}  charcoal\\
 PURPOSE \\
\gll \textbf{aged}  \textbf{na}  \emptyset{}\textbf{-leged-an} ta  lawa  i  ta  Indangan \\
  so.that  \textsc{lk}  \textsc{t.ir}-rub-\textsc{apl} 1\textsc{p.incl.erg}  body  \textsc{def.n}  \textsc{nabs}  surgeonfish \\

RESULT \\
\gll \textbf{aged}  \textbf{kanen}  \textbf{i}  \textbf{mag-mitem}. \\
so.that  3\textsc{s.abs}  \textsc{def.n}  \textsc{i.ir}-black \\
\glt `Let’s get now some charcoal \textbf{in} \textbf{order} \textbf{that} \textbf{we} \textbf{rub} \textbf{(it)} on the body of Surgeon fish \textbf{such} \textbf{that} \textbf{as} \textbf{for} \textbf{him} \textbf{(he)} \textbf{will} \textbf{become} \textbf{black}.’ [JCON-L-07 15.8]
\z
\ea
Dayon  kanen  pupo  ta  kasoy  tak  atag  din  ta  mga  kana \textbf{aged}  \textbf{na}  \textbf{makaparani}  \textbf{kanen}. \smallskip\\
\gll Dayon  kanen  pupo  ta  kasoy  tak  \emptyset{}-atag  din  ta  mga  kana \\
right.away  3\textsc{s.abs}  pick  \textsc{nabs}  cashew  because  \textsc{t.ir}-give  3\textsc{s.erg}  \textsc{nabs}  \textsc{pl}  foreigner \\

PURPOSE \\
\gll \textbf{aged}  \textbf{na}  \textbf{maka-pa-dani}  \textbf{kanen}. \\
so.that  \textsc{lk}  \textsc{i.hap.ir}-\textsc{caus}-close  3\textsc{s.abs} \\
\glt `Right away she picked some cashews because she would give (them) to the foreigners \textbf{in} \textbf{order} \textbf{that} \textbf{she} \textbf{can} \textbf{approach} \textbf{(them)}.’ [DBON-T-10 1.7]
\z

Example \REF{bkm:Ref115100547} is an extended excerpt that contains three clauses (bolded) introduced with \textit{aged}:
\ea
\label{bkm:Ref115100547}
Sabat  man  ta  bubuo  ya,  “Miad  gani  daw  pakpaken  no tudtod  ko  i  \textbf{aged}  \textbf{magļapad}.”  Tapos,  ambaļ  ta  amo  ya  a, “A,  dili  ta kaw  nang  en  pagpakpaken.  Daw madakep  ta kaw,  tekteken  ta kaw.”  Sabat  man ta  bubuo  ya,  “Miad  gani  daw  tekteken  a  no \textbf{aged}  \textbf{magtama}  \textbf{a}.”  Dayon  ambaļ  amo  i  na, “A,  dili  ta kaw  pagtekteken.  Daw  madakep ta kaw  kani,  pilak  ta kaw  naan  ta  suba  na  sikad daļem  \textbf{aged}  \textbf{malemmes}  \textbf{ka}.” \\
    \ea
    \gll Sabat  man  ta  bubuo  ya,  “Miad  gani  daw  pakpak-en  no tudtod  ko  i \\
    reply  also  \textsc{nabs}  tortoise  \textsc{def.f}  good  truly  if/when  pound.on-\textsc{t.ir} 2\textsc{s.erg} back  1\textsc{s.gen}  \textsc{def.n} \\
    \glt `The tortoise answered, “It is truly good if you pound on my back ... \\
    \ex
    \gll \textbf{aged}  \textbf{mag-ļapad}.”  Tapos,  ambaļ  ta  amo  ya  a, \\
    so.that  \textsc{i.ir}-wide  then  say  \textsc{nabs}  monkey  \textsc{def.f}  \textsc{inj} \\
    \glt so that it becomes wide.” Then, the monkey said, ...
    \ex
    \gll “A,  dili  ta kaw  nang  en  pag-pakpak-en.  Daw ma-dakep  ta kaw,  tektek-en  ta kaw.”  Sabat  man ta  bubuo  ya,  “Miad  gani  daw  tektek-en  a  no \\
    ah  \textsc{neg.ir}  1\textsc{s.erg} 2\textsc{s.abs}  only  \textsc{cm}  \textsc{nr.act}-pound.on-\textsc{t.ir}  if/when \textsc{a.hap.ir}-catch  1\textsc{s.erg} 2\textsc{s.abs} chop.up-\textsc{t.ir}  1\textsc{s.erg} 2\textsc{s.abs}  reply  also \textsc{nabs}  tortoise  \textsc{def.f}  good  truly  if/when  chop.up-\textsc{t.ir}  1\textsc{s.abs}  2\textsc{s.erg} \\
    \glt “Ah, I will not pound on you. If I catch you, I will chop you up.” The tortoise replied, “It is truly good if you chop me up ... \\
    \ex
    \gll \textbf{aged}  \textbf{mag-tama}  \textbf{a}.”  Dayon  ambaļ  amo  i  na, “A,  dili  ta kaw  pag-tektek-en.  Daw  ma-dakep ta kaw  kani,  pilak  ta kaw  naan  ta  suba  na  sikad daļem \\
    so.that  \textsc{i.ir}-many  1\textsc{s.abs}  then  say  monkey  \textsc{def.n}  \textsc{lk} ah  \textsc{neg.ir}  1\textsc{s.erg} 2\textsc{s.abs}  \textsc{nr.act}-pound.on-\textsc{t.ir}  if/when  \textsc{a.hap.ir}-catch 1\textsc{s.erg} 2\textsc{s.abs} later  throw.away\textsc{} 1\textsc{s.erg} 2\textsc{s.abs}  \textsc{spat.def}  \textsc{nabs} river \textsc{lk} very deep \\
    \glt so I will become many." Then the monkey said, “Ah, I will not chop you up. If I catch you later I will throw you in the river that is very deep ... \\
    \ex
    \gll \textbf{aged}  \textbf{ma-lemmes}  \textbf{ka}.” \\
    so.that  \textsc{a.hap.ir}-drown  2\textsc{s.abs} \\
    \glt so that you drown.”' [CBWN-C-16 9.15]
    \z
\z

Example \REF{bkm:Ref115099741} seems to be an exception to the tendency for \textit{aged} alone to express result. In this example, the subjunctive clause in \REF{bkm:Ref115099741b} clearly expresses the intended purpose of the activity described in the main clause, yet no linker follows \textit{aged}. In any case, purpose and result are similar concepts, and it is not surprising that their modes of expression overlap:

\ea
\label{bkm:Ref115099741}
Listo  ambaļ  kuti  i  na,  “Matay  ki  naan  ta kaoy  \textbf{aged}  \textbf{mabantaw}  ta  baļay  an  daw may  ittaw. \smallskip\\
    \ea
    \gll Listo  ambaļ  kuti  i  na,  “M-katay  ki  naan  ta kaoy   \\
    promptly  say  cat  \textsc{def.n  lk  i.v.ir}-climb  1\textsc{p.incl.abs}  \textsc{spat.def}  \textsc{nabs} tree/wood \\
    \glt `In a little time the cat said, “Let’s climb a tree \\\smallskip
\ex
    \label{bkm:Ref115099741b}
    PURPOSE \\
    \gll \textbf{aged}  \textbf{ma-bantaw}  ta  baļay  an  daw may  ittaw. \\
    so.that  \textsc{a.hap.ir}-look.far.off  1\textsc{p.incl.erg}  house  \textsc{def.m}  if/when \textsc{ext.in} person \\
    \glt so that we can look (from afar) at the house (to see) if there are people (there). [7.5]
    \z
\z

\largerpage
Purpose can also be expressed by \textit{para na} or simply \textit{para} before a subjunctive adverbial clause (\ref{bkm:Ref115100950} and \ref{bkm:Ref115100953}):

\ea
\label{bkm:Ref115100950}
Pag-abot  ta  Maynila  gadisisyon  a  pa  gid  na mangita  a  ta  ubra  \textbf{para}  \textbf{na}  \textbf{malipatan} \textbf{ko}  tanan-tanan  yon  na  mga  prublima. \smallskip\\
\gll Pag-abot  ta  Maynila  ga-disisyon  a  pa  gid  na ma-ng-ngita  a  ta  ubra  \textbf{para}  \textbf{na}  \textbf{ma-lipat-an} \textbf{ko}  tanan\sim{}tanan  yon  na  mga  prublima. \\
\textsc{nr.act}-arrive  \textsc{nasb}  Manila  \textsc{i.r}-decision  1\textsc{s.abs}  \textsc{inc}  \textsc{int}  \textsc{lk}
\textsc{a.hap.ir}-\textsc{pl}-search  1\textsc{s.abs}  \textsc{nabs}  work/do  for  \textsc{lk}  \textsc{a.hap.ir}-forget-\textsc{apl}
1\textsc{s.abs}  \textsc{red}\sim{}all  \textsc{d3abs}  \textsc{lk}  \textsc{pl}  problem \\
\glt `When arriving in Manila, I really decided that I will look for work \textbf{in} \textbf{order} \textbf{that} \textbf{I} \textbf{forget} completely all those problems.’ [JBON-J-01 4.1]
\z
\ea
\label{bkm:Ref115100953}
Duma  an  kalabanan  gabugtaw  mga  alas  kwarto  imidya asta  alas  singko  \textbf{para}  \textbf{magsin-ad}  \textbf{ta}  \textbf{kan-en}  \textbf{danen}. \smallskip\\
\gll Duma  an  ka-laban-an  ga-bugtaw  mga  alas  kwarto  imidya asta  alas  singko  \textbf{para}  \textbf{mag-sin-ad}  \textbf{ta}  \textbf{kan-en}  \textbf{danen}. \\
some  \textsc{def.m}  \textsc{nr}-most-\textsc{nr}  \textsc{i.r}-wake.up  \textsc{pl}  o’clock  four  thirty
until  o’clock  five  for  \textsc{i.ir}-cook.grain  \textsc{nabs}  cooked.rice  3\textsc{p.gen} \\
\glt `Almost all wake up about four thirty o’clock until five o’clock \textbf{in} \textbf{order} \textbf{to} \textbf{cook} \textbf{their} \textbf{rice}.’ [DBOE-C-03 1.2]
\z

\textit{Bag-o} ‘before/new/newly’ may also introduce subjunctive adverbial clauses:

\ea
\label{ex:govisiting}
Pag-abot  nay  ta  baļay  na  dayunan  nay,  gapuay kay  ta  pila  na  oras  \textbf{bag-o}  \textbf{manaw}  \textbf{na} \textbf{mamasyar}. \smallskip\\
\gll Pag-abot  nay  ta  baļay  na  dayon-an  nay,  ga-puay kay  ta  pila  na  oras  \textbf{bag-o}  \textbf{m-panaw}  \textbf{na} \textbf{ma-ng-pasyar}. \\
\textsc{nr.act}-arrive  1\textsc{p.excl.gen}  \textsc{nabs}  house  \textsc{lk}  continue-\textsc{nr}  1\textsc{p.excl.gen}  \textsc{i.r}-rest
1\textsc{p.excl.abs}  \textsc{nabs}  few  \textsc{lk}  hour/time  before  \textsc{i.v.ir}-go/walk  \textsc{lk}
\textsc{a.hap.ir-pl}-go.visiting \\
\glt `When we arrived at the house where we will stay, we rested \textbf{before} \textbf{leaving} \textbf{to} \textbf{go} \textbf{visiting}.’ [AGWN-L-01 3.2]
\z

\textit{Imbis} \textit{(na)} ‘instead of’ (pronounced \textit{imbís,} with stress on the final syllable) may also introduce subjunctive adverbial clauses, as in \REF{ex:gavethewinetopedro} and \REF{ex:driedupinthesun}:

\ea
\label{ex:gavethewinetopedro}
\textbf{Imbis}  \textbf{na}  \textbf{ambaļen  din}  iya  na   suguon  na  atagan Pedro  an,  kanen  i  gapanaog  ta  geddan  daw kanen  gid  gaatag  ta  aļak. \smallskip\\
\gll \textbf{Imbis}  \textbf{na}  \textbf{ambaļ-en}  \textbf{din}  iya  na   sugo-en  na  \emptyset{}-atag-an Pedro  an,  kanen  i  ga-panaog  ta  geddan  daw kanen  gid  ga-atag  ta  aļak. \\
instead.of \textsc{lk}  say-\textsc{t.ir} 3\textsc{s.erg}  3\textsc{s.gen}  \textsc{lk}  order-\textsc{t.ir}  \textsc{lk}  \textsc{t.ir}-give-\textsc{apl}
Pedro  \textsc{def.m}  3\textsc{s.abs}  \textsc{def.n}  \textsc{i.r}-go.downstairs  \textsc{nabs}  stairs  and
3\textsc{s.abs}  \textsc{int}  \textsc{i.r}-give  \textsc{nabs}  wine \\
\glt `\textbf{Instead} \textbf{of} \textbf{him} saying to his servant to give Pedro (some wine), as for him, (he) went down the stairs and he really gave the wine (to Pedro).’ [BEWN-T-01 3.4]
\z
\ea
\label{ex:driedupinthesun}
\textbf{Imbis}  \textbf{na}  tallo  na  mag-arey  maglangoy,  naubos  danen  ega  ta  adlaw. \smallskip\\
\gll \textbf{Imbis}  \textbf{na}  tallo  na  mag-arey  mag-langoy,  na-ubos  danen  ega  ta  adlaw. \\
instead  \textsc{lk}  three  \textsc{lk}  \textsc{rel}-friend  \textsc{i.ir}-bathe  \textsc{a.hap.r}-all  3\textsc{p.erg} dry  \textsc{nabs}  sun/day \\
\glt ‘\textbf{Instead} \textbf{of} the three friends bathing, they were all dried up by the sun.’ (This is a story abut a hermit crab, dragonfly and wasp who went to the river to bathe. But they fought over who would use the soap first and so they did not get to bathe but dried up in the sun.) [CDWN-T-01 3.9]
\z

Example \REF{bkm:Ref115874703} illustrates a construction with two subjunctive clauses, one embedded within the other. The clause introduced by \textit{daw} is a subjunctive adverbial clause. This clause is modified by another subjunctive adverbial clause introduced by \textit{na}:

\ea
\label{bkm:Ref115874703}
Kaon  nang  pagustuon  \textbf{daw}  \textbf{miling}  \textbf{ka}  \textbf{ta}  \textbf{Manila} \textbf{na}  \textbf{magtudlo}  ki  kanen  ta  ambaļ  ta. \smallskip\\
\gll Kaon  nang  pa-gusto-en  \textbf{daw}  \textbf{m-iling}  \textbf{ka}  \textbf{ta}  \textbf{Manila} \textbf{na}  \textbf{mag-tudlo}  ki  kanen  ta  ambaļ  ta. \\
2\textsc{s.abs}  only  \textsc{caus}-want/like-\textsc{t.ir}  if/when  \textsc{i.v.ir}-go  2\textsc{s.abs}  \textsc{nabs}  Manila
\textsc{lk}  \textsc{i.ir}-teach  \textsc{obl.p}  3s  \textsc{nabs}  say  1\textsc{p.incl.gen} \\
\glt ‘You are the one to do what you want \textbf{if} \textbf{you} \textbf{will} \textbf{go} \textbf{to} \textbf{Manila} \textbf{to} \textbf{teach/teaching/and} \textbf{teach} her our language.’ [RCON-L-03 4.4]
\z

The following are additional subjunctive adverbial clauses from the corpus:

\ea
Galangkag  a  \textbf{daw}  \textbf{muli}  kaw  di. \smallskip\\
\gll Ga-langkag  a  \textbf{daw}  \textbf{m-uli}  kaw  di. \\
\textsc{i.r}-expect  1\textsc{s.abs}  if/when  \textsc{i.v.ir}-go.home  2\textsc{p.abs}  \textsc{d}1\textsc{loc} \\
\glt ‘I am expectantly waiting whether/if/when you would/will come home here.’
\z
\ea
Gaduwa-duwa  a  \textbf{daw}  \textbf{mabang}  kanen  an  ki  yaken. \smallskip\\
\gll Ga-duwa-duwa  a  \textbf{daw}  \textbf{m-tabang}  kanen  an  ki  yaken. \\
\textsc{i.r}-waver/doubt  1\textsc{s.abs}  if/when  \textsc{i.v.ir}-help  3\textsc{s.abs}  \textsc{def.m}  \textsc{obl.p}  1s \\
\glt ‘I am wavering/doubting \textbf{whether/if/when} \textbf{s/he} \textbf{would/will} \textbf{help} me.’
\z

Some examples of verbs that do not allow subjunctive adjuncts introduced by \textit{daw} include:

\ea
*Gaplano kanen daw magbakasyon. \\
(‘S/he planned whether to go on vacation.’) \smallskip\\
*Gadisisyon kanen daw miling to Manila. \\ (‘S/he decided whether to go to Manila.’) \smallskip\\
*Gaandem a daw mag-iskwila a. \\
(‘I am desiring whether to go to school.’) \smallskip\\
*Lagpat din nang daw muli ka. \\
(‘S/he guessed whether you will go home.’)
\z

Finally, in \REF{bkm:Ref474828944} there are two recursively embedded complement clauses. The first (CC1) is the subjunctive complement of \textit{gusto ko} ‘I want’, while the second (CC2) is the action nominalization complement of \textit{tapuson ko} ‘I will finish’:

\ea
\label{bkm:Ref474828944}
Piro  gaprusigir  kay  gid  para  na  makaiskwila  a  nang tak  gusto  ko  \textbf{na}  \textbf{tapuson}  \textbf{ko} \textbf{ake}  \textbf{na}  \textbf{pag-iskwila}. \smallskip\\

\gll Piro  ga-prusigir  kay  gid  para  na  maka-iskwila  a  nang \\
but  \textsc{i.r}-persevere  1\textsc{p.exc.abs}  \textsc{int}  \textsc{purp}  \textsc{lk}  \textsc{i.hap.ir}-school  1\textsc{s.abs}  only  \\
\glt \hspace{3cm}[\hspace{1.2cm}CC1\hspace{1.2cm} [\hspace{1.6cm}CC2\hspace{1.6cm}]] \\
\gll tak  gusto  ko  \textbf{na}  \textbf{tapuson}  \textbf{ko}  \textbf{ake}  \textbf{na}  \textbf{pag-iskwila}. \\
because  want  1\textsc{s.erg}     \textsc{lk}  finish-\textsc{t.ir}  1\textsc{s.erg}  1\textsc{s.gen}  \textsc{lk}  \textsc{nr.act}-school \\
\glt `But we just really persevere in order that I can just go to school because I want \textbf{to} \textbf{finish} \textbf{my} \textbf{education}.’ [PBWL-T-06 4.2] \\
\is{subjunctive adverbial clauses|)}
\is{adverbial clauses!subjunctive|)}
\is{subjunctive|)}
\is{dependent clauses!subjunctive|)}
\is{subjunctive clauses|)}
\z
\section{Finite dependent clauses}
\label{bkm:Ref474473773}
\is{finite dependent clauses|(}
\is{dependent clauses!finite|(}There are several dependent clause constructions in Kagayanen that are grammatically fully finite in that they directly express all the transitivity and modality information that fully independent clauses do. However, we still want to say they are “dependent” because they do not normally stand on their own as independent assertions. They are always introduced with a complementizer, conjunction or introducer that links them to a fully independent clause in the near vicinity. In this section we will describe the various functions of finite dependent clauses.


\subsection{Finite complement clauses}

There are two words that function as \textit{complementizers}\is{complementizers} in Kagayanen. These are \textit{na} and \textit{daw} both of which have other functions as well. The particle \textit{na} is, of course, the “linker” that joins modifiers to their heads in noun phrases, while \textit{daw} is a conjunction that introduces conditional clauses, as well as simply coordinating two structures of the same construction type (e.g. RP+RP, CL+CL, etc.). In this section we illustrate the use of these forms to introduce fully finite clausal arguments, also known as complement clauses.


\subsubsection{Finite \textit{na} complement clauses}
\label{sec:na-complementclauses}

Clausal arguments may be headed by fully finite verbs introduced with the linker \textit{na} functioning as a complementizer. Matrix verbs that take finite complement clauses introduced by \textit{na} include the following:

\ea
Verbs of utterance \\
Verbs of perception \\
Phasal verbs (to begin to finish) \\
Many cognition verbs \\
\textit{dengngan}   ‘to do something at the same time as X’ \\
\textit{kuyog}   ‘to go with someone to do X’  \\
\textit{pakita}   ‘to show someone that you are doing X’  \\
\textit{pasalamat}   ‘to give thanks that X happened’  \\
\textit{sadya}   ‘to be happy that X happened’  \\
\textit{una}   ‘to do X first’ \\
\textit{untat}   ‘to stop doing X’
\z

Without the complementizer, the two clauses are understood as just two independent clauses. The following conversational examples illustrate that these clausal arguments are truly finite in that they may be headed by realis or irrealis verbs. They are also argumental, in that they fill the absolutive role in a grammatically transitive matrix construction:

\ea
Paumpisaan  din  en  \textbf{na}  \textbf{gatanem}  ta  kamuti. \smallskip\\
\gll Pa-umpisa-an  din  en  \textbf{na}  \textbf{ga-tanem/ga-ng-tanem}  ta  kamuti. \\
\textsc{t.r}-start-\textsc{apl}  3\textsc{s.erg}  \textsc{cm}  \textsc{lk}  \textsc{i.r}-plant/\textsc{i.r}-\textsc{pl}-plant  \textsc{nabs}  cassava \\
\glt ‘S/he started planting cassava.’
\z
\ea
Paumpisaan  din  en  \textbf{na}  \textbf{magtanem}  ta  kamuti. \smallskip\\
\gll Pa-umpisa-an  din  en  \textbf{na}  \textbf{mag-tanem/ma-ng-tanem}  ta  kamuti. \\
\textsc{t.r}-start-\textsc{apl}  3\textsc{s.erg}  \textsc{cm}  \textsc{lk}  \textsc{i.ir}-plant\textsc{/a.hap.ir}-\textsc{pl}-plant  \textsc{nabs}  cassava \\
\glt ‘S/he is just starting to plant cassava (but might not have planted anything yet).’
\z
\ea
Gaumpisa  en  kanen  an  \textbf{na}  \textbf{gatanem}  ta  kamuti. \smallskip\\
\gll Ga-umpisa  en  kanen  an  \textbf{na}  \textbf{ga-tanem/ga-ng-tanem}  ta  kamuti. \\
\textsc{i.r}-start  \textsc{cm}  3\textsc{s.abs}  \textsc{def.m}  \textsc{lk}  \textsc{i.r}-plant\textsc{/i.r}-\textsc{pl}-plant  \textsc{nabs}  cassava \\
\glt ‘S/he started planting cassava.’ (This means she is now already planting it.)
\z
\ea
Gaumpisa  en  kanen  an  \textbf{na}  \textbf{magtanem}  ta  kamuti. \smallskip\\
\gll Ga-umpisa  en  kanen  an  \textbf{na}  \textbf{mag-tanem/ma-ng-tanem}  ta  kamuti. \\
\textsc{i.r}-start  \textsc{cm}  3\textsc{s.abs}  \textsc{def.m}  \textsc{lk}  \textsc{i.ir}-plant\textsc{/a.hap.ir}-\textsc{pl}-plant  \textsc{nabs}  cassava \\
\glt ‘S/he is just starting to plant cassava.’ (This means s/he is just now beginning, but possibly hasn’t planted anything yet.)
\z

In addition to aspect/modality CTPs such as \textit{umpisa} ‘start’,  and \textit{gauntat} ‘stop’, many cognition, emotion, perception and utterance constructions may also contain finite clausal arguments introduced by \textit{na}:

\ea
Cognition: \\
Nļaman  ko  \textbf{na}  \textbf{gaiskwila}  ka  en. \smallskip\\
\gll Na-aļam-an  ko  \textbf{na}  \textbf{ga-iskwila}  ka  en. \\
\textsc{a.hap.r}-know-\textsc{apl}  1\textsc{s.erg}  \textsc{lk}  \textsc{i.r}-school  2\textsc{s.abs}  \textsc{cm} \\
\glt ‘I know that you already went to school.’
\z
\ea
Nļaman  no  \textbf{na}  \textbf{kaselled}  aren  ta  ubra  unti … \smallskip\\
\gll Na-aļam-an  no  \textbf{na}  \textbf{ka-selled}  aren  ta  ubra  unti … \\
\textsc{a.hap.r}-know-\textsc{apl}  2\textsc{s.erg}  \textsc{lk}  \textsc{i.exm}-inside  1\textsc{s.abs}  \textsc{nabs}  work  \textsc{d}1\textsc{loc.pr} \\
\glt ‘You know that  I got to enter into work here ...’ [DBWL-T-20 8.5]
\z
\ea
… uļa  kay  naļam  na  galarga  danen  an. \smallskip\\
\gll … uļa  kay  na-aļam  na  ga-larga  danen  an. \\
{} \textsc{neg.r}  1\textsc{p.exc.abs}  \textsc{a.hap.r}-know  \textsc{lk}  \textsc{i.r}-depart  3\textsc{p.abs}  \textsc{def.m} \\
\glt ‘… we did not know that  they are departing.’ [PBWL-T-05 3.4]
\z
\ea
Direct speech: \\
Dayon  kamati  kanen  i  ta  gatagbaļay  na gaambaļ  \textbf{na}  “\textbf{Dili}  \textbf{ka}  \textbf{madlek}  bļai  tak yaken  magduaw  a  nang  ta  sapping  yan.” \smallskip\\
\gll Dayon  ka-mati  kanen  i  ta  ga-tagbaļay  na ga-ambaļ  \textbf{na}  “\textbf{Dili}  \textbf{ka}  \textbf{ma-adlek}  bļai  tak yaken  mag-duaw  a  nang  ta  sapping  yan.” \\
right.away  \textsc{i.exm}-hear  3\textsc{s.abs}  \textsc{def.n}  \textsc{nabs}  \textsc{i.r}-owner.of.house  \textsc{lk}
\textsc{i.r}-say  \textsc{lk}  \textsc{neg.ir}  2\textsc{s.abs}  \textsc{a.hap.ir}-afraid  fellow.parents.in.law  because
1\textsc{s.abs}  \textsc{i.ir}-visit  1\textsc{s.abs}  only  \textsc{nabs}  twins  \textsc{def.m} \\
\glt `Right away she heard one calling the owner of the house saying, “Don’t be afraid fellow mother in law because \textsc{as for me} I am only visiting the twins.”' [MBON-T-05 3.9]
\z
\ea
Indirect speech: \\
… sabaten  no  \textbf{na}  \textbf{gabalyo}  ki  ta  bayo tak  natignawan  ka. \smallskip\\
\gll … sabat-en  no  \textbf{na}  \textbf{ga-balyo}  ki  ta  bayo tak  na-tignaw-an  ka. \\
{} reply-\textsc{t.ir}  2\textsc{s.erg}  \textsc{lk}  \textsc{i.r}-change  1\textsc{p.incl.abs}  \textsc{nabs}  clothes
because  \textsc{a.hap.r}-cold-\textsc{apl}  2\textsc{s.erg} \\
\glt ‘… answer that we exchanged clothes because you were cold.’ [CBWN-C-25 10.5]
\z
\ea
Perception: \\
Nakita  din  \textbf{na}  \textbf{gaitem}  inog  an  na  lumboy. \smallskip\\
\gll Na-kita  din  \textbf{na}  \textbf{ga-item}  inog  an  na  lumboy. \\
\textsc{a.hap.r}-see  3\textsc{s.erg}  \textsc{lk}  \textsc{i.r}-black  ripe  \textsc{def.m}  \textsc{lk}  java.plum \\
\glt ‘He saw that the ripe java plums were becoming black.’ [JCWN-L-38 7.2]
\z

\subsubsection{Finite \textit{daw} complement clauses}

Finite clauses that function as arguments of certain stative roots expressing cognition and emotion may be introduced by \textit{daw} `if/when' functioning as a complementizer. In many cases, these may be understood as \textit{indirect questions}\is{indirect questions} For example, in \REF{bkm:Ref472922620}, the complement (bolded) expresses the content of what the Actor knows, namely the answer to the question “Did you go to school?”:

\ea
\label{bkm:Ref472922620}
Nļaman  ko  \textbf{daw}  \textbf{gaiskwila}  ka  en. \smallskip\\
\gll Na-aļam-an  ko  \textbf{daw}  \textbf{ga-iskwila}  ka  en. \\
\textsc{a.hap.r}-know-\textsc{apl}  1\textsc{s.erg}  \textsc{comp}  \textsc{i.r}-go.to.school  2\textsc{s.abs}  \textsc{cm} \\
\glt ‘I know whether/if/when you went to school.’
\z
\ea
Nļaman  ko    \textbf{daw}  \textbf{miskwila}  ka. \smallskip\\
\gll Na-aļam-an  ko    \textbf{daw}  \textbf{m-iskwila}  ka. \\
\textsc{a.hap.r}-know-\textsc{apl}  1\textsc{s.erg}    \textsc{comp}  \textsc{i.v.ir}-go.to.school  2\textsc{s.abs} \\
\glt ‘I know whether/if/when you will go to school.’
\z

The stative root \textit{salig} ‘think wrongly’ often occurs with a complement clause introduced with \textit{daw} ‘if/when’:

\ea
Salig  ko  \textbf{daw}  \textbf{galarga}  kaw  en. \smallskip\\
\gll Salig  ko  \textbf{daw}  \textbf{ga-larga}  kaw  en. \\
think.wrongly  1\textsc{s.erg}  if/when  \textsc{i.r}-depart/travel  2\textsc{s.abs}  \textsc{cm} \\
\glt ‘I thought wrongly that you already departed.’
\z
\ea
Salig  ko  \textbf{daw}  \textbf{marga}  kaw  kani. \smallskip\\
\gll Salig  ko  \textbf{daw}  \textbf{m-larga}  kaw  kani. \\
think.wrongly  1\textsc{s.erg}  if/when  \textsc{i.v.ir}-depart/travel  2\textsc{s.abs}  later \\
\glt ‘I thought wrongly that you will depart later.’
\z

The verb \textit{insa/ansa/ensa} ‘to ask a question’ may express a direct or indirect question in its complement clause. When the complement clause is an indirect question, the complementizer is \textit{daw} (exx. \ref{bkm:Ref474410675} and \ref{bkm:Ref474410677}). If it is a direct question, then \textit{na} may occur, or the complementizer may be dropped \REF{bkm:Ref116120188}.

\ea
\label{bkm:Ref474410675}
Dayon  ka  din  insaan  \textbf{daw}  \textbf{naļam}  ka  en  maggalang. \smallskip\\
\gll Dayon  ka  din  \emptyset{}-insa-an  \textbf{daw}  \textbf{na-aļam}  ka  en  mag-galang. \\
right.away  2\textsc{s.abs}  3\textsc{s.erg}  \textsc{t.ir}-ask-\textsc{apl}  \textsc{comp}  \textsc{a.hap.r}-know  2\textsc{s.abs}  \textsc{cm}  \textsc{i.ir}-respect \\
\glt ‘Right away he asks you whether you know how to give respect.’\footnote{When \textit{dayon} occurs after the verb it means ‘immediately’, but when it occurs sentence initially, it is working more on the discourse level to mark the following clause as heightened tension leading up to the climax. In such examples we translate \textit{dayon} as ‘right away.’ It occurs too often in narratives for all such examples to be accurately translated with ‘immediately,’ though we consistently gloss \textit{dayon} in that way.} [JCWN-T-20 5.2]
\z
\ea
\label{bkm:Ref474410677}
Dayon  ka  din  insaan  \textbf{daw}  \textbf{misab}  ka  pa. \smallskip\\
\gll Dayon  ka  din  insa-an  \textbf{daw}  \textbf{m-isab}  ka  pa. \\
right.away  2\textsc{s.abs}  3\textsc{s.erg}  ask-\textsc{apl}  \textsc{comp}  \textsc{i.v.ir}-do.again  2\textsc{s.abs}  \textsc{inc} \\
\glt ‘Right away he asks you whether you will still do it again.’ [JCWN-T-20 6.3]
\z
\ea
\label{bkm:Ref116120188}
Dayon  ka  din  insaan  \textbf{(na),}  \textbf{“Naļam}  ka  en maggalang?” \smallskip\\
\gll Dayon  ka  din  insa-an  \textbf{(na),}  \textbf{“Na-aļam}  ka  en mag-galang?” \\
right.away  2\textsc{s.abs}  3\textsc{s.erg}  ask-\textsc{apl}  \textsc{comp}  \textsc{a.hap.r}-know  2\textsc{s.abs}  \textsc{cm}
\textsc{i.ir}-respect \\
\glt `Right away he asks you “Do you know how to give respect?”’
\z
\subsection{Finite adverbial clauses}
\label{bkm:Ref460483264} \label{sec:finiteadverbialclauses}\is{adverbial clauses!finite|(}

In addition to finite complement clauses, there are adjunct clauses introduced by a complementizer or other introducer. Such clauses are “adverbial” in that they are normally optional from a syntactic point of view, and they do not fill any argument roles. They also tend strongly to express \textit{presuppositions}\is{presuppositions} rather than \textit{assertions}\is{assertions}. In other words, the information they express is usually background and already established, rather than the main new information in the construction. Finite adverbial clauses may be introduced by one of the introducers listed in \REF{bkm:Ref116282225}. Many of these also introduce other types of dependent clauses, as discussed in previous sections. The last four of these are borrowings from Spanish, and all retain roughly their Spanish meanings:

\ea
\label{bkm:Ref116282225}
\begin{tabbing}
Introducer \hspace{.3cm} \= linker/complementizer \hspace{.4cm}\= \kill
Introducer \>  Gloss \>         Interpropositional Relation \\
\textit{na  } \>  \textsc{linker/complementizer}   (various) \\
\textit{daw  } \>  ‘if/when’  \> Condition (and others) \\
\textit{tak  } \>  ‘because’    \>      Reason (+realis)/Purpose (+irr) \\
\textit{tenged } \>   ‘because’   \>       Reason \\
\textit{kumo  } \>  ‘because’       \>    Reason \\
\textit{gani na } \>   ‘and so’       \>    Result \\
\textit{samtang } \>   ‘while’      \>    Simultanaity \\
\textit{bisan/baskin} \> ‘even though'/'even if’ \>  Concession \\
\textit{maskin}  \\
\textit{asta } \>   ‘until’/‘and so’      \>     Result, bounded time \\
\textit{tapos (na) } \>  ‘after’         \>  Posteriority \\
\textit{antes } \>   ‘before’/‘while’      \>     Anteriority, Simultanaity \\
\textit{mintras } \>   ‘while’ \>          Simultanaity \\
\textit{para  } \>  ‘for’/'purpose'    \>      Purpose
\end{tabbing}
\z

The following are examples of various finite adverbial clauses from the corpus, organized according to the interpropositional relations expressed:

\ea
Finite time adverbial clause with \textit{na} introducer and realis verb: \\
\textbf{Na}  \textbf{gapanaw}  en  iya  na  manong  gabalikid  kanen  ta iya  na  utod  tak  naadlek  kanen  basi  may  matabo. \smallskip\\
\gll \textbf{Na}  \textbf{ga-panaw}  en  iya  na  manong  ga-balikid  kanen  ta iya  na  utod  tak  na-adlek  kanen  basi  may  ma-tabo. \\
\textsc{lk}  \textsc{i.r}-go/walk  \textsc{cm}  3\textsc{s.gen}  \textsc{lk}  older.brother  \textsc{i.r}-look.back  3\textsc{s.abs}  \textsc{nabs}
3\textsc{s.gen}  \textsc{lk}  sibling  because  \textsc{a.hap.r}-afraid  3\textsc{s.abs}  perhaps  \textsc{ext.in}  \textsc{a.hap.ir}-happen \\
\glt `When his other brother left, he looked back to his sibling because he was afraid that perhaps something will happen.’ [EDWN-T-04 2.3]
\z
\ea
\textbf{Na}  \textbf{nalebbeng}  en  ake  na  bayaw  gaimes  kay en  na  muli  ta  Manila. \smallskip\\
\gll \textbf{Na}  \textbf{na-lebbeng}  en  ake  na  bayaw  ga-imes  kay en  na  m-uli  ta  Manila. \\
\textsc{lk}  \textsc{a.hap.r}-bury  \textsc{cm}  1\textsc{s.gen}  \textsc{lk}  sibling.in.law  \textsc{i.r}-prepare  1\textsc{p.excl.abs}
\textsc{cm}  \textsc{lk}  \textsc{i.v.ir}-go.home  \textsc{nabs}  Manila \\
\glt `\textbf{When} \textbf{my} \textbf{brother-in-law} \textbf{had} \textbf{been} \textbf{buried} we were preparing to go home to Manila.’ [VAWN-T-15  6.8]
\z
\ea
Finite adverbial clause with \textit{na} introducer and irrealis verb: \\
\textbf{Na}  \textbf{mag-adlaw-adlaw}  en  nabantawan  ta  mga  ittaw naan  ta  iran  na  lugar  sakayan  i. \smallskip\\
\gll \textbf{Na}  \textbf{mag-adlaw\sim{}adlaw}  en  na-bantaw-an  ta  mga  ittaw naan  ta  iran  na  lugar  sakayan  i. \\
\textsc{lk}  \textsc{i.ir-red}\sim{}sun/day  \textsc{cm}  \textsc{a.hap.r-}look.far.off\textsc{-apl}  \textsc{nabs}  \textsc{pl}  person
\textsc{spat.def}  \textsc{nabs}  3\textsc{p.gen}  \textsc{lk}  place  veheicle  \textsc{def.n} \\
\glt `\textbf{When} \textbf{it} \textbf{was} \textbf{about} \textbf{to} \textbf{become} \textbf{day} the people in their place  could look far off (and see) the vehicle.’ [DBWN-T-33 2.28]
\z
\ea
Finite time adverbial clause with \textit{na} introducer sentence finally: \\
Paambargo  din  pawikan,  daing  ta  nanagat  alin ta  Cavili  daw  Maļayang  \textbf{na}  \textbf{gauli}  \textbf{en}  \textbf{ta}  Cagayancillo. \smallskip\\
\gll Pa-ambargo  din  pawikan,  daing  ta  na-ng-dagat  alin ta  Cavili  daw  Maļayang  \textbf{na}  \textbf{ga-uli}  \textbf{en}  \textbf{ta}  Cagayancillo. \\
\textsc{t.r}-take.by.force  3\textsc{s.erg}  sea.turtle  dried.fish  \textsc{nabs}  \textsc{a.hap.r-pl}-sea  from
\textsc{nabs}  Cavili  and  Malayang  \textsc{lk}  \textsc{i.r}-go.home  \textsc{cm}  \textsc{nabs}  Cagayancillo \\
\glt `He took by force sea turtle, dried fish of the ones who fish from Cavili and Malayang \textbf{when} \textbf{going} \textbf{home} \textbf{to} \textbf{Cagayancillo}.’ [BEWN-T-01 2.8]
\z
\ea
Ame  na  Lolo  galarga  \textbf{na}  \textbf{naabutan}    \textbf{ta}  \textbf{bagyo}. \smallskip\\
\gll Ame  na  Lolo  ga-larga  \textbf{na}  \textbf{na-abot-an}    \textbf{ta}  \textbf{bagyo}. \\
1\textsc{p.excl.gen}  \textsc{lk}  grandfather  \textsc{i.r}-depart  \textsc{lk}  \textsc{a.hap.r}-arrive-\textsc{apl}  \textsc{nabs}  typhoon \\
\glt ‘Our grandfather departed \textbf{when} \textbf{a} \textbf{typhoon} \textbf{came} \textbf{upon} \textbf{him}.’ [CBWN-T-27 2.1]
\z
\ea
Finite time adverbial clause with \textit{daw} introducer and irrealis verb: \\
\textbf{Daw}  \textbf{mag-iling}  kay  ta  Cebu  ame  na  sakayan pambot  daw  lansa  man. \smallskip\\
\gll \textbf{Daw}  \textbf{mag-iling}  kay  ta  Cebu  ame  na  sakay-an pambot  daw  lansa  man. \\
if/when  \textsc{i.ir}-go  1\textsc{p.excl.abs}  \textsc{nabs}  Cebu  1\textsc{p.excl.gen}  \textsc{lk}  ride-\textsc{nr} motorboat  and  launch  too \\
\glt `When we go to Cebu our vehicle is a motorboat and launch too.’ [DBWN-T-23 2.1]
\z
\ea
Finite time adverbial clause with \textit{daw} introducer and realis verb: \\
\textbf{Daw}  \textbf{gakaang}  luy-a  kamangen  din  luy-a  an  daw  paid din  ta  gettek … \smallskip\\
\gll \textbf{Daw}  \textbf{ga-kaang}  luy-a  kamang-en  din  luy-a  an  daw  paid din  ta  gettek … \\
if/when  \textsc{i.r}-hot.taste  ginger  get-\textsc{t.ir}  3\textsc{s.erg}  ginger  \textsc{def.m}  and  wipe
3\textsc{s.erg}  \textsc{nabs}  stomach \\
\glt `When the ginger becomes hot, he gets the ginger and he wipes (it) on the stomach …’ [VAOE-J-04 VAOE-J-04 3.7]’
\z

\largerpage
\ea
Finite time adverbial clause with \textit{asta} ‘until’ and irrealis verb: \\
Ta  pag-abot  en  ta  alas  dos  na  mapon  gabantay kay  en  ta  sakayan  na  pauli  en  ta  ame na  ulian,  \textbf{asta}  \textbf{na}  \textbf{makasakay}  \textbf{kay}  \textbf{en}. \smallskip\\
\gll Ta  pag-abot  en  ta  alas  dos  na  mapon  ga-bantay kay  en  ta  sakay-an  na  pa-uli  en  ta  ame na  uli-an,  \textbf{asta}  \textbf{na}  \textbf{maka-sakay}  \textbf{kay}  \textbf{en}. \\
so  \textsc{nr.act}-arrive  \textsc{cm}  \textsc{nabs}  o’clock  two  \textsc{lk}  afternoon  \textsc{i.r}-watch
1\textsc{p.excl.abs}  \textsc{cm}  \textsc{nabs}  ride-\textsc{nr}  \textsc{lk}  \textsc{caus}-go.home  \textsc{cm}  \textsc{nabs}  1\textsc{p.excl.gen}
\textsc{lk}  go.home-\textsc{nr}  until  \textsc{lk}  \textsc{i.hap.ir}-ride  1\textsc{p.excl.abs}  \textsc{cm} \\
\glt `So when two o’clock afternoon arrived we were watching for a vehicle to go homewards to the place where we go home to \textbf{until} \textbf{we} \textbf{will} \textbf{be} \textbf{able} \textbf{to} \textbf{ride}.’ [RMWN-L-01 9.1]
\z
\ea
Finite time adverbial clause with \textit{asta} ‘until’ and realis verb: \\
Pag-arya  nang  pa  ta  layag  danen  pabumbaan  ta  mga Pilipino  mga  sakayan  ya  danen  \textbf{asta}  \textbf{na}  \textbf{napatay}  danen  ya  tanan. \smallskip\\
\gll Pag-arya  nang  pa  ta  layag  danen  pa-bumba-an  ta  mga Pilipino  mga  sakay-an  ya  danen  \textbf{asta}  \textbf{na}  \textbf{na-patay}  danen  ya  tanan. \\
\textsc{nr.act}-raise  just  \textsc{inc}  \textsc{nabs}  sail  3\textsc{p.abs}  \textsc{t.r}-bomb-\textsc{apl}  \textsc{nabs}  \textsc{pl}
Filipino  \textsc{pl}  ride-\textsc{nr}  \textsc{def.f}  3\textsc{p.gen}  until  \textsc{lk}  \textsc{a.hap.r}-kill  3\textsc{p.abs}  \textsc{def.f}  all \\
\glt `When just still raising their sail, Filipinos bombed their vehicles (boats) \textbf{until} \textbf{they} \textbf{all} \textbf{died}.’ [ICWN-T-04 3.13]
\z

The introducer \textit{asta}, normally glossed as 'until', can also express ideas similar to ‘and also’/ ‘and even’ or it may introduce a result of the activity in the main clause.

\ea
Ta,  bayad  ta  ambaļ  no  ya  na  nadlek  a \textbf{asta}  man  en  na  buksoļ  man  takong  ko  i. \smallskip\\
\gll Ta,  bayad  ta  ambaļ  no  ya  na  na-adlek  a \textbf{asta}  man  en  na  buksoļ  man  takong  ko  i. \\
so  pay  \textsc{nabs}  say  2\textsc{s.erg}  \textsc{def.f}  \textsc{lk}  \textsc{a.hap.r}-afraid  1\textsc{s.abs}
until  \textsc{emph}  \textsc{cm}  \textsc{lk}  lump  also  forehead  1\textsc{s.gen}  \textsc{def.n} \\
\glt `So, you paid for what you said so that I was afraid and the \textbf{result} \textbf{(was)} a lump on my forehead.’ [CON-L-07 13.1]
\z
\ea
Pagkatapos  tan,  dayon  ka  nang  dļagan  en  \textbf{asta}  ka  nang tumbalik  ta  dļagan. \smallskip\\
\gll Pag-ka-tapos  tan,  dayon  ka  nang  dļagan  en  \textbf{asta}  ka  nang tumbalik  ta  dļagan. \\
\textsc{nr.act}-\textsc{vr}-finish  \textsc{d}1\textsc{nabs}  right.away  2\textsc{s.abs}  only  run  \textsc{cm}  until  2\textsc{s.abs}  only
fall.backward  \textsc{nabs}  run \\
\glt `After that, then you ran \textbf{until} \textbf{the} \textbf{result} (was) you fell backward from running.’ [JCOE-T-06 8.14]
\z
\ea
Finite adverbial Clause with \textit{gani na} ‘and so’ and realis verb: \\
Kanen  gatabang  man  yaken  ta  pagtabog  ta  sidda pilis  \textbf{gani}  \textbf{na}  \textbf{tama}  \textbf{kamang}  \textbf{ko}. \smallskip\\
\gll Kanen  ga-tabang  man  yaken\footnotemark{}  ta  pag-tabog  ta  sidda pa-ilis  \textbf{gani}  \textbf{na}  \textbf{tama}  \textbf{kamang}  \textbf{ko}. \\
3\textsc{s.abs}  \textsc{i.r}-help  \textsc{emph}  1s  \textsc{nabs}  \textsc{nr.act}-drive.away  \textsc{nabs}  fish \textsc{caus}-close.to.shore  so  \textsc{lk}  many  get  1\textsc{s.erg} \\
\footnotetext{Here \textit{ki} ‘\textsc{oblique, personal name}.’ has been dropped before \textit{yaken} in conversation. Speakers agree that \textit{ki} would precede \textit{yaken} in more careful speech.}
\glt ‘As for him, (he) even helped me driving the fish going close to shore \textbf{and} \textbf{so} \textbf{I} \textbf{got} \textbf{many}.’ [JCWN-L-31 25.3]
\z

\ea
\label{bkm:Ref116285228}
Finite adverbial clause with \textit{antes} ‘while’ and irrealis verb sentence initially:  \\
\textbf{Antes}  \textbf{malin}  \textbf{kay}  ta  dagsayan  mwa  ta  ļangan dasig  ļagan  ta  lunday  tak  may  angin  na  gaeyep. \smallskip\\
\gll \textbf{Antes}  \textbf{m-alin}  \textbf{kay}  ta  dagsayan  m-gwa  ta  ļangan dasig  dļagan  ta  lunday  tak  may  angin  na  gaeyep. \\
while  \textsc{i.v.ir}-from  1\textsc{p.excl.abs}  \textsc{nabs}  shore  \textsc{i.v.ir}-go.out  \textsc{nabs}  coral.reef
fast  run  \textsc{nabs}  outriggor.canoe  because  \textsc{ext.in}  wind  \textsc{lk}  \textsc{i.r}-blow \\
\glt `\textbf{While} \textbf{we} \textbf{were} \textbf{about} \textbf{to} \textbf{leave} the shore going out to the coral reef, our outriggor canoe was running fast because there was a wind blowing.’ [EDWN-T-05 2.5]
\z
\ea
Finite adverbial clause with \textit{antes} ‘while’ and realis verb sentence initially: \\
\textbf{Antes}  \textbf{galin}  \textbf{kay}  ta  dagsayan  mwa  ta  ļangan dasig  ļagan  ta  lunday  tak  may  angin  na  gaeyep. \smallskip\\
\gll \textbf{Antes}  \textbf{ga-alin}  \textbf{kay}  ta  dagsayan  m-gwa  ta  ļangan dasig  dļagan  ta  lunday  tak  may  angin  na  ga-eyep. \\
while  \textsc{i.r}-from  1\textsc{p.excl.abs}  \textsc{nabs}  shore  \textsc{i.v.ir}-go.out  \textsc{nabs}  coral.reef
fast  run  \textsc{nabs}  outrigger.canoe  because  \textsc{ext.in}  wind  \textsc{lk}  \textsc{i.r}-blow \\
\glt `\textbf{When} \textbf{we} \textbf{left} the shore going out to the coral reef, our outriggor canoe was running fast because there was a wind blowing.’ (Elicited example contrasting with \ref{bkm:Ref116285228}.)
\z

\ea
Finite adverbial clause with \textit{mintras} ‘while’ and realis clause: \\
… daw  pirmi  kay  balik-balik  iran  na  baļay  \textbf{mintras}  \textbf{na} \textbf{gatagad}  \textbf{kay}  ta  sakayan  na  muli  ta  Cagayancillo. \smallskip\\
\gll … daw  pirmi  kay  balik-balik  iran  na  baļay  \textbf{mintras}  \textbf{na} \textbf{ga-tagad} \textbf{kay}  ta  sakay-an  na  m-uli  ta  Cagayancillo. \\
{} and  always  1\textsc{p.excl.abs}  \textsc{red}-return  3\textsc{p.gen}  \textsc{lk}  house  while  \textsc{lk}
\textsc{i.r}-wait  1\textsc{p.excl.abs}  \textsc{nabs}  ride-\textsc{nr}  \textsc{lk}  \textsc{i.v.ir}-go.home  \textsc{nabs}  Cagayancillo. \\
\glt `… and we always kept returning to their place \textbf{while} \textbf{we} \textbf{were} \textbf{waiting} for a vehicle going home to Cagayancillo.’ [NEWN-T-05 2.16]
\z

\ea
Finite adverbial clause with \textit{samtang} ‘while’ and realis clause in initial position: \\
Piro,  \textbf{samtang}  \textbf{gatamba}  \textbf{daen}  \textbf{i},  bakod  gid  iran  i  na katingaļa  tak  paryo  ta  sikad  en  tama  daen  i na  galabet  ta  pagtamba. \smallskip\\
\gll Piro,  \textbf{samtang}  \textbf{ga-tamba}  \textbf{daen}  \textbf{i},  bakod  gid  iran  i  na ka-tingaļa  tak  paryo  ta  sikad  en  tama  daen  i na  ga-labet  ta  pag-tamba. \\
but  while  \textsc{i.r}-slap.water  3\textsc{p.abs}  \textsc{def.n}  big  \textsc{int}  3\textsc{p.gen}  \textsc{def.n}  \textsc{lk}
\textsc{nr}-amaze  because  like  \textsc{nabs}  very  \textsc{cm}  many  3\textsc{p.abs}  \textsc{def.n}
\textsc{lk}  \textsc{i.r}-participate  \textsc{nabs}  \textsc{nr.act}-slap.water \\
\glt ‘But, \textbf{while} \textbf{they} \textbf{were} \textbf{slapping} \textbf{the} \textbf{seawater}, their amazement was really big because it was like there were many of them who were participating in slapping the sea water.’ (People slap the sea water to drive fish into their nets. In this text, there were only four who went fishing but while they were slapping the water, suddenly many more appeared. The others were evil spirits.) [JPWN-L-01 4.4]
\z

\ea
Finite adverbial clauses with \textit{tak} ‘because’ indicating purpose or reason:   \\
Manong  megbeng  ka  \textbf{tak}  \textbf{mamati}  ka  anay ta  miting  ta  PTA  \textbf{tak}  \textbf{yaken}  \textbf{i}  \textbf{magbantay} ta  baļay  \textbf{tak}  \textbf{yaken}  \textbf{gasakit}  ake  na  uļo. \smallskip\\
\gll Manong  m-tegbeng  ka  \textbf{tak}  \textbf{ma-mati}  ka  anay ta  miting  ta  PTA  \textbf{tak}  \textbf{yaken}  \textbf{i}  \textbf{mag-bantay} ta  baļay  \textbf{tak}  \textbf{yaken}  \textbf{ga-sakit}  ake  na  uļo. \\
older.brother  \textsc{i.v.ir}-go.downhill  2\textsc{s.abs}  because  \textsc{a.hap.ir}-hear  2\textsc{s.abs}  first/for.a.while
\textsc{nabs}  meeting  \textsc{nabs}  PTA  because  1\textsc{s.abs}  \textsc{def.n}  \textsc{i.ir}-watch/guard
\textsc{nabs}  house  because  1\textsc{s.abs}  \textsc{i.r}-pain  1\textsc{s.gen}  \textsc{lk}  head \\
\glt `Older brother you go downhill, \textbf{because} \textbf{you} \textbf{will} \textbf{listen} for a while to the meeting of PTA, \textbf{because} \textbf{as} \textbf{for} \textbf{me} \textbf{I} \textbf{will} \textbf{watch} \textbf{the} \textbf{house}, \textbf{because} \textbf{as} \textbf{for} \textbf{me} \textbf{my} \textbf{head} \textbf{hurts}.’ (The speaker is giving the reason why the older brother should go down hill and also the reason why he will not go to the meeting.) [EDWN-T-04 2.2]
\z
\ea
 … \textbf{tak}  \textbf{pabor}  \textbf{en}  \textbf{angin}  \textbf{an},  kalitan    ta  ta  paglayag. \smallskip\\
\gll … \textbf{tak}  \textbf{pabor}  \textbf{en}  \textbf{angin}  \textbf{an},  \emptyset{}-kalit-an    ta  ta  pag-layag. \\
{} because  favorable  \textsc{cm}  wind  \textsc{def.m}  \textsc{t.ir}-swift-\textsc{apl}  1\textsc{p.incl.erg}  \textsc{nabs}  \textsc{nr.act}-sail \\
\glt ‘… \textbf{because the wind is favorable}, let’s make swift sailing.’ [CTWN-L-01 2.9]
\z

\newpage
\ea
Gabakak  kay  na  gatuļo  ame  na  luwa  \textbf{tak} \textbf{nļaman  nay} na  Dios  an  gaduma  ki  kami. \smallskip\\
\gll Ga-bakak  kay  na  ga-tuļo  ame  na  luwa  \textbf{tak} \textbf{na-aļam-an}  \textbf{nay} na  Dios  an  ga-duma  ki  kami. \\
\textsc{i.r}-glad  1\textsc{p.exc.abs}  \textsc{lk}  \textsc{i.r}-drip  1\textsc{p.exc.gen}  \textsc{lk}  tear  because
\textsc{a.hap.r}-know-\textsc{apl}  1\textsc{p.excl.erg}  \textsc{lk}  God  \textsc{def.m}  \textsc{i.r}-companion  \textsc{obl.p}  1\textsc{p.exc} \\
\glt `We were glad with our tears dripping \textbf{because} \textbf{we} \textbf{knew} that GOD was accompanying us.’ [VAWN-T-18 5.12]
\z
\ea
Uyo   nya  sikad  gid  ake  na  kulba  \textbf{tak}  \textbf{salig} \textbf{ko}  gid  daw  mataring  aren  ta  ake  na  gian. \smallskip\\
\gll U-yo   nya  sikad  gid  ake  na  kulba  \textbf{tak}  \textbf{salig} \textbf{ko}  gid  daw  ma-taring  aren  ta  ake  na  agi-an. \\
\textsc{emph-d4abs} \textsc{d4pr}  \textsc{d}4\textsc{abs}  very  \textsc{int}  1\textsc{s.gen}  \textsc{lk}  nervous  because  think.wrongly
1\textsc{s.erg}  \textsc{int}  if/when  \textsc{a.hap.ir}-go.wrong.way  1\textsc{s.abs}  \textsc{nabs}  1\textsc{s.gen}  \textsc{lk}  pass-\textsc{nr} \\
\glt `That (was the reason) my nervousness was really very much \textbf{because} \textbf{I} \textbf{thought} \textbf{wrongly} that I will/would go the wrong way from my path.’ [DBON-C-08 2.15]
\z
\ea
Finite adverbial clauses with \textit{kumo} ‘because’: \\
Dili  aren  magpangaran-ngaran  ta  mga  kaoy \textbf{kumo}  \textbf{nļaman  ko}  \textbf{man}  na  nļaman  nyo  gid. \smallskip\\
\gll Dili  aren  mag-pa-ngaran\sim{}-ngaran  ta  mga  kaoy \textbf{kumo}  \textbf{na-aļam-an}  \textbf{ko}  \textbf{man}  na  na-aļam-an  nyo  gid. \\
\textsc{neg.ir}  1\textsc{s.abs}  \textsc{i.ir}-\textsc{caus}-\textsc{red}-name  \textsc{nabs}  \textsc{pl}  tree
because  \textsc{a.hap.r}-know-\textsc{apl}  1\textsc{s.erg}  too  \textsc{lk}  \textsc{a.hap.r}-know-\textsc{apl}  2\textsc{s.erg}  \textsc{int} \\
\glt `I will not name the trees \textbf{because} \textbf{I} \textbf{also} \textbf{know} that you really know (the names of the  trees).’ [ROOB-T-01 8.25]
\z
\ea
Gani  a  mga  ittaw  naan  ta  Bario  nakulian  ta pagnubig  \textbf{kumo}  \textbf{waig}  \textbf{naan}  \textbf{pa}  \textbf{kamangen}  \textbf{ta} \textbf{Barangay}   \textbf{Wahig}. \smallskip\\
\gll Gani  a  mga  ittaw  naan  ta  Bario  na-kuli-an  ta pag-nubig  \textbf{kumo}  \textbf{waig}  \textbf{naan}  \textbf{pa}  \textbf{kamang-en}  \textbf{ta} \textbf{Barangay}   \textbf{Wahig}. \\
so  \textsc{inj}  \textsc{pl}  person  \textsc{spat.def}  \textsc{nabs}  Bario  \textsc{a.hap.r}-difficult-\textsc{apl}  \textsc{nabs}
\textsc{nr.act}-haul.water  because  water  \textsc{spat.def}  \textsc{inc}  get-\textsc{t.ir}  \textsc{nabs}
Community  Wahig \\
\glt `So hauling water difficults/slows the people in Bario \textbf{because} \textbf{the} \textbf{water} \textbf{was} \textbf{even} \textbf{gotten} \textbf{in} \textbf{Community} \textbf{Wahig}.’ [VPWE-T-01 2.7]
\z
\ea
Finite adverbial clauses with tenged (na) ‘because’: \\
Tapos  pagilekan  kanen  i  ta  nanay  din \textbf{tenged}  \textbf{na}  \textbf{patudlo}  \textbf{din}  makina  ya. \smallskip\\
\gll Tapos  pa-gilek-an  kanen  i  ta  nanay  din \textbf{tenged}  \textbf{na}  \textbf{pa-tudlo}  \textbf{din}  makina  ya. \\
then  \textsc{t.r}-angry-\textsc{apl}  3\textsc{s.abs}  \textsc{def.n}  \textsc{nabs}  mother  3\textsc{s.gen}
because  \textsc{lk}  \textsc{t.r}-point.out/teach  3\textsc{s.erg}  motor  \textsc{def.f} \\
\glt `Then his mother got angry with him \textbf{because} \textbf{he} \textbf{pointed} \textbf{out} the engine (to the raiders).’ [BCWN-C-04 7.7]
\z
\ea
Sikad  na  kalised  ta  mga  ittaw  \textbf{tenged}  \textbf{ta}  \textbf{isya}  \textbf{na}  \textbf{adlaw} \textbf{dili}  \textbf{magminos}  \textbf{ta}  \textbf{10}  \textbf{na}  \textbf{mapatay}. \smallskip\\
\gll Sikad  na  ka-lised  ta  mga  ittaw  \textbf{tenged}  \textbf{ta}  \textbf{isya}  \textbf{na}  \textbf{adlaw} \textbf{dili}  \textbf{mag-minos}  \textbf{ta}  \textbf{10}  \textbf{na}  \textbf{ma-patay}. \\
very  \textsc{lk}  \textsc{nr}-distress  \textsc{nabs}  \textsc{pl}  person  because  \textsc{nabs}  one  \textsc{lk}  day/sun
\textsc{neg.ir}  \textsc{i.ir}-minus  \textsc{nabs}  10  \textsc{lk}  \textsc{a.hap.ir}-kill \\
\glt `The distress of the people was very much \textbf{because} \textbf{in} \textbf{one} \textbf{day} \textbf{it} \textbf{never} \textbf{was} \textbf{less} \textbf{than} \textbf{10} \textbf{who} \textbf{died}.’ [HEWN-L-03 4.2]
\z

Of the three reason clause introducers, \textit{tak}, \textit{kumo}, and \textit{tenged}, only \textit{tenged} ‘can introduce a non-finite, nominalized, clause (\ref{bkm:Ref116287582}; see \sectref{bkm:Ref115861741}) or a RE \REF{bkm:Ref116287610}:

\ea
\label{bkm:Ref116287582}
May  sise  man  en  na  improvar  ame  na  lugar \textbf{tenged}  \textbf{ta}  \textbf{pagtanem}  \textbf{ta}  \textbf{guso}.\smallskip\\
\gll May  sise  man  en  na  improvar  ame  na  lugar \textbf{tenged}  \textbf{ta}  \textbf{pag-tanem}  \textbf{ta}  \textbf{guso}. \\
\textsc{ext.in}  small  too  \textsc{cm}  \textsc{lk}  improve  1\textsc{p.excl.gen}  \textsc{lk}  place
because  \textsc{nabs}  \textsc{nr.act}-plant  \textsc{nabs}  seaweed. \\
\glt `Our place has a little improvement \textbf{because of planting agar seaweed}.’  [HEWN-L-03 8.21]
\z

\newpage
\ea
\label{bkm:Ref116287610}
\textbf{Tenged}  \textbf{ta}  \textbf{selleg}  na  gabuļong  nakita  din  nang  en na  gaeļeng-eļeng  Maria  ya  ta  dagat  daw  uļa  din  en  nakita. \smallskip\\
\gll \textbf{Tenged}  \textbf{ta}  \textbf{selleg}  na  ga-buļong  na-kita  din  nang  en na  ga-eļeng\sim{}-eļeng  Maria  ya  ta  dagat  daw  uļa  din  en  na-kita. \\
because  \textsc{nabs}  current  \textsc{lk}  \textsc{i.r}-crash.together  \textsc{a.hap.r}-see  3\textsc{s.erg}  nang  \textsc{cm}
\textsc{lk}  \textsc{i.r}-\textsc{red}-revolve  Maria  \textsc{def.f}  \textsc{nabs}  sea  and  \textsc{neg.r}  3\textsc{s.erg}  \textsc{cm}  \textsc{a.hap.r}-see \\
\glt `\textbf{Because} \textbf{of} \textbf{the} \textbf{current} that was crashing together, she just saw that Maria kept revolving around in the sea and she did not see (her) any more.’ [EMWN-T-06 5.6] \smallskip\\
*\textbf{Tak/kumo} \textbf{ta} \textbf{selleg} na gabulong nakita din nang en …
\z

Adverbial clauses with \textit{tak} usually do not occur before the main clause, but adverbial clauses with \textit{kumo} or \textit{tenged} do more frequently. \textit{Kumo} is the least common adverbial clause introducer, and is considered archaic by many speakers.

\ea
Finite adverbial clause with \textit{tapos na} introducer with realis verb: \\
\textbf{Tapos}  \textbf{na}  \textbf{gasandok}  a  ta  waig,  panggat  a  ta duma  ko  na  mangali  ta  kamas. \smallskip\\
\gll \textbf{Tapos}  \textbf{na}  \textbf{ga-sandok}  a  ta  waig,  pa-anggat  a  ta duma  ko  na  ma-ng-kali  ta  kamas. \\
then  \textsc{lk}  \textsc{i.r}-carry.water  1\textsc{s.abs}  \textsc{nabs}  water  \textsc{t.r}-invite.to.go.with  1\textsc{s.abs}  \textsc{nabs}
companion  1\textsc{s.gen}  \textsc{lk}  \textsc{a.hap.ir}-\textsc{pl}-dig.up  \textsc{nabs}  hingkamas \\
\glt `\textbf{Then} \textbf{when} \textbf{I} \textbf{was} \textbf{carrying} water, my companions invited me to go with them to dig up hingkamas.’ [BMON-C-02 1.7]
\z
\ea
Finite adverbial clause with \textit{tapos na} introducer with irrealis verb: \\
\textbf{Tapos}  \textbf{na}  \textbf{tampekan}  ta  pantad,  plantsa  an  trapuan ta  gaming  na  limpyo … \smallskip\\
\gll \textbf{Tapos}  \textbf{na}  \textbf{tampek-an}  ta  pantad,  plantsa  an  trapo-an ta  gaming  na  limpyo … \\
then  \textsc{lk}  pack.on-\textsc{apl}  \textsc{nabs}  sand  iron  \textsc{def.m}  wipe.off-\textsc{apl}
\textsc{nabs}  cloth  \textsc{lk}  clean \\
\glt `\textbf{Then} \textbf{when} \textbf{(the} \textbf{leftover} \textbf{coals)} \textbf{will} \textbf{be} \textbf{packed} with sand, as for the iron, wipe it with a clean cloth …’ (This is a procedural text about how  to  prepare a coal iron before ironing clothes.) [BMOP-C-07 2.5]
\z
\ea
Finite adverbial clause with \textit{asta na} introducer and realis verb: \\
Pagiran  ko  pag-asod  \textbf{asta}  \textbf{na}  \textbf{nakita}  \textbf{ko} na  naleg-as  en. \smallskip\\
\gll Pa-gid-an  ko  pag-asod  \textbf{asta}  \textbf{na}  \textbf{na-kita}  \textbf{ko} na  na-leg-as  en. \\
\textsc{t.r}-\textsc{int}-\textsc{apl}  1\textsc{s.erg}  \textsc{nr.act}-pound  until  \textsc{lk}  \textsc{a.hap.r}-see  1\textsc{s.erg}
\textsc{lk}  \textsc{a.hap.r}-smash  \textsc{cm} \\
\glt `I intensified pounding \textbf{until} \textbf{I} \textbf{saw} that (it) had been smashed fine.’ (This is about pounding rice.) [JCWE-T-13 2.7]
\z
\ea
Finite adverbial clause with \textit{asta na} introducer and irrealis verb: \\
Paryo  a  nagayya  tak  drayber  an  sigi  tan-aw ki  kami  naan  ta  may  ispiyo  na  sigi  kay  suka \textbf{asta}  \textbf{na}  \textbf{mabot}  ta  Puerto. \smallskip\\
\gll Paryo  a  na-gayya  tak  drayber  an  sigi  tan-aw ki  kami  naan  ta  may  ispiyo  na  sigi  kay  suka \textbf{asta}  \textbf{na}  \textbf{m-abot}  ta  Puerto. \\
like  1\textsc{s.abs}  \textsc{a.hap.r}-embarrass  because  driver  \textsc{def.m}  continue  look.at
\textsc{obl.p}  1\textsc{p.excl}  \textsc{spat.def}  \textsc{nabs}  \textsc{ext.in}  mirror  \textsc{lk}  continue  1\textsc{p.excl.abs}  vomit
until  \textsc{lk}  \textsc{i.v.ir}-arrive  \textsc{nabs}  Puerto \\
\glt `(It was) like I was embarrassed because, as for the driver, (he) kept looking at us (in the backseat) in the mirror as we kept on vomiting \textbf{until} \textbf{we} \textbf{were} \textbf{about} \textbf{to} \textbf{arrive} in Puerto.’ [YBWN-T-06 2.4]
\z
\ea
Finite adverbial clause with \textit{bisan, maskin, baskin (na/daw)} ‘even though / even if’ introducer and realis and irrealis verbs: \\
\textbf{Bisan}  \textbf{pa}  \textbf{na}  uļa  kanen  gabangon  tak  natignawan man  gaambaļ. \smallskip\\
\gll \textbf{Bisan}  \textbf{pa}  \textbf{na}  uļa  kanen  ga-bangon  tak  na-tignaw-an man  ga-ambaļ. \\
even.though  \textsc{emph}  \textsc{lk}  \textsc{neg.r}  3\textsc{s.abs}  \textsc{i.r}-get.up  because  \textsc{a.hap.r}-cold-\textsc{apl}
also  \textsc{i.r}-say \\
\glt `\textbf{Even} \textbf{though} she did not even get up because (she) also was cold (she) said …’ [JCWN-L-31 3.5]
\z
\ea
Dili  ko  gusto  na  ake  na  mga  kabataan  magpaadyo  ki  kami \textbf{bisan}  \textbf{daw}  imol  ki  nang  o  maan  ki  nang ta  mga  gamot  ta  kaoy … \smallskip\\
\gll Dili  ko  gusto  na  ake  na  mga  ka-bata-an  mag-pa-adyo  ki  kami \textbf{bisan}  \textbf{daw}  imol  ki  nang  o  m-kaan  ki  nang ta  mga  gamot  ta  kaoy … \\
\textsc{neg.ir}  1\textsc{s.erg}  want  \textsc{lk}  1\textsc{s.gen}  \textsc{lk}  \textsc{pl}  \textsc{nr}-child-\textsc{nr}  \textsc{i.ir}-\textsc{caus}-far  \textsc{obl.p}  1\textsc{p.excl}
even.though  if/when  poor  1\textsc{p.incl.abs}  only  or  \textsc{i.v.ir}-eat  1\textsc{p.incl.abs}  only
\textsc{nabs}  \textsc{pl}  root  \textsc{nabs}  tree \\
\glt `I do not want my children to go far from us \textbf{even} \textbf{if} we are only hallelujah poor or we will eat only roots of trees …’ [CBWE-C-05 4.2]
\z

Adverbial clauses may be non-verbal. The following is a locational adverbial clause introduced by \textit{na}. Recall that locational constructions do not include a verb (see \chapref{chap:non-verbalclauses}, \sectref{sec:locationalclauses}). A literal English translation of the adverbial clause in \REF{bkm:Ref474745770} would be “being there again as before in the place ahead ...”:

\ea
\label{bkm:Ref474745770}
\textbf{Na}  \textbf{naan}  \textbf{eman}  \textbf{kon}  \textbf{ta}  \textbf{unaan}  \textbf{ya}  may  nakita eman  kon  danen  an  na  wasay. \smallskip\\
\gll \textbf{Na}  \textbf{naan}  \textbf{eman}  \textbf{kon}  \textbf{ta}  \textbf{una-an}  \textbf{ya}  may  na-kita eman  kon  danen  an  na  wasay. \\
\textsc{lk}  \textsc{spat.def}  again.as.before  \textsc{hsy}  \textsc{nabs}  first-\textsc{nr}  \textsc{def.f}  \textsc{ext.in}  \textsc{a.hap.r}-see
again.as.before  \textsc{hsy}  3\textsc{p.abs}  \textsc{def.m}  \textsc{lk}  axe \\
\glt `\textbf{When} \textbf{(being)} \textbf{as} \textbf{before} \textbf{in} \textbf{the} \textbf{place} \textbf{on} \textbf{ahead} they saw something again that was an axe.’ [CBWN-C-10 3.3]
\z
\ea
\textbf{Na}  \textbf{nyaan}  \textbf{en}  \textbf{dani}  \textbf{ta}  \textbf{karwa}  \textbf{ya}  \textbf{na}  \textbf{bungyod},  gasinggit eman  isab  Pwikan  i,  “Umang,  indi  ka  yan  en? \smallskip\\
\gll \textbf{Na}  \textbf{nyaan}  \textbf{en}  \textbf{dani}  \textbf{ta}  \textbf{karwa}  \textbf{ya}  \textbf{na}  \textbf{bungyod},  ga-singgit eman  isab  Pwikan  i,  “Umang,  indi  ka  yan  en? \\
\textsc{lk}  \textsc{spat.def}  \textsc{cm}  near  \textsc{nabs}  second  \textsc{def.f}  \textsc{lk}  hill  \textsc{i.r}-shout
again.as.before  again  sea.turtle  \textsc{def.n}  hermit.crab  where  2\textsc{s.abs}  \textsc{def.m}  \textsc{cm} \\
\glt `\textbf{When} \textbf{(being)} \textbf{near} \textbf{to} \textbf{the} \textbf{second} \textbf{hill} the Sea Turtle shouted again as before, “Hermit Crab, where are you now?”' [JCON-L-08 42.1]
\z

To conclude this section we present one example with two finite adverbial clauses, and one subjunctive clause. Example \REF{bkm:Ref474847117} begins with a finite time adverbial locational clause introduced with \textit{na}, followed by the main clause ``we continually cried". This in turn is followed by a finite adverbial clause introduced by \textit{tak}, which itself contains an embedded subjunctive clause, \textit{na patayen} ‘(we) would be killed.’ The three dependent clauses are indicated with brackets to illustrate the embedding relationships:

\newpage
\ea
\label{bkm:Ref474847117}
Na  naan  kami  i  ta  lansa  ya  sigi  kami  agaļ tak  adlek  kami na  patayen. \smallskip\\
Time/Location adverbial clause \\
\gll {}[ Na  naan  kami  i  ta  lansa  ya  ]  sigi \\
{} \textsc{lk}  \textsc{spat.def}  1\textsc{p.excl.abs}  \textsc{def.n}  \textsc{nabs}  launch  \textsc{def.f} {}   continually \\
\glt Main clause \hspace{1cm} Reason adv. cl. \hspace{1.6cm}  Subjunctive clause \\
\gll Kami  agaļ [ tak  adlek  kami    [ na  patay-en ] ]. \\
1\textsc{p.excl.abs}  cry {} because  afraid  1\textsc{p.excl.abs} {} \textsc{lk}  kill-\textsc{t.ir} \\
\glt `[When we were on the launch], we continually cried [ because we were afraid [(we) would be killed]].’ [BCWN-C-04 4.4]
\is{adverbial clauses!finite|)}
\is{dependent clauses!finite|)}
\is{finite dependent clauses|)}
\z

\section{Relative clauses}
\label{bkm:Ref460483291} \label{sec:relativeclauses}
\is{relative clauses|(}
Relative clauses (RCs) are clauses that serve a modifying function within Referring Phrases \citep{comrie1989, payne1997}. They are clauses because they consist of a predicator plus one or more of its arguments. Semantically, relative clauses represent a complete discourse “scene”. Here are some examples of Referring Phrases containing relative clauses in English:


\ea
 \label{bkm:Ref116885896}
    \ea
    \label{bkm:Ref116885896a}
    a \textbf{helper} who [ will watch their child ] ... \\
    \ex
    \label{bkm:Ref116885896b}
    the \textbf{lentils} that [ I dried ] ... \\
    \ex
    \label{bkm:Ref116885896c}
    the \textbf{stage} where [ we held our program ] ...
    \z
\z

Each of these examples is a Referring Phrase in which the bolded noun is the head. The bracketed portion describes a scene that specifies the head in some way. In \REF{bkm:Ref116885896a}, the bracketed portion describes a scene involving someone watching a child. This portion can be thought of as a reduction of the clause “a helper will watch their child”. The subject of this clause refers to the same referent, \textit{a helper}, as the head of the Referring Expression. Similarly, the bracketed portion of \REF{bkm:Ref116885896b} can be thought of as a reduction of “I dried the lentils”, and the bolded portion of \REF{bkm:Ref116885896c} can be thought of as a reduction of “we held our program on the stage”. In each example, an element of the modifying clause is omitted because it is coreferential with (it refers to the same referent as) the head of the larger Referring Phrase.

Close Kagayanen equivalents to the examples in \REF{bkm:Ref116885896} are given in \REF{bkm:Ref116889490}:

\ea
\label{bkm:Ref116889490}
    \ea
    \label{bkm:Ref116889490a}
    isya  na  saluguon  na  magbantay  ta  iran  na  bata \\\smallskip
\gll \textbf{isya}  \textbf{na}  \textbf{saluguon}  na [ mag-bantay  ta  iran  na  bata ] \\
    one  \textsc{lk}  servant  \textsc{lk}  {} \textsc{i.ir}-watch/guard  \textsc{nabs}  3\textsc{p.gen}  \textsc{lk}  child {} \\
    \glt ‘a helper who will watch their child’
    \ex
    \label{bkm:Ref116889490b}
    ta  pabaļad  ko  na  dawa \\\smallskip
\gll [ ta  pa-baļad  ko ] na \textbf{dawa} \\
    {} \textsc{nabs}  \textsc{t.r}-dry  1\textsc{s.gen}  {} \textsc{lk}  lentils \\
    \glt ‘lentils that I dried’
    \ex
    \label{bkm:Ref116889490bc}
    ta  stage  na  naiwasan  ta  ate  na  prugrama \\\smallskip
\gll ta  \textbf{stage}  na [ na-iwas-an  ta  ate  na  prugrama ] \\
    \textsc{nabs}  stage  \textsc{lk} {} \textsc{a.hap.r}-hold/present-\textsc{apl}  \textsc{nabs}  1\textsc{p.incl.gen}  \textsc{lk}  program {} \\
    \glt ‘the stage on which we held our program’
    \z
\z


Like the English examples, the examples in \REF{bkm:Ref116889490} are all Referring Phrases with a clause-like element serving as a modifier of the head noun. Also, as in the English examples, the modifying clause (bolded) can be thought of as a shortened clause in which the omitted element is an RP which is coreferential with the head of the phrase. Unlike the English examples, in each of the Kagayanen examples the omitted element is the \textit{absolutive} of the truncated clause. In Kagayanen grammar only the absolutive within a relative clause may be coreferential with the head noun of the phrase. Another difference between these examples and corresponding English examples is that in English, all the relative clauses follow the head noun of the phrase. In Kagayanen, relative clauses usually follow the head, but they also may precede the head, as in \REF{bkm:Ref116889490b}. This pattern is consistent with the syntax of Modification in general in Kagayanen. Recall from \chapref{chap:referringexpressions}, there are two modifier positions in Kagayanen Referring Phrases; the first, MOD1, occurs before the head, and the second, MOD2, follows the head. In general shorter relative clauses tend to occur in MOD1 position, while longer relative clauses tend to occur in MOD2 position. As with most modifiers within Referring Phrases, relative clauses are separated from their Heads with the linker \textit{na}, as represented in \REF{bkm:Ref474761906}:

\ea
\label{bkm:Ref474761906}
RC \textit{na} HEAD / HEAD \textit{na} RC
\z

We will divide our description of Kagayanen relative clauses between those that are based on dedicated participant nominalizations (\sectref{bkm:Ref115159779}) and those that are based on fully finite Inflected verbs (\sectref{bkm:Ref116818852}). Additional examples of relative clauses in Referring Phrases are presented in \chapref{chap:referringexpressions}, \sectref{sec:rpscontainingrelativeclauses}.

\subsection{Dedicated participant nominalizations filling a relative clause function}
\label{bkm:Ref115159779}
Most relative clauses in Kagayanen are headed by inflected verb forms. However, some \isi{participant nominalizations} described in \chapref{chap:referringexpressions} may also serve as modifiers within Referring Phrases. Since these forms have no function as verbal inflections, we refer to them as \textit{dedicated} participant nominalizing processes. In this section, we describe relative clauses based on such dedicated noun-stem forming processes. These include the \isi{resultative} nominalizer \textit{<in>}, and the future patient nominalizer \textit{{}-én/-ón} (see \chapref{chap:referringexpressions}, \sectref{sec:in} and \sectref{sec:en}).

Verbs nominalized with the resultative nominalizing infix <\textit{in}> may function as modifiers within the RP (see \chapref{chap:referringexpressions}, \sectref{sec:in}). As such, they may be understood as filling a relative clause function. The following are examples of post-head \REF{bkm:Ref474824085} and pre-head \REF{bkm:Ref474824092} relative clauses formed with \textit{<in>} nominalizations:

\ea
\label{bkm:Ref474824085}
Head  \textit{na}  Relative clause   \\
Naliagan  din  dilis  an  na  \textbf{linuto}  \textbf{ko}. \smallskip\\
\gll Na-liag-an  din  dilis  an  na  \textbf{l<in>uto}  \textbf{ko}. \\
\textsc{a.hap.r}-like/want-\textsc{apl}  3\textsc{s.erg}  anchovies \textsc{def.m}  \textsc{lk} <\textsc{nr.res}>cook  1\textsc{s.erg}   \\
\glt ‘S/he liked the anchovies \textbf{I} \textbf{have} \textbf{cooked}.’
\z
\ea
\label{bkm:Ref474824092}
Relative clause         \textit{na}  Head   \\
Naliagan  din  \textbf{linuto}  \textbf{ko}  \textbf{an}  na  dilis. \smallskip\\
\gll Na-liag-an  din  \textbf{l<in>uto}  \textbf{ko}  \textbf{an}  na  dilis. \\
\textsc{a.hap.r}-like/want-\textsc{apl}  3\textsc{s.erg}  <\textsc{nr.res}>cook  1\textsc{s.erg}  \textsc{def.m}  \textsc{lk}  anchovies \\
\glt ‘S/he liked the anchovies \textbf{I} \textbf{have} \textbf{cooked}.’
\z

Examples \REF{bkm:Ref474824085} and \REF{bkm:Ref474824092} may both be translated literally as “S/he liked my cooked anchovies.” However, the meaning of the Kagayanen examples must be that the speaker is the Actor of the action of cooking, and not simply the possessor of the anchovies. In the English RP “my cooked anchovies” the speaker asserts that the anchovies are in a cooked state, but the identity of the person who cooked them is not part of the assertion. In Kagayanen, however, the genitive/ergative element inside the RP is strongly asserted to be the Actor of the nominalized verb. Thus these resultative nominalizations are slightly more “relative clause like” than the literal English translations.

Examples \REF{bkm:Ref474827251}-\REF{bkm:Ref360193876} illustrate participant nominalizations functioning in this way in the text corpus.

\ea
\label{bkm:Ref474827251}
... daw  timpo  kurisma  \textbf{iran}  \textbf{na}  \textbf{kinutkot}  na  waig  dili  en magamit  tak  mangngod  nang  ta  dagat  na  sikad  masin. \smallskip\\
\gll ... daw  timpo  kurisma  \textbf{iran}  \textbf{na}  \textbf{k<in>utkot}  na  waig  dili  en ma-gamit  tak  mangngod  nang  ta  dagat  na  sikad  masin. \\
{} if/when  time/season  drought  3\textsc{p.gen}  \textsc{lk}  <\textsc{nr.res}>dig  \textsc{lk}  water  \textsc{neg.ir}  \textsc{cm}
\textsc{a.hap.ir}-use  because  younger.sibling  only  \textsc{nabs}  sea  \textsc{lk}  very  salty \\
\glt `… when it is the time of drought the well \textbf{they} \textbf{dug} (lit. “their dug water”) cannot be used because (it is) the younger sibling of the sea being very salty.’ [VPWE-T-01 2.4]
\z
\ea
Ake,  kinangļan  na  gapang-insa  a  daw  kino  gauļa  ta \textbf{binaļad  ko} na  dawa. \smallskip\\
\gll Ake,  kinangļan  na  ga-pang-insa  a  daw  kino  ga-uļa  ta \textbf{b<in>aļad}  \textbf{ko} na  dawa. \\
1\textsc{s.gen}  need  \textsc{lk} \textsc{i.r}-\textsc{pl}-ask  1\textsc{s.abs}  if/when  who  \textsc{i.r}-spill  \textsc{nabs}
<\textsc{nr.res}>dry  1\textsc{s.gen}  \textsc{lk}  lentils \\
\glt `As for me, (it was) necessary that I ask (some people) who spilled \textbf{my} \textbf{dried} lentils (i.e., the lentils that I dried).’ [TPWN-J-01 5.6]
\z
\ea
Maria,  ginakasebe   ko  ta  buļan  ta  Disyimbri  pitsa  15  ta taon  1987  \textbf{pinalangga}  \textbf{ko}  na  nanay  na  imo  na  lola  napatay. \smallskip\\
\gll Maria,  ginaka-sebe\footnotemark{}   ko  ta  buļan  ta  Disyimbri  pitsa  15  ta taon  1987  \textbf{p<in>alangga}  \textbf{ko}  na  nanay  na  imo  na  lola  na-patay. \\
Maria  \textsc{t.r}-sad  1\textsc{s.erg}  \textsc{nabs}  month  \textsc{nabs}  December  date  15  \textsc{nabs}
year  1987  <\textsc{nr.res}>love  1\textsc{s.gen}  \textsc{lk}  mother  \textsc{lk}  2\textsc{s.gen}  \textsc{lk}  grandmother  \textsc{a.hap.r}-kill \\
\footnotetext{This complex prefix, \textit{ginaka}{}-, is code switching from \isi{Hiligaynon}.}
\glt `Maria, I was sad in the month of December date 15 of the year 1987 \textbf{my} \textbf{beloved} mother who is your grandmother died.’ [EFWL-T-08 8.1]
\z
\ea
… dugang  ko  ta  minsahi  ta  ate  na  \textbf{pinalangga} ta  na  ex-mayor … \smallskip\\
\gll … dugang  ko  ta  minsahi  ta  ate  na  \textbf{p<in>alangga} ta  na  ex-mayor … \\
{}   add  1\textsc{s.erg}  \textsc{nabs}  message  \textsc{nabs}  1\textsc{p.incl.gen}  \textsc{lk}  <\textsc{nr.res}>love
1\textsc{p.incl.gen}  \textsc{lk}  ex-mayor \\
\glt `… I will add to the message of our \textbf{beloved} ex-mayor …’ [ROOB-T-01 3.1]
\z

The following is a common salutation formula at the beginning of letters:

\ea
\textbf{Pinalangga}  \textbf{ko}  na  mga  mangngod \smallskip\\
\gll \textbf{P<in>alangga}  \textbf{ko}  na  mga  mangngod \\
<\textsc{nr.res}>love  1\textsc{s.gen}  \textsc{lk}  \textsc{pl}  younger.sibling \\
\glt ‘\textbf{My} \textbf{beloved} younger siblings’ [ICWL-T-05 3.1]
\z

Example \REF{bkm:Ref474761456} illustrates that the head of the RP within which the participant Nominalization functions must be coreferential with the absolutive of the nominalized verb. The verb \textit{alin} ‘to come’ must appear in the applicative form \textit{alinan} ‘to come from’ in which the origin is the absolutive. This is because the unmentioned head of the modifier is the “old ways”:

\ea
\label{bkm:Ref474761456}
Baliken  ta  ate  \textbf{na}  inalinan. \smallskip\\
\gll Balik-en  ta  ate  \textbf{na}  <\textbf{in}>\textbf{alin-an}. \\
return-\textsc{t.ir}  1\textsc{p.incl.erg}  1\textsc{p.incl.gen}  \textsc{lk}  \textsc{<nr.res>}from-\textsc{apl} \\
\glt ‘We will go back (to the old ways) from where we \textbf{came}.’ [JCWO-L-29 43.1] \smallskip\\
*Baliken ta ate na inalin.
\z

The future patient nominalizer -\textit{én/-ón} described in \chapref{chap:referringexpressions}, \sectref{sec:en} may also mark the verb in a nominalization functioning as a relative clause:

\ea
\label{bkm:Ref360193876}
… may  idugang  nang  ta  ake  na  \textbf{deén}  \textbf{na}  kwarta. \smallskip\\
\gll … may  i-dugang  nang  ta  ake  na  \textbf{dala-én}  \textbf{na}  kwarta. \\
{} \textsc{ext.in}  \textsc{t.deon}-add  only  \textsc{nabs}  1\textsc{s.gen}  \textsc{lk}  take-\textsc{nr}  \textsc{lk}  money \\
\glt ‘… there is something to add to my money \textbf{I} \textbf{am} \textbf{taking}.’ [PBWL-T-06 4.3]
\z
\ea
Daw  tagan  ka  ta  kan-en  o  mga  bagay  \textbf{na}  \textbf{kinangļanén}  \textbf{no}, kinangļan  magpasalamat  ka. \smallskip\\
\gll Daw  \emptyset{}-atag-an  ka  ta  kan-en  o  mga  bagay  \textbf{na}  \textbf{kinangļan-én}  \textbf{no}, kinangļan  mag-pa-salamat  ka. \\
if/when  \textsc{t.ir}-give-\textsc{apl}  2\textsc{s.abs}  \textsc{nabs}  cooked.rice  or  \textsc{pl}  thing  \textsc{lk}  need-\textsc{nr}  2\textsc{s.gen}
necessary  \textsc{i.ir}-\textsc{caus}-thanks  2\textsc{s.abs} \\
\glt `If you are given cooked rice or things \textbf{that} \textbf{you} \textbf{need}, it is necessary to give thanks.’ [ETOP-C-09 3.2]
\z

\newpage
\ea
Prengngan:  Yi ni  mga  pagkaan  \textbf{na}  \textbf{kan-enén}  \textbf{ta}  \textbf{mga}   \textbf{gabagnes} daw  mabata  danen  an. \smallskip\\
\gll Prengngan:  Yi ni  mga  pagkaan  \textbf{na}  \textbf{kan-en-én}  \textbf{ta}  \textbf{mga}   \textbf{ga-bagnes} daw  ma-bata  danen  an. \\
postpartum.food\footnotemark{}  \textsc{d1abs} \textsc{d1pr}  \textsc{pl}  food  \textsc{lk}  cooked.rice-\textsc{nr}  \textsc{nabs}  \textsc{pl}  \textsc{i.r}-pregnant
if/when  \textsc{a.hap.ir}-child  3\textsc{p.abs}  \textsc{def.m} \\
\footnotetext{\textit{Prengngan} refers to foods that a woman needs to eat after giving birth to prevent \textit{beggat}-`relapse' to weaknesses and sicknesses associated with pregnancy. \textit{Prengngan} includes pork, fish, food with thorns (e.g. sea urchins), sour food, or foods that change color when cooked.}
\glt `Prengngan: This very one is food \textbf{that} \textbf{a} \textbf{pregnant} \textbf{(woman)} \textbf{eats} when she gives birth.’ [VAOE-J-05 4.1]
\z

\subsection{Finite clauses filling a relative clause function}
\label{bkm:Ref116818852}
As mentioned earlier, almost any inflected verb form can function as a noun referring to the absolutive argument of the inflected verb. As such, they may also serve as predicates in relative clauses. When the head describes a non-referential entity, the relative clause (RC) is often irrealis, as in \REF{bkm:Ref474746763}, and the second RC in \REF{bkm:Ref474761945}:

\ea
\label{bkm:Ref474746763}
Danen  i  daw  may  bata  abi  dady  daw  mamy,  mamang gid  ta  isya  na  saluguon  na  \textbf{magbantay}  \textbf{ta}  \textbf{iran}  \textbf{na}  \textbf{bata} … \smallskip\\
\gll Danen  i  daw  may  bata  abi  dady  daw  mamy,  m-kamang gid  ta  isya  na  saluguon  na  \textbf{mag-bantay}  \textbf{ta}  \textbf{iran}  \textbf{na}  \textbf{bata} … \\
3\textsc{p.abs}  \textsc{def.n}  if/when  \textsc{ext.in}  child  for.example  daddy  and  mammy  \textsc{i.v.ir}-get
\textsc{int}  \textsc{nabs}  one  \textsc{lk}  servant  \textsc{lk}  \textsc{i.ir}-watch/guard  \textsc{nabs}  3\textsc{p.gen}  \textsc{lk}  child \\
\glt `As for them, if the dad and mom have a child for example, (they) will really get a helper \textbf{who} \textbf{will} \textbf{watch} their child…’ [RZWE-J-01  14.4]
\z
Example \REF{bkm:Ref116914459} illustrates two relative clauses modifying one head:

\ea
\label{bkm:Ref116914459}\label{bkm:Ref474761945}
Uļa  man  waig  na  \textbf{gailig}  na  \textbf{magbunyag}  ta  mga  tanem tak  uļa  man  suba. \smallskip\\
\gll Uļa  man  waig  na  \textbf{ga-ilig}  na  \textbf{mag-bunyag}  ta  mga  tanem tak  uļa  man  suba. \\
\textsc{neg.r}  also  water  \textsc{lk}  \textsc{i.r}-flow  \textsc{lk}  \textsc{i.ir}-irrigate  \textsc{nabs}  \textsc{pl}  plant
because  \textsc{neg.r}  also  river \\
\glt `There is also no water that is flowing \textbf{that} \textbf{will} \textbf{irrigate} the plants because there also is no river.’ [JCWE-L-32 1.6]
\z

We do not consider this use of irrealis modality in an RC to be subjunctive in the sense described in \sectref{bkm:Ref477523359}, since the irreality of the action expressed is consistent with the semantics of the construction, and realis modality is also grammatical in the same context.

Finite relative clauses are marked as realis when referential entities are involved, as in examples \REF{bkm:Ref474747019}-\REF{bkm:Ref474907453}:

\ea
\label{bkm:Ref474747019}
Naliagan  din  dilis  na  \textbf{paluto}  \textbf{ko}  \textbf{an}. \smallskip\\
\gll Na-liag-an  din  dilis  na  \textbf{pa-luto}  \textbf{ko}  \textbf{an}. \\
\textsc{a.hap.r}-like/want-\textsc{apl}  3\textsc{s.erg}  anchovies  \textsc{lk}  \textsc{t.r}-cook  1\textsc{s.erg}  \textsc{def.m} \\
\glt ‘S/he liked the anchovies \textbf{I} \textbf{cooked}.’
\z
\ea
\label{bkm:Ref474749882}
Nakita  din  isya  na  bai  na  \textbf{ganegga}  \textbf{ta}  \textbf{duyan}. \smallskip\\
\gll Na-kita  din  isya  na  bai  na  \textbf{ga-negga}  \textbf{ta}  \textbf{duyan}. \\
\textsc{a.hap.r}-see  3\textsc{s.erg}  one  \textsc{lk}  woman  \textsc{lk}  \textsc{i.r}-lie  \textsc{nabs}  hammock \\
\glt ‘He saw a woman \textbf{lying} \textbf{in} \textbf{a} \textbf{hammock}.’ [EDWN-T-03  2.9]
\z
\ea
Nakita  din  ake  na  mangngod  na  \textbf{nalemmes}. \smallskip\\
\gll Na-kita  din  ake  na  mangngod  na  \textbf{na-lemmes}. \\
\textsc{a.hap.r}-see  3\textsc{s.erg}  1\textsc{s.gen}  \textsc{lk}  younger.sibling  \textsc{lk}  \textsc{a.hap.r}-drown \\
\glt ‘He saw my younger sibling \textbf{who} \textbf{drowned}.’ [LCWN-T-01]
\z
\ea
\label{bkm:Ref474907453}
Nakita  din  ake  na  mangngod  na  \textbf{sise  nang  nalemmes}. \smallskip\\
\gll Na-kita  din  ake  na  mangngod  na  \textbf{sise} nang  \textbf{na-lemmes}. \\
\textsc{a.hap.r}-see  3\textsc{s.erg}  1\textsc{s.gen}  \textsc{lk}  younger.sibling  \textsc{lk}  little only \textsc{a.hap.r}-drown \\
\glt ‘He saw my younger sibling \textbf{who} \textbf{almost} \textbf{drowned}.’
\z

For all relative clauses, the absolutive argument in the RC must be coreferential with the head. This is evidenced by the ungrammaticality of the following examples, based on \REF{bkm:Ref474747019} and \REF{bkm:Ref474749882} above:

\ea
*Naliagan  din  dilis  na  \textbf{galuto}  \textbf{a}. \smallskip\\
\gll *Na-liag-an  din  dilis  na  \textbf{ga-luto}  \textbf{a}. \\
\textsc{a.hap.r}-like/want-\textsc{apl}  3\textsc{s.erg}  anchovies  \textsc{lk}  \textsc{i.r}-cook  1\textsc{s.abs} \\
\glt (‘S/he liked anchovies I cook.’)
\z
\ea
*Nakita  din  isya  na  bai  na  \textbf{paluto}  \textbf{dilis}. \smallskip\\
\gll *Na-kita  din  isya  na  bai  na  \textbf{pa-luto}  \textbf{dilis}. \\
\textsc{a.hap.r}-see  3\textsc{s.erg}  one  \textsc{lk}  woman  \textsc{lk}  \textsc{i.r}-cook  anchovies \\
\glt (‘S/he saw the woman who cooked the anchovies.’)
\z

The following are some additional examples of pre-nominal finite relative clauses from the corpus:

\ea
Na  gakawas  kay  ta  barka,  \textbf{nakita}  \textbf{ko}  \textbf{ya} na  ittaw  gatindeg  ta  isya  na  tyanggi  na  gatalikod. \smallskip\\
\gll Na  ga-kawas  kay  ta  barka,  \textbf{na-kita}  \textbf{ko}  \textbf{ya} na  ittaw  ga-tindeg  ta  isya  na  tyanggi  na  ga-talikod. \\
\textsc{lk}  \textsc{i.r}-disembark  1\textsc{p.excl.abs}  \textsc{nabs}  rowboat  \textsc{a.hap.r}-see  1\textsc{s.erg}  \textsc{def.f}
\textsc{lk}  person  \textsc{i.r}-stand  \textsc{nabs}  one  \textsc{lk}  store  \textsc{lk}  \textsc{i.r}-backside \\
\glt `When we disembarked from the rowboat, the person \textbf{I} \textbf{saw} was standing at the store with back turned.’ [DBWN-T-23 9.3]
\z
\ea
Dayon  kon  pilak  ta  piang  \textbf{ibitan}  \textbf{din}  \textbf{ya}  na  kaļat. \smallskip\\
\gll Dayon  kon  pilak  ta  piang  \textbf{…-ibit-an}  \textbf{din}  \textbf{ya}  na  kaļat. \\
right.away  \textsc{hsy}  throw.away  \textsc{nabs}  lame  \textsc{t.r}-hold-\textsc{apl}  3\textsc{s.erg}  \textsc{def.f}  \textsc{lk}  rope \\
\glt ‘Right away it was said the lame one threw the rope \textbf{he} \textbf{held}.’ [CBWN-C-10 75]
\z

As mentioned earlier, longer relative clauses, that is, those containing two full RP core arguments, or at least one oblique argument, may not precede their Heads:

\ea
*Nakita  din  \textbf{ganegga}  \textbf{ta}  \textbf{duyan}  na  bai. \smallskip\\
\gll *Na-kita  din  \textbf{ga-negga}  \textbf{ta}  \textbf{duyan}  na  bai. \\
\textsc{a.hap.r}-see  3\textsc{s.erg}  \textsc{i.r}-lie  \textsc{nabs}  hammock  \textsc{lk}  woman \\
\glt (‘S/he saw a lying in a hammock woman.’)
\z

Sometimes the head of a relative clause may be internal to the clause itself, as in example \REF{bkm:Ref474907787}:

\ea
\label{bkm:Ref474907787}
Nakita  din  \textbf{ganegga}  na  bai  naan  \textbf{ta}  \textbf{duyan}. \smallskip\\
\gll Na-kita  din  \textbf{ga-negga}  na  bai  naan  \textbf{ta}  \textbf{duyan}. \\
\textsc{a.hap.r}-see  3\textsc{s.erg}  \textsc{i.r}-lie  \textsc{lk}  woman  \textsc{spat.def}  \textsc{nabs}  hammock \\
\glt ‘S/he saw the woman \textbf{lying} \textbf{in} \textbf{a} \textbf{hammock}.’
\z

In this example, the woman is the absolutive argument of \textit{nakita} ‘saw’, and \textit{ganegga,} ‘lying’. The linker \textit{na} intervenes between the RC verb, \textit{ganegga}, and the head of the RC. Possible literal translations of this example may include “S/he saw the lying woman in a hammock,”  or “S/he saw the lying one, that is the woman, in a hammock.”
\is{relative clauses|)}
\section{Coordinate Clauses}
\label{sec:coordinateclauses}
\is{coordinate clauses|(}
Clauses may be conjoined with the \is{conjunctions} \textit{daw, asta, daw dili,} or \textit{o}. In \sectref{sec:daw} through \sectref{sec:o} we describe and illustrate each of these conjunctions with multi-clause examples from the corpus. In \sectref{sec:culminativeuse} we illustrate the \textit{culminative}\is{culminative} usage of irrealis modality in clause coordination.

Omission of arguments\is{argument omission} in discourse is common when the referents of the omitted elements can easily be recovered from the context. In particular, in coordinate clauses, there do not seem to be any strictly syntactic constraints on omission of arguments, or on which arguments must be coreferential. There is a tendency for coordinate clauses to share absolutive arguments, but this is not a rigid requirement, as can be seen in the following examples. In the examples in the following four sections, we present the conjunction and any overt coreferential arguments in bold.

\subsection{\textit{Daw}, ‘and’}
\label{sec:daw}
As discussed in \sectref{sec:finiteadverbialclauses}, \textit{daw} may introduce adverbial time clauses. As such we have glossed it as ‘when’. It also functions as a general conjunction that stands between units of equal syntactic rank, e.g., two nouns, two referring phrases, two adverbs, two dependent clauses, or two fully inflected clauses. In this usage, it may be glossed as ‘and’.  If two clauses coordinated with \textit{daw} share an  argument, the second reference may or may not be omitted. In example \REF{ex:hefellover}, the absolutive, \textit{kanen an}, is not omitted in the second clause, though it would be fully grammatical if the second reference to the absolutive were omitted:
\ea
\label{ex:hefellover}
Dayon kon \textbf{kanen} \textbf{i} salamat ta Ginuo \textbf{daw} natumba \textbf{kanen} \textbf{an} … \smallskip\\
\gll Dayon kon \textbf{kanen} \textbf{i} salamat ta Ginuo \textbf{daw} na-tumba \textbf{kanen} \textbf{an} … \\
right.away \textsc{hsy} 3\textsc{s.abs} \textsc{def.n} thank \textsc{nabs} Lord and \textsc{a.hap.r}-fall.over 3\textsc{s.abs} \textsc{def.m} \\
\glt ‘Right away he thanked the Lord and he fell over …’ [CBWN-C-21 4.8]
\z

Example \REF{ex:drinkvitamins} omits mention of the coreferential absolutive in the second clause:

\ea
\label{ex:drinkvitamins}
Daw bagnes en isya na nanay kinangļan \textbf{kanen} magkaan ta gulay \textbf{daw} mag-inem ta bitamina …\\\smallskip
\gll Daw bagnes en isya na nanay kinangļan \textbf{kanen} mag-kaan ta gulay \textbf{daw} mag-inem ta bitamina … \\
when pregnant \textsc{cm} one \textsc{lk} mother should 3\textsc{s.abs} \textsc{i.ir}-eat \textsc{nabs} vegetables and \textsc{i.ir}-drink \textsc{nabs} vitamins \\
\glt ‘When a mother is pregnant she should eat vegetables and take (lit. drink) vitamins …’ [LBOP-C-03 11.3]
\z

If there are two or more distinct participants in two coordinate clauses, all are normally retained. The Actor is usually not omitted unless the clause is in the peak of a narrative. Coreferential undergoers are more likely to be omitted in the second clause when they are absolutives. In example \REF{ex:onekilo}, the absolutive (\textit{sidda} 'fish') is omitted in two clauses because it is set up as the undergoer in the previous clause:

\ea
\label{ex:onekilo}
Yan pļa \textbf{sidda} \textbf{na} \textbf{sikad} \textbf{bakod}. Padaļa danen ta baybay \textbf{daw} baligya singko isya kilo.\\\smallskip
\gll Yan pļa \textbf{sidda} \textbf{na} \textbf{sikad} \textbf{bakod}. Pa-daļa danen ta baybay \textbf{daw} ...-baligya singko isya kilo. \\
\textsc{d2abs} surprise fish \textsc{lk} very big \textsc{t.r}-carry 3p{erg} \textsc{nabs} beach and \textsc{t.r}-sell five one kilogram \\
\glt ‘That surprise was a fish that was very big. They took (it) to the beach and sold (it) for five (pesos for) one kilo.’ [DBOE-C-05 1.5-6]
\z

Undergoers are less-often omitted when they are non-absolutive. This is because non-absolutive undergoers tend to be non-topical in the discourse, and therefore not as easily recovered as participants expressed in the absolutive. Example \REF{ex:attheschool} illustrates a retained non-absolutive undergoer, \textit{ti} `\textsc{d1nabs}', referring back to  \textit{baļon} `packed lunch':

\ea
\label{ex:attheschool}
Gadaļa \textbf{a} nang \textbf{baļon ko} \textbf{daw} naan \textbf{a} nang gakaan \textbf{ti} iskwilahan i.\\\smallskip
\gll Ga-daļa \textbf{a} nang \textbf{baļon} \textbf{ko} \textbf{daw} naan \textbf{a} nang ga-kaan \textbf{ti} iskwila-an i. \\
\textsc{i.r}-carry 1\textsc{s.abs} only/just packed.lunch 1\textsc{s.gen} and \textsc{spat.def} 1\textsc{s.abs} only/just \textsc{i.r}-eat \textsc{d1nabs} school-\textsc{nr} \textsc{def.n}\\
\glt ‘I carried my packed lunch and I just ate (it) at the school.’ [DBON-C-07 2.1]
\z

Example \REF{ex:andateit} illustrates two omitted absolutive undergoers, and one omitted non-absolutive undergoer. The Actor, \textit{nay} `1\textsc{p.incl.erg}', is also omitted in the second and third clauses:

\ea
\label{ex:andateit}
Gani	dayon		\textbf{nay} anien 			\textbf{daw} lutuon		\textbf{daw}	magkaan. \smallskip\\
\gll Gani	dayon		\textbf{nay} ani-en			\textbf{daw} luto-en	\textbf{daw}	mag-kaan\\
So 	right.away	1\textsc{p.incl.erg} harvest-\textsc{t.ir} 	and cook-\textsc{t.ir} 		 and	\textsc{i.ir}-eat \\
\glt `So, right away we harvest (previously mentioned coconut, corn, sorghum) and cook (it) and eat (it).' [SFOE-T-06 5.10]
\z

In example \REF{ex:shewateredit} the absolutive undergoer, \textit{niog} `coconut (palm), is omitted in the second clause:

\ea
\label{ex:shewateredit}
Gani, papalangga \textbf{din} man \textbf{yi} \textbf{na} \textbf{niog} \textbf{daw} adlaw-adlaw \textbf{din} pabunyagan.\\\smallskip
\gll Gani, pa-palangga \textbf{din} man \textbf{yi} \textbf{na} \textbf{niog} \textbf{daw} adlaw-adlaw \textbf{din} pa-bunyag-an. \\
so \textsc{t.r}-have.affection 3\textsc{s.erg} also \textsc{d1abs} \textsc{lk} coconut.palm and \textsc{red}-day 3\textsc{s.erg}   \textsc{t.r}-irrigate-\textsc{apl} \\
\glt ‘So, she had affection for this coconut tree and every day she watered (it).’ [PBWN-C-12 13.4]
\z

Example \REF{ex:topofthetree} illustrates two intransitive clauses coordinated with \textit{daw}:

\ea
\label{ex:topofthetree}
Gakatay en \textbf{kuti} \textbf{i} \textbf{daw} galayog man \textbf{manok} \textbf{i} naan punta ta ugbos ta kaoy.\\\smallskip
\gll Ga-katay en \textbf{kuti} \textbf{i} \textbf{daw} ga-layog man \textbf{manok} \textbf{i} naan punta ta ugbos ta kaoy. \\
\textsc{i.r}-climb \textsc{cm} cat \textsc{def.n} and \textsc{i.r}-fly also chicken \textsc{def.i} \textsc{spat.def} going.to \textsc{nabs} top \textsc{gen} tree \\
\glt ‘The cat climbed and also the chicken flew going to the very top of the tree.’ [CBWN-C-18 7.11]
\z

Examples \REF{ex:alsotheyhelped} and \REF{ex:wearingasuit} each illustrate a grammatically transitive and an intransitive clause in a coordinate construction with different actors and different absolutives.

\newpage
\ea
\label{ex:alsotheyhelped}
Dayon \textbf{ko} pilak mga kaoy an \textbf{daw} gatabang man \textbf{danen} \textbf{an}.\\\smallskip
\gll Dayon \textbf{ko} ...-pilak mga kaoy an \textbf{daw} ga-tabang man \textbf{danen} \textbf{an}. \\
right.away 1\textsc{s.erg} \textsc{t.r}-throw.away \textsc{pl} wood \textsc{def.m} and \textsc{i.r}-help also 3\textsc{p.abs} \textsc{def.m} \\
\glt `Right away I threw away the wood and also they helped.’ [CBWN-C-11 4.27]
\z

\ea
\label{ex:wearingasuit}
Pag-uļog din ya, tag-iya ya ta bļangay, dayon \textbf{kanen} ya tugpa \textbf{daw} peseb \textbf{din} bata ya na Pedro na nuļog na gaamirikana kanen an. \\
\gll Pag-uļog din ya, tag-iya ya ta bļangay, dayon \textbf{kanen} ya ...-tugpa \textbf{daw} pa-eseb \textbf{din} bata ya na Pedro na na-uļog na ga-amirikana kanen an. \\
\textsc{nr.act}-fall 3\textsc{s.erg} \textsc{def.f} owner \textsc{def.f}  \textsc{gen} two.masted.boat right.away 3\textsc{s.abs} \textsc{def.f} \textsc{i.r}-jump and \textsc{t.r}-dive.to.get 3\textsc{s.erg} child \textsc{def.f} \textsc{lk} Pedro \textsc{lk} \textsc{a.hap.r}-fall \textsc{lk} \textsc{i.r}-suit 3\textsc{s.abs} \textsc{def.m} \\
\glt ‘When he fell, as for the owner of the two-masted boat, right away he jumped (into the sea) and dove underwater (to get) the child Pedro who had fallen (and) was wearing a suit.’ [PCON-C-01 3.11]
\z

Example \REF{ex:gotsomecookedrice} illustrates three clauses conjoined with \textit{daw}:

\ea
\label{ex:gotsomecookedrice}
Tapos kay kaan, listo \textbf{kay} eman lisinsya na manaw \textbf{daw} nanay ta barkada ko gakamang ta tinapaan na sidda na deen \textbf{nay} muli naan ta Sintro \textbf{daw} bai na duma nay, gakamang man ta kan-en …\\\smallskip
\gll Tapos kay ...-kaan, listo \textbf{kay} eman ...-lisinsya na m-panaw \textbf{daw} nanay ta barkada ko ga-kamang ta t<in>apa-an na sidda na daļa-en \textbf{nay} m-uli naan ta Sintro \textbf{daw} bai na duma nay, ga-kamang man ta kan-en … \\
after 1\textsc{p.exc.abs} \textsc{i.r}-eat promptly  1\textsc{p.exc.abs} again.as.before \textsc{i.r}-ask.permission \textsc{lk} \textsc{i.v.ir}-leave/walk and mother \textsc{gen} friend 1\textsc{s.gen} \textsc{i.r}-get \textsc{nabs} <\textsc{nr.res}>smoke-\textsc{apl} \textsc{lk} fish \textsc{lk} carry-\textsc{t.ir} 2\textsc{p.excl.erg} \textsc{i.v.r}-go.home \textsc{spat.def} \textsc{nabs} Central  and woman \textsc{lk} companion 2\textsc{p.excl.gen} \textsc{i.r}-get also \textsc{nabs} cooked.rice \\
\glt ‘After we ate, promptly we again as before requested permission to leave  and the mother of my friend got some smoked fish which we would take to Central and as for the woman our companion, she got some cooked rice …’ [CBWN-C-11 3.3]
\z

\subsection{\textit{asta}, ‘until’}
\label{sec:asta}
\textit{Asta} (a Spanish word that means ‘until’) usually means ‘until a certain time, place or event.’  As a conjunction between clauses, it can have a resultative sense-the situation expressed in clause B is a result of the situation expressed in clause A:

\ea
Pelles en angin an. Darko baļed \textbf{asta} en mga layag ni ubos en ta gisi.\\\smallskip
\gll Pelles en angin an. Darko baļed \textbf{asta} en mga layag ni ubos en ta gisi. \\
strong.wind \textsc{cm} wind/air \textsc{def.m} big.\textsc{pl} wave until \textsc{cm} \textsc{pl} sail \textsc{d1abs} all \textsc{cm} \textsc{nabs} tear \\
\glt ‘The wind was strong. The waves were big and (as a result) as for the sails, these were all torn.’ [PCON-C-01 2.16]
\z

\ea
Pabatangan no ta tellek saging i a \textbf{asta} nang en na tama tellek lawa ko i.\\\smallskip
\gll Pa-batang-an no ta tellek saging i a \textbf{asta} nang en na tama tellek lawa ko i. \\
\textsc{t.r}-put-\textsc{apl} 2\textsc{s.erg} \textsc{nabs} thorn banana \textsc{def.n} \textsc{inj} until only/just \textsc{cm} \textsc{lk} many thorn body 1\textsc{s.gen} \textsc{def.n} \\
\glt ‘You put thorns on (the trunk of) the banana plant and (the result is) my body has lots of thorns.’ [CBWN-C-16 9.4]
\z

\subsection{\textit{daw dili}, `if not/and not'}
\label{sec:dawdili}
\textit{Daw dili} as a fixed expression usually means ‘if not’, ‘and not’, or ‘but rather’ depending on the context. However, sometimes it simply coordinates independent clauses, with a conjunction or disjunction sense. We consider example \REF{ex:usetwomastedboats} to illustrate clause coordination even though the verb (\textit{mabyai} ‘travel’) is omitted in the second clause due to coreferentiality:

\ea
\label{ex:usetwomastedboats}
Pagamit nang pa \textbf{ta} \textbf{mga} \textbf{ittaw} unti, daw mabyai naan Iloilo \textbf{daw} \textbf{dili} gani naan ta minland Palawan, pagamit \textbf{danen} bļangay.\\\smallskip
\gll Pa-gamit nang pa \textbf{ta} \textbf{mga} \textbf{ittaw} unti, daw mabyai naan Iloilo \textbf{daw} \textbf{dili} gani naan ta minland Palawan, pa-gamit \textbf{danen} bļangay. \\
\textsc{t.r}-use only/just still \textsc{nabs} \textsc{pl} person here and travel \textsc{spat.def} Iloilo and \textsc{neg} truly \textsc{spat.def} \textsc{nabs} mainland Palawan \textsc{t.r}-use 3\textsc{p.erg} two.masted.boat \\
\glt ‘People here still use, when traveling to Iloilo if not (or) to mainland Palawan, they use two-masted boats.’
\z

\ea
… kalabanan ta mga mamy daw dady ubos may ubra ta upisina ta darko na mga kumpanya paryo abi manigir ka \textbf{daw} \textbf{dili} gaubra ka ta Municipyo.\\\smallskip
\gll … kalabanan ta mga mamy daw dady ubos may ubra ta upisina ta darko na mga kumpanya paryo abi manigir ka \textbf{daw} \textbf{dili} ga-ubra ka ta Municipyo. \\
{} most \textsc{nabs} \textsc{pl} mom and dad all \textsc{ext.in} work \textsc{nabs} office \textsc{gen} big \textsc{lk} \textsc{pl} company like for.example manager 2\textsc{s.abs} and \textsc{neg} \textsc{i.r}-work 2\textsc{s.abs} \textsc{nabs} town.hall \\
\glt ‘… most of the moms and dads all have work in offices of big companies like for example you are a manager if not (or) you work in the town hall.’ (RZWE-J-01 15.3)
\z

\subsection{\textit{o} disjunction ‘or’}
\label{sec:o}
\is{disjunction|(}
The Kagayanen word \textit{o} is from Spanish ‘or’. It usually expresses alternatives between two conjuncts of equal syntactic status. As a conjunction between clauses, it sometimes presents a reiteration or paraphrase of an idea, as in \REF{ex:tookcareofthem}:

\ea
\label{ex:tookcareofthem}
Ta pugya na timpo \textbf{kabaw} \textbf{i} \textbf{daw} \textbf{baka} isya nang ta istaran \textbf{o} isya nang ta tag-iya na gasagod \textbf{ki} \textbf{danen}.\\\smallskip
\gll Ta pugya na timpo \textbf{kabaw} \textbf{i} \textbf{daw} \textbf{baka} isya nang ta istar-an \textbf{o} isya nang ta tag-iya na ga-sagod \textbf{ki} \textbf{danen}. \\
\textsc{nabs} long \textsc{lk} time water.buffalo \textsc{def.n} an cow one only/just \textsc{nabs} live-\textsc{nr} or one only/just \textsc{nabs} owner \textsc{lk} \textsc{i.r}-care.for \textsc{obl.p} 3p \\
\glt ‘A long time ago the water buffalo and the cow had only one place to live or had only one owner who took care of them.’ [CBWN-C-25 2.1]
\is{disjunction|(}
\z

\subsection{Culminative use of irrealis modality in clause coordination}
\label{sec:culminativeuse}
\is{culminative|(}
In \chapref{chap:verbstructure}, \sectref{sec:intransitiveirrealis} we briefly discussed what we describe as the \textit{culminative} use of irrealis modality in narrative chains of events. In this section we will provide additional examples and discussion. Recall that irrealis modality is one of the inflectional values in Kagayanen. Verbs in irrealis modality can be considered fully inflected, and therefore fully finite. However, when a clause terminates a narrative chain of events, it may appear in irrealis modality, even though semantically-according to the content of the narrative-the event is presented as an accomplished fact. In this section we will present the irrealis marked verbs in bold.

Example \REF{ex:weslepttherealso} is from a long narrative in which the narrator and company climb a very high mountain. Example \REF{ex:weslepttherealso} describes what they did when they finally arrived at a house on the mountain. The use of irrealis modality in the last two verbs highlights the fact that this is the culmination of an arduous journey.

\ea
\label{ex:weslepttherealso}
Naan kay dya anay gadayon, \textbf{magpuay} kay daw naan kay man \textbf{magtunuga}.\\\smallskip
\gll Naan kay dya anay ga-dayon, \textbf{mag-puay} kay daw naan kay man \textbf{mag-tunuga}. \\
\textsc{spat.def} 1\textsc{p.excl.abs} \textsc{d4loc} for.awhile \textsc{i.r}-stay \textsc{i.ir}-rest 1\textsc{p.excl.abs} and \textsc{spat.def} 1\textsc{p.excl.abs} also \textsc{i.ir}-sleep \\
\glt ‘There we stayed for a while, we rested and we slept there also.’ [PCON-C-01 6.3]
\z

Examples \REF{ex:lyingflatagain} and \REF{ex:tothetown} are both from a long sad narrative. Example \REF{ex:lyingflatagain} is the final sentence of a long episode in which a father tricked his child into going with him to the mountains where the father killed the child and buried her. The paragraph following this one introduces a new episode describing what the mother of the child did when she found their child missing.

\ea
\label{ex:lyingflatagain}
Pagtapos din tampek ta lungag ya, dayon kanen uli daw \textbf{magdapa-dapa} isab.\\\smallskip
\gll Pag-tapos din tampek ta lungag ya, dayon kanen ...-uli daw \textbf{mag-dapa-dapa} isab. \\
\textsc{nr.act}-finish 3\textsc{s.gen} pack.soil \textsc{nabs} hole \textsc{def.f} right.away 3\textsc{s.abs} \textsc{i.r}-return.home and \textsc{i.ir}-lie.flat again\\
\glt ‘After he packed (soil) in the hole, right away he went home and kept on lying flat again. [PBWN-C-12 7.1]
\z

Example \REF{ex:tothetown} describes what happened when the police arrested the man for murder. Again, this excerpt terminates an episode, though it is not the end of the story:

\ea
\label{ex:tothetown}
Pag-abot danen naan ta kapitto na bukid, dayon din man tudlo daw indi din dapit palebbenga. Pagkita danen ta lebbengan ya, gamandar dayon mga pulis an ki kanen na kutkuton din iya na bata. Pagkita danen, dayon danen pati na kanen matuod nakapatay ta bugtong danen na bata. Uļa en maimo iya na sawa daw dili sigi nang en agaļ daw sigi nang man en na pababawi. Gani, patampekan isab danen daw \textbf{muli} \textbf{naan ta banwa}.\\\smallskip
\gll Pag-abot danen naan ta ka-pitto na bukid, dayon din man tudlo daw indi din dapit
pa-lebbeng-a. Pag-kita danen ta lebbeng-an ya, ga-mandar dayon mga pulis an ki kanen na
kutkot-en din iya na bata. Pag-kita danen, dayon danen pati na kanen matuod naka-patay ta
bugtong danen na bata. Uļa en ma-imo iya na sawa daw dili sigi nang en agaļ daw sigi nang man en na pa-ba~bawi. Gani, pa-tampek-an isab danen daw \textbf{m-uli} \textbf{naan}
\textbf{ta} \textbf{banwa}. \\
\textsc{nr.act}-arrive 3\textsc{p.abs} \textsc{spat.def} \textsc{nabs} \textsc{ord}-seven \textsc{lk} mountain right.away 3\textsc{s.erg} also point and direction 3\textsc{s.erg} direction
dig-\textsc{t.ir} 3\textsc{s.erg} 3\textsc{s.gen} \textsc{lk} child \textsc{nr.act}-see 3\textsc{p.gen} right.away 3\textsc{p.erg}  believe \textsc{lk} 3\textsc{p.abs} truly \textsc{i.hap.r}-die
\textsc{t.r}-bury-\textsc{xc} \textsc{nr.act}-see 3\textsc{p.gen} \textsc{nabs} bury-\textsc{nr} \textsc{def.f} \textsc{i.r}-command immediately \textsc{pl} police \textsc{def.m} \textsc{obl.p} 3s \textsc{lk}
\textsc{nabs} only 3\textsc{p.gen} \textsc{lk} child \textsc{neg} \textsc{cm} \textsc{i.r}-do 3\textsc{s.gen} \textsc{lk} spouse and \textsc{neg} continue only/just \textsc{cm} \textsc{i.ir}-cry and continue only/just also
\textsc{cm} \textsc{lk} \textsc{t.r}-\textsc{red}~revive then \textsc{t.r}-pack.soil-\textsc{apl} again 3\textsc{p.erg} and \textsc{i.v.ir}-return.home \textsc{spat.def} \textsc{nabs} town \\
\glt ‘When they (the police with the father who killed his child) arrived on the seventh mountain, he (the father) pointed out in what direction he buried (the child). When they saw the grave, the police immediately commanded him to dig up his child. When they saw (the child), right away they believed that he truly had killed their one and only child. There was nothing his spouse could do, except to keep on crying and keep on being revived. So, they packed (soil on it) again and \textbf{went home to the town}.’ [PBWN-C-12 12.2]
\z

Example \REF{ex:andthenbathed} is from a different narrative in which a water buffalo and a cow have a discussion of how they will go to the river and swim without their owner knowing it. Example \REF{ex:andthenbathed} describes what they do when they arrive at the river. The next paragraph gives more details of what they did when they were swimming.

\ea
\label{ex:andthenbathed}
Pag-abot danen naan ta suba, palubbas danen iran na bayo daw pabatang danen naan ta kilid ta suba daw \textbf{maglangoy}.\\\smallskip
\gll Pag-abot danen naan ta suba, pa-lubbas danen iran na bayo daw pa-batang danen naan ta kilid ta suba daw \textbf{mag-langoy}.\\
\textsc{nr.act}-arrive 3\textsc{p.abs} \textsc{spat.def} \textsc{nabs} river \textsc{t.r}-remove 3\textsc{p.erg} 3\textsc{p.gen} \textsc{lk} clothes and \textsc{t.r}-put 3\textsc{p.erg} \textsc{spat.def} \textsc{nabs} bank \textsc{lk} river and \textsc{i.ir}-swim \\
\glt ‘When they arrived at the river, they took off their clothes and put (them) on the bank of the river and then bathed.’ [CBWN-C-25 5.4]
\z

There is also a tendency for irrealis modality to be used at points of high tension or episodic climax of a narrative. Since such points tend to be characterized by multiple closely linked events, this usage often overlaps with the culminative usage. A future discourse study is needed to elucidate the factors that contribute to the use of irrealis modality to express realis events at certain points in narrative discourse. Example \REF{ex:andkilledhim} illustrates irrealis modality in the second conjunct of two conjoined clauses at a point of high tension in a story:

\ea
\label{ex:andkilledhim}
Lugar na gatago kanen i nakita din iya na magulang na galebbeng naan ta bļawan daw padakep ta mga ittaw magulang ya daw \textbf{gapuson} daw \textbf{patayen}.\\\smallskip
\gll lugar na ga-tago kanen i na-kita din iya na magulang an ga-lebbeng naan ta  bļawan daw pa-dakep ta mga ittaw magulang ya daw \textbf{gapos-en} daw \textbf{patay-en}. \\
then \textsc{lk} \textsc{i.r}-hide 3\textsc{s.abs} \textsc{def.n} \textsc{a.hap.r}-see 3\textsc{s.erg} 3\textsc{s.gen} \textsc{lk} older.sibling \textsc{def.m} \textsc{i.r}-bury \textsc{spat.def} \textsc{nabs} gold and \textsc{t.r}-capture \textsc{nabs} \textsc{pl} person older.sibling \textsc{def.f} and tie.up-\textsc{t.ir} and die-\textsc{t.ir} \\
\glt ‘Then when he was hiding, he saw his older sibling (brother) being buried in gold and the people captured the older sibling and \textbf{tied (him) up} and \textbf{killed (him)}.’ [CBWN-C-22 12.4]
\is{culminative|)}
\is{coordinate clauses|)}
\z
