\chapter{Modification}
\label{chap:modification}
\section{Introduction}
\label{sec:introduction-4}

In this chapter we discuss the very general communicative function of \textit{modification}\is{modification|(}.  Modification is a function that can be performed within any construction type, and contrasts with other functions such as \textit{syntactic} \textit{headship},\is{heads, syntactic} \textit{predication}\is{predication}, \textit{participant}\is{reference} \textit{reference}\is{}\is{},\is{}\is{} and \textit{complementation}\is{complementation}.

We define modification as the function of any “optional” substantive element\footnote{By “substantive element” \is{substantive element}we mean words or phrases that express semantically rich meanings. Such elements contrast with grammatical functors\is{grammatical functors}\is{functors, grammatical}, such as pronouns, articles, deictic elements, and affixes, which tend to express limited and relatively precise grammatical meanings. This distinction was probably first proposed by \citet{fries1952}, and has been largely assumed in subsequent linguistic research. See, for example, \citet[66--67]{payne2011}.} that clarifies or adds detail to the discourse scene\is{discourse scenes|(}\is{scenes|(} evoked by a particular grammatical construction, without changing the construction’s basic propositional meaning. This definition is based on the metaphor of communication as a “stage” on which communicators depict and interpret scenes being communicated. The scenes evoked by individual lexical items and basic constructions tend to be sparse and \textit{schematic}\is{schematic (characteristic of discourse scenes)}. Sometimes the communication situation invites speakers to enrich the discourse scene in order to develop a more explicit picture of the idea being expressed. For example, consider the following imaginary English utterances:

\ea
\label{bkm:Ref441656793}
    \ea 
    \label{bkm:Ref441656793-a}
    We had dinner. \\
    \ex
    \label{bkm:Ref441656793-b}
    Honestly, we had the best dinner at a new Italian restaurant last night.
    \z
\z
Example \REF{bkm:Ref441656793-a} is a complete, grammatical sentence that is likely to evoke a certain scene in the minds of English speakers. However, the scene depicted is very vague and, frankly, uninteresting. In a Western context there may be a table and chairs, at least two diners and some unspecified food, but little else “on stage”.  On the other hand, a construction like \REF{bkm:Ref441656793-b} evokes a much more enriched version of possibly the same scene. Now we have a better idea of what kind and quality of food may be present, something about the surroundings, the presence of servers, money, the approximate time that this event is alleged to have occurred, and also the speaker’s attitude toward this scene (“honestly”). The scene is still vague, to be sure. We don’t know, for example, how many other diners were present, exactly what menu items were ordered, and so on. Nevertheless, \REF{bkm:Ref441656793-b} is much more explicit and communicative than \REF{bkm:Ref441656793-a}.

The difference between \REF{bkm:Ref441656793-a} and \REF{bkm:Ref441656793-b} is mostly modification. First there is modification of the noun “dinner”.  This very general notion has been restricted to a certain class of dinners, those  portrayed by the speaker as “the best”, “new” and``Italian” similarly enrich the scene by modifying the noun “restaurant”. Then there are modifying elements that restrict the scene to a place (“at a new Italian restaurant”) and a time (“last night”). Finally, there is one modifying element that specifies the speaker’s attitude toward the speech act itself (“honestly”). None of this information is technically required by the grammar, but all of it enriches and clarifies the scene in ways that make it more consistent with the communicative intentions of the speaker.\is{discourse scenes|)}\is{scenes|)}

In Kagayanen, modifying elements occur in many construction types. In this chapter we describe elements and constructions that serve as modifiers within the Referring Phrase (\sectref{bkm:Ref422117197}{}-\ref{bkm:Ref52774536}), the Modifier Phrase (\sectref{bkm:Ref418142991}), the Predicate  (\sectref{bkm:Ref420904083}), and the Clause (\sectref{bkm:Ref441599506} and \sectref{bkm:Ref481473714}).\is{modification|)}

\section{Adjectives}
\label{bkm:Ref422117197} \label{sec:adjectives} \is{adjectives|(}

Adjectives are a grammatically distinct class of words that express \is{property concepts|(}“property concepts”, i.e., semantic notions that describe the properties of items \citep{thompson2004}. \citet[3--4]{dixon2004} categorizes the semantic properties that tend to underlie the word class of adjectives in any language into core types and peripheral types as follows:

\ea
Core semantic property types often expressed by adjectives: \\
\textsc{Dimension}: big, small, long, tall, short, wide, deep, etc. \\
\textsc{Age}: new, young, old, etc. \\
\textsc{Value}: good, bad, lovely, atrocious, perfect, proper/real, odd, strange, curious, necessary, crucial, important, lucky, etc. \\
\textsc{Color}: black, white, red, etc.
\z

\newpage
\ea
Peripheral semantic property types often expressed by adjectives: \\
\textsc{Human} \textsc{propensity} {}- jealous, happy, kind, clever, generous, cruel, proud, ashamed, eager, etc. \\
\textsc{Physical property}: hard, soft, heavy, wet, strong, clean, hot, sour, etc. And a   subclass referring to corporeal properties: well, sick, dead, absent. \\
\textsc{Speed}: fast, quick, slow, etc.
\z

Because the notion of “property concept” is semantic (it has to do with the ideas being expressed in language), its boundaries are “fuzzy” and not necessarily comparable from one language to the next. For this reason, we must look at language-internal evidence in order to identify adjectives, rather than simply assume that words that translate as adjectives in one language (say, English) will necessarily be expressed with a distinct category that can be called “adjectives” in another language.

\citet{dixon2004} contends that every language does have a grammatically distinct class of forms whose prototypical function is to refer to property concepts, though the morphosyntactic criteria for distinguishing this class may be subtle, and are different from one language to the next. \citet{thompson2004} shows that in some languages property concept words are grammatically similar to referring words (“nouns”), while in other languages they are similar to predicating words (“verbs”). Still other languages express property concepts in a more noun-like way when they are used \textit{attributively}\is{attributive!use of property concept words} (i.e., within a Referring Phrase to modify the head noun---“the \textit{red} barn”), but in a more verb-like way when used \textit{predicatively}\is{predicative!use of property concept words} (i.e., to assert a property of some item---“the barn is \textit{red}”).

The grammar of Kagayanen treats adjectives and nouns similarly, in both the attributive and predicative usages. For example, the bare forms of prototypical nouns\is{nouns!prototypical}, like \textit{baļay} ‘house’, and prototypical property concept words, like \textit{bakod} ‘big’, can both function as the Head of a Referring Phrase (examples in \ref{ex:bighouse}, and as the main predicator in a non-verbal predication \is{predicative!use of nouns}(examples in \ref{bkm:Ref422575828}; see also \chapref{chap:non-verbalclauses}):

\ea 
\label{ex:bighouse}
    \ea
        \gll bakod  na  baļay / baļay na bakod \\
        big  \textsc{lk}  house \\
        \glt ‘a big house’
   \ex 
       \gll bakod nai \\
            big \textsc{d1adj} \\
       \glt ‘this big (one)’
   \z
\z

\ea \label{bkm:Ref422575828} 
    \ea
    \label{bkm:Ref422575828-a}
        \gll Baļay  ni. \\
        house \textsc{d1abs} \\
        \glt ‘This is a house.’
    \ex
    \label{bkm:Ref422575828-b}
        \textit{Bakod na baļay ni. / Baļay na bakod ni.} \\
        ‘This is a big house.’
    \ex
    \label{bkm:Ref422575828-c}
        \textit{Bakod ni.} \\
        ‘This is big/a big one.’
    \z
\z

Despite these important grammatical similarities, there are subtle ways in which property concept words are treated differently than nouns. First, while property concept words can always function as Heads\is{heads, syntactic} or Modifiers in \isi{Referring Phrases}, not all nouns can freely function as attributive Modifiers\is{attributive!use of nouns}:

\ea 
\label{bkm:Ref422580186}
    \ea[*]{ \label{bkm:Ref422580186-a}
        \gll kaoy  na  baļay / *baļay na kaoy \\
          wood  \textsc{lk}  house \\
        \glt (‘wood(en) house’)
        }
    \ex \label{bkm:Ref422580186-b}
        \gll Kaoy  baļay  nai. / ?Balay nai kaoy. \\
        wood  house   \textsc{d}1\textsc{adj} \\
        \glt ‘This house is wood(en).’
    \ex \label{bkm:Ref422580186-c}
        \gll baļay  na  buat  ta  kaoy \\
        house  \textsc{lk}  make  \textsc{nabs}  wood \\
        \glt ‘house that is/was made of wood’
    \z
\z
\ea
\label{bkm:Ref422580189}
    \ea[*]{
    \label{bkm:Ref422580189-a}
        \gll kawayan  na  saag  /  *saag na kawayan \\
            bamboo  \textsc{lk} floor \\
        \glt (‘bamboo floor’)
    }
    \ex
    \label{bkm:Ref422580189-b}
        \gll Kawayan  saag  nai. / Saag nai kawayan. \\
            bamboo  floor  \textsc{d}1\textsc{adj} \\
        \glt  ‘This floor is (made of) bamboo.’
    \ex 
    \label{bkm:Ref422580189-c}
        \textit{Kaoy saag nai. / Saag nai kaoy.} \\
        ‘This floor is wood(en).’
    \z
\z

Examples \REF{bkm:Ref422580186-a} and \REF{bkm:Ref422580189-a} show that the nouns \textit{kaoy} ‘wood’ and \textit{kawayan} ‘bamboo’ may not function as modifiers of the head \textit{baļay}. \REF{bkm:Ref422580186-b}, \REF{bkm:Ref422575828-b} and \REF{bkm:Ref422575828-c} show that \textit{kaoy} and \textit{kawayan} may function predicatively\is{predicative!use of nouns} to express a material composition. However, a relative clause structure more commonly expresses the idea of ‘house made of wood’ \REF{bkm:Ref422580186-c}. Some other “material composition” words may directly modify the head of an RP, as shown in \REF{bkm:Ref422580415}:

\ea
\label{bkm:Ref422580415}
    \ea   
        \gll siminto  na  baļay / baļay na siminto \\
        cement  \textsc{lk}  house \\
        \glt ‘cement house’ (house made of cement)
    \ex
        \gll plastik  na  baso / baso na plastik \\
          plastic  \textsc{lk}  glass \\
        \glt ‘plastic drinking glass’
    \ex
        \gll bagoļ  na  luag  / luag na bagoļ \\
          coconut.shell  \textsc{lk}  serving.spoon \\
        \glt ‘coconut shell serving spoon’
    \ex
        \gll bato  na  padir / padir na bato \\
        stone  \textsc{lk} wall \\
        \glt ‘stone wall’
    \z  
\z

Example \REF{bkm:Ref422581440} shows that even \textit{kawayan} can be used as a modifier with head nouns other than \textit{saag}:

\ea
\label{bkm:Ref422581440}
\gll kawayan na kuraļ / kuraļ na kawayan \\
bamboo \textsc{lk} fence \\
\glt ‘bamboo fence’
\is{property concepts|)}
\z

The exact patterns of which nouns can modify which other nouns seem to be lexically determined. Perhaps some noun-noun pairs in which one noun modifies the other are compound words (\ref{bkm:Ref422581440}). However, we have no independent reason for calling these compounds. They exhibit the same intonational characteristics and variability as Referring Phrases, and their meanings are as compositional and transparent as those of ordinary RPs. Furthermore, the fact that the order of elements may be reversed, as in most modifier+noun constructions, is an argument against the analysis of these noun-noun pairs as lexicalized compounds. The point of these illustrations, however, is to show that property concept words can always be Modifiers or Heads of RPs, while referring words (nouns) can always be Heads, but only sometimes Modifiers of RPs. This is one subtle grammatical property that distinguishes \isi{adjectives} and \isi{nouns}.

A second grammatical property that distinguishes adjectives as a class is their ability to be modified by the \isi{degree adverbs} \textit{sikad} ‘very’, \textit{segeng} ‘extremely’, \textit{gid} ‘really’, \textit{tise (nang)} ‘a little bit’ and \textit{midyo} ‘somewhat’. The following are a few examples. Additional examples from the text corpus are found in \sectref{bkm:Ref418142991} below.

\ea
\gll tise  nang  gisi  na  bayo \\
little  just  torn  \textsc{lk}  clothes/shirt \\
\glt ‘slightly torn clothes/shirt’ /
*tise nang bayo na gisi\footnote{The word \textit{tise} as a noun modifier means small in number or size. As a modifier of a gradable quality, such as spoiled, red or smelly, it means slightly. As a modifier of an activity it means almost (see \sectref{bkm:Ref480610640}). The word \textit{sise} means small in size. But sometimes Kagayanen speakers use these two words interchangeably. When the word \textit{tise} means small in number (‘a few’), or a little bit of a quality, as in example \REF{bkm:Ref441658460}, then it is always followed by \textit{nang}. In its other usages, \textit{nang} is optional. The form \textit{tise (nang)}, like most adverbial elements, may also function as a noun modifier. However, in that case the linker \textit{na} is required: \textit{tise nang na bayo} ‘a few clothes’, ‘small clothes’. This usage, however, is far less common than the usage as a degree adverb.}
\z

\ea
\label{bkm:Ref441658460}
\gll tise  nang  baļ{}-es  na  kan-en \\
little  just  spoiled  \textsc{lk}  cooked.rice \\
\glt ‘slightly spoiled cooked rice’
\z
\ea
\gll sikad  bakod  na  baļay \\
very  big  \textsc{lk}  house \\
\glt ‘a very big house’
\z
\ea
\gll baļay  na  sikad  bakod \\
house  \textsc{lk}  very  big \\
\glt ‘a very big house’ \\\smallskip

\textit{*sikad baļay na bakod \\
*sikad siminto na baļay \\
*sikad bato na padir} \\
  etc.
\z
\ea
\gll sikad gid  gwapa  na  dļaga \\
very  really  attractive  \textsc{lk}  young.woman \\
\glt ‘a really very attractive young woman’
\z
\ea
\gll dļaga  na  sikad  gid  gwapa \\
young.woman  \textsc{lk}  very \textsc{int}  attractive \\ 
\glt ‘a really very attractive young woman’ / *sikad dļaga gid na gwapa
\z
\ea
\textit{segeng/sikad minog na bayo} ‘extremely/very red clothes’ \\
\textit{segeng/sikad ammot na buro}  ‘extremely/very good-smelling salted-fish’ \\
\textit{segeng/sikad bao na buro}  ‘extremely/very bad-smelling salted-fish’ \\
\textit{segeng/sikad sakit na nina}  ‘extremely/very continual wound’ \\
\textit{segeng/sikad bugnaw na waig}  ‘extremely/very cold water’ \\
\textit{segeng/sikad darko na baļed}   ‘extremely/very big waves’ \\
\textit{segeng/sikad pelles na angin}  ‘extremely/very strong wind’ \\
\textit{segeng/sikad sakit na adlaw}  ‘extremely/very continual (hot) sun’ \\
\textit{segeng/sikad dangga na adlaw}  ‘extremely/very hot day’ \\
\textit{segeng/sikad langaet na adlaw}  ‘extremely/very humid day’ \\
etc.
\z

This property of course only applies to gradable property concepts, like \textit{bakod} ‘big’, \textit{gwapa} ‘attractive’ and the others illustrated in these examples. In fact, we may say that all true adjectives in Kagayanen express gradable property concepts\is{property concepts!gradable}. Non-gradable concepts\is{property concepts!non-gradable} such as \textit{mama} ‘male’ and \textit{bai} ‘female’ exhibit no properties that distinguish them from nouns:

\ea
    \ea
        \gll mama  na  bata  /  bata na mama \\
        male  \textsc{lk} child \\
        \glt `male child’
    \ex
        \gll bai  na  bata  /  bata na bai \\
          female  \textsc{lk}  child \\
        \glt ‘female child’
    \z
\z
\ea
    \ea
        \textit{mama  na  manok   /  manok na mama}  {} `male chicken/rooster.'
    \ex 
        \textit{bai na manok / manok na bai} ‘female chicken/hen’
    \z
\z
\ea
    \ea
        \textit{mama  na  ingkantado\footnotemark / ingkantado na mama}  ‘male fairy’ \\
    \ex
        \textit{bai na ingkantada / ingkantada na bai}   ‘female fairy’
        \footnotetext{There are a few words in Kagayanen, all borrowings from Spanish, which reflect a masculine/feminine distinction. In each such pair, -\textit{o} expresses the masculine and -\textit{a} the feminine gender, as in Spanish.}
    \z
\z

Finally, the last morphosyntactic property that may be said to distinguish a word class of adjectives is that this class can be the \textit{target} of certain word-forming patterns. The resulting forms may function as Heads of RPs, Modifiers of RPs, or as non-verbal Predicators. Furthermore, most of them may occur with intensity and degree adverbs (examples in \ref{bkm:Ref422804351}), and so may be considered true adjectives. In the following subsections we describe ten adjective-forming processes that occur in the corpus for this study.\is{adjectives|)}

\section{Adjective forming morphological patterns}
\label{sec:adjectiveformingprocesses} \is{adjective-forming morphological patterns|(} \is{adjectives!derived|(}

In this section we describe several regular morphological patterns that derive gradable property concept words (or``adjectives"). Some of these patterns also have roles in nominal and verbal derivations. In this section we concentrate on morphology that derives \isi{adjectives}, making note of similarities to other derivational categories where applicable.

\subsection{\textit{ma}{}- derivation}
\label{sec:ma-derivation}

An adjective can be formed with a prefix \textit{ma}{}- and/or suffix \nobreakdash-\textit{én} (described in \sectref{bkm:Ref117512242}). The resulting form refers to a salient quality associated with the root concept:

\ea
\begin{tabbing}
\hspace{5cm} \= \kill 
\textbf{Root}   \>   \textbf{Derivation} \\
\textit{asin } ‘salt’  \>  \textit{masin } ‘salty' \\
\textit{lised } ‘hard/difficult’ \> \textit{malised } ‘distressful/sorrowful’ \\
\textit{ligna}\footnotemark{}  ‘filthy/dirty/evil thing’ \> \textit{maligna } ‘filthy/defiled/disgusting’ \\
\textit{lain } ‘not good/bad’ \> \textit{malain } ‘evil’ \\
\textit{iseg } ‘aggressive/brave/fearless’ \> \textit{maiseg } ‘mean/cruel/deadly’ \\
\textit{law-ay } ‘unsightly/ugly’ \> \textit{malaw-ay } ‘obscene/indecent’ \\
\textit{gastos } ‘expense’ \> \textit{magastos } ‘expensive/costly’ \\
\textit{ļangkaw } ‘long’ \>   \textit{mļangkaw  } ‘longer than usual/elongated’ \\
\textit{las-ay } ‘tasteless’ \> \textit{malas-ay } ‘tasteless/unappetizing’ \\
\textit{lineng } ‘quiet/peaceful’ \> \textit{malineng } ‘quiet/orderly/calm’ \\
\textit{lipo } ‘short’ \>   \textit{malipo } ‘shorter than usual/expected’ \\
\textit{adyo } ‘far distance’ \> \textit{madyo } ‘distant’ \\
\textit{abeļ } ‘dull for a blade  or tool’ \> \textit{mabeļ } ‘dull, not sharp’ \\
\textit{ingaw } ‘drunk’ \>   \textit{mingaw } ‘lonely/bored’ \\
\textit{sadya } ‘to have fun/to enjoy’ \> \textit{masadya } ‘enjoyable/cheerful/fun’ \\
\textit{tam-is } ‘sweet’  \>  \textit{matam-is } ‘sweeter than usual’ \\
\textit{tawway } ‘peaceful/no strife’ \> \textit{matawway } ‘peaceful in mind and soul’
\end{tabbing}
\footnotetext{This is probably a backformation from \textit{maligna} (column 2), which is the feminine form of the Spanish adjective meaning ‘evil’. This pair provides one piece of evidence that adjectivization with \textit{ma}{}- is a productive morphological process.}
\z

\subsection{\textit{-én/-ón} derivation}
\label{bkm:Ref117512242}

Certain roots employ the suffix -\textit{én} (with the allomorph -\textit{ón}) to express property concepts based on the root concept. Since this suffix is stressed, we write the stress in the following examples, even though it is not written in the official orthography. Stress distinguishes this derivation from the transitive irrealis verbal suffix {}-\textit{en} (see \chapref{chap:verbstructure}, \sectref{sec:transitiveirrealis}). It is similar to the nominalizing suffix {}-\textit{én/-ón} that creates future patient nouns (see \chapref{chap:referringexpressions}, \sectref{sec:en}), though there is no futurity associated with the adjective-forming usage. The allomorph {}-\textit{ón} occurs when the last vowel in the root is /u/.

\ea
\begin{tabbing}
\hspace{5cm} \= \kill
\textbf{Root}   \>    \textbf{Derived form} \\
\textit{béļbeļ} ‘body hair/fur/feathers’ \> \textit{bèļbeļén} ‘hairy/feathery/furry’ \\
\textit{katéļ} ‘itchy’   \> \textit{kàteļén} ‘itchy with sores’ \\
\textit{mutá} ‘rheum’ \>   \textit{mùtaén} ‘rheumy-eyed’ \\
\textit{lasáy} ‘moving around’ \> \textit{làsayén} ‘always moving around’ \\
\textit{báto} ‘rock/stone’ \>   \textit{bàtuón} ‘rocky’ \\
\textit{masákit } ‘sick/sickness’ \> \textit{masàkitén} ‘sickly/unhealthy’
\end{tabbing}
\z
\subsection{\textit{ma}{}- … -\textit{én} derivation}
\label{sec:ma-en-derivation}

Certain other roots employ both \textit{ma}{}- and -\textit{én} to form adjectives that express permanent characteristics.

\ea
\begin{tabbing}
\hspace{5cm} \= \kill
\textbf{Root}  \>  \textbf{Derived form} \\
\textit{lipáy} ‘joy’ \>   \textit{malìpayén} ‘joyful’ \\
\textit{sadyá} ‘enjoy/fun’ \> \textit{masàdyaén} ‘always enjoying’ \\
\textit{luóy} ‘mercy/compassion/pity’ \> \textit{malùuyón} ‘always compassionate/pitying’ \\
                                    \> \textit{malùluy-ón} ‘always helping/giving \\
                                    \> because of compassion/pity’ \\
\textit{égtas} ‘irritated/annoyed’ \> \textit{maègtasén} ‘always irritable’ \\
\textit{isá} ‘selfish’ \>   \textit{maìsaén} ‘always selfish’ \\
\textit{paébes} ‘humble’ \> \textit{mapaèbesén/mapainèbesén  } ‘always \\
                        \> humble’ \\
\textit{pasínsya} ‘patient’ \> \textit{mapasìnsyaén} ‘always patient’ \\
\textit{patáwad} ‘forgive’  \>  \textit{mapatàwarén} ‘always forgiving’ \\
\textit{palángga} ‘affection’ \> \textit{mapalànggaén} ‘affectionate’ \\
\textit{adlék} ‘fear’ \>   \textit{maàdlekén} ‘always afraid’ \\
\textit{bugál} ‘proud’  \>  \textit{mabùgalén} ‘always proud’ \\
\textit{gáyya} ‘embarrass/shame’ \> \textit{magàyyaén} ‘always shy’ \\
\textit{ínggit} ‘envious’ \>   \textit{maìnggitén} ‘always envious’ \\
\textit{ímon} ‘jealous’  \>  \textit{maìmunón} ‘always jealous’ \\
\textit{lális} ‘to defy’  \>  \textit{malàlisén} ‘always defying’ \\
\textit{táod} ‘to respect/honor’ \>  \textit{matàurón/matinàurón} ‘always \\
                                \> respectful/polite’ \\
\textit{nļáman(áļam)} ‘to know’ \> \textit{màļamen/mànļamén} ‘wise/knowledgeable’ \\
\textit{dépet} ‘diligent’ \> \textit{madèpetén} ‘always diligent’ \\
\textit{dayá} ‘to deceive/cheat’ \> \textit{madàyaén} ‘always deceiving/cheating’ \\
\textit{iyá} ‘his/hers’ \>   \textit{maìyaén} ‘always concerned with self’ \\
\textit{tábang/tábyang} ‘to help’ \> \textit{matàbangén/matàbyangén} ‘always helpful’
\end{tabbing}
\z

All these derived property concept words may occur with intensity and degree adverbs,\is{adverbs!intensity}\is{adverbs!degree} indicating that they fall into the category of adjectives:

\ea
\label{bkm:Ref422804351}
\begin{tabbing}
\hspace{5cm} \= \kill
\textit{Sikad (gid) masin na mga sidda} \> ‘some (really) very salty fish’ \\
\textit{Sikad (gid) beļbeļén na mama}
\> ‘a (really) very hairy man’ \\
\textit{Sikad (gid) mutaén na bata} \> ‘a (really) very rheumy-eyed child’ \\
\textit{Sikad (gid) masakitén na mama} \> ‘a (really) very sickly man’ \\
\textit{Sikad (gid) matabangén na bai} \> ‘a (really) very helpful woman’ \\
\textit{Sikad (gid) maisaén na bata} \> ‘an always (really) very selfish child’ \\
\textit{Sikad (gid) malipayén na bata} \> ‘an always (really) very joyful child’ \\
etc.
\end{tabbing}
\z
\subsection{\textit{ka- … -én/-ón} derivation}
\label{sec:ka-en-derivation}
\textit{Ka}{}- … -\textit{én/-ón} occurs on a few verbal roots, notably \textit{tanuga} ‘sleep’, \textit{kaan} ‘eat’, \textit{agaļ} ‘cry’ and other body activity verbs. The meaning of the resultant adjective is the property of feeling like one is about to \textsc{verb} or has the urge to \textsc{verb}: This is very similar to the nominalizing affix having the same form \textit{ka}{}- … -\textit{én/-ón}. It attracts the main stress of the word, and it is pronounced as \textit{ka}{}- … \textit{{}-ón} when the vowel of the final syllable of the root is /u/ (see \chapref{chap:referringexpressions} \sectref{sec:ka-en}).

\ea
\begin{tabbing}
\hspace{5cm} \= \kill
\textbf{Root}    \>  \textbf{Derived form} \\
\textit{ágaļ} ‘to cry’ \> \textit{kàgaļén} ‘having the urge to cry’ \\
\textit{íi} ‘to urinate’ \> \textit{kìién  } ‘having the urge to urinate’ \\
\textit{iném} ‘to drink’ \> \textit{kaìnemén} ‘having the urge to drink’ \\
\textit{táwa} ‘to laugh’ \> \textit{katàwaén} ‘having the urge to laugh’ \\
\textit{pánaw} ‘go/walk’ \> \textit{kapànawén} ‘having the urge to go’ \\
\textit{tanúga} ‘to sleep’ \> \textit{katanùgaén/tanùgaén} ‘… to sleep’ \\
\textit{kaán} ‘to eat’ \> \textit{kakàn-enén/kàn-enén} ‘… to eat’ \\
\textit{índis} ‘to defecate’ \> \textit{kaìndisén/kindisén } ‘… to defecate’ \\
\textit{kéngkeng} ‘to hold in the arms’ \>  \textit{kakèngkengén} ‘… to hold in the arms’
\end{tabbing}
\z

As with most adjective-forming derivational affixes, the forms \textit{ka}{}- and -\textit{én} are also related to inflectional verb affixes (\chapref{chap:verbstructure}). \textit{ka}{}- is a happenstantial mood prefix, while \nobreakdash-\textit{en} (without stress) is a dynamic mood, transitive, irrealis suffix. As verb affixes, these forms are distinct members of the same paradigm, and can never simultaneously inflect the same root. Therefore, the functions of these forms as adjectivizers are clearly distinct from their uses as verbal inflectional affixes. The following are some additional examples of \textit{ka- … -én} formations used as adjectives:

\ea
\textbf{Katanugaén}  kanen  ya  na  ittaw. \\\smallskip
 \gll \textbf{Ka-tanuga-én}  kanen  ya  na  ittaw.\footnotemark \\
\textsc{adj}-sleep-\textsc{adj}  3\textsc{s.abs}  \textsc{def.f}  \textsc{lk}  person \\
\footnotetext{The Predicate in this construction is \textit{katanugaén na ittaw} ‘a sleepy person’. The absolutive argument, \textit{kanen ya} ‘s/he (far away)’ is in the position after the first element in the predicate. We describe this phenomenon as “pronoun intrusion” in \chapref{chap:non-verbalclauses} and beyond.}
\glt ‘S/he is a sleepy person.’ (‘S/he is a person who often has the urge to sleep’)
\z
\ea
Kanen  ya  sikad  \textbf{katanugaén}  na  ittaw. \\\smallskip
 \gll Kanen  ya  sikad  \textbf{ka-tanuga-én}  na  ittaw. \\
3\textsc{s.abs}  \textsc{def.f}  very  \textsc{adj}-sleep-\textsc{adj}  \textsc{lk}  person \\
\glt ‘S/he is a very sleepy person.’
\z
\ea
Pirmi  a  nang  gabatyag  ta  \textbf{kiién}  daw  paryo  a man  ta  \textbf{kindisén}. \\\smallskip
 \gll Pirmi  a  nang  ga-batyag  ta  \textbf{kiién}  daw  paryo  a man  ta  \textbf{ka-indis-én}. \\
always  1\textsc{s.abs}  just  \textsc{i.r}-feel  \textsc{nabs}  \textsc{adj}-urinate-\textsc{adj}  and  same  1\textsc{s.abs} also  \textsc{nabs}  \textsc{adj}-defecate-\textsc{adj} \\
\glt `I kept feeling the urge to urinate and it was like I had the urge to defecate.’ (This is a story of a man who had a severe asthma attack and almost died) [JCWN-T-22 3.10].
\z
\ea
Na  yaken  i  naan  ta  annem  na  taon,  yaken  may inagian  asta  anduni  daw  mademdeman  ko  paryo a  pa  \textbf{kagaļén}. \\\smallskip
 \gll Na  yaken  i  naan  ta  annem  na  taon,  yaken  may <in>agi-an  asta  anduni  daw  ma-demdem-an  ko  paryo a  pa  \textbf{ka-agaļ-én}. \\
when  1\textsc{s.abs}  \textsc{def.n}  \textsc{spat.def}  \textsc{nabs}  six  \textsc{lk}  year  1\textsc{s.abs}  \textsc{ext.in}
<\textsc{nr.res}>pass-\textsc{nr}  until  now/today  if/when  \textsc{a.hap.ir}-remember-\textsc{apl}  1\textsc{s.erg}  same 1\textsc{s.abs}  \textsc{inc}  \textsc{adj}-cry-\textsc{adj} \\
\glt `When I was six years old, I had an experience that until now when I remember it, it is like I still feel like crying.’ [HBWN-T-01 1.1]
\z
\ea
\textbf{Katawaén} waļeng  din  an. \\\smallskip
 \gll \textbf{Ka-tawa-én} waļeng  din  an. \\
\textsc{adj}-laugh/smile-\textsc{adj}  face  3\textsc{s.gen}  \textsc{def.m} \\
\glt ‘His face (looks like he) has the urge to laugh.’\footnote{This sentence was used in a conversation to refer to a picture depicting several people. It is marginally acceptable out of context. Usually a person would be the Head of the RP modified by a \textit{ka-…-én} adjective, but in this case it is the person’s face.}
\z

\subsection{\textit{pinaka}- derivation – superlative}\is{superlative|(}
\label{sec:pinaka-derivation}

The complex prefix \textit{pinaka}{}- may derive the superlative form of some adjectives:
\ea
Yi  \textbf{pinakamaadlek}  na  masakit  na  gabot  ta  Cagayancillo. \\\smallskip
 \gll Yi  \textbf{pinaka-ma-adlek}  na  masakit  na  ga-abot  ta  Cagayancillo. \\
\textsc{d}1\textsc{abs}  \textsc{superl}-\textsc{adj}-fear  \textsc{lk}  sickness  \textsc{lk}  \textsc{i.r}-arrive  \textsc{nabs}  Cagayancillo \\
\glt ‘This is  the most fearsome sickness that arrived on Cagayancillo.’ [JCWN-T-21 20.1]
\z
\ea
Kanen  gid  \textbf{pinakagwapa}  ta  iran  na  lugar. \\\smallskip
 \gll Kanen  gid  \textbf{pinaka-gwapa}  ta  iran  na  lugar. \\
3\textsc{s.abs}  \textsc{int}  \textsc{superl}-attractive  \textsc{nabs}  3\textsc{p.gen}  \textsc{lk}  place \\
\glt ‘She is really the most attractive in their place.’ [VAWN-T-20 2.5]
\z
\ea
\textbf{Pinakamadali}  ta  pagkamang  ta  sidda  bungbong. \\\smallskip
 \gll \textbf{Pinaka-ma-dali}  ta  pag-kamang  ta  sidda  bungbong. \\
\textsc{superl}-\textsc{adj}-quick/easy  \textsc{nabs}  \textsc{nr.act}-get  \textsc{nabs}  fish  dynamite \\
\glt ‘Dynamite is the quickest/easiest (way) of getting fish.’ [EFOB-C-01 6.3]
\z
\ea
Danen  gid  \textbf{pinakauryan}  uli  na  naan  ta  patyo. \\\smallskip
 \gll Danen  gid  \textbf{pinaka-uryan}  uli  na  naan  ta  patyo. \\
3\textsc{p.abs}  \textsc{int}  \textsc{superl}-last  go.home  \textsc{lk}  \textsc{spat.def}  \textsc{nabs}  graveyard \\
\glt ‘They are the very last to go home when (they are) in the graveyard.’ [CBWE-C-06 10.3]
\z

\subsection{\textit{ka- … -an} derivation – absolute extent}
\label{sec:ka-an-derivation}

For other adjectives, the prefix-suffix combination \textit{ka- … -an} forms an adjective or noun (see \chapref{chap:referringexpressions}, \sectref{sec:ka-an}) that refers to the greatest possible extent of a quality. Sometimes words so derived may be interpreted as superlative adjectives, as in \REF{bkm:Ref52463441}:
\ea
\label{bkm:Ref52463441}
Ake  na  \textbf{kamaguļangan}  na  bata  na  bai  nagatapos unduni ta  sais  grado. \\\smallskip
 \gll Ake  na  \textbf{ka-maguļang-an}  na  bata  na  bai  naga-tapos\footnotemark{} unduni ta  sais  grado. \\
1\textsc{s.gen}  \textsc{lk}  \textsc{nr}-older.sibling-\textsc{nr}  \textsc{lk}  child  \textsc{lk}  female  \textsc{i.r}-finish  now/today
\textsc{nabs}   six  grade \\
\footnotetext{The prefix \textit{naga}{}- on this word is code switching from Ilonggo. The Kagayanen form would be \textit{gatapos}.}
\glt `My oldest daughter is now finishing grade six.’ [VPWL-T-04 76.5]
\z

Other times, the \textit{ka- … -an} derivation forms a \isi{comparative} nominalization, as described in \chapref{chap:referringexpressions} \sectref{sec:ka-an}. For example \textit{katumanan} ‘the absolute end point or boundary’, \textit{katapusan} `absolute end', \textit{kaumpisaan} ‘absolute beginning place’ and \textit{kaunaan} ‘very first place’, and example \REF{bkm:Ref113278812} from the corpus:

\ea
\label{bkm:Ref113278812}
Uyi  na  isturya  ni  \textbf{kamatuuran}  gid. \\\smallskip
 \gll U-yi  na  isturya  ni  \textbf{ka-matuod-an}  gid. \\
\textsc{emph-d}1\textsc{abs}  \textsc{lk}  story  \textsc{d}1\textsc{pr}  \textsc{nr}-true-\textsc{nr}  \textsc{int} \\
\glt ‘This very story is the \textbf{absolute} \textbf{truth}.’ [MBON-T-07 61.-2]
\is{superlative|)}
\z

\subsection{\textit{pala}{}- derivation}
\label{sec:pala-derivation}

The prefix \textit{pala-} on verbal roots derives an adjective that refers to the property of always liking to do the action described by the verb. This is not a common derivation, and no examples of it occur in the text corpus. All the following examples come from conversation or elicitation. Words derived with \textit{pala}{}- are clearly adjectives as evidenced by the fact that they may occur with degree adverbs (\ref{bkm:Ref439853390}):

\ea
\begin{tabbing} 
\hspace{3.5cm} \= \kill
inem  ‘to drink’ \> palainem  ‘likes to always drink’ \\
ingaw  ‘drunk’ \>   palaingaw  ‘likes to always get drunk’ \\
kaan  ‘to eat’  \>  palakaan  ‘likes to always eat’ \\
butaļ  ‘to fight/argue’ \> palabutaļ  ‘likes to always fight/argue’ \\
sayaw  ‘to dance’ \> palasayaw  ‘likes to always dance’
\end{tabbing}
\z
\ea
\label{bkm:Ref439853390}
Sikad  \textbf{palabutaļ}  na  bai  nan. \\\smallskip
 \gll Sikad  \textbf{pala-butaļ}  na  bai  nan. \\
very  likes.always-fight/argue  \textsc{lk}  woman  \textsc{d}3\textsc{abs} \\
\glt ‘That one is a woman who really likes to fight/argue.’
\z
\subsection{\textit{maka}{}- derivation}
\label{sec:maka-derivation}

The prefix \textit{maka-} on verbal roots describes the quality of allowing, enabling or causing the activity described by the verb. This is also a relatively uncommon derivation occurring only eight times in the text corpus (\ref{bkm:Ref125305525}):

\ea
\label{bkm:Ref125305525}
\begin{tabbing}
\hspace{4cm} \= \kill
\textit{adlek} ‘to fear/be afraid’ \> \textit{makaadlek} ‘scary/frightening’ \\
\textit{patay} ‘to die/be dead’ \> \textit{makapatay} ‘deadly, able to kill’ \\
\textit{luoy} ‘pity/compassion’ \> \textit{makaluluoy} ‘pitiful’ \\
\textit{ilo} ‘poison'   \>   \textit{makailo} ‘poisonous’ \\
\textit{sablag} ‘to disturb’ \> \textit{makasablag} ‘able to disturb’ \\
\textit{tabang} ‘to help’  \>  \textit{makatabang} ‘helpful’ \\
\textit{biskeg} ‘strong’  \>  \textit{makabiskeg} ‘strengthening’
\end{tabbing}
\z

The meaning of \textit{maka}{}- as an adjectivizer is clearly related to its meaning as a verbal inflectional prefix as described in \chapref{chap:verbstructure}, \sectref{sec:ambitransitivehappenstantialirrealis} as happenstantial, often abilitative modality. Unlike the inflectional verbal prefix, however, this usage of \textit{maka}{}- does not express irrealis modality.  Examples \REF{bkm:Ref117581373} through \REF{bkm:Ref117580692} illustrate additional examples of \textit{maka}{}- from the corpus:

\ea
\label{bkm:Ref117581373}
Makaadlek  na  Masakit.  Masakit  na  kulira  makaadlek. \\\smallskip
 \gll Maka-adlek  na  Masakit.  Masakit  na  kulira  maka-adlek. \\
\textsc{adj}-afraid  \textsc{lk}  sickness  sickness  \textsc{lk}  cholera  \textsc{adj}-afraid \\
\glt ‘Frightening Sickness (the title of the text). The sickness cholera is frightening.' (first sentence of the same text) [JCWN-T-21 1.1-2]
\z
\ea
…  agian  din  na  bilang  sikad  \textbf{makaadlek} \\\smallskip
 \gll …  agi-an  din  na  bilang  sikad  \textbf{maka-adlek} \\
{} pass-\textsc{nr}  3\textsc{s.erg}  \textsc{lk}  as  very  \textsc{adj}-afraid \\
\glt ‘…his/her experience that was as (something) very scary’ [JCOE-T-06 2.5]
\z

\newpage
\ea
\label{bkm:Ref117580692}
Yi  \textbf{makailo}  ig  gapanipis  ta  langgit  ta  lima  no  na  gapalot  ka. \\\smallskip
 \gll Yi  \textbf{maka-ilo}  ig\footnotemark{}  ga-pa-nipis  ta  langgit  ta  lima  no  na  ga-palot  ka. \\
\textsc{d}1\textsc{abs}  \textsc{adj}-poison  and  \textsc{i.r}-\textsc{caus}-thin  \textsc{nabs}  skin  \textsc{nabs}  hand  2\textsc{s.gen}
\textsc{lk}  \textsc{i.r}-peel  2\textsc{s.abs} \\
\footnotetext{The use of \textit{ig} as a conjunction is code-switching from \isi{Cuyonon}. The Kagayanen form here would be \textit{daw}. We will not mention this in subsequent examples, as it is very frequent in conversation and in writing.}
\glt `This is \textbf{poisonous} and causes the skin of your hand to become thin when you are peeling (it).’ (This is about a kind of root crop called \textit{kallot} that is poisonous before it is processed the right way). [JCWE-L-32 5.8]
\z
\subsection{Root reduplication}
\label{sec:rootreduplication-derivation}

Complete or partial root reduplication may derive downtoning adjectives from other adjectives, similar to the adjective suffix -\textit{ish} in English (\textit{greenish, smallish}, etc.). It may also express a slight comparative degree (see example \ref{bkm:Ref117581975}). For partial reduplication, the first consonant and vowel are reduplicated, and if the root ends in a consonant, the final consonant is also reduplicated. All roots that reduplicate allow complete reduplication, but not all allow partial reduplication. In particular, roots with closed initial syllables do not allow partial reduplication, for example, \textit{darko}, \textit{derse}, \textit{tambek} and \textit{sagbak} (see \ref{bkm:Ref481477458} below).

\ea
\begin{tabbing}
\hspace{4cm} \= \hspace{1cm} \= \hspace{1cm} \= \hspace{.1cm} \kill
Complete Reduplication: \>  Root\> → Root-Root \\
       \> bakod\> → bakod-bakod\footnotemark \\
       \> darko\> → darko-darko \\
Partial Reduplication, V-final root:  C\textsubscript{1}V\textsubscript{1}XV\textsubscript{2}\# → C\textsubscript{1}V\textsubscript{1}C\textsubscript{1}V\textsubscript{1}XV\textsubscript{2}\# \\
 \>             sise\> → sisise \\
Partial Reduplication, C-final root:  C\textsubscript{1}V\textsubscript{1}XC\textsubscript{2}\# → C\textsubscript{1}V\textsubscript{1}C\textsubscript{2}XC\textsubscript{2}\# \\
\>              dalem\> → damdalem
\footnotetext{It is conventional in Kagayanen spelling to insert a dash between completely reduplicated roots. In such cases, the dash does not indicate a glottal stop.}
\end{tabbing}
\z

When property-concept roots are reduplicated, either completely or partially, they downtone\is{downtoning} the meaning of the root, expressing the idea of ‘kind of X’ or ‘sort of X’ (examples in \ref{bkm:Ref423363798}).

\newpage
\ea
\label{bkm:Ref423363798}\label{bkm:Ref481477458}
\begin{tabbing}
\hspace{3cm} \= \hspace{2.8cm} \= \hspace{2.5cm} \= \kill   
\textbf{Root} \> \textbf{Complete} \> \textbf{Partial} \>   \textbf{Meaning} \\
\>  \textbf{reduplication}     \> \textbf{reduplication} \\
\textit{sise}   ‘small'        \> \textit{sise-sise    } \> \textit{sisise  }                \> ‘kind of small (sg.)’ \\
\textit{daļem}  ‘deep’         \> \textit{daļem-daļem  } \> \textit{damdaļem}                \>  ‘kind of deep’ \\
\textit{deļem}  ‘dark’         \> \textit{deļem-deļem  } \> \textit{demdeļem}                \>  ‘kind of dark’ \\
\textit{daan}   ‘old (things)’ \> \textit{daan-daan    } \> \textit{dandaan }                \>  ‘kind of old’ \\
\textit{layog}  ‘tall’         \> \textit{layog-layog  } \> \textit{laglayog}                \>  ‘kind of tall’ \\
\textit{bakod}  ‘big’          \> \textit{bakod-bakod  } \> \textit{bakbakod}\footnotemark   \> ‘kind of big’ \\
\textit{darko}  ‘big (pl)’     \> \textit{darko-darko  } \>                                  \>  ‘kind of big (pl.)’ \\
\textit{derse}  ‘small (pl)’   \> \textit{derse-derse  } \>                                  \>  ‘kind of small (pl.)’ \\
\textit{tambek} ‘fat’          \> \textit{tambek-tambek} \>                                  \>  ‘kind of fat’ \\
\textit{sagbak} ‘noisy’        \> \textit{sagbak-sagbak} \>                                  \>  ‘kind of noisy’
\end{tabbing}
\footnotetext{This is an exception to the partial reduplication rule described above. The predicted form would be *\textit{badbakod}, but this does not exist.}
\z

The quantifier \textit{tanan} ‘all’ and the adjective \textit{matuod} ‘true’ when reduplicated have a \isi{superlative} meaning ‘completely/absolutely X’:

\ea
\begin{tabbing}
\hspace{3cm} \= \hspace{2.8cm} \= \hspace{2.5cm} \= \kill   
\textit{tanan} ‘all' \>  \textit{tanan-tanan} \> \textit{tantanan}   \> ‘absolutely all’ \\
\textit{matuod} ‘true’ \> \textit{matuod-tuod} \> \textit{matudtuod}  \>  ‘absolutely true’
\end{tabbing}
\z

\hspace*{-4.9pt}Reduplicated adjectives may also be used in \isi{comparative constructions} in which the standard and the item compared are not very distinct:
\ea
\label{bkm:Ref117581975}
Iya  na  labbot  bakod-bakod  pa  ta  iya  na  lawa …. \\\smallskip
 \gll Iya  na  labbot  bakod\sim{}bakod  pa  ta  iya  na  lawa …. \\
3\textsc{s.gen}  \textsc{lk}  bottom  \textsc{red}\sim{}big  \textsc{inc}  \textsc{nabs}  3\textsc{s.gen}  \textsc{lk}  body \\
\glt ‘His bottom is kind of bigger yet than his body…’ (This is about a coconut crab.) [DBWE-T-27 2.3]
\z

\subsection{\textit{di}{}- derivation}
\label{di-derivation}

The prefix \textit{di}{}- is probably borrowed from Spanish \textit{de} ‘of’. The meaning of the resulting form is ‘of a certain kind’. Occasionally, the resulting form can function as a Modifier, as in examples \REF{bkm:Ref481478039}{}-\REF{bkm:Ref481478100}:


\ea
\label{bkm:Ref481478039}
Ta  una  na  timpo  naan  ta  banwa  Cagayancillo  uļa anay  mga  sakayan  na  \textbf{dimakina}. \\\smallskip
 \gll Ta  una  na  timpo  naan  ta  banwa  Cagayancillo  uļa anay  mga  sakay-an  na  \textbf{di-makina}. \\
\textsc{nabs}  first/for.awhile  \textsc{lk}  time  \textsc{spat.def}  \textsc{nabs}  town/country  Cagayancillo  \textsc{neg.r} for.a.while \textsc{pl} ride-\textsc{nr}  \textsc{lk}  \textsc{adj}-motor/machine \\
\glt `In the earliest times in the town of Cagayancillo there were at first no \textbf{motorized} vehicles.’ [VPWN-T-06 1.3]
\z
\ea
Napalit  din  ki  kami  tenga  sako  na  beggas daw  mga  \textbf{dilata}  na  sid-anan. \\\smallskip
 \gll Na-palit  din  ki  kami  tenga  sako  na  beggas daw  mga  \textbf{di-lata}  na  sid-anan. \\
\textsc{a.hap.r}-buy  3\textsc{s.erg}  \textsc{obl.p}  1\textsc{p.exc}  half  sack  \textsc{lk}  milled.rice
and  \textsc{pl} \textsc{adj}-can  \textsc{lk}  viand\footnotemark \\
\footnotetext{‘Viand’ is a word in Philippine English that refers to any meat, fish or vegetable dish served over rice. It comes from French, through 19\textsuperscript{th}{}-20\textsuperscript{th} century English, and came to be used in the Philippines during the American colonial period (1898--1946).}
\glt `He bought for us half a sack of rice and some \textbf{canned} viand.’ [DBWN-T-22 7.13]
\z
\ea
\label{bkm:Ref481478100}
Labi  na  gid  ta  \textbf{dipamilya}  rispunsibilidad… \\\smallskip
 \gll Labi  na  gid  ta  \textbf{di-pamilya}  rispunsibilidad \\
especially  \textsc{lk}  \textsc{int}  \textsc{nabs}  \textsc{adj}-family  responsibility… \\
\glt ‘Especially the \textbf{family} \textbf{kind} of responsibility…' [TTOB-L-03 9.12]
\z

In example \REF{bkm:Ref117583999} a form derived with \textit{di-} is used in a predicative context. \textit{Serbidor} is a noun meaning ‘servant’; \textit{diserbidor} is a modifier meaning ‘serving one’:
\ea
\label{bkm:Ref117583999}
Nyaan  sa  en  nļaman  danen  na  Pedro \textbf{disirbidor} gid  ta  Dios. \\\smallskip
 \gll Nyaan  sa  en  na-aļam-an  danen  na  Pedro \textbf{di-sirbidor} gid  ta  Dios. \\
\textsc{spat.def}  \textsc{d4nabs}  \textsc{cm}  \textsc{a.hap.r-}know-\textsc{apl}  3\textsc{p.erg}  \textsc{lk}  Pedro \textsc{adj}-servant \textsc{int}  \textsc{nabs}  God \\
\glt`There they knew that Pedro was truly the kind of person who is a \textbf{servant} of God.’ [TTOB-L-03 7.25]
\z

The \textit{di}- prefix occurs mostly with roots borrowed from Spanish (as in \ref{bkm:Ref481478039}{}-\ref{bkm:Ref117583999}), but may occur on original Kagayanen roots, as in \REF{bkm:Ref113279533}:
\ea
\label{bkm:Ref113279533}
Gasakay  danen  an  ta    \textbf{dilayag}  na  lunday. \\\smallskip
 \gll Ga-sakay  danen  an  ta    \textbf{di-layag}  na  lunday. \\
\textsc{i.r}-ride  3\textsc{p.abs}  \textsc{def.m}  \textsc{nabs}  \textsc{adj}-sail \textsc{lk} outrigger.canoe \\
\glt ‘They rode a \textbf{sail} \textbf{kind} of outrigger canoe.’
\is{adjectives!derived|)}
\is{adjective-forming morphological patterns|)}
\z

\section{Numbers}
\label{sec:numbers} \is{numbers|(}

The second type of modification in the Referring Phrase that we will discuss is the numbers---one, two, three; first, second, third, and so on. As with most Philippine languages, numbers in Kagayanen are complicated by the fact that there are two number systems---the original Austronesian-based Kagayanen numbers and the Spanish-based numbers\is{numbers!Spanish-based}, spelled according to the Kagayanen pronunciations. Furthermore, since math is taught mostly in English, English numbers are also quite common. However, in this grammar, we will discuss only the Austronesian and Spanish-based number systems. While both systems are decimal (base ten), they differ in their structure and use.

The Kagayanen numbers\is{numbers!Austronesian} are either cardinal with no affixes (\sectref{bkm:Ref479572983}), ordinal \REF{bkm:Ref479339690}, or distributive \REF{bkm:Ref479572999}. Ordinal\is{ordinal numbers}\is{numbers!ordinal} and distributive numbers\is{distributive numbers}\is{numbers!distributive} above one are derived by affixation from the cardinals, as described in \sectref{bkm:Ref479339690} and \sectref{bkm:Ref479572999}. Numbers normally occur in the pre-Head (MOD\textsubscript{1}) position in the Referring Phrase, though they may appear following the Head (in MOD\textsubscript{2}). See \sectref{bkm:Ref418107822} below for a detailed description of \isi{Number Phrases} (NumPs):

\ea
darwa  na  adlaw \\\smallskip
 \gll darwa  na  adlaw \\
two  \textsc{lk}   day \\
\glt `two days'
\z
\ea
pangarwa   na  adlaw \\\smallskip
 \gll pang-darwa   na  adlaw \\
\textsc{ord}-two  \textsc{lk}  day \\
\glt ‘second day’
\z

Like other modifying elements, numbers and Number Phrases may also function as predicates of non-verbal clauses (\ref{bkm:Ref329087374}). See also \chapref{chap:non-verbalclauses} on non-verbal clauses):
\ea
\label{bkm:Ref329087374}
Darwa  nang  ame   na  bata. \\\smallskip
 \gll Darwa  nang  ame   na  bata. \\
two  only  1\textsc{p.excl.gen}  \textsc{lk}  child \\
\glt ‘We have only two children.’ (lit. ‘Our children are only two.’)
\z
\subsection{Cardinal numbers}
\label{bkm:Ref479572983}
\is{cardinal numbers|(}\is{numbers!cardinal|(}

The original Kagayanen cardinal numbers are used for counting, especially up to ten. After ten the Spanish-based number\is{Spanish-based numbers}\is{numbers!Spanish-based} set is more common, probably because the names of Spanish numbers are shorter. For counting money and for \isi{clock time}, the Spanish-based numbers are usually used.
\ea
\begin{tabbing}
\hspace{4cm} \= \hspace{4cm} \= \kill 
\textbf{Cardinal numbers:} \\
\textbf{Kagayanen numbers   }                   \> \textbf{Spanish-based numbers} \> \textbf{Gloss} \\
\textit{isya/isļa              }       \> \textit{uno/una } \>  ‘one’ \\
\textit{darwa                  }       \> \textit{dos} \> ‘two’ \\
\textit{tallo                  }       \> \textit{tris } \>  ‘three’ \\
\textit{appat                  }       \> \textit{kwattro } \>  ‘four’ \\
\textit{limma                  }       \> \textit{singko } \>  ‘five’ \\
\textit{annem                  }       \> \textit{sais } \>  ‘six’ \\
\textit{pitto                  }       \> \textit{syiti } \>  ‘seven’ \\
\textit{waļļo                  }       \> \textit{utso } \>  ‘eight’ \\
\textit{isyam/syam             }       \> \textit{nwibi } \>  ‘nine’ \\
\textit{sampuļo                }       \> \textit{dyis } \>  ‘ten’ \\
\textit{sampuļo (daw) isya     }       \> \textit{unsi } \>  ‘eleven’ \\
\textit{sampuļo (daw) darwa    }       \> \textit{dusi } \>  ‘twelve’ \\
\textit{sampuļo (daw) tallo    }       \> \textit{trisi } \>  ‘thirteen’ \\
\textit{sampuļo (daw) appat    }       \> \textit{katursi } \>  ‘fourteen’ \\
\textit{sampuļo (daw) limma    }       \> \textit{kinsi } \>  ‘fifteen’ \\
\textit{sampuļo (daw) annem    }       \> \textit{dyisisais } \>  ‘sixteen’ \\
\textit{sampuļo (daw) pitto    }       \> \textit{dyisisyiti } \>  ‘seventeen’ \\
\textit{sampuļo (daw) waļļo    }       \> \textit{dyisiutso } \>  ‘eighteen’ \\
\textit{sampuļo (daw) (i)syam  }       \> \textit{dyisinwibi } \>  ‘nineteen’ \\
\textit{kaļuan/kuļuan          }       \> \textit{binti/byinti } \>  ‘twenty’ \\
\textit{katluan                }       \> \textit{trinta } \>  ‘thirty’ \\
\textit{kappatan               }       \> \textit{kwarinta } \>  ‘forty’ \\
\textit{kalim-an               }       \> \textit{singkwinta } \>  ‘fifty’ \\
\textit{kanneman               }       \> \textit{saisinta } \>  ‘sixty’ \\
\textit{kappituan              }       \> \textit{syitinta } \>  ‘seventy’ \\
\textit{kawaļļuan              }       \> \textit{utsinta } \>  ‘eighty’ \\
\textit{kasyaman               }       \> \textit{nubinta  } \>  ‘ninety’ \\
\textit{isya gatos             }       \> \textit{syintos } \>  ‘one hundred’ \\
\textit{darwa gatos            }       \> \textit{dos syintos } \>  `two hundred' \\
\textit{libo                   }       \> \textit{mil     } \>  ‘one thousand’ \\
\textit{sampuļo libo           }       \> \textit{dyis mil } \>  ‘ten thousand’ \\
                                       \> \textit{milyon  }  \> ‘one million’
\end{tabbing}
\z

\largerpage
Clock time\is{clock time|(} is expressed in the Spanish way with the word \textit{ala} for `o’clock' (from Spanish \textit{a la} ‘at the.\textsc{fem.sg}’) and \textit{alas} ‘plural’ (Spanish \textit{a las} ‘at the.\textsc{fem.pl}’) for the other numbers, as in \REF{bkm:Ref329088549}:

\ea
\label{bkm:Ref329088549}
\begin{tabbing}
\hspace{3cm} \= \kill
\textit{ala una } \>  ‘one o’clock’ \\
\textit{alas dos } \>  ‘two o’clock’ \\
\textit{alas tris } \>  ‘three o’clock’ \\
\textit{alas kwattro } \>  ‘four o’clock’ \\
\textit{alas singko } \>  ‘five o’clock’ \\
\textit{alas sais } \>  ‘six o’clock’ \\
\textit{alas syiti/siti } \>  ‘seven o’clock’ \\
\textit{alas utso } \>  ‘eight o’clock’ \\
\textit{alas nwibi } \>  ‘nine o’clock’ \\
\textit{alas dyis } \>  ‘ten o’clock’ \\
\textit{alas unsi } \>  ‘eleven o’clock’ \\
\textit{alas dusi } \>  ‘twelve o’clock’
\end{tabbing}
\z

\is{clock time|)}With periods of time the Kagayanen numbers occur before the head noun that describes the period, for example, \textit{minuto}/\textit{minutos} ‘minute/minutes’, \textit{oras} ‘hour’, \textit{adlaw} ‘day’, \textit{kilem} ‘night’, \textit{duminggo} ‘week’, \textit{buļan} ‘month’, and \textit{taon} ‘year’. The linker is often dropped in such constructions. The time period expressions \textit{mapit madlaw} ‘right before first light’, \textit{sellem} ‘morning’, \textit{mapon} ‘afternoon’, and \textit{kadlaw} ‘all night’ cannot be used in this way with numbers as modifiers.

\ea
\begin{tabbing}
\hspace{3.5cm} \= \kill
\textit{isya (na) buļan     }\> ‘one month/moon’ \\
\textit{darwa (na) taon     }\> ‘two years’ \\
\textit{tallo (na) duminggo }\> ‘three weeks’ \\
\textit{*kwattro (na) sellem} \> (‘four mornings’) \\
\textit{*singko (na) mapon  } \> (‘five afternoons’) \\
etc.
\end{tabbing}
\z

When \textit{mga} ‘plural’ occurs before the number word or before a time phrase like \textit{mga alas dos} ‘two o’clock’, then it means ‘approximately’:

\ea
mga  darwa  (na)   duminggo \\\smallskip
 \gll mga  darwa  (na)   duminggo \\
\textsc{pl}  two  \textsc{lk}  week \\
\glt ‘approximately two weeks’
\z
\ea
mga  alas  dyis \\\smallskip
 \gll mga  alas  dyis \\
\textsc{pl}  o’clock  ten \\
\glt ‘approximately ten o’clock'
\is{numbers!cardinal|)}\is{cardinal numbers|)}
\z

\subsection{Ordinal numbers}
\label{bkm:Ref479339690} \label{sec:ordinalnumbers} \is{ordinal numbers|(} \is{numbers!ordinal|(}

There are two sets of \textit{ordinal numbers} in Kagayanen, both based on Austronesian cardinal number\is{numbers!Austronesian} roots above the first ordinal. One type is formed with the prefix \textit{pang}{}- and the other is formed with the prefix \textit{ka}{}-. The first ordinal for both types is borrowed from Spanish\is{numbers!Spanish-based} (exx. \ref{bkm:Ref479573487} and \ref{bkm:Ref479573710}).

\subsubsection{\textit{pang}{}- general ordinal}
\label{sec:generalordinalnumbers}

The \textit{pang}{}- ordinals are unspecified for the kind of order such as order in space, in time or in importance. The ordinal referring to ‘first’ in this system is the Spanish cardinal number \textit{una} (exx. \ref{ex:unmarkedordinal} and \ref{bkm:Ref479573487}). For the numbers \textit{darwa} ‘two’, \textit{tallo} ‘three’, and those beginning with a glottal stop (orthographic vowel-initial roots), the root-initial consonant is omitted following \textit{pang}{}- (see the examples in \ref{ex:unmarkedordinal}). The initial consonant is not omitted with \textit{limma} ‘five’, \textit{pitto} ‘seven’, \textit{waļļo} ‘eight’, and \textit{sampuļo} ‘ten’.

Since the instrumental nominalizer \textit{pang}- (\chapref{chap:referringexpressions}, \sectref{sec:pang}) may also occur with number roots, we include examples of these constructs in \REF{ex:unmarkedordinal} to illustrate the contrast between these two stem-forming prefixes. As mentioned in \chapref{chap:referringexpressions}, the instrumental \textit{pang}- prefix never causes root-initial consonant omission:

\ea
\label{ex:unmarkedordinal}
\begin{tabbing}
\hspace{4cm} \= \kill
\textbf{Unmarked ordinal} \> \textbf{Instrumental} \\
\textit{una} ‘first’ \> \textit{pang-isya} ‘used/good for one' \\
\textit{pangarwa}  ‘second’ \>  \textit{pangdarwa} ‘used/good for two' \\
\textit{pangallo}  ‘third’ \>   \textit{pangtallo} ‘used/good for three' \\ 
\textit{pangappat}  ‘fourth’ \>  \textit{pang-appat} ‘used/good for four' \\
\textit{panglimma}  ‘fifth’ \> \textit{panglimma} ‘used/good for five' \\ 
\textit{pangannem}  ‘sixth’ \> \textit{pang-annem} ‘used/good for six' \\
\textit{pangpitto}  ‘seventh’ \> \textit{pangpitto} ‘used/good for seven' \\
\textit{pangwaļļo}  ‘eighth’ \> \textit{pangwaļļo} ‘used/good for eight' \\
\textit{pangsyam}  ‘ninth’ \> \textit{pang-isyam} ‘used/good for nine' \\
\textit{pangsampuļo}  ‘tenth’ \> \textit{pangsampuļo} ‘used/good for ten'
\end{tabbing}
\z

The following are some examples of unmarked ordinal numbers from the corpus.

\ea
\label{bkm:Ref479573487}
Yon  \textbf{una}  an  na  pagpati  ta  mga  ittaw. \\\smallskip
 \gll Yon  \textbf{una}  an  na  pag-pati  ta  mga  ittaw. \\
\textsc{d}3\textsc{abs}  first  \textsc{def.m}  \textsc{lk}  \textsc{nr.act}-believe  \textsc{nabs}  \textsc{pl}  person \\
\glt ‘That was the first belief of the people.’ [JCOE-C-03 4.1]
\z
\ea
Ta  \textbf{pangarwa}  na  kilem  ta  ame  na  pagtinir  ta  yo na  baļay  Maria  i  gasakit  iya  na  uļo  kag  tudo  suka  din  an. \\\smallskip
 \gll Ta  \textbf{pang-darwa}  na  kilem  ta  ame  na  pag-tinir  ta  yo na  baļay  Maria  i  ga-sakit  iya  na  uļo  kag  tudo  suka  din  an. \\
   \textsc{nabs}  \textsc{ord}-two  \textsc{lk}  night  \textsc{nabs}  1\textsc{p.excl.gen}  \textsc{lk}  \textsc{nr.act}-stay  \textsc{nabs}
 \textsc{d}4\textsc{adj}
\textsc{lk}  house  Maria  \textsc{def.n}  \textsc{i.r}-pain  3\textsc{s.gen}  \textsc{lk}  head  and  intense  vomit  3\textsc{s.gen}  \textsc{def.n} \\
\glt `On the second night of our staying in that house, as for Maria, her head became painful and her vomiting was intense.’ [EMWN-T-09 7.1]
\z
\ea
Apang  ta  \textbf{pangallo}  ko  na  pag-eseb magtunga a  dagat  naan  tuman  ta  ake  na  ilek. \\\smallskip
 \gll Apang  ta  \textbf{pang-tallo}  ko  na  pag-eseb mag-tunga a  dagat  naan  tuman  ta  ake  na  ilek. \\
but  \textsc{nabs}  \textsc{ord}-three  1\textsc{s.gen}  \textsc{lk}  \textsc{nr.act}-immerse \textsc{i.ir}-surface.from.water 1\textsc{s.abs}  sea  \textsc{spat.def}  come.up.to  \textsc{nabs}  1\textsc{s.gen}  \textsc{lk}  armpit \\
\glt `But on my third dive, I surfaced in the sea that came up to my armpits.’ [EFWN-T-11 14.7]
\is{numbers!temporal ordinal|(}
\z

\subsubsection{\textit{ka}{}- temporal ordinal}
\label{sec:temporalordinalnumbers}

Ordinals derived with \textit{ka}{}- have a more specific function, referring only to order in time. For example, \textit{kadarwa na gira} ‘the second war.’ Also \textit{katallo na bukid} ‘the third mountain (that someone came upon)’. It can also mean the first, second, and so on in a series of actions \REF{bkm:Ref479574197}. The first ordinal for the \textit{ka}{}- derivation set is borrowed from Spanish ‘primero’ (exx. \ref{ex:temporalordinals} and \ref{bkm:Ref479573710}). For the second to tenth ordinals, the Kagayanen numbers are used, but after the tenth, the Spanish-based numbers are used as in example \REF{bkm:Ref481151157}. The second ordinal in this set sometimes reduces to \textit{karwa} (exx. \ref{ex:temporalordinals} and \ref{bkm:Ref479574075}).

\ea
\label{ex:temporalordinals}
\begin{tabbing}
\hspace{3.1cm} \= \kill
\textbf{Temporal ordinal} \> \textbf{Gloss} \\
primiro \> ‘first’ \\
kadarwa/karwa \> ‘second’ \\
katallo \> ‘third’ \\
kaappat \> ‘fourth’ \\
kalimma \> ‘fifth’ \\
kaannem \> ‘sixth’ \\
kapitto \> ‘seventh’ \\
kawaļļo \> ‘eighth’ \\
kaisyam \> ‘ninth’ \\
kasampuļo \> ‘tenth’
\end{tabbing}\z

The following are some examples of temporal ordinal numbers from the corpus:

\ea
\label{bkm:Ref479573710}
Tak  \textbf{primiro}  ko  nang  pa  sakay  ta  kalisa… \\\smallskip
 \gll Tak  \textbf{primiro}  ko  nang  pa  sakay  ta  kalisa… \\
because  first.time  1\textsc{s.gen}  only/just  \textsc{inc}  ride  \textsc{nabs}  horse.cart \\
\glt ‘Because (it was) still my \textbf{first} (time) to ride a horse cart…' [DBWN-L-23 5.10]
\z
\ea
\label{bkm:Ref479574075}
Piro  sigi  man  gyapon  dļagan  danen  asta  na  kalambot en  danen  ta  \textbf{karwa}  ya  na  bungyod. \\\smallskip
 \gll Piro  sigi  man  gyapon  dļagan  danen  asta  na  ka-lambot en  danen  ta  \textbf{ka-darwa}  ya  na  bungyod. \\
but  continual  also   just.the.same  run  3\textsc{p.abs}  until  \textsc{lk}  \textsc{i.exm}-reach \textsc{cm}  3\textsc{p.abs}  \textsc{nabs}  \textsc{ord}-two  \textsc{def.f}  \textsc{lk}  hill \\
\glt `But they kept running just the same until they reached the \textbf{second} (in time) hill.’ [JCON-T-08 42.1]
\z
\ea
\label{bkm:Ref479574197}
Ig  \textbf{una}  yaken  patagan  a  nyo  ta  tallo  na  ubra  ta kilem  nai  na  nakauna  ti  piling  i  na  minsai,  ig \textbf{dason} kunpirmasyon,  ig  \textbf{katallo}  distribution  ta  inyo  na  certificate. \\\smallskip
 \gll Ig  \textbf{una}  yaken  pa-atag-an  a  nyo  ta  tallo  na  ubra  ta kilem  nai  na  naka-una  ti  pa-iling  i  na  minsai,  ig \textbf{dason} kunpirmasyon,  ig  \textbf{ka-tallo}  distribution  ta  inyo  na  certificate. \\
and  first  1\textsc{s.abs}  \textsc{t.r}-give-\textsc{apl}  1\textsc{s.abs}  2\textsc{p.erg}  \textsc{nabs}  three  \textsc{lk}  work  \textsc{nabs}
night  \textsc{d}1\textsc{adj}  \textsc{lk}  \textsc{i.hap.r}-first  \textsc{d}1\textsc{nabs}  \textsc{t.r}-say  \textsc{def.n}  \textsc{lk} message  and  next confirmation  and  \textsc{ord}-three  distribution  \textsc{nabs}  2\textsc{p.gen}  \textsc{lk}  certificate \\
\glt`And \textbf{first}, as for me, you gave me three things to do this night which are the one that is first of these is called the message, and \textbf{next} confirmation and \textbf{third} (in time) distribution of your certificates.’ [SFOB-L-02 3.1]
\z
\ea
\label{bkm:Ref481151157}
Nakamang  ko  piro  lugay  betengay  nay  tak  sikad  baked na  bawlo.  Bali  \textbf{kaunsi}  ko  en  na  subbad.  Ta pagpadayon  ko  daw  ino  na  katingaļaan,  subbad aren  man  isab  bilang  \textbf{kadusi}  na  kaan  ki  yaken. \\\smallskip
 \gll Na-kamang  ko  piro  lugay  beteng-ay  nay  tak  sikad  baked na  bawlo.  Bali  \textbf{ka-unsi}  ko  en  na  subbad.  Ta pag-pa-dayon  ko  daw  ino  na  ka-tingaļa-an,  subbad aren  man  isab  bilang  \textbf{ka-dusi}  na  kaan  ki  yaken. \\
\textsc{a.hap.r}-get  1\textsc{s.erg}  but  long.time  pull-\textsc{rec}  1\textsc{p.excl.erg}  because  very  big \textsc{cm}  jack.fish  equal  \textsc{ord}-eleven  1\textsc{s.gen}  \textsc{cm}  \textsc{lk}  catch  \textsc{nabs}
\textsc{nr.act}-\textsc{caus}-continue  1\textsc{s.gen}  if/when  what  \textsc{lk}  \textsc{nr}-wonder-\textsc{nr}  catch 1\textsc{s.abs+cm}  too  again  as  \textsc{ord}-twelve  \textsc{lk}  eat  \textsc{obl}  1s \\
\glt `I got (the fish) but for a long time we pulled against each other because (it was) a very big jack fish. (It) equalled the \textbf{eleventh} of my catch. During my continuing on, what a wonder, I caught another again as the \textbf{twelfth} one that bit for me.’ [EFWN-T-11 9.8, 9]
\is{numbers!temporal ordinal|)}
\is{numbers!ordinal|)} \is{ordinal numbers|)}
\z

\subsection{Distributive numbers}
\label{bkm:Ref479572999}  \is{distributive numbers|(} \is{numbers!distributive|(} 

The distributive numbers describe the distribution of numbers of items in groups. They usually function as clause-level adverbial modifiers. There are two types of distributive numbers in Kagayanen---those involving reduplication alone, and those involving the prefix \textit{tag}-/\textit{tig}-. Each type has its own function, as described in the following paragraphs.

\subsubsection{Reduplication distributive numbers}
\label{sec:reduplicationdistributivenumbers} \is{numbers!reduplication distributive|(}

The original Kagayanen numbers\is{numbers!Austronesian} may be completely reduplicated, indicating groups of N participants in an action. For example, \textit{isya-isya} ‘one by one’ or ‘one at a time’ and \textit{darwa-darwa} ‘two by two’ or ‘two at a time’.

\ea
\textbf{Isya-isya}  ki  nang  manaw. \\\smallskip
 \gll \textbf{Isya\sim{}-isya}  ki  nang  m-panaw. \\
\textsc{red}\sim{}one  1\textsc{p.incl.abs}  just/only  \textsc{i.v.ir}-go/walk \\
\glt ‘We will leave \textbf{one} \textbf{by} \textbf{one}.’
\z


\ea
Insa  iya  ta  \textbf{isya-isya}  ki  kami, ``Kan-o  ki  isab  balik di?" \\\smallskip
 \gll Insa  iya  ta  \textbf{isya\sim{}-isya}  ki  kami, ``Kan-o  ki  isab  balik di?" \\
ask  3\textsc{s.erg}  \textsc{nabs}  \textsc{red}\sim{}one  \textsc{obl.p}  1\textsc{p.excl}  when  1\textsc{p.incl.abs}  again  return \textsc{d}1\textsc{loc} \\
\glt ‘She asked us \textbf{one} \textbf{by} \textbf{one},``When will we return here again?"’ [EMWN-T-09 10.3]
\z
\ea
\textbf{Darwa-darwa}  gid  na  ļunan  gamit  danen. \\\smallskip
 \gll \textbf{Darwa\sim{}-darwa}  gid  na  ļunan  gamit  danen. \\
\textsc{red}\sim{}two  \textsc{int}  \textsc{lk}  pillow  use  3\textsc{p.erg} \\
\glt ‘They really used two pillows each.’
\z

The form \textit{isya-isya} can also function to modify an RP, in which case it expresses the idea of ‘each and every one’.

\ea
… \textbf{isya-isya}  kiten  mag-atag  kayaran ta  masigkaittaw ta. \\\smallskip
 \gll … \textbf{isya\sim{}-isya}  kiten  mag-atag  ka-ayad-an ta  masigka-ittaw ta. \\
{} \textsc{red}\sim{}one  1\textsc{p.incl.abs}  \textsc{i.ir}-give  \textsc{nr}-good-\textsc{nr}
\textsc{nabs}  fellow-person 1\textsc{p.incl.gen} \\
\glt`… \textbf{each} \textbf{and} \textbf{every} one of us ought ’to give goodness to our fellow humans.’ [JCOB-L-02 10.4]
\is{numbers!reduplication distributive|)}
\z

\subsubsection{\textit{tag}{}-/\textit{tig}{}- distributive numbers}
\label{sec:tag-distributivenumbers} \is{numbers!\textit{tag}{}-/\textit{tig}{}- distributive|(}

The prefix \textit{tag}{}- or \textit{tig}{}- occurs with the original Kagayanen\is{numbers!Austronesian} cardinal numbers to indicate how many items go into each group. For example, \textit{tag}{}- plus \textit{darwa} becomes \textit{tagdarwa} meaning ‘two each’; \textit{tag}{}- plus \textit{pitto} ‘seven’ becomes \textit{tagpitto} ‘seven each’, and so on. There are two different ways to say ‘one each’---\textit{tag-isya} or \textit{tagsaļa}:

\ea
Mga  gamiten  ta  pagmikaw  pitto  pungpong  na  puso na  ummay  na  derse  na  paryo  nang  ta  bunga  ta  biid  na \textbf{tagpitto}  buok.  Piro  kawaļļo  na  pungpong  sampuļo  buok na  puso  na  derse  man. \\\smallskip
 \gll Mga  gamit-en  ta  pag-mikaw  pitto  pungpong  na  puso na  ummay  na  derse  na  paryo  nang  ta  bunga  ta  biid  na \textbf{tag-pitto}  buok.  Piro  ka-waļļo  na  pungpong  sampuļo  buok na  puso  na  derse  man. \\
\textsc{pl}  use-\textsc{t.ir}  \textsc{nabs}  \textsc{nr.act}-food.offering  seven  cluster  \textsc{lk}  rice.in.leaves \textsc{lk}  unmilled.rice  \textsc{lk}   small.\textsc{pl}  \textsc{lk}  like  just/only  \textsc{nabs}  fruit  \textsc{nabs}  hog.plum  \textsc{lk}
each-seven  piece  but  \textsc{ord}-eight  \textsc{lk}  cluster  ten  piece
unmilled.rice  rice.in.leaves  \textsc{lk}  small.\textsc{pl}  \textsc{emph} \\
\glt `The things used in food offering are seven clusters of small coconut leaf pouches with rice inside like the (size) of the hog plum fruit, which are \textbf{seven} \textbf{pieces} \textbf{each}. But the eighth cluster has ten small coconut leaf pouches with rice inside.’ [JCWE-T-16 4.1] 
\z
\ea
Daw  mapukan  en,  utud-uturon  en  lawa  din  an na \textbf{tagsaļa}  duppa. \\\smallskip
 \gll Daw  ma-pukan  en,  utod\sim{}-utod-en  en  lawa  din  an na \textbf{tagsaļa}  duppa. \\
if/when  \textsc{a.hap.ir}-cut.down  \textsc{cm}  \textsc{red}-cut-\textsc{t.ir}  \textsc{cm}  body  3\textsc{s.gen}  \textsc{def.m} \textsc{lk} one.each  arm’s.span \\
\glt `When it will be cut-down, cut its trunk up, one arm’s-span \textbf{each} \textbf{(piece)}.' [JCWE-L-32 6.3]
\z

The word \textit{tagsaļa} in this example can be replaced with \textit{tag-isya} with no difference in meaning.

The prefix \textit{tag-} may also occur with the Spanish-based\is{numbers!Spanish-based} numbers to refer to the price of each item, for example, \textit{tagdyis pisos} ‘ten pesos each’. The word \textit{pisos} may drop out, such that \textit{tagdyis} alone is understood to mean ‘ten pesos each’:

\ea
Duma  unti  daw  magbaligya  ta  karni  ta  tļunon  na  baboy \textbf{tagdyis}  pisos,  piro  iling  ko,  “Atag  nang  ta  \textbf{tagnwibi}  pisos.” \\\smallskip
 \gll Duma  unti  daw  mag-baligya  ta  karni  ta  tļunon  na  baboy \textbf{tag-dyis}  pisos,  piro  iling  ko,  “Atag  nang  ta  \textbf{tag-nwibi}  pisos.” \\
some  \textsc{d}1\textsc{loc.pr}  if/when  \textsc{i.ir}-sell  \textsc{nabs}  meat  \textsc{nabs}  wild  \textsc{lk}  pig each-ten  pesos  but  say  1s  give  just/only  \textsc{nabs}  each-nine  pesos \\
\glt `Some here when selling the meat of wild pig (say it is) \textbf{ten} \textbf{pesos} \textbf{each} (kilogram), but I said, “(I) will just give (it to you) for \textbf{nine} \textbf{pesos} \textbf{each}.”' [RCON-L-01 10.5]
\z

The prefix \textit{tag}{}- has another usage with some roots to indicate a season of the year or a season of some activity (see \chapref{chap:referringexpressions}, \sectref{sec:tag}).
\is{numbers!\textit{tag}{}-/\textit{tig}{}- distributive|)} \is{numbers!distributive|)} \is{distributive numbers|)}
\is{numbers|)}

\section{Non-numeral quantifiers}
\label{bkm:Ref52774536} \label{sec:non-numeralquantifiers} \is{quantifiers|(}

Non-numeral quantifiers in Kagayanen can only occur in the pre-head position (MOD\textsubscript{1}) in the Referring Phrase. They may occur as predicates of clauses, but they do not take affixes. Only \textit{tanan} ‘all’ and \textit{tama} ‘many’ can take complete reduplication; \textit{tanan-tanan}, meaning ‘absolutely all’ and \textit{tama-tama} ‘quite a few’. All the non-numeral quantifiers are listed in \REF{bkm:Ref424336114}:

\ea
\label{bkm:Ref424336114}
\begin{tabbing}
\hspace{3cm} \= \kill
\textit{tanan                        } \>   ‘all’ \\
\textit{tanan-tanan                  } \>  ‘absolutely all’ \\
\textit{kada                         } \>  ‘each’, ‘every’ \\
\textit{kada isya                    } \>  ‘each one’ \\
\textit{duma                         } \>   ‘some’, ‘other’, ‘others’, ‘another’ \\
\textit{bilog                         } \>   ‘whole’, ‘all inclusive’ \\
\textit{tama                          } \>  ‘many’, ‘much’ \\
\textit{tama-tama                      } \>    ‘quite a few’ \\
\textit{tise/sise (nang) }\footnotemark \>  ‘few’ \\
\textit{pila (nang)                 } \>  ‘few’
\end{tabbing}
\footnotetext{The forms \textit{tise} and \textit{sise} are idiolectal variants in this context. We suspect interference from the adjective \textit{sise} meaning ‘small (sg.) in size’. The variant forms \textit{tiset} and \textit{siset} also occur, parallel to the variants \textit{derse} and \textit{derset} for the adjective meaning ‘small (pl.)’. Both of these forms are also predicate adverbs indicating ‘almost do X’ (see \sectref{bkm:Ref420904083}).}
\z
The non-numeral quantifiers \textit{tama} ‘many’, \textit{tise} ‘few’ and \textit{pila} ‘few’ can occur with degree/intensity adverbs, and so qualify as adjectives \REF{bkm:Ref425171721}{}-\REF{bkm:Ref479341772}.

\ea
\label{bkm:Ref425171721}
\textbf{sikad}  \textbf{tama}  na  kwarta \\\smallskip
 \gll \textbf{sikad}  \textbf{tama}  na  kwarta \\
very  much   \textsc{lk}  money \\
\glt ‘very much money’
\z
\ea
\textbf{tanan}  \textbf{gid}  na  tanem \\\smallskip
 \gll \textbf{tanan}  \textbf{gid}  na  tanem \\
all  \textsc{int}  \textsc{lk}  plant \\
\glt ‘really all the plants’
\z
\ea
\textbf{tanan-tanan} \textbf{gid}  na  guso \\\smallskip
 \gll \textbf{tanan\sim{}-tanan} \textbf{gid}  na  guso \\
\textsc{red}\sim{}all    \textsc{int}  \textsc{lk}  agar.seaweed \\
\glt ‘completely really all the agar seaweed’
\z
\ea
\textbf{kada}  \textbf{gid}  na  adlaw \\\smallskip
 \gll \textbf{kada}  \textbf{gid}  na  adlaw \\
each  \textsc{int}  \textsc{lk}  day/sun \\
\glt ‘really each day’/’each and every day’
\z
\ea
\label{bkm:Ref479341772}
\textbf{kada}  isya  \textbf{gid}  na  bata \\\smallskip
 \gll \textbf{kada}  isya  \textbf{gid}  na  bata \\
each  one  \textsc{int}  \textsc{lk}  child \\
\glt ‘really each single child’/’each and every child’
\z

RPs containing \textit{tanan} and \textit{kada} often occur clause-initially \REF{bkm:Ref329086952}-\REF{bkm:Ref424539971}, \REF{bkm:Ref425584633}{}-\REF{bkm:Ref425234047}, but also occur in other positions \REF{bkm:Ref425584609}, \REF{bkm:Ref424540252}:

\ea
\label{bkm:Ref329086952}
\textbf{Tanan}  na  mga  ittaw   galineng. \\\smallskip
 \gll \textbf{Tanan}  na  mga  ittaw   ga-lineng. \\
all   \textsc{lk}  \textsc{pl}  people  \textsc{i.r}-quiet/peaceful \\
\glt ‘All the people became quiet.’ [JCWN-T-20 19.2]
\z
\ea
\label{bkm:Ref424539971}
\textbf{Tanan-tanan}  ki  kabatyag  ta  kakulian ta  Cagayancillo … \\\smallskip
 \gll \textbf{Tanan\sim{}-tanan}  ki  ka-batyag  ta  ka-kuli-an ta  Cagayancillo … \\
\textsc{red}\sim{}all    1\textsc{p.incl.abs}  \textsc{i.exm}-feel  \textsc{nabs}  \textsc{nr}-difficult/hard-\textsc{nr} \textsc{nabs}  Cagayancillo {} \\
\glt ‘\textbf{Absolutely} \textbf{all} of us feel extreme difficulties of Cagayancillo…’ [FDOE-T-01 3.3]
\z
\ea
\label{bkm:Ref425584609}
Barangay  kapitan  gaumaw  ta  \textbf{tanan}  na  mga  ittaw. \\\smallskip
 \gll Barangay  kapitan  ga-umaw  ta  \textbf{tanan}  na  mga  ittaw. \\
community  captain  \textsc{i.r}-call  \textsc{nabs}  all  \textsc{lk}  \textsc{pl}  people \\
\glt ‘The community captain called \textbf{all} \textbf{the} \textbf{people}.’ [JCWN-T-21 9.4]
\z

\textit{Kada} ‘each’ occurs before the head noun without a linker as in \textit{kada mama} ‘each man’ \REF{bkm:Ref425584633}, \textit{kada baļay} ‘each house’ \REF{bkm:Ref424540252} and \textit{kada isya} ‘each one’ \REF{bkm:Ref425234047}.

\ea
\label{bkm:Ref425584633}
\textbf{Kada}  mama  ta  bilog  na  banwa  papaubra na  uļa  gid  swildo. \\\smallskip
 \gll \textbf{Kada}  mama  ta  bilog  na  banwa  pa-pa-ubra na  uļa  gid  swildo. \\
each  man  \textsc{nabs}  whole  \textsc{lk}  town/country  \textsc{t.r}-\textsc{caus}-work
\textsc{lk}  \textsc{neg.r}  \textsc{int}  wage \\
\glt `\textbf{Every} \textbf{man} in the whole town was made to work without any wage.’ [JCWN-T-20 11.1]
\z
\ea
\label{bkm:Ref425234047}
… daw  \textbf{kada}  isya  ki  danen  may  bitbit  na  pusil  daw  sundang na  sikad  ļangkaw. \\\smallskip
 \gll … daw  \textbf{kada}  isya  ki  danen  may  bitbit  na  pusil  daw  sundang na  sikad  ļangkaw. \\
{} and  each  one  \textsc{obl.p}  3p  \textsc{ext.in}  hold  \textsc{lk}  gun  and  machete
\textsc{lk}  very  long \\
\glt `…and \textbf{each} \textbf{one} of them  held a gun and machete that was very long.’ [BCWN-C-04 9.13]
\z
\ea
\label{bkm:Ref424540252}
Tapos  gatagtag  kay  ta  kindi  ta  \textbf{kada}  \textbf{baļay.} \\\smallskip
 \gll Tapos  ga-tagtag  kay  ta  kindi  ta  \textbf{kada}  \textbf{baļay.} \\
then  \textsc{i.r}-distribute  1\textsc{p.excl.abs}  \textsc{nabs}  candy  \textsc{nabs}  each  house \\
\glt ‘Then we distributed candy to \textbf{each} house.’ [VAOE-J-02 3.7]
\z

The root \textit{duma} is particularly multi-functional. As a noun it means ‘companion’. As a verb, it means ‘to be with (someone)’. As a quantifier in a non-contras\-tive context, it means ‘some’ (i.e., an indefinite number of). In a contrastive context it tends to mean ‘other’ or ‘others’ \REF{bkm:Ref329087006}.

\ea
\label{bkm:Ref329087006}
\textbf{duma}  na  mga  bataan \\\smallskip
 \gll \textbf{duma}  na  mga  bata-an \\
some  \textsc{lk}  \textsc{pl}  child-\textsc{nr} \\
\glt ‘\textbf{some} children’ or ‘other children’
\z

\textit{duma} may also occur with singular nouns to express ideas similar to ‘another’ or some non-specific item:

\ea
\textbf{duma} na bata \\\smallskip
 \gll \textbf{duma} na bata \\
some \textsc{nabs} child \\
\glt ‘another child’, ‘some child’
\z

In contexts when it is known who or what is being talked about, the non-numeral quantifiers, except for \textit{kada} ‘each’, can occur as the head noun in a Referring Phrase \REF{bkm:Ref329087044}.

\ea
\label{bkm:Ref329087044}
Kamang  din  \textbf{tanan}. \\\smallskip
 \gll Kamang  din  \textbf{tanan}. \\
get  3\textsc{s.erg}  all \\
\glt  ‘S/he took \textbf{all} (of them/it).’
\is{quantifiers|)}
\z

\section{Modifier Phrases}
\label{bkm:Ref418142991} \label{sec:modifierphrases} \is{Modifier Phrases|(}

We refer to syntactic constructions headed by Modifiers as \textit{Modifier Phrases} (MPs). There are three situations in which MPs may function in Kagayanen clauses:

\newpage
\begin{enumerate}
\item \textit{Attributively}\is{attributive use of Modifier Phrases}\is{Modifier Phrases!attributive use} in Referring Phrases" (\textit{the \textbf{very thick} rain})
\item \textit{Predicatively}\is{predicative!use of Modifier Phrases}\is{Modifier Phrases!predicative use} as non-verbal Predicates (\textit{The rain is \textbf{very thick}}.)
\item \textit{Adverbially}\is{adverbial use of Modifier Phrases} as clause-level Modifiers (\textit{It is raining} \textbf{very thickly}.)
\end{enumerate}

In this section, we will describe the internal structure of MPs serving each of these three major functions. The distribution of MPs in their larger contexts will be described in other sections: attributive functions of MPs are discussed in \chapref{chap:referringexpressions}; predicative functions are discussed in \chapref{chap:non-verbalclauses}; and adverbial functions are discussed in \sectref{bkm:Ref420904083} below.

Modifier Phrases (MP) consist minimally of a Head, which may be a lexical adjective, a derived adjective, a noun, a precategorial root, a numeral, one of a small set of quantifiers, a genitive case RP, or a relative clause. We argue that when MPs function predicatively or adverbially they are nominalized (see \sectref{bkm:Ref418108091}).

The basic template for Modifier Phrases is given in \REF{bkm:Ref343768497}:

\ea
\label{bkm:Ref343768497}
\label{bkm:Ref117605153}
MP = (ADV1\textsubscript{degree}) (ADV2\textsubscript{intensity/aspect}) Head
\z

As shown in \REF{bkm:Ref343768497}, there are generally two groups of adverbial elements (ADV) that may occur within a Modifier Phrase. Group one, \textit{degree adverbs}\is{degree adverbs}\is{adverbs!degree}, always precede the head. Group two, \textit{intensity}\is{intensity adverbs}\is{adverbs!intensity} and \textit{aspect adverbs}\is{aspect adverbs}\is{adverbs!aspectual}, are second-position enclitics. They occur following the first major constituent in the phrase. Thus, if no degree adverb occurs, any intensity/aspect adverbs follow the Head (exx. \ref{bkm:Ref479575080}, and \ref{bkm:Ref479575151}):

\ea
sikad  gid  tama    na  ittaw \\\smallskip

  ADV1  ADV2  HEAD \\
\gll {}[ sikad  gid  tama  ]  na  ittaw \\
{}  very  \textsc{int}  many {} \textsc{lk}  people \\
\glt ‘very many people’
\z
\ea
\label{bkm:Ref479575080}
dani  nang  man  na  baryo \\\smallskip

  Head  ADV2 \\
\gll {}[ dani  nang  ]  man  na  baryo \\
{}  close  only {}     too  \textsc{lk}  barrio \\
\glt ‘a very close barrio too’  [DDWN-C-01 3.6]
\z

\newpage
\ea
\label{bkm:Ref479575151}
dayad  gid    na  ittaw \\\smallskip

  Head  ADV2 \\
\gll {}[ dayad  gid   ]  na  ittaw \\
 {} good  \textsc{int} {}   \textsc{lk}  person \\
\glt ‘a really good person’ [JCOE-T-06 1.1]
\z

The intensity adverb \textit{gid} and the aspectual adverbs may occur after the Head even if a degree adverb appears before the head. This suggests a “tighter” link between the ADV1 and the HEAD, as indicated by the bracketing in examples \REF{bkm:Ref439938209} and \REF{bkm:Ref425605536}. Such special constructions indicate higher intensity, as indicated by upper case in the free translations:

\ea
  \label{bkm:Ref439938209}
sikad  tama   gid   na  ittaw \\\smallskip

  ADV1  Head  ADV2 \\
\gll {}[ [ sikad  tama ]  gid  ] na  ittaw \\
     {} {} very  many {} \textsc{int} {} \textsc{lk}  people \\
\glt ‘really \textsc{very many} people’
\z
\ea
\label{bkm:Ref425605536}
sikad  lawig      pa  na  iksplikar \\\smallskip

ADV1  Head      ADV2 \\
\gll {}[ [ sikad  lawig    ]   pa ]  na  iksplikar \\
  {} {}   very  long.time {} \textsc{inc} {} \textsc{lk}  explanation \\
\glt ‘very \textsc{long} explanation’ [SBWL-C-03 6.4]
\z

The central degree adverbs are \textit{sikad} ‘very’, \textit{segeng} ‘extremely’ and \textit{tise (nang)} ‘slightly/a little bit/almost’. A fourth degree adverb, \textit{midyo} ‘kind of/sort of’ may also be included in this set, though it has slightly different syntactic properties, as discussed below.

Intensity adverbs include the intensifier \textit{gid} ‘truly’, and the downtoner \textit{nang} ‘just/only’ (see examples \ref{bkm:Ref425250007}{}-\ref{bkm:Ref424739679} below). What we are calling “aspectual” adverbs are \textit{pa} ‘incompletely’ or ‘still’ (exx. \ref{bkm:Ref424739679}, \ref{bkm:Ref424800529}) and \textit{en} ‘completely’ (\ref{bkm:Ref425694314}). \isi{Tagalog} also employs a particle \textit{pa} with senses comparable to the same particle in Kagayanen. The \textit{en} adverb is comparable to Tagalog \textit{na}, and is sometimes represented in the free translations as “already” or “now”.

\hspace*{-1.3pt}In the following sections we will describe MPs functioning attributively (\sectref{bkm:Ref424719800}), predicatively (\sectref{bkm:Ref418108091}), and adverbially (\sectref{bkm:Ref439924168}).  Following this, we will describe two subtypes of MPs---MPs headed by numbers (NumP) and non-numeral quantifiers (\sectref{bkm:Ref439924202}).
\is{distributive numbers|)} \is{numbers!distributive|)}

\subsection{MPs functioning attributively}
\label{bkm:Ref424719800}\label{sec:attributivemps} \is{ordinal numbers|(}

As mentioned above, MPs consist minimally of a single modifying element. Several examples of minimal MPs in MOD\textsubscript{1} and MOD\textsubscript{2} positions are presented in \sectref{bkm:Ref422117197} above. Examples \REF{bkm:Ref417997093}{}-\REF{bkm:Ref425250007} illustrate slightly more complex MPs functioning attributively in the MOD\textsubscript{1} position in an RP. In these examples the MP is indicated in brackets:

\ea
\label{bkm:Ref417997093}  
sikad  dakmeļ  na  mga  lamunon \\\smallskip

ADV1\hspace{6pt}Head \\
\gll {}[  sikad  dakmeļ ]  na  mga  lamon-én \\
{}  very  thick  {}  \textsc{lk}  \textsc{pl}  to.weed-\textsc{nr} \\
\glt ‘very thick weeds’ [PBWN-C-13 10.2]
\z
\ea
\label{bkm:Ref418000563}
segeng  bao  na  tangkaļ  ta  baboy \\\smallskip

ADV1\hspace{24pt}Head \\
\gll {}[ segeng  bao ]  na  tangkaļ  ta  baboy \\
{}  extremely  odor {} \textsc{lk}  pen  \textsc{nabs}  pig \\
\glt ‘extremely smelly pen of  the pig’
\z
\ea
\label{bkm:Ref418000663}
midyo  inog  na  nangka \\\smallskip

ADV1\hspace{14pt}Head \\
\gll {}[  midyo  inog ]  na  nangka \\
  {} kind.of  ripe {} \textsc{lk}  jackfruit \\
\glt ‘kind of ripe jackfruit’
\z
\ea
\label{bkm:Ref425693139}\label{bkm:Ref425250007}
tise  nang  bao   na  tangkaļ  ta  baboy \\\smallskip

ADV1\hspace{16pt}ADV2\hspace{16pt}Head \\
\gll {}[ tise  nang  bao  ]  na  tangkaļ  ta  baboy \\
 {} slightly  just/only  odor {}   \textsc{lk}  pen  \textsc{nabs}  pig \\
\glt ‘just slightly smelly pen of the pig’
\z

Examples \REF{bkm:Ref425694314}{}-\REF{bkm:Ref424739679} illustrate MPs with intensity/aspect adverbs in the ADV2 position.

\ea
\label{bkm:Ref425694314}
subļa  en   na  dikstros. \\\smallskip

Head\hspace{24pt}ADV2 \\
\gll [ subļa  en  ]  na  dikstros. \\
{} too.much  \textsc{cm} {} \textsc{lk}  dextrose \\
\glt ‘too much dextrose now’ [JCWN-T-21 17.3]
\z
\ea
bugtong  nang   na  bata \\\smallskip

  Head\hspace{20pt}ADV2 \\
\gll {}[ bugtong  nang  ]  na  bata \\
 {} sole  only  {}  \textsc{lk}  child \\
\glt ‘one and only child’ [CBWN-C-17 2.2]
\z
\ea
\label{bkm:Ref418078036}
sikad  gid  tama  na  mga  pilak  na  dayad \\\smallskip

ADV1  ADV2  Head \\
\gll [ sikad  gid  tama ]  na  mga  pilak  na  dayad \\
 {} very  \textsc{int}  many {} \textsc{lk}  \textsc{pl}  silver  \textsc{lk}  good \\
\glt ‘really very much silver that is good’  [AGWN-L-01 5.8]
\z
\ea
sikad  tama gid    na  mga  pilak  na  dayad \\\smallskip

  \hspace{12pt}ADV1  Head\hspace{12pt}ADV2 \\
\gll [ [ sikad  tama ]\footnotemark {} gid   ]  na  mga  pilak  na  dayad \\
  {} {}  very  many {} \textsc{int} {} \textsc{lk}  \textsc{pl}  silver  \textsc{lk}  good \\
\footnotetext{We have used multiple brackets in this example to illustrate the tight syntactic relationship between \textit{sikad} and the following Head. This is mentioned above, and in the following discussion.}
\glt ‘really very MUCH silver that is good’
\z
\ea
\label{bkm:Ref425326439}\label{bkm:Ref424739679}
sikad  pa  lawig  na  iksplikar \\\smallskip

ADV1  ADV2  Head \\
\gll {}[ sikad  pa  { } lawig ]  na  iksplikar \\
{}  very  \textsc{inc}  { } long.time {}  \textsc{lk}  explanation \\
\glt ‘a \textsc{very} long explanation’
\z

The following examples illustrate an aspectual plus an intensity adverb following the HEAD \REF{bkm:Ref424800529} and preceding the HEAD \REF{bkm:Ref481481587}. The combinations \textit{pa gid} ‘truly very’ and \textit{pa man} ‘very (emphatic)’ may be lexicalized compounds. They are quite common, while other combinations of aspectual and intensity adverbs do not occur:

\ea
\label{bkm:Ref424800529}
matama  \textbf{pa}  \textbf{gid}  na  ambaļanen \\\smallskip

\hspace{6pt}Head\hspace{18pt}ADV2  ADV2 \\
\gll {}[ ma-tama  \textbf{pa}  \textbf{gid} ]  na  ambaļ-anen \\
 {} \textsc{adj}-many  \textsc{inc}  \textsc{int} {} \textsc{lk}  say-\textsc{nr} \\
\glt ‘really \textsc{many} more sayings’ [SFOB-L-02 4.5]
\z
\ea
\label{bkm:Ref481481587}
sikad  \textbf{pa}  \textbf{man}  gwapa  na  bata \\\smallskip

ADV1 ADV2 ADV2 Head \\
\gll [ sikad  \textbf{pa}  \textbf{man}  gwapa ]  na  bata \\
{}  very  \textsc{inc}  \textsc{emph}  attractive {} \textsc{lk}  child \\
\glt ‘a \textsc{very} attractive child’ [CBOE-C-01 1.9]
\z

As mentioned above, and illustrated in \REF{bkm:Ref418078036}--\REF{bkm:Ref425326439}, in general when a degree adverb occurs in an MP, the intensity and aspectual adverbs occur after the degree adverb or after the Head of the MP. However, the downtoner \textit{nang} is an exception. It only occurs after the degree adverb \textit{tise}. If there is no degree adverb, \textit{nang} may appear after the Head, but it does not occur with any degree adverb other than \textit{tise}. For this reason we sometimes consider \textit{tise nang}, ‘just a little bit’, to be a fixed compound adverbial expression (see \ref{bkm:Ref425693139}).

As is the case at all levels of Kagayanen syntax, various “intrusive” elements may occur after the first syntactic element in the MP. These may be clause-level enclitic pronouns or clausal adverbial particles. Examples \REF{bkm:Ref401111978} through \REF{bkm:Ref422823012} illustrate MPs with an intrusive clausal attitude marker, pronoun and/or clause-level enclitic (bolded).

\ea
\label{bkm:Ref424739555}\label{bkm:Ref401111978}
sikad  \textbf{ya}  tama  na  sagbetan \\\smallskip
 \gll [ sikad  \textbf{ya}  tama ]  na  sagbet-an \\
  {} very  \textsc{att}  many {} \textsc{lk}  weeds/trash-\textsc{nr} \\
\glt ‘the very many weeds (far away, out of sight)’
\z
\ea
\label{bkm:Ref424739557}
mataļem  \textbf{kaw}  \textbf{man}  na  maistro  daw  maistra \\\smallskip
 \gll [ ma-taļem  \textbf{kaw}  \textbf{man} ]  na  maistro  daw  maistra \\
{} \textsc{adj}-sharp  2\textsc{p.abs}  \textsc{emph} {} \textsc{lk}  teacher.\textsc{m}  and  teacher.\textsc{f} \\
\glt ‘\textbf{also} \textbf{you} (will be) skillful teachers (male) and teachers (female)’ [SFOB-L-01 8.7]…
\z
\ea
sikad  \textbf{kon}  gwapa  na  dļaga \\\smallskip
 \gll {}[ sikad  \textbf{kon}  gwapa ]  na  dļaga \\
 {} very  \textsc{hsy}  attractive {} \textsc{lk}  single.woman \\
\glt ‘a very, \textbf{they} \textbf{say}, attractive young woman’ [PBWN-C-12 15.3]
\z
\ea
… sikad  \textbf{ka}  \textbf{kon}  miad  na  dļaga. \\\smallskip
 \gll [ … sikad  \textbf{ka}  \textbf{kon}  miad ]  na  dļaga. \\
  {} {}  very  2\textsc{s.abs  hsy}  kind {} \textsc{lk}  single.woman \\
\glt ‘… \textbf{they} \textbf{say} you  are a very kind single woman.’ [PBWN-C-12 15.3]
\z
\ea
\label{bkm:Ref422823012}
sikad  \textbf{man}  \textbf{kon}  tama   na  kwarta … \\\smallskip
 \gll [  sikad  \textbf{man}  \textbf{kon}  tama  ]  na  kwarta … \\
{}  very  \textsc{emph}  \textsc{hsy}  many  {}  \textsc{lk}  money \\
\glt  ‘\textbf{it} \textbf{is} \textbf{said} (there was) very much money …' [PBWN-C-13 14.2]
\z

As mentioned in \chapref{chap:referringexpressions}, \sectref{sec:basicreferringphrases} there is an element we’ve termed “{Є}" that consists of a genitive enclitic and/or a deictic determiner. The genitive enclitic occurs in the second position in the Referring Phrase. Of course, if the MP is in MOD\textsubscript{1}, and consists of a single word, the genitive enclitic follows that word. However, if the MP is complex, the genitive enclitic intrudes after the first element of the MP. Examples \REF{bkm:Ref418078298} and \REF{bkm:Ref418078380} illustrate MPs with such intrusive genitive enclitic pronouns.

\ea
\label{bkm:Ref418078298}
tallo  \textbf{ko}  nang  na  mga  mangngod  na  mga  dirset \\\smallskip
 \gll [ tallo  \textbf{ko}  nang ]  na  mga  mangngod  na  mga  dirset \\
{} three  1\textsc{s.gen}  only {} \textsc{lk}  \textsc{pl}  younger.sib  \textsc{lk}  \textsc{pl}  small.\textsc{pl} \\
\glt ‘just my three small younger siblings’ [ETOP-C-08 4.1]
\z
\ea
\label{bkm:Ref418078380}
ta tanan-tanan  \textbf{no}  gid  na  pamilya \\\smallskip
 \gll ta  [  tanan\sim{}-tanan  \textbf{no}  gid ]  na  pamilya \\
\textsc{nabs} {} \textsc{red}\sim{}all  2\textsc{s.gen}  \textsc{int} {} \textsc{lk}  family \\
\glt ‘truly all your family’ [ETOP-C-08]
\z

As mentioned above, the degree adverbs seem to be more tightly bound syntactically to the following Head than other modifiers, since genitive enclitics may not intrude between them and the Head of the MP. Instead the only possible order in this case is for the genitive enclitic to follow the Head of the MP, thus indicating that \textit{sikad bakod} in example \REF{bkm:Ref52474400} is treated as a unitary substantive element for purposes of {Є} placement:

\ea
\label{bkm:Ref52474400}
\textit{sikad bakod no na baļay}   ‘your very big house’ \\
\textit{*sikad no bakod na baļay}
\z

Interestingly, this restriction only applies to genitive enclitics. Clausal pronouns, attitude markers and adverbs may intrude between \textit{sikad} and the Head of the MP (exx. \ref{bkm:Ref424739555}, \ref{bkm:Ref424739557}, and \ref{bkm:Ref422823012}). Also, intensity/aspectual clausal adverbs may occur after \textit{sikad} (exx. \ref{bkm:Ref418078036} and \ref{bkm:Ref424739679}).

\subsection{MPs functioning as predicates}
\label{bkm:Ref418108091}

Every language has ways of predicating concepts referring to properties. For example, in English one can say “she is very tall” or “he is taller than his father”. The predicates in these sentences are sometimes called “adjectival predicates” or “predicate adjective phrases”. Such adjectival predicates have much in common with Modifier Phrases within Referring Phrases, but they are not exactly the same. For example, in English one can use “very tall” attributively, “A very tall woman”, or predicatively “the woman is very tall”. However “taller than his father” can only be used predicatively. The attributive use is not allowed  {}- “*A taller than his father boy”. Even some individual modifiers may be used attributively but not predicatively, and vice versa:

\ea
\begin{tabbing}
\hspace{5.5cm} \= \kill   
\textbf{Attributive use}    \>    \textbf{Predicative use} \\
The dolphin is a \textit{marine} mammal. \> *Some mammals are \textit{marine}. \\
They caught the \textit{serial} killer.   \>   *That killer is \textit{serial}. \\ \\
*An \textit{ablaze} house   \>   The house was \textit{ablaze}. \\
*An \textit{ashamed} teacher  \>    I am \textit{ashamed}.
\end{tabbing}
\z

For this reason, the notion of “Adjective Phrase” in English needs to be qualified as to whether it is the type of phrase that may be used attributively, predicatively or both.

Something similar is true in Kagayanen as well. We propose that MPs functioning as Predicates are actually nominalizations, and that therefore there is no essential difference between nominal predicates and adjectival predicates. Furthermore, there is a reasonable functional explanation for this fact.

To illustrate, compare examples \REF{bkm:Ref417997171} and \REF{bkm:Ref113343326} with \REF{bkm:Ref417997093} and \REF{bkm:Ref418000563} (repeated here as \ref{bkm:Ref479412653} and \ref{bkm:Ref418087692}):

\ea
\label{bkm:Ref417997171}\label{bkm:Ref418088532}
Predicative use of modifier phrases: \\
Sikad  na  dakmeļ  mga  lamunon. \\\smallskip
 \gll {}[ Sikad  na  dakmeļ ]  mga  lamon-én. \\
 {} very  \textsc{lk}  thick {} \textsc{pl}  to.weed-\textsc-\textsc{nr} \\
\glt ‘The weeds are very thick.’ Or ‘Very thick ones are the weeds.’
\z
\ea
\label{bkm:Ref113343326}
Segeng  na  bao    tangkaļ  ta  baboy. \\\smallskip
 \gll [ Segeng  na  bao   ]  tangkaļ  ta  baboy. \\
 {} extremely  \textsc{lk}  odor {} pen  \textsc{nabs}  pig \\
\glt ‘The pig pen is extremely smelly.’
\z
\ea
\label{bkm:Ref418087690}\label{bkm:Ref479412653}
Attributive use of modifier phrases: \\
sikad  dakmeļ na  mga  lamunon \\\smallskip
 \gll {}[ sikad  dakmeļ ]  na  mga  lamunon \\
 {} very  thick {} \textsc{lk}  \textsc{pl}  weeds \\
\glt ‘very thick weeds’ [PBWN-C-13 10.2]
\z
\ea
\label{bkm:Ref418087692}
segeng  bao   na  tangkaļ  ta  baboy \\\smallskip
 \gll [ segeng  bao ]  na  tangkaļ  ta  baboy \\
{} extremely  odor {} \textsc{lk}  pen  \textsc{nabs}  pig \\
\glt ‘extremely smelly pig pen’
\z

Note that the only overt indication of the distinction between the predicative versus attributive functions is the position of the linker, \textit{na}.\footnote{This generalization is true in careful speech. In relaxed speech, all linkers are subject to omission, thus potentially obscuring the grammatical distinction between Clause and RP. However, in context speakers are easily able to replace the dropped linkers, thus disambiguating the structures in exactly the manner described in this section.} In the predicative use (exx. \ref{bkm:Ref417997171} and \ref{bkm:Ref113343326}), the linker occurs inside the Modifier Phrase, whereas in the attributive use (exx. \ref{bkm:Ref479412653} and \ref{bkm:Ref418087692}), the linker occurs between the Modifier Phrase and the Head of the nominal predicate RP. This is entirely parallel to simple predicate nominal constructions in which a modified RP functions as a predicate:

\ea
Layen  gid  na  bata  Pedro  ya. \\\smallskip

Predicate \hspace{70pt} Absolutive \\\smallskip
 \gll Layen  gid  na  bata  Pedro  ya. \\
mischievious  \textsc{int}  \textsc{lk}  child  Pedro  \textsc{def.f} \\
\glt ‘Pedro is really a mischievous child.’
\z
\ea
Beet  na  ittaw    kanen  an. \\\smallskip

Predicate \hspace{40pt}   Absolutive \\\smallskip
 \gll Beet  na  ittaw    kanen  an. \\
behaved  \textsc{lk}  person  3\textsc{s.abs}  \textsc{def.m} \\
\glt ‘S/he is a well-behaved person.’
\z

\largerpage
Thus if we consider modifier phrases functioning as predicates as being nominalizations (e.g. “very thick ones” in \ref{bkm:Ref418088532}, and “extremely smelly one” in \ref{bkm:Ref113343326}, there is no need to posit an “optional” linker that only shows up when the phrase functions as a predicate. These modifying predicates are just RPs with a property concept word as the Head and the degree adverbs in the MOD\textsubscript{1} position. The RP structure explains the presence of the linker.

A reasonable functional explanation for this fact is simply that, if the predicating modifiers were not nominalized, there would be no structural distinction between attribution and predication, for example \textit{sikad dakmeļ mga lamunon} would mean both “the very thick weeds" and “the weeds are very thick”.

Just as illustrated in English above, in Kagayanen certain modifying expressions may occur attributively, but not predicatively. This is the difference between \textit{midyo} ‘kind of/sort of’ and the other degree adverbs, \textit{sikad} ‘very’ \textit{segeng} ‘extremely’ and \textit{tise (nang)} ‘a little bit’. As illustrated in example \REF{bkm:Ref418000663}, repeated here as \REF{bkm:Ref418089153}, \textit{midyo} may function to modify a property concept word such as \textit{inog} ‘ripe’.

\ea
\label{bkm:Ref418089153}
midyo  inog  na  nangka \\\smallskip
 \gll [  midyo  inog ]  na  nangka \\
 {} kind.of  ripe {} \textsc{lk}  jackfruit \\
\glt  ‘kind of ripe jackfruit’
\z

However, \textit{midyo} by itself may not appear in a predicate modifier:

\ea
\textit{[ * Midyo na inog ] nangka an.} \\
(‘The jackfruit is kind of ripe.’)
\z

The other degree adverbs may freely function in either context (see exx. \ref{bkm:Ref418088532} through \ref{bkm:Ref418087692}). For this reason, we say that \textit{midyo} is a non-central member of the class of degree adverbs. Of course this may be because it is a recently borrowed word, coming originally from Spanish \textit{medio} ‘half/partly’.

Similar to attributive MPs when MPs occur as predicates, an enclitic pronoun referring to the absolutive argument of the clause may intrude in second position. In example \REF{bkm:Ref418091763}, the second clause, “we were very tired” consists of a predicate modifier \textit{sikad}\textbf{ }\textit{gid bellay} ‘very tired’, with the linker \textit{na} dropped out before the word \textit{bellay}. The absolutive argument of this predicate is \textit{kay} ‘we’ which appears in second position in the clause:

\ea
\label{bkm:Ref418091763}
Mam,  ta  ame  na  pag-uli  di {}[ sikad  \textbf{kay}  gid  bellay … \\\smallskip
 \gll Mam,  ta  ame  na  pag-uli  di {}[ sikad  \textbf{kay}  gid  bellay ] … \\
ma’am  \textsc{nabs}  1\textsc{p.excl.gen}  \textsc{lk}  \textsc{nr.act}-go.home  \textsc{d}1\textsc{loc} {} very  1\textsc{p.excl.abs}  \textsc{int}  tired {} \\
\glt `Ma’am, when we went home here, we were really very tired…' [AFWL-L-01]
\z

The second clause of example \REF{bkm:Ref418091763}, with pronoun intrusion, is the most normal way of expressing this type of clause. The “non-intrusive” version is used in unusual situations, such as excited speech \REF{bkm:Ref422836404} (see \chapref{chap:non-verbalclauses}):

\ea
\label{bkm:Ref422836404}
Sikad  gid  (na)  bellay  kami  i. \\\smallskip
 \gll Sikad  gid  (na)  bellay  kami  i. \\
very  \textsc{int}  \textsc{lk} tired  1\textsc{p.excl.abs} \textsc{def.n} \\
\glt ‘We were very tired (excitedly).’
\z

As mentioned above, the linker \textit{na} often drops out in relaxed speech. However, there is no apparent communicative function for this omission of \textit{na}.

Similarly, in example \REF{bkm:Ref418091931}, the predicate is \textit{sikad (na) sadya} ‘very happy.’ In this case both the absolutive enclitic pronoun and a clause-level adverbial particle appear after the degree adverbial:

\ea
\label{bkm:Ref418091931}
Sikad  \textbf{kaw}  \textbf{taan}  sadya. \\\smallskip
 \gll [  Sikad  \textbf{kaw}  \textbf{taan}  sadya. ] \\
 {} very  2\textsc{p.abs} perhaps  happy {} \\
\glt ‘Y’all are perhaps very happy.’ [SBWL-C-01 4.2]
\z

Finally, \REF{bkm:Ref418092305} illustrates two coordinate predicate modifier clauses. In the second clause, the absolutive enclitic \textit{kay} ‘we exclusive’ intrudes in the MP functioning as the predicate.

\ea
\label{bkm:Ref418092305}
Sikad  tama  ame  na  kaoy  daw  sikad  \textbf{kay}  na  teggeb. \\\smallskip
 \gll Sikad  tama  ame  na  kaoy  daw  [ sikad  \textbf{kay}  na  teggeb.{ }] \\
very  many  1\textsc{p.excl.gen}  \textsc{lk}  wood  and {} very  1\textsc{p.excl.abs}  \textsc{lk}  overloaded \\
\glt ‘Our wood was very much and we were very overloaded…’ (This is a story about riding in an outrigger canoe.) [CBWN-C-11 2.6]
\z

\subsection{MPs functioning as clause-level modifiers}
\label{bkm:Ref418107822} \label{bkm:Ref439924168}

MPs functioning as clause-level modifiers mostly occur with a non-absolutive marker \textit{ta} preceding them, and there is usually no linker between any degree adverb and the Head. Examples \REF{bkm:Ref113345573} and \REF{bkm:Ref113345575} illustrate basic MPs in this function, with both degree and intensity/aspectual adverbs:

\ea
\label{bkm:Ref113345573}
Guran  \textbf{ta}  \textbf{sikad} \textbf{gid}  \textbf{dakmeļ}. \\\smallskip
 \gll Ga-uran  \textbf{ta}  \textbf{sikad} \textbf{gid}  \textbf{dakmeļ}. \\
\textsc{i.r}-rain  \textsc{nabs} very  \textsc{int}  thick \\
\glt ‘It is/was really raining \textbf{very} \textbf{thickly}.’
\z
\ea
\label{bkm:Ref113345575}
Bukasan  din  \textbf{ta}  \textbf{tise}  \textbf{nang.} \\\smallskip
 \gll \emptyset{}-Bukas-an  din  \textbf{ta}  \textbf{tise}  \textbf{nang}. \\
\textsc{t.r}-open-\textsc{apl}  3\textsc{s.erg}  \textsc{nabs}  little.bit  only/just \\
\glt ‘S/he opened it \textbf{just} \textbf{a} \textbf{little} \textbf{bit}.’
\z

Examples \REF{bkm:Ref52476459} through \REF{bkm:Ref479413467} from the corpus illustrate MPs of various compositions functioning as clause-level modifiers:

\ea
\label{bkm:Ref52476459}
Sikad  gatingaļa  gid  \textbf{ta}  \textbf{bakod}  mga  ittaw  an  man-o  tak ittaw  i  gatallog. \\\smallskip
 \gll Sikad  ga-tingaļa  gid  \textbf{ta}  \textbf{bakod}  mga  ittaw  an  man-o  tak ittaw  i  ga-tallog. \\
very  \textsc{i.r}-wonder  \textsc{int}  \textsc{nabs}  big  \textsc{pl}  person  \textsc{def.m}  why  because person  \textsc{def.n}  \textsc{i.r}-egg \\
\glt ‘People really were wondering \textbf{greatly} (lit. big) why a person was laying eggs.’ [MBON-T-07 10:7]
\z
\ea
Dayon  kon  papakang  buļag  tamboļ  ya  \textbf{ta}  \textbf{sikad}  \textbf{tudo}. \\\smallskip
 \gll Dayon  kon  pa-pakang  buļag  tamboļ  ya  \textbf{ta}  \textbf{sikad}  \textbf{tudo}. \\
right.away  \textsc{hsy}  \textsc{t.r}-hit  blind  drum  \textsc{def.f}  \textsc{nabs} very  intense \\
\glt ‘Right away the blind one hit the drum \textbf{very} \textbf{hard}.’ [CBWN-C-10 7.12]
\z
\ea
Ta  buļan  ta  Abril  gulpi  nang  uran  \textbf{ta}  \textbf{sikad}  \textbf{dakmeļ}. \\\smallskip
 \gll Ta  buļan  ta  Abril  gulpi  nang  uran  \textbf{ta}  \textbf{sikad}  \textbf{dakmeļ}. \\
\textsc{nabs} month/moon  \textsc{nabs} April  suddenly  only/just  rain  \textsc{nabs} very  thick \\
\glt ‘In the month of April (it) suddenly rains \textbf{very} \textbf{heavily}.’ [JCWE-T-14 3.5]
\z
\ea
Kada  adlaw  mag-istudyo  ta  iran  na  liksyon  \textbf{ta}  \textbf{miad}. \\\smallskip
 \gll Kada  adlaw  mag-istudyo  ta  iran  na  liksyon  \textbf{ta}  \textbf{miad}. \\
each  day  \textsc{i.ir}-study  \textsc{nabs}  3\textsc{p.gen}  \textsc{lk}  lessson  \textsc{nabs}  well/good \\
\glt ‘Each day (they) studied their lessson \textbf{well}.’ [CFWE-T-01 2.7]
\z
\ea
Mga  kan-en  na  sinama  dapat  na  takļeban  \textbf{ta}  \textbf{usto}. \\\smallskip
 \gll Mga  kan-en  na  s<in>ama  dapat  na  \emptyset{}-takļeb-an  \textbf{ta}  \textbf{usto}. \\
\textsc{pl}  cooked.rice  \textsc{lk}  <\textsc{nr.res}>leftover  must  \textsc{lk}  \textsc{t.ir}-cover-\textsc{apl}  \textsc{nabs} well \\
\glt ‘Leftover rice must be covered \textbf{well}.’ [JCWN-T-21 12.6]
\z

Sometimes in relaxed speech the non-absolutive marker drops out (as in \ref{bkm:Ref479413467}).

\ea
\label{bkm:Ref479413467}
Tenged  ta  adlek  din,  gadļagan  \textbf{tudo}  kabaw  ya. \\\smallskip
 \gll Tenged  ta  adlek  din,  ga-dļagan  \textbf{tudo}  kabaw  ya. \\
because  \textsc{nabs}  afraid  3\textsc{s.gen}  \textsc{i.r}-run  intense  carabao  \textsc{def.f} \\
\glt ‘Because of his (the rider of the carabao) fear, the carabao ran \textbf{intensely}.’ [RCON-L-02 2.13]
\z

\largerpage[2]
\subsection{MPs headed by numbers and quantifiers}
\label{bkm:Ref439924202}

Number Phrases (NumP) are a subtype of Modifier Phrases. They may function as RP modifiers in MOD\textsubscript{1} or MOD\textsubscript{2} position, or as non-verbal predicates. However, they have a slightly different structural template than other Modifier Phrases, and so we are treating them distinctly. A Numeral Phrase may consist of just a numeral, a numeral accompanied by adverbs, and/or an optional classifier or measure word. Unlike other MPs, only one adverbial particle can precede the numeral, namely the indefinite plural adverbial \textit{mga}. The basic structure of the NumP is given in \REF{bkm:Ref343924122}:

\ea
\label{bkm:Ref343924122}  NumP = (PL) NUM (ADV2) (\{CLASS/MEASURE\})
\z

The Numeral Phrase consists of an optional adverb \textit{mga} (glossed PL, as it is identical to the plural marker at the RP level) indicating “approximately”, followed by the cardinal or ordinal numeral. Following the numeral are optional intensity and aspectual adverbs as in the MP. Following the adverb(s), an optional \textit{noun classifier}, such as \textit{buok} ‘piece’, \textit{bilog} ‘round’, \textit{panid} ‘flat sheet’, \textit{puon} ‘stem’, \textit{naet} ‘strand’, or an optional measurement, such as \textit{akep} ‘handful’, \textit{kilo} ‘kilogram, \textit{litsi} ‘8oz.can’, \textit{gantang} ‘ganta’\footnote{A \textit{ganta} is a dry measure in the Philippines mostly for rice that equals about 3 quarts.}, \textit{pungpong} ‘cluster’, \textit{begkes} ‘bundle’, \textit{baso} ‘glass’ and many others.


\textit{Mga} ‘approximately’ and the measurement/classifier words do not occur with ordinal numbers in the text corpus, but do occur in conversation (\ref{bkm:Ref365610460} and \ref{bkm:Ref365610463}). When the numeral phrase includes \textit{mga} ‘approximately’ before the numeral, there is usually no \textit{mga} plural marker immediately before the Head of the larger RP, though it sometimes does occur. Examples \REF{bkm:Ref117598318} through \REF{bkm:Ref343776670} illustrate a few Number Phrases occurring in the text corpus:

\ea
\label{bkm:Ref117598318}
NumP in attributive function: \\
\textbf{tallo}  i  na  mag-arey \\\smallskip
 \gll \textbf{tallo}  i  na  mag-arey \\
three  \textsc{def.n}  \textsc{lk}  \textsc{rel}-friend \\
\glt ‘the three friends’ [CBWN-C-10 2.2]
\z
\ea
\label{bkm:Ref117598320}
\textbf{kapitto}   ya  na  bukid \\\smallskip
 \gll \textbf{ka-pitto}   ya  na  bukid \\
\textsc{ord-}seven  \textsc{def.f}  \textsc{lk}  mountain \\
\glt ‘the seventh mountain’ [PBWN-C-12 11.7]
\z
\ea
\textbf{isya}  \textbf{nang}  \textbf{panid}   na  daon \\\smallskip
 \gll \textbf{isya}  \textbf{nang}  \textbf{panid}   na  daon \\
one  only  sheet  \textsc{lk}  leaf \\
\glt ‘only one leaf’ [MBON-T-07 13.4]
\z
\ea
\textbf{isya}  \textbf{nang}  \textbf{litsi}     na  ummay  na  may  passi  pa \\\smallskip
 \gll \textbf{isya}  \textbf{nang}  \textbf{litsi}     na  ummay  na  may  passi  pa \\
one  only  8oz.can  \textsc{lk}  unmilled.rice  \textsc{lk}  \textsc{ext.in}  husk  still \\
\glt ‘just one can of rice still with the husks’ [PBWN-C-12 8.2]
\z
\ea
\label{bkm:Ref365610460}
\textbf{mga}  \textbf{pangappat}   na  adlaw \\\smallskip
 \gll \textbf{mga}  \textbf{pang-appat}   na  adlaw \\
\textsc{pl}  \textsc{ord}-fourth  \textsc{lk}  day \\
\glt ‘approximately the fourth day’
\z
\ea
\label{bkm:Ref365610463}
\textbf{panglimma}  \textbf{bilog}   na  baboy \\\smallskip
 \gll \textbf{pang-limma}  \textbf{bilog}   na  baboy \\
\textsc{ord}-five  piece   \textsc{lk}  pig \\
\glt ‘the fifth pig’
\z


\ea
NumP in predicate function: \\
\textbf{Mga}  \textbf{tallo}  \textbf{nang}  \textbf{en}   duminggo  ame  na  klasi bag-o  eman  magbakasyon. \\\smallskip
 \gll \textbf{Mga}  \textbf{tallo}  \textbf{nang}  \textbf{en}   duminggo\textup{\footnotemark{}}  ame  na  klasi bag-o  eman  mag-bakasyon. \\
\textsc{pl}  three  only  \textsc{cm}  week  1\textsc{p.excl.gen}  \textsc{lk}  class
before  again.as.before  \textsc{i.ir}-vacation \\
\footnotetext{Notice the absence of linker \textit{na} before the Head of the RP, \textit{duminggo}. Speakers report that such time-setting phrases are more natural without the linker. This example is a nominal predicate that sets the time frame for a main assertion.}
\glt `\textbf{Only} \textbf{about} \textbf{three} weeks was before our vacation again.’ [JBWL-J-03 3.4]
\z
\ea
… riliyon  kon  i  danen  dili  kon \textbf{una}  \textbf{gid}   na  riliyon  ta  kalibutan. \\\smallskip
 \gll … riliyon  kon  i  danen  dili  kon \textbf{una}  \textbf{gid}   na  riliyon  ta  kalibutan. \\
{} religion  \textsc{hsy}  \textsc{def.n} 3\textsc{p.gen}  \textsc{neg.ir}  \textsc{hsy}
first  \textsc{int}  \textsc{lk}  religion  \textsc{nabs}  world \\
\glt ‘…their religion is not, they say, the \textbf{very} \textbf{first} religion in the world.’ [JCOE-C-04 8.7]
\z
\ea
\label{bkm:Ref343776670}
\textbf{Isya}  \textbf{nang}  \textbf{en}   na  layag  ame  na  palayag. \\\smallskip
 \gll \textbf{Isya}  \textbf{nang}  \textbf{en}   na  layag  ame  na  pa-layag. \\
one  only  \textsc{cm}  \textsc{lk}  sail  1\textsc{p.excl.gen}  \textsc{lk}  \textsc{t.r}-sail \\
\glt ‘\textbf{Only} \textbf{one} sail \textbf{now} was what we were sailing with.’ [VAWN-T-18 5.2]
\z

As shown in the RP template given in \chapref{chap:referringexpressions}, \sectref{sec:basicreferringphrases}, an {Є} element (enclitic genitive pronoun and/or enclitic demonstrative determiner) may follow any MOD\textsubscript{1} element, though it is less common after NumPs \REF{bkm:Ref342658638}{}-\REF{bkm:Ref343529548} than after other types of modifiers. Like other modifier phrases in MOD\textsubscript{1} position, no examples were found in texts of a Numeral Phrase with a numeral and an adverb in MOD\textsubscript{1} as well as an {Є} element. In the following examples the NumP is in brackets and the {Є} element is bolded.
\ea
\label{bkm:Ref342658638}
darwa  \textbf{ko}  \textbf{ya}  na  mga  utod \\\smallskip
 \gll [ darwa ]  \textbf{ko}  \textbf{ya}  na  mga  utod \\
{}  two  {}  1\textsc{s.gen}  \textsc{def.f}  \textsc{lk}  \textsc{pl}  sibling \\
\glt `my two siblings’ [PBWN-C-13 8.3]
\ex
\label{bkm:Ref343529548}
katallo  \textbf{din}  \textbf{ya}  na  uyok \\\smallskip
 \gll [ ka-tallo ]  \textbf{din}  \textbf{ya}  na  uyok \\
 {} \textsc{ord}-three {} 3\textsc{s.gen}  \textsc{def.f}  \textsc{lk}  whistle \\
\glt ‘his/her third whistle’ [CBWN-C-20 6.5]
\z

As with Modifier Phrases, the genitive enclitic of the referring phrase follows the first substantive word in a Numeral Phrase. This means that when a Numeral Phrase consists of more than one word, the genitive enclitic intrudes inside the phrase:

\ea
\label{bkm:Ref117598763}
tallo  \textbf{ko}   nang  na  mga  mangngod  na  mga  dirset \\\smallskip
 \gll [ tallo  \textbf{ko}   nang ]  na  mga  mangngod  na  mga  dirset \\
 {} three  1\textsc{s.gen}  only {} \textsc{lk}  \textsc{pl}  younger.sibling  \textsc{lk}  \textsc{pl}  small.\textsc{pl} \\
\glt ‘only my three younger siblings who are small’ [ETON-C-07 1.3]
\z

The demonstrative determiner for the RP, however, occurs after the whole numeral phrase (see also examples \ref{bkm:Ref117598318} and \ref{bkm:Ref117598320} above):

\ea
tallo  \textbf{ko}   nang   \textbf{ya}    na  mga  mangngod  na  mga  dirset \\\smallskip
 \gll {}[ tallo  \textbf{ko}   nang ]  \textbf{ya}    na  mga  mangngod  na  mga  dirset \\
 {}  three  1\textsc{s.gen} only {} \textsc{def.f}  \textsc{lk}  \textsc{pl}  younger.sibling  \textsc{lk}  \textsc{pl}  small.\textsc{pl} \\
\glt ‘only my three small younger siblings’ (variation of \ref{bkm:Ref117598763})
\z

The fraction \textit{tenga} ‘half’ occurs before the RP Head when there is no other numeral (\ref{bkm:Ref329278116}) and following the RP Head (in MOD\textsubscript{2} position) when there is another numeral \REF{bkm:Ref329278135}.

\ea
\label{bkm:Ref329278116}
\textbf{tenga}  \textbf{sako}   na  beggas \\\smallskip
 \gll \textbf{tenga}  \textbf{sako}   na  beggas \\
half  sack  \textsc{lk}  milled.rice \\
\glt ‘half a sack of rice’ [DBWN-L-22 6.13]
\z
\ea
\label{bkm:Ref329278135}
\textbf{isya}  \textbf{nang}  \textbf{pa}   buļan  \textbf{tenga} \\\smallskip
 \gll \textbf{isya}  \textbf{nang}  \textbf{pa}   buļan  \textbf{tenga} \\
one  only  \textsc{inc}  month/moon  half \\
\glt ‘still only one and a half months’ [ETOP-C-08 3.2]
\z

Quantifiers may also occur as heads of Modifier Phrases that may function as RP modifiers in MOD\textsubscript{1}, but not in MOD\textsubscript{2} position. They may also function as non-verbal predicators. See \sectref{sec:non-numeralquantifiers} above for description of the quantifier words and examples in such \textit{quantifier phrases}\is{quantifier phrases}. They have the same structural template as other Modifier Phrases (\ref{bkm:Ref117605153}), except that the Head is a quantifier.

\ea
Sikad  tama  man  iya  na  mga  payong. \\\smallskip
 \gll {}[ Sikad  tama  man ]  iya  na  mga  payong. \\
{} very  many  \textsc{emph} {} 3\textsc{s.gen}  \textsc{lk}  \textsc{pl}  umbrella \\
\glt ‘Her umbrellas were \textbf{very} \textbf{many} \textbf{too}.’ [BMON-C-05 4.6]
\z
\ea
Tama  pa  man  mga  darko  na  mga  buli. \\\smallskip
 \gll {}[ Tama  pa  man ]  mga  darko  na  mga  buli. \\
{} many  \textsc{inc}  \textsc{emph}  {} \textsc{pl}  large.\textsc{pl}  \textsc{lk}  \textsc{pl}  buri \\
\glt ‘Large buri plants are \textbf{\textsc{many}} \textbf{yet.}’ [JCWE-L-32 3.7]
\z
\ea
Sikad  gid  tama   na  mga  pilak  na  dayad. \\\smallskip
 \gll [ Sikad  gid  tama ]  na  mga  pilak  na  dayad. \\
{} very  \textsc{inc}  many {} \textsc{lk}  \textsc{pl}  silver  \textsc{lk}  good \\
\glt ‘Really \textbf{very} \textbf{much} good silver!’ (This is an exclamation and so has the structure of an RP.) [AGWN-L-01 5.8]
\z
\ea
Sikad  na  tama  na  iitaw  naan  ta  pantalan  ya  na gabot kay. \\\smallskip
 \gll {}[ Sikad  na  tama ]  na  iitaw  naan  ta  pantalan  ya  na ga-abot kay. \\
{} very  \textsc{lk}  many {} \textsc{lk}  person  \textsc{spat.def}  \textsc{nabs}  pier  \textsc{def.f}  \textsc{lk}
\textsc{i.r}-arrive 1\textsc{p.excl.abs} \\
\glt ‘\textbf{Very} \textbf{many} people were on the pier when we arrived.’ [BCWN-C-04 9.5]
\z
\ea
Patubbas  nay   tiset  nang,  kulang  pa  kan-enen. \\\smallskip
 \gll Pa-tubbas  nay  [ tiset  nang ],  kulang  pa  kan-en-en. \\
\textsc{t.r}-corn.harvest  1\textsc{p.excl.erg} {} few  just/only {} lack  \textsc{inc}  cooked.rice-\textsc{nr} \\
\glt ‘What we harvested was only \textbf{a} \textbf{few} (ears of corn), not enough to eat.’ [JAWE-T-01 3.8]
\z


\ea
Bisan  pa  na  tise  kay  nang di tuytuyi kay Amay. \\\smallskip
 \gll Bisan  pa  na  [ tise  kay  nang ]  di tuytoy-i kay Amay. \\
even.though  \textsc{inc}  \textsc{lk} {} few  1\textsc{p.excl.abs}  just/only {} \textsc{d}1\textsc{loc} guide-\textsc{xc.apl}  1\textsc{p.excl.abs}  father \\
\glt `Even though we are \textbf{only} \textbf{a} \textbf{few} here, guide us Father.’ (This is part of a prayer at the end of a sermon.) [TTOB-L03 12.3]
\z
\ea
Pila  nang  naet na  nit  nabilin. \\\smallskip
 \gll {}[ Pila  nang  naet ]  na  nit  na-bilin. \\
{} few  just/only  strand {} \textsc{lk}  net  \textsc{a.hap.r}-leave.behind \\
\glt ‘\textbf{Only} \textbf{a} \textbf{few} nets were left.’ (This is about a big typhoon that hit and damaged their livelihood including fishing nets.) [JCWL-T-18 5.6]
\z

The following expressions are frequently heard at the end of public speeches, but the first is more frequent in the corpus.

\ea
Matama  nang  na  salamat. \\\smallskip
 \gll {}[ Ma-tama  nang ]  na  salamat. \\
{} \textsc{adj}-many  just/only {} \textsc{lk}  thanks \\
\glt ‘\textbf{Just} \textbf{many} thanks.’ [BFOB-T-01 8.1]
\z
\ea
Matama  gid  na  salamat.\\\smallskip
\gll {}[ Ma-tama  gid ]  na  salamat. \\
{} \textsc{adj}-many  \textsc{int} {} \textsc{lk}  thanks \\
\glt ‘\textbf{Really} \textbf{many} thanks.’ [MCOB-C-01 13.1]
\z

Sometimes only the degree adverb and the intensifier adverb can occur as the non-verbal predicate.

\ea
Uyo nya  sikad  gid  ake  na  kulba. \\\smallskip
 \gll U-yo nya  [ sikad  gid ]  ake  na  kulba. \\
\textsc{emph}-\textsc{d4abs}  \textsc{d4pr} {} very  \textsc{int} {} 1\textsc{s.gen}  \textsc{lk}  nervous \\
\glt ‘That (is) the very thing/event/reason that my being-nervous was \textbf{really} \textbf{very} \textbf{(much)}.’ [DBON-C-08 2.15]
\is{Modifier Phrases|)}
\z


\section{Adverbs}
\largerpage
\label{bkm:Ref420904083}\label{sec:adverbs} \is{adverbs|(}

Adverbs are general-purpose modifiers. They are words or lexicalized multiword expressions that qualify properties, actions or states within a clause, or express epistemic, modal or evidential value for an entire speech act.


We recognize six main groups of adverbs in Kagayanen. It must be kept in mind, however, that, except for locative adverbs (\sectref{bkm:Ref52477189}), these groups are not lexical subclasses of the lexical category “adverb”. The first four types are adverbial “positions” in a clause, that may be filled by any number of lexical adverbs. Which adverb goes into which position or positions is a matter of lexical semantics in combination with a specific communicative situation and the creativity of the speaker. Often the semantic or discourse-pragmatic contribution of a particular lexical adverb varies depending on which position it occurs in, as will be seen in the following discussion. The fifth type, prepositional adverbs, is defined by the presence of the non-absolutive case marker \textit{ta} appearing before any lexical adverb. Finally, locational adverbs are a distinct lexical class that have much in common with demonstrative pronouns and adjectives (\chapref{chap:referringexpressions}, \sectref{sec:deicticpronouns}). In this section, we will describe the six groups, and will illustrate adverbs commonly occurring in each group.

\textit{Disjunct adverbs}\is{disjunct adverbs}\is{adverbs!disjunct} (\sectref{bkm:Ref441599506}) are those that occur preverbally outside the clause and normally express the speaker’s attitude or evaluation of the proposition expressed in the clause. \textit{Canonical adjunct adverbs}\is{canonical adjunct adverbs}\is{adverbs!canonical adjunct} occur inside the clause. They may be in clause-initial position preceding the predicating word, second position following all other second-position clitics, or they may occur at any major constituent boundary, including clause-finally (\sectref{bkm:Ref52688127}). \textit{Non-canonical adjunct adverbs}\is{non-canonical adjunct adverbs}\is{adverbs!non-canonical adjunct} also occur inside a clause, but they normally occur in clause-initial position, though a few may occur immediately post-verbally. When occurring in clause-initial position, these adverbs impose formal restrictions on the form of the main verb (\sectref{bkm:Ref480610640}). The fourth group, \textit{second-position adverbs}\is{second-position adverbs}\is{adverbs!second-position}, always occur after the first major constituent of the clause, be it a verb, an adjunct adverb, or some other fronted clause constituent (\sectref{bkm:Ref479858962}). The fifth group, \textit{prepositional adverbs}\is{prepositional adverbs}\is{adverbs!prepositional}, always follow the non-absolutive case marker, \textit{ta} (\sectref{bkm:Ref480640397}), and distribute like \isi{canonical adjunct adverbs} except they do not occur clause-initially. The disjunct adverb, second-position adverb, and prepositional adverb positions are filled by small and relatively well-defined sets of lexical items. Both types of adjunct \isi{adverbs} constitute large, and semantically heterogeneous classes. However, as mentioned above, just about any lexical adverb \textit{may} function in just about any position, depending on the communicative needs and creativity of the speaker. Finally, \textit{locative adverbs}\is{locative adverbs}\is{adverbs!locative} are deictic forms referring to places, such as \textit{here} and \textit{there} in English (\sectref{bkm:Ref52477189}). This is the most restricted and lexically well-defined set of adverbs.

\subsection{Disjunct adverbs}
\label{bkm:Ref441599506}
\is{disjunct adverbs}
\textit{Disjunct adverbs} \is{disjunct adverbs|(}\is{adverbs!disjunct|(}are syntactically outside of, or “disjoined”, from the clause (see, e.g., \citealt{greenbaum1996} on this term as applied to English). Evidence for this in Kagayanen is that disjunct adverbs occur before the predicating word, are potentially followed by a pause, and do not attract the second-position \isi{enclitic pronouns} and adverbs. Furthermore, verbal predicates that follow a disjunct adverb are fully finite---they may appear in any inflectional form. Disjunct adverbs mostly express the speaker’s attitude or confidence in the truth of the whole utterance. The canonical lexical adverbs that often function in this position are \textit{sigurado} ‘surely’, \textit{siguro} ‘perhaps’, \textit{basi} ‘maybe’, \textit{ti}/\textit{ta} ‘so’, \textit{inta} ‘(I) wish/wishfully’ (glossed \textsc{opt} for`optative mood'), \textit{lugay/lugar/tapos} ‘then’ and \textit{dason} ‘next’. \textit{Sigurado} ‘surely’, \textit{siguro} ‘perhaps’, and \textit{tapos} can also function as inflected comple\-ment-taking predicates (see \chapref{chap:clausecombining}, \sectref{sec:nominalizationsasdirectcomplementclauses}, and \sectref{sec:subjunctivecomplementclauses}). But \textit{basi}, \textit{ti}/\textit{ta}, \textit{inta}, and \textit{lugay/lugar} do not function in this way.
\textit{Inta} can also occur as a \is{second-position adverbs}\is{adverbs!second-position}second-position adverb (see \sectref{bkm:Ref441599628} below). Other adverbs may also appear in the disjunct position, but we consider the eight mentioned above as \textit{canonical disjunct adverbs}\is{canonical disjunct adverbs}\is{adverbs!canonical disjunct}, because they are far more common and frequent than any of the others.

The following are some examples of disjunct adverbs in context:

\ea
\label{bkm:Ref441903321}
\textit{Sigurado} ‘surely’ \\
\textbf{Sigurado},  muli  danen  anduni  tak  linaw  en. \\\smallskip
 \gll \textbf{Sigurado},  m-uli  danen  anduni  tak  linaw  en. \\
surely  \textsc{i.v.ir}-go.home  3\textsc{p.abs}  now/today  because  calm  \textsc{cm} \\
\glt ‘\textbf{Surely}, they will come home now because (winds and/or waves are) calm.’
\z

In example \REF{bkm:Ref441903321}, if the adverb \textit{sigurado} were a clause constituent, the absolutive enclitic \textit{danen} would occur immediately following it. Furthermore, there is a noticeable pause after \textit{sigurado}, as signaled by the comma. The following are a few more examples of disjunct adverbs from the corpus:

\ea
\textit{Siguro} ‘perhaps’ (The speaker is fairly certain of the truth of the utterance.) \\
Maria,  \textbf{siguro},  namatian  no  mga  inpurmasyon  na  tenged  ta natabo  ki  kami  di  ta  ate  i  na  banwa. \\\smallskip
 \gll Maria,  \textbf{siguro,}  na-mati-an  no  mga  inpurmasyon  na  tenged  ta na-tabo  ki  kami  di  ta  ate  i  na  banwa. \\
Maria  perhaps  \textsc{a.hap.r}-hear-\textsc{apl}  2\textsc{s.erg}  \textsc{pl}  information  \textsc{lk}  about  \textsc{nabs}
\textsc{a.hap.r}-happen  \textsc{obl.p}  1\textsc{p.excl}  \textsc{d1loc}  \textsc{nabs}  1\textsc{p.excl.gen}  \textsc{def.n}  \textsc{lk}  town/country \\
\glt `Maria, \textbf{perhaps}, you have heard information about what happened to us here in our town.’ [JCWL-T-18 5.1]
\z
\ea
\textit{Basi} ‘maybe’ (Speaker is less certain about  the truth of the utterance) \\
\textbf{Basi},  daw  katapos  a  tanem  guso,  muli  a. \\\smallskip
 \gll \textbf{Basi,}  daw  ka-tapos  a  tanem  guso,  m-uli  a. \\
maybe  if/when \textsc{i.exm}-finish  1\textsc{s.abs}  plant  agar.seaweed  \textsc{i.v.ir}-go.home  1\textsc{s.abs} \\
\glt ‘\textbf{Maybe,} when I finish planting agar seaweed, I will come home.’ [ATWL-J01 4.7]
\z
\ea
\textbf{Basi},  magdugang  imo  na  sakit \\\smallskip
 \gll \textbf{Basi},  mag-dugang  imo  na  sakit \\
maybe  \textsc{i.ir}-add  2\textsc{s.gen}  \textsc{lk}  pain. \\
\glt ‘\textbf{Maybe}, your pain will increase.’ [BCWL-C-03 4.7]
\z
\ea
\textit{Ta/Ti} ‘so/resuming’ (Speaker signals a major development in the story or resumption of the mainline of the story after a digression or background information.) \\
\textbf{Ti},  daw  mamatian  nyo  gani  en  singgit  an  ta  Pwikan … \\\smallskip
 \gll \textbf{Ti},  daw  ma-mati-an  nyo  gani  en  singgit  an  ta  Pwikan … \\
so  if/when  \textsc{a.}\textsc{hap}\textsc{.ir}-hear-\textsc{apl}  2\textsc{p.erg}  truly  \textsc{cm}  shout  \textsc{def.m}  \textsc{nabs}  sea.turtle \\
\glt ‘\textbf{So}, if you hear truly the shout of Sea Turtle, …' [JCON-T-08 24.2]
\z
\ea
\textit{Inta} ‘wishfully’ (Speaker wishes that the utterance were true, but it is not, or speaker wants it to be true in the future. \textit{Inta} also may have a deontic sense) \\
\textbf{Inta},  miling  a  ta  uma  mangaoy. \\\smallskip
 \gll \textbf{Inta},  m-iling  a  ta  uma  ma-ng-kaoy. \\
\textsc{opt}  \textsc{i.v.ir}-go  1\textsc{s.abs}  \textsc{nabs}  field  \textsc{a.}\textsc{hap}\textsc{.ir}-\textsc{pl}-wood \\
\glt ‘(I) \textbf{ought/wish}, to go to the field to gather firewood.’
\z


\ea
\textbf{Inta},  giling  a  ta  uma  mangaoy. \\\smallskip
 \gll \textbf{Inta},  ga-iling  a  ta  uma  ma-ng-kaoy. \\
\textsc{opt}  \textsc{i.r}-go  1\textsc{s.abs}  \textsc{abs}  field  \textsc{a.hap.ir}-\textsc{pl}-wood \\
\glt ‘(I) \textbf{wish}, I had gone to the field to gather firewood.’
\z
\ea
\textbf{Inta},  magsuļat  ka  man  ki  nanay  no  tak  sikad gid  kasebe  din  na  sikad  ka  madyo  en  ki  kanen. \\\smallskip
 \gll \textbf{Inta},  mag-suļat  ka  man  ki  nanay  no  tak  sikad gid  ka-sebe  din  na  sikad  ka  madyo  en  ki  kanen. \\
\textsc{opt}  \textsc{i.ir}-write  2\textsc{s.abs}  \textsc{emph}  \textsc{obl.p}  mother  2\textsc{s.gen}  because  very
\textsc{int}  \textsc{nr}-sad  3\textsc{s.gen}  \textsc{lk}  very  2\textsc{s.abs}  far  \textsc{cm}  \textsc{obl.p}  3s \\
\glt ‘\textbf{Hopefully/It-ought-to-be}, you will write to your mother because her sadness is really much that you are very far from her.’ [PMWL-T-06 2.3]
\z

\newpage
\ea
\textit{Lugay} ‘then’/ ‘long time’\footnotemark \\
\textbf{Lugay},  ambaļ  din,  “Pari,  miad  ka  nang  daw  may  pagkaan.” \\\smallskip
 \gll \textbf{Lugay},  ambaļ  din,  “Pari,  miad  ka  nang  daw  may  pagkaan.”\\
then  said  3\textsc{s.gen}   friend  kind  2\textsc{s.abs}  only  if/when  \textsc{ext.in}  food\\
\footnotetext{\textit{Lugay} signals a new development in a narrative. As a disjunct adverb, it just indicates that an unspecified time has lapsed. However, in other positions, it expresses the idea that a significant length of time has passed between clauses. \textit{Lugar} as a clause-initial adverb has essentially the same syntactic and semantic profile, and so may be considered to be a variant of \textit{lugay}. In other environments it is a noun that means `place' (as in Spanish).}
\glt ‘\textbf{Then}, he said, “Friend, you are kind only if/when there is food.”' [RBWN-T-02 5.3]
\z
\ea
\textit{Lugar} ‘then’/‘place’ (This may be a variant of \textit{lugay} \\
\textbf{Lugar},  dili  ko  mapegengan  ake  na  luwa. \\\smallskip
 \gll \textbf{Lugar},  dili  ko  ma-pegeng-an  ake  na  luwa. \\
then  \textsc{neg.ir}  1\textsc{s.erg}  \textsc{a.}\textsc{hap}\textsc{.ir}-control-\textsc{apl}  1\textsc{s.gen}  \textsc{lk}  tear \\
\glt ‘\textbf{Then}, I couldn’t control my tears.' (This is  a story about a student whose teacher accused him wrongly of cheating on a test.) [BCWN-T-06 2.8]
\z
\ea
\textit{Tapos} ‘after’/’finish’ (Speaker signals a new development in a narrative) \\
\textbf{Tapos},  guļi  kay  naan  ta  baļay … \\\smallskip
 \gll \textbf{Tapos},  ga-uļi  kay  naan  ta  baļay … \\
then  \textsc{i.r}-go.home  1\textsc{p.excl.abs}  \textsc{spat.def} \textsc{nabs}  house \\
\glt ‘Then, we went home to the house … ’ [AGWN-L-01 3.13]
\z
\ea
Dason ‘next’ (Speaker signals the next event in a story, next step in a procedure or next point in an expository or behavioral text) \\
\textbf{Dason},  painsaan  man  Umang  i,  “Umang,  anda  ka  en ta  inyo  na  palumbaanay?” \\\smallskip
 \gll \textbf{Dason},  pa-insa-an  man  Umang  i,  “Umang,  anda  ka  en ta  inyo  na  pa-lumba-anay?” \\
next  \textsc{t.r}-ask-\textsc{apl}  also  hermit.crab  \textsc{def.n}  hermit.crab  ready  2\textsc{s.abs}  \textsc{cm} \textsc{nabs}  2\textsc{p.gen}  \textsc{lk}  \textsc{t.r}-race-\textsc{rec} \\
\glt “Next, the Hermit Crab was asked, “Hermit Crab, are you ready for your racing each  other?” [DBWN-T-26 4.8]
\z
\subsection{Canonical adjunct adverbs}
\label{bkm:Ref52688127} \label{sec:adjunctadverbs} \is{canonical adjunct adverbs|(}\is{adverbs!canonical adjunct|(}

\textit{Canonical adjunct adverbs} usually precede the predicating word. They may occur in second position following all other second-position clitics, or they may occur at any major syntactic boundary later in the clause, including clause-final position. When they appear before the predicating word, they are not followed by a pause, and they do attract the enclitic pronouns and adverbs. For these reasons, it is clear that adjunct adverbs are syntactically adjoined to the clause \citep{greenbaum1996}. Canonical adjunct adverbs\is{canonical adjunct adverbs} tend to express temporal notions, and all have Austronesian etymologies. A selection of adverbs that tend to function as canonical adjunct adverbs is given in \REF{bkm:Ref479947563}:

\ea
\label{bkm:Ref479947563}
\begin{tabbing}
    
Adverbs appearing in canonical adjunct positions \\
\hspace{3.5cm} \= \kill
\textit{kis-a/kaysan } \>  ‘sometimes’ \\
\textit{tagsa } \>  ‘seldom/once in a while’ \\
\textit{kisyem } \>  ‘tomorrow’ \\
\textit{anduni } \>  ‘now/today’ \\
\textit{gibii/gabii/kibii } \>  ‘yesterday’ \\
\textit{sanadlaw/sinadlaw } \>  ‘two days ago or in future’ \\
\textit{gina } \>  ‘earlier’ \\
\textit{kani } \>  ‘later’ \\
\textit{kan-o } \>  ‘previously’ \\
\textit{pugya } \>  ‘very long ago’ \\
\textit{gina ugtu-adlaw } \>  ‘earlier midday' \\
\textit{kani ugtu-adlaw } \>  ‘later midday’ \\
\textit{sellem pa } \>  ‘early morning’ \\
\textit{gina sellem } \>  ‘earlier in the morning’ \\
\textit{kisyem sellem } \>  ‘tomorrow morning’ \\
\textit{gina mapon } \>  ‘earlier afternoon’ \\
\textit{kani mapon } \>  ‘later afternoon’ \\
\textit{gibii kilem } \>  ‘yesterday evening/night’ \\
\textit{kani kilem } \>  ‘later evening/night’ \\
\textit{gina kadlunon } \>  ‘earlier in the small hours of morning 2--3 am’ \\
\textit{kani kadlunon } \>  ‘later in the small hours of morning 2--3 am’ \\
\textit{gina mapit madlaw } \>  ‘earlier in the early morning 3--4 am’ \\
\textit{kani mapit madlaw } \>  ‘later in the early morning 3--4 am’ \\
\textit{gina kaagaen } \>  ‘earlier right before dawn’ \\
\textit{kani kaagaen } \>  ‘later right before dawn’ \\
\textit{inta } \>   ‘wish/intent/optative’ \\
\textit{pirmi } \>  ‘always’ \\
\textit{dayon } \>  ‘right away’ / ‘immediately’
\end{tabbing}
\z

There are clearly different nuances associated with these adverbs depending on whether they appear preverbally, in second position, or in clause-final position. A thorough study of the use of adverbs in discourse will be necessary to tease apart all such nuances and variable usages.

The following are examples from the corpus of adjunct adverbs in context. Example \REF{bkm:Ref480191239} illustrates \textit{kani} appearing clause-initially:

\ea
\label{bkm:Ref480191239}
Nasugid  na  gakumpanyar  an  ta  paglebbeng, \textbf{kani}  kanen  en  man  ilebbeng. \\\smallskip
 \gll Na-sugid  na  ga-kumpanyar  an  ta  pag-lebbeng, \textbf{kani}  kanen  en  man  i-lebbeng. \\
\textsc{a.hap.r}-tell  \textsc{lk}  \textsc{i.r}-funeral.procession  \textsc{def.m}  \textsc{nabs}  \textsc{nr.act}-bury later  3\textsc{s.abs}  \textsc{cm}  too  \textsc{t.deon}-bury \\
\glt `(It was) told that the one going with the funeral procession to the burial, \textbf{later} he too must be buried.’ (This is a story about a very bad epidemic.) [JCWN-T-21 2.4]
\z

Note that in \REF{bkm:Ref480191239} the enclitics \textit{kanen}, \textit{en} and \textit{man} follow \textit{kani}. This shows that \textit{kani} is adjoined to the clause (i.e. it is the first element inside the clause boundary). This is the property that distinguishes adjunct adverbs from disjunct adverbs, which do not attract any second-position elements.

In \REF{bkm:Ref481496721} \textit{kani} appears following the main predicating word, which is in second position in this clause (\textit{daw} and the other conjunctions are outside the clause boundary, and therefore do not attract second position elements):

\ea
\label{bkm:Ref481496721}
Daw  miyag  \textbf{kani} en  ginikanan  i  ta  mga  bai  en, yo  en  mangagon  en  mama  i. \\\smallskip
 \gll Daw  miyag  \textbf{kani} en  ginikanan  i  ta  mga  bai  en, yo  en  ma-ng-kagon  en  mama  i. \\
if/when  agree/want  later  \textsc{cm}   parents  \textsc{def.n}  \textsc{nabs}  \textsc{pl}  woman  \textsc{cm}  \textsc{d4abs}  \textsc{cm}  \textsc{a.hap.ir}-\textsc{pl}-engage  \textsc{cm}  man  \textsc{def.n} \\
\glt `If \textbf{later} the parents of the women agree (to the men marrying their daughters), that (is when) the men will be engaged (to the women).’ [DBOE-C-01 1.3]
\z

In \REF{bkm:Ref481497209}, \textit{kani} follows a fronted RP:
\ea
\label{bkm:Ref481497209}
Bata  an  \textbf{kani}  magsabat  na  pabunaļ  tak  may  saļa. \\\smallskip
 \gll Bata  an  \textbf{kani}  mag-sabat  na  pa-bunaļ  tak  may  saļa. \\
child  \textsc{def.m}  later  \textsc{i.ir}-answer  \textsc{lk}  \textsc{t.r}-spank  because  \textsc{ext.in}  mistake/wrong \\
\glt ‘The child \textbf{later} will answer that (s/he) was spanked because (s/he had done something) wrong.’ [EMWE-T-01 2.8]
\z

Example \REF{bkm:Ref481496953} illustrates \textit{kani} at the end of a clause.

\ea
\label{bkm:Ref481496953}
Manaog  ka,  Pedro.  Masunog  ka  \textbf{kani}. \\\smallskip
 \gll M-panaog  ka,  Pedro.  Masunog  ka  \textbf{kani}. \\
\textsc{i.v.ir}-go.down  2\textsc{s.abs}  Pedro  \textsc{a.hap.ir}-burn  2\textsc{s.abs}  later \\
\glt ‘Come down Pedro. You will get burned \textbf{later}.’ (A man was on the roof of a burning house trying to put out the fire.) [RZWN-T-02 3.7-8]
\z

The following are a few examples of other canonical adjunct adverbs from the corpus.

\ea
\textbf{Kisyem}  ki  nang  en  mag-isturya  ta  adlaw. \\\smallskip
 \gll \textbf{Kisyem}  ki  nang  en  mag-isturya  ta  adlaw. \\
tomorrow  1\textsc{p.incl.abs}  just  \textsc{cm}  \textsc{i.ir}-talk  \textsc{nabs}  sun/day \\
\glt ‘\textbf{Tomorrow} let’s just talk in the day (time).’ [ETON-C-07 3.11]
\z


\ea
Nanay  man  \textbf{kis-a}  ganii  man  para  sid-anan  ta mga  bata  din  daw  sawa  din  man. \\\smallskip
 \gll Nanay  man  \textbf{kis-a}  ga-nii  man  para  sid-anan  ta mga  bata  din  daw  sawa  din  man. \\
mother  \textsc{emph}  sometimes  \textsc{i.r}-gather.shellfish  also  for  viand  \textsc{nabs}
\textsc{pl}  child  3\textsc{s.gen}  and  spouse  3\textsc{s.gen}  also \\
\glt `Mother even \textbf{sometimes} gathers shellfish also for the viand of her children and her spouse also.’ [IC0E-C-01 9]
\z
\ea
Ganii  \textbf{kis-a}  nanay  para  sid-anan  danen. \\\smallskip
 \gll Ga-nii  \textbf{kis-a}  nanay  para  sid-anan  danen. \\
\textsc{i.r}-gather.shellfish  sometimes  mother  \textsc{purp}  viand  3\textsc{p.nabs} \\
\glt ‘Mother sometimes gathers shellfish for their viand.’
\z
\ea
Bilog  na  banwa  gatib-ong  ki  kaon  \textbf{anduni}. \\\smallskip
 \gll Bilog  na  banwa  ga-tib-ong\footnotemark{}  ki  kaon  \textbf{anduni}. \\
whole  \textsc{lk}  town  \textsc{i.r}-lift.up  \textsc{obl.p}  2s  now/today \\
\footnotetext{This verb is \isi{Hiligaynon} meaning to lift someone up, to raise to higher social position, elevate, promote, advance.}
\glt ‘The whole town is lifting you up \textbf{now/today}.’ [JCWO-T-30 24.1)
\z

Example \REF{bkm:Ref425515410} illustrates a threat that commonly employs the word \textit{kani} at the end with rising intonation and an extra long vowel on the last syllable.

\ea
\label{bkm:Ref425515410}
Daw  gamiten  no  sundang  Tatay,  magilek  kanen  ki  kaon  \textbf{kani}. \\\smallskip
 \gll Daw  gamit-en  no  sundang  Tatay,  ma-gilek  kanen  ki  kaon  \textbf{kani}. \\
if/when  use-\textsc{t.ir}  2\textsc{s.erg}  machete  Dad  \textsc{a.hap.ir}-angry  3\textsc{s.abs}  \textsc{obl.p}  2s  later \\
\glt ‘If you use Father’s machete, he will be angry with you \textbf{later}.’
\z

Any canonical adjunct adverb must occur in the clause it is modifying and not after dependent clauses or RPs containing relative clauses \REF{bkm:Ref329779114}.

\ea
\label{bkm:Ref329779114}
Apiten  din  nang  en  \textbf{kani}  bata  an  na  gatagad  naan  ta  inyo. \\\smallskip
 \gll Apit-en  din  nang  en  \textbf{kani}  bata  an  na  ga-tagad  naan  ta  inyo. \\
stop.off\textsc{-t.ir}  3\textsc{s.erg}  just  \textsc{cm}  later  child  \textsc{def.m}  \textsc{lk}  \textsc{i.r}-wait  \textsc{spat.def}  \textsc{nabs}  2\textsc{p.gen} \\
\glt ‘S/he will stop off \textbf{later} for the child who is waiting at your (place).’
\z

In examples \REF{bkm:Ref329779247} and \REF{bkm:Ref329779234}, the adverb \textit{anduni} ‘now/today’ occurs clause-ini\-tial\-ly. However, the function of this adverb in each of the two examples is very different because they illustrate two distinct constructions. In \REF{bkm:Ref329779247} the adverb is in a clause with a verb that has an exclamatory suffix (see \chapref{chap:stemformingprocesses} \sectref{sec:exclamatory}). The adverb is very closely tied to the clause itself and it modifies the whole clause. In contrast the adverb in \REF{bkm:Ref329779234} is set off from the clause by a slight pause. Also, it is functioning at a higher, discourse level, as a disjunct adverb (see \sectref{bkm:Ref441599506} above). In this case, the speaker is providing a setting for this clause and those that follow.

\ea
\label{bkm:Ref329779247}
\textbf{Anduni}  din  nang  nļami  na  kapasar  kanen ta  board exam! \\\smallskip
 \gll \textbf{Anduni}  din  nang  na-aļam-i  na  ka-pasar  kanen ta  board exam! \\
now/today  3\textsc{s.erg}  just/only  \textsc{a.hap.r}-know-\textsc{xc.apl}  \textsc{lk}  \textsc{i.exm}-pass  3\textsc{s.abs}
 \textsc{nabs}  board exam \\
\glt `Only \textbf{now/today} s/he knows that s/he passed the board exam!’
\z

\newpage

\ea
\label{bkm:Ref329779234}
\textbf{Anduni},  nļaman  din  en  na  kapasar  kanen ta board exam. \\\smallskip
 \gll \textbf{Anduni,}  na-aļam-an  din  en  na  ka-pasar  kanen ta board exam. \\
now/today  \textsc{a.hap.r}-know-\textsc{apl}  3\textsc{s.erg}  \textsc{cm}  \textsc{lk}  \textsc{i.exm}-pass  3\textsc{s.abs}
\textsc{nabs} board exam \\
\glt `\textbf{Now/toda}y, s/he knows already that s/he passed the board exam.’
\z

In fact all canonical adjunct adverbs may function in the disjunct adverb position with varying semantic scope effects. Such constructions are rather unusual, however, and a thorough study of their usages will have to await future research.
\is{adverbs!canonical adjunct|)}\is{canonical adjunct adverbs|)}

\subsection{Non-canonical adjunct adverbs}
\label{bkm:Ref480610640} \label{sec:non-canonicaladjunctadverbs}\is{non-canonical adjunct adverbs|(}\is{adverbs!non-canonical adjunct|(}

There exists a group of words in Kagayanen that exhibit many properties of adjunct adverbs, but differ in three important ways:

\begin{enumerate}
\item They only appear in pre-predicate position.
\item  When a non-canonical adjunct adverb is present, the form of the following verb is restricted in some way. Some require nominalized verbs, others bare-form or irrealis verbs. Others permit bare forms, but also allow other forms.
\item More than one-third of non-canonical adjunct adverbs are from Spanish (perhaps via other Philippine languages). This is in clear contrast with the class of canonical adjunct adverbs, none of which are from Spanish (see \ref{bkm:Ref479947563}). This suggests that the non-canonical adverb position itself may be a recent innovation.
\end{enumerate}

Non-canonical adverbs tend to express manner, modal, aspectual or epistemic notions. One may argue that non-canonical adjunct adverbs are somewhere on a continuum between adverb and auxiliary for the following reasons:

\begin{enumerate}
\item It is common in the world’s languages for auxiliaries to “govern”, or impose grammatical restrictions on the semantically main verb in the construction.
\item Aspectual, modal, and epistemic notions are the sort likely to be expressed by auxiliaries universally.
\item Preverbal position is the most likely location for auxiliaries in a VO language (\citealt{greenberg1963}; \citealt[90]{dryer2007}).
\end{enumerate}


Though non-canonical adjunct adverbs exhibit these properties known to hold of auxiliaries in other languages, there are also arguments against this analysis. First, there is little to no evidence that these elements are basically verbs (though some may be used as verbs in other constructions, as can most lexical roots in Kagayanen). As adverbs, they do not take verbal inflection. Second, there is no well-established and uncontroversial auxiliary position in Kagayanen that may serve as the target for grammaticalization of these adverbs. Further research may reveal additional arguments for and against the auxiliary hypothesis for non-canonical adjunct adverbs.

Words that commonly occur in this category are listed in \REF{bkm:Ref441309176}, sub-classified according to the restrictions they impose on the following verb. For the most part, these formal sub-classes seem to have semantic motivations. Our general impressions of the semantics of each class are given in parentheses, though these characterizations are not meant to be absolute.

\ea
\label{bkm:Ref480616640}\label{bkm:Ref441309176}
Non-canonical adjunct adverbs \\
\ea  Those requiring irrealis verbs in the following predicate (\textit{modal adverbs}\is{modal adverbs})\is{adverbs!modal}: \\
\begin{tabbing}
\hspace{2cm} \= \hspace{5cm} \= \hspace{1.65cm} \= \kill
\textbf{Adverb} \>   \textbf{Gloss}   \>       \textbf{Spanish Source (if any)} \\
\textit{dapat}  \>  ‘must do X’ \\
\textit{kinangļan} \> ‘need to do X’ \\
\textit{bawal}  \>  ‘forbidden to do X’ \>     bagual? \> ‘untamed/wild’ \\
\textit{pwidi}  \>  ‘permitted to do X’  \>     puede \> ‘able to do X’ \\
\end{tabbing}
\ex  Those that allow either bare verbs, realis or occasionally irrealis modality \\ verbs (\textit{punctuality adverbs}\is{punctuality adverbs}\is{adverbs!punctuality}, often with a following \textit{nang}: \\
\begin{tabbing}
\hspace{2cm} \= \hspace{5cm} \= \hspace{1.65cm} \= \kill
\textbf{Adverb} \>   \textbf{Gloss}   \>       \textbf{Spanish Source (if any)} \\
\textit{gulpi}  \>   ‘suddenly do X’   \>   (de) golpe \> ‘suddenly’ \\
\textit{nali}  \>   ‘abruptly do X’  \>    dale?  \>  ‘hurry up’ \\
\textit{listo} \>  ‘promptly/without delay do X’ \>  listo  \>  ‘ready’ \\
\textit{bag-o}  \>   ‘recently/newly do X’\footnotemark{} \\
\textit{una}  \>   ‘first do X’     \>   una\footnotemark{}  \>  ‘one’ \\
\end{tabbing}
\ex  Those that require bare-form verbs (\textit{durational manner adverbs}\is{durational manner adverbs}\is{adverbs!durational manner}): \\
\begin{tabbing}
\hspace{2cm} \= \hspace{5cm} \= \hspace{1.65cm} \= \kill
\textbf{Adverb} \>   \textbf{Gloss}   \>       \textbf{Spanish Source (if any)} \\
\textit{diritso}  \>   ‘straightaway do X’ \>     derecho \>  ‘straight’ \\
\textit{sigi}  \>   ‘continuously do X’   \>   sigue  \>  ‘continue’ (3sg) \\
\textit{inay-inay} \>  ‘slowly do X’ \\
\textit{sali} (\textit{ta} \>‘persistently do X’ \\
\end{tabbing}
\ex  Those that prefer either irrealis verbs or nominalized clauses, often \\
\begin{tabbing}
\hspace{2cm} \= \hspace{5cm} \= \hspace{1.65cm} \= \kill
\textbf{Adverb} \>   \textbf{Gloss}   \>       \textbf{Spanish Source (if any)} \\
with the prefix \textit{pag}{}- (\textit{velocity adverbs}\is{velocity adverbs}\is{adverbs!velocity}): \\
\textit{dali} \> ‘did/will do X hurriedly' \> dale  \>  ‘hurry up’, \\ 
\>    Or ‘X will soon/easily happen’\>\>`do X quickly' \\
\textit{daļas} \> ‘do X swiftly/quickly’ \\
\textit{dasig} \>‘do X fast quickly’ \\
\textit{sikad} \>   ‘do X a lot’ \\
\end{tabbing}
\ex  Those that allow bare verb stems, or inflected verbs (\textit{general adverbs})\is{general adverbs}\is{adverbs!general} \\
\begin{tabbing}
\hspace{2cm} \= \hspace{5cm} \= \hspace{1.65cm} \= \kill
\textbf{Adverb} \>   \textbf{Gloss}   \>       \textbf{Spanish Source (if any)} \\
\textit{lugay}  \>     ‘do X for a long time’ \\
\textit{uļa (nang) lugay} \> \hspace{1cm} ‘do X for a brief time' \\
\textit{dengngan}  \>  ‘do X at the same time (as a previously mentioned event)’ \\
\textit{sise/tise}   \> ‘almost do X/do X a little bit’ \\
\textit{laka}   \>   ‘rarely do X’
\end{tabbing}
\footnotetext[19]{The word \textit{bag-o} is a multi-purpose word that can function as an adjective meaning ‘new’, and as a subordinate clause introducer that means ‘before doing X’, as in \textit{galangoy kay ta isya na oras} \textbf{\textit{bag-o}} \textit{megbeng naan ta ebes} ‘We bathed for one hour before going down below.’ With this meaning it always precedes an irrealis verb. We consider this to be a different function than \textit{bag-o} as an adverb meaning ‘recently’.}
\footnotetext{This is the only ordinal number that can function as a non-canonical adjunct adverb.}
\z
\z

Examples of each subtype of non-canonical adjunct adverb in context are given below. In these examples, the adverb and the following main verb (when present) are in bold. The ungrammatical examples illustrate syntactic positions that these adverbs may not appear in, thus demonstrating their grammatical distinctness from canonical adjunct adverbs.

\newpage
\ea Class a: Modal adverbs
\ea
\textbf{dapat}  na  isya-isya  kiten  \textbf{mag-atag}  (ta)   kayaran ta  masigkaittaw  ta. \\\smallskip
 \gll \textbf{dapat}  na  isya\sim{}-isya  kiten  \textbf{mag-atag}  (ta)\footnotemark{}   ka-ayad-an ta  masigka-ittaw  ta. \\
must  \textsc{lk}  \textsc{red}-one  1\textsc{p.incl.abs}  \textsc{i.ir}-give  \textsc{nabs}  \textsc{nr}-good/well-\textsc{nr}
\textsc{nabs}  \textsc{nr.fellow}-person  1\textsc{p.incl.gen} \\
\footnotetext[21]{The speaker left this \textit{ta} out, which is common in conversation. However, it is clear that \textit{kayaran} is non-absolutive because the verb is marked as intransitive, and the actor, \textit{kiten}, is in the absolutive case.} 
\glt `… each and every one of us \textbf{must} \textbf{give} goodness to our fellow humans.’ (The meaning of give goodness is to do good things.) [JCOB-L-02 10.4] \\\smallskip

*\textbf{Mag-atag} \textbf{dapat} isya-isya kiten. \\
*\textbf{Mag-atag} isya-isya kiten \textbf{dapat}.

\ex
Duma  an  uļa  en  nanangget  tak  \textbf{bawal}  \textbf{mag-inem}  ta  duma. \\\smallskip
 \gll Duma  an  uļa  en  na-ng-sangget  tak  \textbf{bawal}  \textbf{mag-inem}  ta  duma. \\
some  \textsc{def.m}  \textsc{neg.r}  \textsc{cm}  \textsc{a.hap.r}-\textsc{pl}-sickle.for.coconut.sap  because  forbidden  \textsc{i.ir}-drink  \textsc{nabs}  some \\
\glt `Some did not gather coconut sap because (it is) \textbf{forbidden} for some \textbf{to} \textbf{drink} (coconut wine).’ [JCWN-T-21 13.3] \\\smallskip

*… \textbf{mag-inem} \textbf{bawal} ta duma. \\
*… \textbf{mag-inem} ta duma \textbf{bawal}.
\ex
Kanen  \textbf{pwidi}  din  man  ittaw  i  \textbf{maimo}  na  pawikan. \\\smallskip
 \gll Kanen  \textbf{pwidi}  din  man  ittaw  i  \textbf{ma-imo}  na  pawikan. \\
3\textsc{s.abs}  can  3\textsc{s.erg}  too  person  \textsc{def.n}  \textsc{a.}\textsc{hap}\textsc{.ir}-make/do  \textsc{lk}  sea.turtle \\
\glt “As for him, he \textbf{can} too \textbf{make} the person into a sea turtle.’ [MBON-T-07a 6.2] \\
*Kanen \textbf{maimo} din man \textbf{pwidi} ittaw i na pawikan. \\
*Kanen \textbf{maimo} din man ittaw i na pawikan \textbf{pwidi}.
\z


\newpage
\ex Class b: Punctuality adverbs
\ea
Ta  iya  na  paglaya  kanen  \textbf{gulpi}  nang  \textbf{naruwad}. \\\smallskip
\gll Ta  iya  na  pag-laya  kanen  \textbf{gulpi}  nang  \textbf{na-duwad}. \\
\textsc{nabs}  3\textsc{s.gen}  \textsc{lk}  \textsc{nr.act}-cast.net  3\textsc{s.abs}  suddenly  just/only  \textsc{a.hap.r}-disappear \\
\glt ‘While he was cast net fishing he \textbf{suddenly} \textbf{disappeared}.’ [VAWN-T-17 2.3] \\\smallskip

*Ta iya na paglaya \textbf{narwad/duwad} \textbf{gulpi} nang kanen an. \\
*Ta iya na paglaya \textbf{narwad/duwad} kanen \textbf{gulpi} nang.
\ex
\label{bkm:Ref480616181}
\textbf{Nali}  \textbf{nang}  \textbf{nawigit}  ake  na  manilya. \\\smallskip
 \gll \textbf{Nali}  \textbf{nang}  \textbf{na-wigit}  ake  na  manilya. \\
abruptly  just  \textsc{a.hap.r}-fall.without.notice  1\textsc{s.gen}  \textsc{lk}  bracelett \\
\glt \textbf{‘Abruptly} my bracelet \textbf{fell} without noticing it.’ [BMON-C-05 7.5] \\
*\textbf{Nawigit/wigit} \textbf{nali} nang ake na manilya. \\
*\textbf{Nawigit/wigit} ake na manilya \textbf{nali} nang.
\ex
… \textbf{listo}  \textbf{kamang}  Pedro  an  ta  pana. \\\smallskip
 \gll … \textbf{listo}  \textbf{kamang}  Pedro  an  ta  pana. \\
{}  promptly  get    Pedro  \textsc{def.m}  \textsc{nabs}  spear \\
\glt ‘Pedro \textbf{promptly} \textbf{got} the spear.’ [CBON-C-02 3.1] \\\smallskip

*\textbf{Gakamang/kamang} \textbf{listo} Pedro an ta pana. \\
*\textbf{Gakamang/kamang} Pedro an ta pana \textbf{listo}.
\z


\ex Class c: Durational manner adverbs
\ea
\textbf{Diritso}  ko  \textbf{pas-an}  baboy  ya  dļaen ta    baļay. \\\smallskip
 \gll \textbf{Diritso}  ko  \textbf{pas-an}  baboy  ya  daļa-en ta    baļay. \\
straight.away  1\textsc{s.erg}  carry.on.shoulder  pig  \textsc{def.f}  take-\textsc{t.ir} \textsc{nabs}   house \\
\glt ‘I \textbf{straight} \textbf{away} \textbf{carried} the pig on my shoulder taking (it) to (my) house.’ [RCON-L-01 9.5] \\\smallskip

\textbf{*Papas-an/pas-an} ko \textbf{diritso} baboy ya dļaen ta baļay. \\
\textbf{*Papas-an/pas-an} ko baboy ya \textbf{diritso} dļaen ta baļay.

\newpage

\ex
Na  naan  kami  i  ta  lansa  ya  \textbf{sigi}  kami \textbf{agaļ}  tak  adlek  kami  na  patayen  daw  \textbf{sigi}  pa  na  \textbf{uran}. \\\smallskip
 \gll Na  naan  kami  i  ta  lansa  ya  \textbf{sigi}  kami \textbf{agaļ}  tak  adlek  kami  na  patay-en  daw  \textbf{sigi}  pa  na  \textbf{uran}. \\
\textsc{lk}  \textsc{spat.def} 1\textsc{p.excl.abs}  \textsc{def.n}  \textsc{nabs}  launch  \textsc{def.f}  continual  1\textsc{p.excl.abs}
cry  because  afraid  1\textsc{p.excl.abs}  \textsc{lk}  dead-\textsc{t.ir}  and  continual  \textsc{inc}  \textsc{lk}  rain \\
\glt ‘When we were on the launch we \textbf{continually} \textbf{were} \textbf{crying} because we were afraid that we will be killed and (it was) \textbf{continually} raining still.’ (The people in this story are hiding on a launch from pirates that are attacking them.) [BCWN-C-04 4.4] \\\smallskip

*\textbf{gaga}ļ\textbf{/aga}ļ \textbf{sigi} kami \\
*gagaļ/agaļ kami \textbf{sigi} \\
*\textbf{guran/uran} pa (na) \textbf{sigi} \\
*\textbf{guran/uran} (na) \textbf{sigi} pa
\z
\z

We consider \textit{sali ta} ‘persistently’ to be a compound adverb. The word \textit{sali} is usually followed by the non-absolutive marker \textit{ta} and then the unaffixed verb stem.

\ea
Tak  primiro  ko  nang  pa  sakay  ta  kalisa  sigi a  tawa-tawa  na  nakita  ko  kabayo  an  na  \textbf{sali} \textbf{ta}  \textbf{anges}  daw  gabuļa  iya  na  baba. \\\smallskip
 \gll Tak  primiro  ko  nang  pa  sakay  ta  kalisa  sigi a  tawa\sim{}-tawa  na  na-kita  ko  kabayo  an  na  \textbf{sali} \textbf{ta}  \textbf{anges}  daw  ga-buļa  iya  na  baba. \\
because  first.time  1\textsc{s.gen}  only/just  \textsc{inc}  ride  \textsc{nabs}  horse.cart  continual
1\textsc{s.abs}  \textsc{red}\sim{}laugh  \textsc{lk}  \textsc{a.hap.r}-see  1\textsc{s.erg}  horse  \textsc{def.m}  \textsc{lk}  persistently
\textsc{nabs}  breathe.hard  and  \textsc{i.r}-bubbles  3\textsc{s.gen}  \textsc{lk}  mouth \\
\glt ‘Because (it was) my first time still to ride a horse cart, I continually was laughing when I saw the horse that was \textbf{persistently} \textbf{breathing} \textbf{hard} and his/her mouth was foaming.’ [DBWN-L-23 5.10] \\\smallskip

\textbf{*ganges/anges} \textbf{ta} \textbf{sali}
\z
\ea
\label{bkm:Ref113431626}
Dili  ka  \textbf{sali}  \textbf{ta}  \textbf{panaw}  daw  kilem.  Dili  ka  \textbf{sali} \textbf{ta}  \textbf{ingaw-ingaw}. \\\smallskip
 \gll Dili  ka  \textbf{sali}  \textbf{ta}  \textbf{panaw}  daw  kilem.  Dili  ka  \textbf{sali} \textbf{ta}  \textbf{ingaw\sim{}-ingaw}. \\
\textsc{neg.ir}  2\textsc{s.abs}  persistently  \textsc{nabs}  go/walk  if/when  night  \textsc{neg.ir}  2\textsc{s.abs}  persistently \textsc{nabs}  \textsc{red}\sim{}drunk \\
\glt `Do not \textbf{keep} \textbf{going} (places) at night. Do not \textbf{keep} \textbf{being} \textbf{drunk}.’ [JCOE-T-06 15.8]
\z
Class d: Velocity adverbs
\ea
\label{bkm:Ref113431770}
Lugar  na  gasakay  kay  traysikil,  \textbf{sikad}  \textbf{dasig}  \textbf{panaw}  \textbf{din}. \\\smallskip
 \gll Lugar  na  gasakay  kay  traysikil,  \textbf{sikad}  \textbf{dasig}  \textbf{panaw}  \textbf{din}. \\
then  \textsc{lk}  \textsc{i.r}-ride  1\textsc{p.excl.abs}  tricycle  very  fast  go/walk  3\textsc{s.gen} \\
\glt ‘Then when we were riding the tricycle, his \textbf{going} \textbf{was} \textbf{very} \textbf{fast}.’ (The driver was driving very fast in a motorcycle pedicab, called a tricycle in the Philippines.) [BMON-C-05 7.5] \\\smallskip

*\textbf{gapanaw/panaw} kanen \textbf{sikad} \textbf{dasig}. \\
*\textbf{panaw} din \textbf{sikad} \textbf{dasig}.
\z
\ea
\textbf{Sikad}  \textbf{daļas}  {dļagan}  \textbf{ta}  \textbf{bļangay  an}  tak  pelles  angin  an. \\\smallskip
 \gll \textbf{Sikad}  \textbf{daļas}  {dļagan}  \textbf{ta}  \textbf{bļangay  an}  tak  pelles  angin  an. \\
very  swift  run  \textsc{nabs}  2.masted.boat  \textsc{def.m}  because  strong.wind  wind  \textsc{def.m} \\
\glt ‘The \textbf{running} of the two-masted boat was very \textbf{swift} because the wind was strong.’ [JCON-L-07 3.3]
\z
\ea
\label{bkm:Ref481732330}
\textbf{Sikad}  \textbf{iya}  \textbf{na}  \textbf{singgit}  na, … \\\smallskip
 \gll \textbf{Sikad}  \textbf{iya}  \textbf{na}  \textbf{singgit}  na, … \\
very  3\textsc{s.gen}  \textsc{lk}  shout  \textsc{lk} \\
\glt ‘His \textbf{shouting} was \textbf{loud}, …  [DBWN-T-32 2.10]
\z
\ea
Ta  pag-iling  nay  ta  Duļļo  \textbf{dali}  \textbf{nang}  en  \textbf{magsaļep} adlaw  an. \\\smallskip
 \gll Ta  pag-iling  nay  ta  Duļļo  \textbf{dali}  \textbf{nang}  en  \textbf{mag-saļep} adlaw  an. \\
\textsc{nabs}  \textsc{nr.act}-go  1\textsc{p.excl.erg}  \textsc{nabs}  Duļļo  soon  just  \textsc{cm}  \textsc{i.ir}-set un/day  \textsc{def.m} \\
\glt ‘When we were going to Dullo, the sun was \textbf{soon} \textbf{to} \textbf{set}.’ [DBWN-T-24 4.1] \\\smallskip

*\textbf{magsaļep dali} nang en adlaw an. \\
*\textbf{magsaļep} adlaw an \textbf{dali} nang en. \\
?dali nang en pagsaļep ta adlaw \\
?dali nang en saļep adlaw an
\z
Class e: “General” adverbs
\ea
\textbf{Tise}  a  nang  \textbf{lemmes}  ta  dagat  naan  ta  Iba … \\\smallskip
 \gll \textbf{Tise}  a  nang  \textbf{lemmes}  ta  dagat  naan  ta  Iba … \\
almost  1\textsc{s.abs}  just  drown  \textsc{nabs}  sea  \textsc{spat.def}  \textsc{nabs}  Iba \\
\glt ‘I \textbf{almost} \textbf{drowned} in the sea at Iba …’ [VAWN-T-15 6.14]
\z
\ea
\textbf{Tise}  a  nang  \textbf{nalemmes}. \\\smallskip
 \gll \textbf{Tise}  a  nang  \textbf{na-lemmes}. \\
almost  1\textsc{s.abs}  just  \textsc{a.hap.r}-drown \\
\glt`I \textbf{almost} \textbf{happened} \textbf{to} \textbf{drown}.'
\z
\ea
\textbf{Tise}   a  nang \textbf{malemmes} \\\smallskip
 \gll \textbf{Tise}   a  nang \textbf{ma-lemmes} \\
almost  1\textsc{s.abs}  just  \textsc{a.hap.ir}-drown \\
\glt ‘I am \textbf{about} \textbf{to} \textbf{drown}.’ \\\smallskip

*Yaken \textbf{nalemmes/lemmes} ta dagat \textbf{tise} \textbf{nang} naan ta Iba. \\
*Yaken \textbf{nalemmes/lemmes} ta dagat naan ta Iba \textbf{tise} \textbf{nang}. \\
\textit{*}Yaken \textbf{nalemmes/lemmes} \textbf{tise} \textbf{nang} ta dagat. \\
*\textbf{Nalemmes/lemmes} a \textbf{tise} \textbf{nang} ta dagat.
\z
\ea
Yi  na  magsawa  \textbf{lugay}  \textbf{maatagan}  ta  bata. \\\smallskip
 \gll Yi  na  mag-sawa  \textbf{lugay}  \textbf{ma-atag-an}  ta  bata. \\
\textsc{d}1\textsc{adj}  \textsc{lk}  \textsc{rel}-spouse  long.time  \textsc{a.hap.ir}-give-\textsc{apl}  \textsc{nabs}  child \\
\glt ‘This married couple, (it took) \textbf{a} \textbf{long} \textbf{time} for a child \textbf{to} \textbf{be} \textbf{given} to them.’ [YBWN-T-01 2.2] \\\smallskip

*\textbf{maatagan} danen \textbf{lugay} ta bata. \\
*\textbf{maatagan} \textbf{lugay} danen an ta bata. \\
*\textbf{maatagan} danen ta bata \textbf{lugay}.
\z
\ea
\textbf{Lugay}  man  mende  \textbf{tagtagad}  nay  sakayan  gina ta  Maranan  ya … \\\smallskip
 \gll \textbf{Lugay}  man  mende  \textbf{tag-tagad}  nay  sakayan  gina ta  Maranan  ya … \\
long.time  too  1\textsc{p.excl.gen}  \textsc{red}-wait  1\textsc{p.excl.erg}  vehicle  earlier
\textsc{nabs}  Maranan  \textsc{def.f} \\
\glt ‘We ourselves waited \textbf{a} \textbf{long} \textbf{time} for the vehicle to Maranan ….’ [BGON-L-01 3.21] \\\smallskip

*\textbf{tagtagad} nay sakayan an \textbf{lugay} ta Maranan \\
\textbf{*tagtagad} nay \textbf{lugay} sakayan an ta Maranan
\z

\ea
\textbf{Uļa  lugay}  \textbf{napanno}  en  libon  nay  an. \\\smallskip
 \gll \textbf{Uļa}  \textbf{lugay}  \textbf{na-panno}  en  libon  nay  an. \\
\textsc{neg.r}  long.time  \textsc{a.hap.r}-full  \textsc{cm}  basket  1\textsc{p.excl.gen}  \textsc{def.m} \\
\glt ‘Our basket was \textbf{full} \textbf{in} \textbf{not} \textbf{a} \textbf{long} \textbf{time}.’ [JCWN-L-38 15.5] \\\smallskip

\textbf{*napanno} en libon nay an \textbf{uļa lugay}. \\
\textbf{*napanno} en \textbf{uļa lugay} libon nay an.
\z

Most non-canonical adjunct adverbs in classes b through e occur in two constructions. One construction is illustrated in examples \REF{bkm:Ref480616181}{}--\REF{bkm:Ref113431626} the adverb occurs clause-initially and is followed by the main verb that is often unaffixed. Any second-position elements, such as enclitic pronouns and adverbs, are attracted to the position after the adverb. Referring Phrases remain in their post-verbal positions in their normal Absolutive/Ergative case roles, depending on the transitivity of the verb. In this construction, the main verb describes an event that is on the main storyline of a narrative.

In the second construction, the pre-verbal adverb is followed by a nominalized clause. The nominalized verb is usually unaffixed, but may take the \textit{pag}{}- action nominalization prefix (see \chapref{chap:referringexpressions}, \sectref{sec:pag} and \sectref{sec:actionnominalizations}). It is clear that the verb is nominalized because a genitive pronoun, enclitic or full RP refers to the Actor, whether the verb is transitive or not. Also, in examples such as \REF{bkm:Ref329771116} a genitive pronoun precedes the verb with the linker \textit{na} intervening, which is a common structure for RPs with possessors. This construction is illustrated in exx. \REF{bkm:Ref113431770}{}-\REF{bkm:Ref481732330} and \REF{bkm:Ref329771116} through \REF{bkm:Ref481737967}:

\largerpage
\ea
\label{bkm:Ref329771116}
Genitive pronoun + transitive verb \\
\textbf{Sigi}  nang  iya  na  kaan. \\\smallskip
 \gll \textbf{Sigi}  nang  iya  na  kaan. \\
continual  just  3\textsc{s.gen}  \textsc{lk}  eat \\
\glt ‘His/her eating is/was just \textbf{continual}.’
\z
\ea
Transitive verb + genitive enclitic \\
\textbf{Sigi}  kaan  din  an. \\\smallskip
\gll \textbf{Sigi}  kaan  din  an. \\
continual  eat  3\textsc{s.gen}  \textsc{def.m} \\
\glt ‘His/her eating is/was \textbf{continual}.’
\z
\ea
Transitive verb + genitive RP \\
\textbf{Sigi}  kaan  ta  bata. \\\smallskip
\gll \textbf{Sigi}  kaan  ta  bata. \\
continual  eat  \textsc{nabs}  child \\
\glt ‘The child’s eating was \textbf{continual}.’
\z
\ea
Intransitive verb + genitive enclitic \\
\textbf{Sigi}  agaļ  ko  an. \\\smallskip
 \gll \textbf{Sigi}  agaļ  ko  an. \\
continual  cry  1\textsc{s.gen}  \textsc{dem.m} \\
\glt ‘My crying was continual.’ [VAWN-T-16 2.16]
\z
\ea
Intransitive verb + genitive enclitic \\
\textbf{Sigi}  en  indis  din  an. \\\smallskip
\gll \textbf{Sigi}  en  indis  din  an. \\
continual  \textsc{cm}  defecate  3\textsc{s.gen}  \textsc{def.m} \\
\glt ‘His defecating was continual.’ [JCWN-T-21 4.4]
\z

This construction occurs more often in background, descriptive material. Following is a selection of examples from the text corpus.

\ea
Sanglit  Pedro  ya  \textbf{sigi}  \textbf{istadi}  \textbf{din}  an  daw  kilem  daw kaisa  mga  alas  unsi  en  kanen  manuga. \\\smallskip
 \gll Sanglit  Pedro  ya  \textbf{sigi}  \textbf{istadi}  \textbf{din}  an  daw  kilem  daw kaisa  mga  alas  unsi  en  kanen  m-tanuga. \\
because  Pedro  \textsc{def.f}  continual  study  3\textsc{s.gen}  \textsc{def.m}  if/when  night  and
sometimes  \textsc{pl}  o’clock  eleven  \textsc{cm}  3\textsc{s.abs}  \textsc{i.v.ir}-sleep \\
\glt `Because as for Pedro his \textbf{studying} \textbf{was} \textbf{continual} at night and sometimes around eleven o’clock he went to sleep.’ [LGON-L-01 7.2]
\z
\ea
Ta  iran  na  paglayag  pagayungan  man  danen  aged \textbf{dali}  \textbf{iran}  \textbf{na}  \textbf{pag-abot}  ta  iran  na  lugar. \\\smallskip
 \gll Ta  iran  na  pag-layag  pa-gayung-an  man  danen  aged \textbf{dali}  \textbf{iran}  \textbf{na}  \textbf{pag-abot}  ta  iran  na  lugar. \\
\textsc{nabs}  3\textsc{p.gen}  \textsc{lk}  \textsc{nr.act}-sail  \textsc{t.r}-paddle-\textsc{apl}  too  3\textsc{p.erg}  so.that soon  3\textsc{p.gen}  \textsc{lk}  \textsc{nr.act}-arrive  \textsc{nabs}  3\textsc{p.gen}  \textsc{lk}  place \\
\glt `While they were sailing they paddled too so that \textbf{their} \textbf{arriving} to their place would be \textbf{soon}.’ [DBWN-T-33 2.9]
\z
\ea
Na  gasin-ad  a  ta  lub-ong  \textbf{tudo}  \textbf{ake}  \textbf{na}  \textbf{dabok}… \\\smallskip
 \gll Na  ga-sin-ad  a  ta  lub-ong  \textbf{tudo}  \textbf{ake}  \textbf{na}  \textbf{dabok}… \\
\textsc{lk}  \textsc{i.r}-cooking.grain  1\textsc{s.abs}  \textsc{nabs}  cooked.cassava  intense  1\textsc{s.gen}  \textsc{lk}  stoke \\
\glt ‘When I was cooking cassava \textbf{my} \textbf{stoking} \textbf{(the} \textbf{fire)} \textbf{was} \textbf{very} \textbf{intense}…’ [ANWN-T-01 3.1]
\z

The adjective \textit{bakod} ‘big’ can be used in this construction as an adverb meaning ‘greatly’.
\ea
\label{bkm:Ref481737967}
Siguro,  \textbf{bakod}  \textbf{gid}  \textbf{imo}  \textbf{na}  \textbf{pagtingaļa}  na  man-o tak nļaman  ko  imo  na  ngaran. \\\smallskip
 \gll Siguro,  \textbf{bakod}  \textbf{gid}  \textbf{imo}  \textbf{na}  \textbf{pag-tingaļa}  na  man-o tak na-aļam-an  ko  imo  na  ngaran. \\
perhaps  big    \textsc{int}  2\textsc{s.gen}  \textsc{lk}  \textsc{nr.act}-wonder  \textsc{lk}  why
because \textsc{a.hap.r}-know-\textsc{apl}  1\textsc{s.erg}  2\textsc{s.gen}  \textsc{lk}  name \\
\glt ‘Perhaps \textbf{your} \textbf{wondering} \textbf{is} \textbf{great} why I know your name.’ [EMWL-T-04 5.2]
\z

In what may be considered a third construction, a linker occurs between the preverbal adverb and the unaffixed verb (exx. \ref{bkm:Ref425350651}{}-\ref{bkm:Ref425350655}). It seems that when the linker occurs the speaker is slowing down her or his speech and drawing out each word thus highlighting the verb and adverb. This can occur with the adverbs \textit{dapat}, \textit{kinanglan} and \textit{sigi}.
\ea
\label{bkm:Ref425350651}
Piro  \textbf{sigi}  kay  \textbf{na}  \textbf{agaļ}  tak  gakilem  en. \\\smallskip
 \gll Piro  \textbf{sigi}  kay  \textbf{na}  \textbf{agaļ}  tak  ga-kilem  en. \\
but  continual  1\textsc{p.excl.abs}  \textsc{lk}  cry  because  \textsc{i.r}-night  \textsc{cm} \\
\glt ‘But we \textbf{continually} \textbf{were} \textbf{crying} because (it was) already becoming night.’ [CBWN-C-11 4.9]
\z
\ea
\label{bkm:Ref425350655}
Piro  duma  an  \textbf{sigi}  pa  gid  \textbf{na}  \textbf{tanem}. \\\smallskip
 \gll Piro  duma  an  \textbf{sigi}  pa  gid  \textbf{na}  \textbf{tanem}. \\
but  other  \textsc{def.m} continual  \textsc{inc}  \textsc{int}  \textsc{lk}  plant \\
\glt`But others really \textbf{continually} \textbf{kept} \textbf{planting}.’ [ETOP-C-08 3.9]
\z

The following elicited examples illustrate that \textit{lugay} may be followed by a verb in any form, including a bare form. The English free translations are meant to give a sense of the differences in meaning expressed by the different constructions. Similar variation occurs with the other adverbs listed in \textsc{\REF{bkm:Ref480616640}}, though a full study of the syntactic behavior of all adverbs awaits future research:

\ea
\textbf{Lugay}  kay  gaisturyaay. \\\smallskip
 \gll \textbf{Lugay}  kay  ga-isturya-ay. \\
long.time  1\textsc{p.excl.abs}  \textsc{i.r}-talk-\textsc{rec} \\
\glt ‘We were/are conversing with each other for a long time.’
\z
\ea
\textbf{Lugay}  kay  isturyaay. \\\smallskip
 \gll \textbf{Lugay}  kay  isturya-ay. \\
long.time  1\textsc{p.excl.abs}  talk-\textsc{rec} \\
\glt ‘We (sometimes) converse with each other for a long time.’
\z
\ea
\textbf{Lugay}  kay  mag-isturyaay. \\\smallskip
 \gll \textbf{Lugay}  kay  mag-isturya-ay. \\
long.time  1\textsc{p.excl.abs}  \textsc{i.ir}-talk-\textsc{rec} \\
\glt ‘We will converse with each other for a long time.’
\z
\ea
\textbf{Uļa}  kay  \textbf{lugay}  gaisturyaay. \\\smallskip
 \gll \textbf{Uļa}  kay  \textbf{lugay}  ga-isturya-ay. \\
\textsc{neg.r}  1\textsc{p.excl.abs}  long.time  \textsc{i.r}-talk-\textsc{rec} \\
\glt ‘We conversed with each other \textbf{briefly}.’
\z


\ea
Uļa  kay  \textbf{lugay}  isturyaay. \\\smallskip
 \gll Uļa  kay  \textbf{lugay}  isturya-ay. \\
\textsc{neg.r}  1\textsc{p.excl.abs}  long.time  talk-\textsc{rec} \\
\glt ‘We (sometimes) converse with each other \textbf{briefly}.’
\z

\ea
Dili  kay  \textbf{lugay}  mag-isturyaay. \\\smallskip
 \gll Dili  kay  \textbf{lugay}  mag-isturya-ay. \\
\textsc{neg.ir}  1\textsc{p.excl.abs}  long.time  \textsc{i.ir}-talk-\textsc{rec} \\
\glt ‘We will not converse with each other for long.’
\is{adverbs!non-canonical adjunct|)}\is{non-canonical adjunct adverbs|)}
\z

\subsection{Second-position adverbs}
\label{bkm:Ref441599628} \label{bkm:Ref479858962} \label{sec:secondpositionadverbs}

The third group of adverbs consists of those that appear after the first major constituent in the clause. They are mostly monosyllabic, and all have Austronesian etymologies. Up to three of these adverbs may “stack” in second position, in which case there is a fairly strict order in which they may occur (see \tabref{tab:orderofsecondpositionadverbs} below). All second-position adverbs follow any second-position pronominal enclitics. Semantically, second-position adverbs tend to express aspectual and modal notions. Some of the English “translations” are impressionistic. Different speakers use them in different ways, and contextual factors often interact with their lexical meanings to produce nuances that are difficult to describe in a concise English gloss.

\ea
\label{bkm:Ref113436041}
\begin{tabbing}
\hspace{2.2cm} \= \kill
second-position adverbs \\
\textit{gid} \> ‘intensifier’ \\
\textit{pa} \> ‘still/yet/incompletive/emphasis’ \\
\textit{en} \> ‘now/completive’ \\
\textit{daan} \> ‘ahead of time, immediately’ \\
\textit{anay} \> ‘first/for a while’ (polite word) \\
\textit{dagli} \> ‘momentarily’ \\
\textit{nang} \> ‘only/just’ (downtoner) \\
\textit{kon} \> ‘hearsay’ \\
\textit{man} \> ‘also, emphasis/importance/prominence’ \\
\textit{imo} \> ‘attention getter, emphasis/importance’ \\
\textit{inta} \> ‘wishfully/unrealized intension’ \\
\textit{isab} \> ‘again’ (repetition of same activity) \\
\textit{eman} \> ‘again, just as before’ (resumption of earlier activity) \\
\textit{man gyapon} \> ‘just the same’ \\
\textit{tuo} \> ‘intention/please’ \\
\textit{taan} \> ‘perhaps’ (inferential) \\
\textit{gani} \> ‘therefore, truly, attention getter’ \\
\textit{abi/paryo abi} \> ‘for instance’ (softens requests/commands and states \\
\> what another may be thinking) \\
\textit{ba}\footnotemark{} \> ‘teasing, trying to get a reaction, irony/sarcasm’ \\
\textit{i, an, ya} \> ‘various emotions, exclamations’\footnotemark \\
\textit{paran} \> ‘perhaps/isn’t that so?’ (in rhetorical questions) \\
\textit{pļa/pļang} \> ‘unexpected/surprise’
\end{tabbing}
\footnotetext[23]{This looks like the yes/no question particle in \isi{Tagalog} and many other Philippine languages. Kagayanen marks yes/no questions with intonation (see \chapref{chap:pragmaticallymarkedstructures}, \sectref{sec:yesnoquestions}), and does not use a question particle. However, this form is used as a marker of rhetorical irony, as discussed further below and in \chapref{chap:pragmaticallymarkedstructures}, \sectref{sec:rhetoricalconfirmation}.} 
\footnotetext{These second-position adverbs are homophonous with the demonstrative determiners that occur in Referring Phrases (see \chapref{chap:referringexpressions}, \sectref{sec:definiteness}). As second-position adverbs, they express various strong emotions, such as ‘wow!’, ‘oh man!’, ‘what the heck?’ etc. See below for more explanation.}
\z

The following are some examples from the text corpus containing these second-position adverbs:

\ea
\textit{gid} ‘intensifier’ (twice), \textit{man} ‘emphasis’, and \textit{en} ‘completive’ \\
Nabatyagan  ko  \textbf{gid}  dya  na  uļa  \textbf{gid}  \textbf{man}  \textbf{en}  gasakit. \\\smallskip
 \gll Na-batyag-an  ko  \textbf{gid}  dya  na  uļa  \textbf{gid}  \textbf{man}  \textbf{en}  ga-sakit. \\
\textsc{a.hap.r}-feel-\textsc{apl}  1\textsc{s.erg}  \textsc{int}  \textsc{d}4\textsc{loc}  \textsc{lk}  \textsc{neg.r}  \textsc{int}  \textsc{emph}  \textsc{cm}  \textsc{i.r}-pain \\
\glt ‘I \textbf{really} felt that there was \textbf{really no more} pain.' [JCWN-T-22 8.13]
\z
\ea
\textit{nang} ‘just’/’only’ downtoner \\
Piro Maria  i  patandasan  \textbf{nang}  ta  mga  maļbaļ. \\\smallskip
 \gll Piro Maria  i  pa-tandas-an  \textbf{nang}  ta  mga  maļbaļ. \\
but  Maria  \textsc{def.n}  \textsc{t.r}-kick-\textsc{apl}  only  \textsc{nabs}  \textsc{pl}  witch \\
\glt ‘But the witches \textbf{just} kicked Maria.’ (So that she would not be eaten by the witches, the mother of Maria made her into a dog and she was lying under a chair where the witches were sitting with their feet kicking her.) [MBON-T-06 4.7] 
\z


\ea
Daw  ino  liag  no  na  kan-enen,  magsinyas  ka  \textbf{nang}  ki  danen. \\\smallskip
 \gll Daw  ino  liag  no  na  kan-en-en,  mag-sinyas  ka  \textbf{nang}  ki  danen. \\
if/when  what  like  2\textsc{s.erg}  \textsc{lk}  cooked.rice-\textsc{t.ir}  \textsc{i.ir}-signs  2\textsc{s.abs}  only  \textsc{obl.p}  3p \\
\glt ‘Whatever you want to eat \textbf{just} do hand signals to them.’ [VAWN-T-15 9.3]
\z
\ea
\textit{nang} ‘just’/’only’ downtoner, and \textit{en} ‘completive’ \\
Gusto  \textbf{nang}  \textbf{en}  ta    mga  manakem  mungko  \textbf{nang} daw  meļeb-eļeb,  tanod  ta  apo  an  daen. \\\smallskip
 \gll Gusto  \textbf{nang}  \textbf{en}  ta    mga  manakem  m-pungko  \textbf{nang} daw  m-eļeb-eļeb,  tanod  ta  apo  an  daen. \\
want  only  \textsc{cm}  \textsc{nabs}  \textsc{pl}  older  \textsc{i.v.ir}-sit  only
and  \textsc{i.v.ir}-\textsc{red}-slice.thin  watch  \textsc{nabs}  grandchild  \textsc{def.m}  3\textsc{p.gen} \\
\glt`The older people \textbf{only} want to \textbf{just} sit and keep on cutting up (cassava), watching their grandchildren.’ [RZWE-J-01 18.8]
\z

\ea
\textit{daan} ‘ahead of time’ \\
    Nasugat  danen  bata  i  ta  iran  na  maistro  na  ngaran Pedro  na  galin  Iloilo  na  naļam  \textbf{daan}  tak may gapaļam kanen. \\\smallskip
 \gll Na-sugat  danen  bata  i  ta  iran  na  maistro  na  ngaran Pedro  na  ga-alin  Iloilo  na  na-aļam  \textbf{daan}  tak may ga-pa-aļam {...}\footnotemark{} kanen. \\
        \textsc{a.hap.r}-meet  3\textsc{p.erg}  child  \textsc{def.n}  \textsc{nabs}  3\textsc{p.gen}  \textsc{lk}  teacher  \textsc{lk}  name  Pedro  \textsc{lk}  \textsc{i.r}-from  Iloillo  \textsc{lk}  \textsc{a.hap.r}-know  ahead.of.time  because  \textsc{ext.in} \textsc{i.r}-\textsc{caus}-know \textsc{obl.p} 3s \\
\footnotetext{The oblique, personal preposition \textit{ki} has dropped out before \textit{kanen} in this example. Such dropping of prepositions is not uncommon in conversation.}

\newpage
\glt `They happened to meet the son of their teacher whose name was Pedro coming from Iloilo who knew \textbf{already} \textbf{before} \textbf{this} because someone told him.’ [JCWN-T-20 21.2]
\z

\ea
\textit{anay} ‘first’ \\
    Mangugas  ka  \textbf{anay}  ta  lima  no  bag-o  magkaan. Mapunas  ka  \textbf{anay}  ta  mga  lawa  bag-o  manuga  daw  kilem. Daw  manaw  malimpyo        \textbf{anay}  ta  mga  lawa. \\\smallskip
 \gll Ma-ng-ugas  ka  \textbf{anay}  ta  lima  no  bag-o  mag-kaan. Ma-punas  ka  \textbf{anay}  ta  mga  lawa  bag-o  m-tanuga  daw  kilem. Daw  m-panaw  ma-limpyo        \textbf{anay}  ta  mga  lawa. \\
    \textsc{a.hap.ir}-\textsc{pl}-wash  2\textsc{s.abs}  first  \textsc{nabs}  hand  2\textsc{s.gen}  before  \textsc{i.ir}-eat       \textsc{a.hap.ir}-wipe  2\textsc{s.abs}  first  \textsc{nabs}  \textsc{pl}  body  before  \textsc{i.v.ir}-sleep  if/when  night if/when  \textsc{i.v.ir}-go/walk  \textsc{a.hap.ir}-clean  first  \textsc{nabs}  \textsc{pl}  body \\
    \glt `Wash your hands \textbf{first} before eating. Wipe your body \textbf{first} before sleeping when night. When going somewhere clean your body \textbf{first}.' [ETOP-C-10 2.6-8]
\z
\ea
\textit{nang} ‘just/only’ downtoner, \textit{anay} ‘for a while’ \\
Matinir  ka  \textbf{nang}  \textbf{anay}  dyan. \\\smallskip
 \gll Ma-tinir  ka  \textbf{nang}  \textbf{anay}  dyan. \\
\textsc{a.hap.ir}-stay  2\textsc{s.abs}  just  a.while  \textsc{d}2\textsc{loc} \\
\glt ‘\textbf{Just} stay there for \textbf{a} \textbf{while}.’ [JBWL-J-04 6.3]
\z
\ea
\textit{dagli} ‘momentarily’ \\
Miling  a  \textbf{dagli}  naan  baļay  Manang. \\\smallskip
 \gll M-iling  a  \textbf{dagli}  naan  baļay  Manang. \\
\textsc{i.v.ir}-go  1\textsc{s.abs}  monentarily  \textsc{spat.def}  house  Older.sister \\
\glt ‘I will go \textbf{momentarily} to Older Sister’s house.’
\z
\ea
\textit{kon} ‘hearsay’ \\
    Pambaļan  a  ta  mga  manakem  na  dili  a  \textbf{kon} maggwa tak  dilikado  \textbf{kon}  ta  kasaļen  na  sigi  panaw. \\\smallskip
 \gll Pa-ambaļ-an  a  ta  mga  manakem  na  dili  a  \textbf{kon} mag-gwa tak  dilikado  \textbf{kon}  ta  kasaļ-en  na  sigi  panaw. \\
        \textsc{t.r}-say-\textsc{apl}  1\textsc{s.abs}  \textsc{nabs}  \textsc{pl}  older.person  \textsc{lk}  \textsc{neg.ir}  1\textsc{s.abs}  \textsc{hsy}    \textsc{i.ir}-out because  dangerous  \textsc{hsy}  \textsc{nabs}  wedding-\textsc{nr}  \textsc{lk}  continuous  go/walk \\
    \glt `The older people warned me that I will not go out \textbf{they} \textbf{said} because it is dangerous \textbf{they} \textbf{said} for one(s) to be married to keep going somewhere.’ [AWN-T-16 2.7]
\z
\ea
\textit{man} ‘emphasis’, \textit{inta} ‘wish/intention unrealized’, \textit{en} ‘completive’, and  \textit{eman} ‘again as before’ \\
Pagtugpa  nay  ta    landingan  ta  Iloilo  gagwa  aren \textbf{man}  \textbf{inta}  \textbf{en}  daw  gulpi  \textbf{eman}  bellay  ake  na  ginawa. \\\smallskip
 \gll Pag-tugpa  nay  ta    landingan  ta  Iloilo  ga-gwa  aren \textbf{man}  \textbf{inta}  \textbf{en}  daw  gulpi  \textbf{eman}  bellay  ake  na  ginawa. \\
\textsc{nr.act}-land.on  1\textsc{p.excl.gen}  \textsc{nabs}  airstrip  \textsc{nabs}  Iloilo  \textsc{i.ir}-go.out  1\textsc{s.abs+cm}
\textsc{emph}  \textsc{opt}  \textsc{cm}  and  suddenly  again.as.before  difficult  1\textsc{s.gen}  \textsc{lk}  breathing \\
\glt `When we landed on the airstrip in Iloilo I was \textbf{going} to go out (of the plane) and my breathing was suddenly difficult \textbf{again} \textbf{like} \textbf{previously}.’ [JCWN-T-22 8.18]
\z
\ea
\textit{man} ‘emphasis’, \textit{isab} ‘again’ \\
Pagselled  ko  ta  iruplano  ya  dayon  aren \textbf{man}  \textbf{isab}  injection. \\\smallskip
 \gll Pag-selled  ko  ta  iruplano  ya  dayon  aren \textbf{man}  \textbf{isab}  injection. \\
\textsc{nr.act}-go.inside  1\textsc{s.gen}  \textsc{nabs}  airplane  \textsc{def.f}  right.away  1\textsc{s.abs+cm} \textsc{emph}  again  injection \\
\glt `When I went inside the airplane, then right away I was injected \textbf{again}.’ [JCWN-T-22 8.22]
\z
\ea
\textit{man gyapon}  ‘just the same’ \\
Piro  pangka  i  gabalik  \textbf{man  gyapon}  naan  ta  katri  din. \\\smallskip
 \gll Piro  pangka  i  ga-balik  \textbf{man  gyapon}  naan  ta  katri  din. \\
but  frog  \textsc{def.n}  \textsc{i.r}-return  just.the.same   \textsc{spat.def}  \textsc{nabs}  bed  3\textsc{s.gen} \\
\glt ‘But the frog returned \textbf{just} \textbf{the} \textbf{same} to her bed.’ (The girl found the frog in her bed and so she threw the frog out of the room but the frog returned anyway.) [CBWN-C-17 5.3] 
\z

The adverb \textit{imo} usually emphasizes excitement on the part of the speaker. It occurs as the final element in the second-position complex \REF{bkm:Ref329763033}{}-\REF{bkm:Ref443714404}, or final position \REF{bkm:Ref441331337}. It is homophonous with the free genitive/ergative second person singular pronoun, but makes no reference to a second singular referent, even if there is one elsewhere in the sentence.

\ea
\label{bkm:Ref329763033}
\label{bkm:Ref441331337}
Sugat  din  baked  na  kapri  \textbf{imo}. \\\smallskip
 \gll Sugat  din  baked  na  kapri  \textbf{imo}. \\
meet  3\textsc{s.erg}  big  \textsc{lk}  giant  \textsc{emph} \\
\glt ‘Wow!, s/he met a big giant.’ [CBON-T-03 3.2]
\z
\ea
Ambaļ  din,  “Indi  no  \textbf{imo}  kamanga  mga  bļawan  an  Pedro?” \\\smallskip
 \gll Ambaļ  din,  “Indi  no  \textbf{imo}  kamang-a  mga  bļawan  an  Pedro?” \\
say  3\textsc{s.erg}  where  2\textsc{s.erg}  \textsc{emph}  get-\textsc{xc.t}  \textsc{pl}  gold  \textsc{def.m}  Pedro \\
\glt ‘He said, “\textbf{Wow!} Where did you get the gold, Pedro?”‘ [CBWN-C-22 8.3]
\z
\ea
\label{bkm:Ref443714404}
Dili  ka  man  \textbf{imo}  magilek. \\\smallskip
 \gll Dili  ka  man  \textbf{imo}  ma-gilek. \\
\textsc{neg.ir}  2\textsc{s.abs}  \textsc{emph}  \textsc{emph}  \textsc{a.hap.ir}-angry \\
\glt ‘\textbf{Hey}, don’t get angry, man!’ [DBON-C-08 2.24]
\z
\textit{Imo} may also express intensity within an MP:
\ea
Tama  kay  \textbf{imo}  na  bļangay  dya  na  gapundo  dya ta  dawisan  an  na  gabantay  ta  timpo  kumo sali  ta  deļem-deļem. \\\smallskip
 \gll Tama  kay  \textbf{imo}  na  bļangay  dya  na  ga-pundo  dya ta  dawis-an  an  na  ga-bantay  ta  timpo  kumo sali  ta  deļem\sim{}-deļem. \\
many  1\textsc{p.excl.abs}  \textsc{emph}  \textsc{lk}  2.masted.boat  \textsc{d}4\textsc{loc}  \textsc{lk}  \textsc{i.r}-anchor  \textsc{d}4\textsc{loc}
\textsc{nabs}  point.of.island-\textsc{nr}  \textsc{def.m}  \textsc{lk}  \textsc{i.r}-watch  \textsc{nabs}  weather/season  because
persistently  \textsc{nabs}  \textsc{red}\sim{}dark \\
\glt `We were \textbf{\textsc{many}} ships that were anchored there at the point of the island watching the weather because it kept being kind of dark.’ [PCON-C-01 2.3]
\z

\textit{Imo} can be used as an attention getter to point out something of importance that a speaker wants others to notice. For example:
\ea
Pawa  \textbf{imo}  buļan  an. \\\smallskip
 \gll Pawa  \textbf{imo}  buļan  an. \\
bright  \textsc{emph}  moon  \textsc{def.m} \\
\glt ‘\textbf{Look}, the moon is bright!’
\z

\hspace*{-4.1pt}The adverbs \textit{anay} and \textit{tuo} indicate politeness. As mentioned earlier, \textit{anay} means `first' or `for a while'. This meaning is extended to politeness in requests and commands because it expresses the idea that the request is`only for a while'. However, in context \textit{anay} expresses many of the nuances of ‘please’ in English. \textit{Tuo} also indicates politeness, but is used more in statements. \textit{Tuo} may be somewhat archaic.


\ea
Mamati  kaw  \textbf{anay}  ki  kanen  na  mambaļ  ta  Kagayanen. \\\smallskip
 \gll Ma-mati  kaw  \textbf{anay}  ki  kanen  na  m-ambaļ  ta  Kagayanen. \\
\textsc{a.hap.ir}-hear  2\textsc{p.abs}  please  \textsc{obl.p}  3s  \textsc{lk}  \textsc{i.v.ir}-say  \textsc{nabs}  Kagayanen \\
\glt ‘Listen \textbf{please} to her speaking Kagayanen.’ [JCOE-T-06  17.2]
\z
\ea
Unso  ki  \textbf{tuo}  mag-isturya. \\\smallskip
 \gll Unso  ki  \textbf{tuo}  mag-isturya. \\
\textsc{d}4\textsc{loc.pr}  1\textsc{p.incl.abs}  please  \textsc{i.ir}-talk \\
\glt ‘Over there can we talk \textbf{please}.’ (This is grammatically a statement---not a question or imperative. The politeness adverb \textit{tuo} makes it a request rather than the literal “Over there we will talk,” which would be considered rude.)
\z
\ea
Umawen  no  \textbf{tuo}  bata  ko  ya  na  gaampang. \\\smallskip
 \gll Umaw-en  no  \textbf{tuo}  bata  ko  ya  na  ga-ampang. \\
call-\textsc{t.ir}  2\textsc{s.erg}  please  child  2\textsc{s.gen}  \textsc{def.f}  \textsc{lk}  \textsc{i.r}-play \\
\glt \textbf{‘Please} call my child who is playing.’
\z
\ea
\textit{taan} ‘perhaps’ (inference based on some evidence) \\
Manaw  \textbf{taan}  kani  danen  an  na  magnakaw. \\\smallskip
 \gll M-panaw  \textbf{taan}  kani  danen  an  na  mag-nakaw. \\
\textsc{i.v.ir}-go/walk  perhaps  later  3\textsc{p.abs}  \textsc{def.m}  \textsc{lk}  \textsc{i.ir}-steal \\
\glt ‘They will go later \textbf{perhaps} to steal.’ [CBWN-C-18 7.19]
\z

\ea
\label{bkm:Ref113434139}
\textit{gani} ‘truly’ /`in fact' (emphasis especially to correct another’s mistaken thinking or understanding) \\
Galayas  \textbf{gani}  danen  Cagayan Sulu  tak  adlek  na  tulien. \\\smallskip

\gll Ga-layas  \textbf{gani}  danen  Cagayan Sulu  tak  adlek  na  tuli-en. \\
\textsc{i.r}-flee  truly  3\textsc{p.abs}  Cagayan Sulu  because  fear  \textsc{lk}  circumcision-\textsc{t.ir} \\
\glt ‘They \textbf{in fact} fled from Cagayan Sulu because of being afraid of being circumcised.’ [MOOE-C-01 17.1]
\z

Example \REF{bkm:Ref113434139} is from a conversation in which the previous speaker asked where the people from Cagayan Sulu were headed, assuming they had a definite destination in mind. But the truth was they fled without really having any specific place they were trying to get to. So the speaker of this sentence is correcting a mistaken assumption.


\ea
\label{bkm:Ref113434330}
Ti,  daw  mamatian  nyo    \textbf{gani}  en  singgit  an  ta Pwikan, mabat  kaw  en  na  ‘Anen  a  duti.’ \\\smallskip
 \gll Ti,  daw  ma-mati-an  nyo    \textbf{gani}  en  singgit  an  ta Pwikan, m-sabat  kaw  en  na  ``Anen  a  duti". \\
        so  if/when  \textsc{a.hap.ir}-hear-\textsc{apl}  2\textsc{p.erg}  truly  \textsc{cm}  shout  \textsc{def.m}  \textsc{nabs}      sea.turtle \textsc{i.v.ir}-reply  2\textsc{p.abs}  \textsc{cm}  \textsc{lk}  \textsc{ext.g}  1\textsc{s.abs}  \textsc{d}1\textsc{loc.pr} \\
    \glt `So, if you \textbf{really} hear the shout of the Turtle, you reply, “I am here”.' [JCON-T-08 24.2]
\z

Example \REF{bkm:Ref113434330} is from a story about a race between a sea turtle and a hermit crab. The hermit crab is giving instructions to his fellow hermit crabs to distribute themselves along the race course. His instructions are interrupted by a digression about how proud the sea turtle is about being so fast. After the digression when he returns to his instructions about what his fellow crabs will do, he uses the word ‘\textit{gani}’ in order to get their attention focused back to the main point of what he is saying which is what they will do during the race to help him win.

In example \REF{bkm:Ref441345768}, \textit{abi} functions as ‘for instance’, but in examples \REF{bkm:Ref441345813}{}-\REF{bkm:Ref441514069} it functions to soften a request for another person to do something. It can also indicate what other people might be thinking as in \REF{bkm:Ref441514118}.

\ea
\label{bkm:Ref441345768}
Daw  dirse  pa  \textbf{abi}  bataan  din  an, tama  na  prublima  na  gabot  nanay  daw  tatay \\\smallskip
 \gll Daw  dirse  pa  \textbf{abi}  bata-an  din  an, tama  na  prublima  na  ga-abot  nanay  daw  tatay. \\
if/when  small.\textsc{pl}  \textsc{emph}  for.instance  child-\textsc{nr}  3\textsc{s.gen}  \textsc{def.m}
many  \textsc{lk}  problem  \textsc{lk}  \textsc{i.r}-arrive  mother  and  father \\
\glt `If/when his children are still small \textbf{for} \textbf{instance} the mother and father have many problems that arrive.’ [RZWE-J-01 16.11]
\z
\ea
\label{bkm:Ref441345813}
Matian  ta  \textbf{abi}  daw  matuod  na  dayad  kagi no  an  daw  manta  ka. \\\smallskip
 \gll \emptyset{}-Mati-an  ta  \textbf{abi}  daw  matuod  na  dayad  kagi no  an  daw  m-kanta  ka. \\
\textsc{t.ir}-hear-\textsc{apl}  1\textsc{p.incl.erg}  for.instance  if/when  true  \textsc{lk}  good  voice  2\textsc{s.gen}  \textsc{def.m}  if/when  \textsc{i.v.ir}-sing  2\textsc{s.abs} \\
\glt `Let’s listen \textbf{for} \textbf{instance} if it is true that your voice is good when you sing.’ (This is an indirect way to ask the person to sing.)
\z


\ea
\label{bkm:Ref441514069}
Sugid  no  \textbf{abi}. \\\smallskip
 \gll \emptyset{}-Sugid  no  \textbf{abi}. \\
\textsc{t.ir}-tell  2\textsc{s.erg}  for.instance \\
\glt \textbf{‘Please} tell for instance.' [JCON-T-08/ 21.2]
\z
\ea
\label{bkm:Ref441514118}
Pirmi  ngasebe  tak  uļa  \textbf{abi}  ittaw. \\\smallskip
 \gll Pirmi  ngasebe  tak  uļa  \textbf{abi}  ittaw. \\
always  sad  because  \textsc{neg.r}  for.instance  person \\
\glt ‘(The previously mentioned two people) were always sad because they \textbf{thought} there are no people.’ (This is a story about the first people to arrive on Cagayancillo and were lonely because there was no-one else around.) [MOOE-C-01 22.4] 
\z

The second-position adverb \textit{ba} is used as a marker of rhetorical irony (see \chapref{chap:pragmaticallymarkedstructures}, \sectref{sec:rhetoricalconfirmation}rhetoricalconfirmation. In Kagayanen, \textit{ba} implies the speaker is trying to get the addressee to react somehow by gently teasing or joking. It can express irony and sarcasm---saying one thing but meaning the opposite. \textit{Sikad dayaw ba.} ‘Wow how beautiful!’ (teasing). It can also imply real or false admiration.

\ea
Bellay   \textbf{ba}  bag-o  utod  buok  din  an. \\\smallskip
 \gll Bellay   \textbf{ba}  bag-o  utod  buok  din  an. \\
difficult  \textsc{irn}  newly  cut  hair  3\textsc{s.gen}  \textsc{def.m} \\
\glt ‘Oh so difficult, his/her hair is newly cut.’ (The speaker is referring to the hearer’s hair. Many times in irony the third person is used in place of second person.)
\z

The second-position adverbs \textit{i}, \textit{an}, and \textit{ya} express intensity of attitude and emotion of the speaker, and give emphasis as exclamations \REF{bkm:Ref441519758}{}-\REF{bkm:Ref441519766}. They are homophonous with the demonstrative determiners, but distribute as second-position adverbs (see \chapref{chap:referringexpressions}, \sectref{sec:definiteness} on the demonstrative determiners). They seem to retain the deictic properties of the determiners, however, insofar as \textit{i} ‘near’ is used to describe situations close to the speaker \REF{bkm:Ref441519758}-\REF{bkm:Ref52607199}, whereas \textit{an} \REF{bkm:Ref52607927}{}-\REF{bkm:Ref52607929} and \textit{ya} describe situations more distant from the speaker \REF{bkm:Ref52607977}{}-\REF{bkm:Ref52607980}:

\ea
\label{bkm:Ref441519758}
Sakit  man  \textbf{i}    gettek  ko  i  a! \\\smallskip
 \gll Sakit  man  \textbf{i}    gettek  ko  i  a! \\
continual  \textsc{emph}  \textsc{att}  stomach  1\textsc{s.gen}  \textsc{def.n}  \textsc{inj} \\
\glt ‘Oh, my stomach is continual!’ [PBON-T-01 3.8]
\z
\ea
\label{bkm:Ref52607199}
Miling  a  \textbf{i}  ta  pari  ko  ya! \\\smallskip
 \gll M-iling  a  \textbf{i}  ta  pari  ko  ya! \\
\textsc{i.v.ir}-go  1\textsc{s.abs}  \textsc{att}  \textsc{nabs}  friend  1\textsc{s.gen}  \textsc{def.f} \\
\glt ‘Man, I am going to my friend!’ [RBWN-T-02 5.1]
\z

Example \REF{bkm:Ref52607199} expresses strong emotion because the speaker’s friend did not help him when he really needed help. So he is going to his friend to break the friendship. Just as the English interjections ‘man’ or ‘oh!’ are used to express strong emotion, \textit{i}, \textit{an}, and \textit{ya} as second-position adverbs in Kagayanen express similar, but generic, strong emotions.
\ea
\label{bkm:Ref52607927}
Daw  sikad  man  \textbf{an}  inog  nangka  i,  lugay makamang  iya  an  na  bao. \\\smallskip
 \gll Daw  sikad  man  \textbf{an}  inog  nangka  i,  lugay ma-kamang  iya  an  na  bao. \\
if/when  very  \textsc{emph}  \textsc{att}  ripe  jackfruit  \textsc{def.n}  long.time
\textsc{a.hap.ir}-remove  3\textsc{s.gen}  \textsc{def.m}  \textsc{lk}  odor \\
\glt `\textbf{Man/oh} \textbf{boy} when the jackfruit is very ripe, it takes a long time for the odor to be removed.’ [VPWE-T-02 2.2]
\z
\ea
\label{bkm:Ref52607929}
Paryo  abi  ta  “kaļaw”  an  kan-o  yan  dili gid  \textbf{an}  masuļat. \\\smallskip
 \gll Paryo  abi  ta  “kaļaw”  an  kan-o  yan  dili gid  \textbf{an}  ma-suļat. \\
like  for.instance  \textsc{nabs}  winnowing.basket  \textsc{def.m}  previously  \textsc{d}2\textsc{abs}  \textsc{neg.ir}
\textsc{int}  \textsc{att}  \textsc{a.hap.ir-}write \\
\glt `Like for instance “winnowing basket”, previously that one, \textbf{oh} \textbf{man}, really could not be written!’ (The word \textit{kaļaw} was hard to write since it has the Kagayanen interdental approximant /ð̞/ which has no symbol in the major Philippine languages or in English.) [MOOE-C-01 33.1]
\z
\ea
\label{bkm:Ref52607977}
Piro,  daw  kino  man  \textbf{ya}  na  gatunuga  en  ta  oras  ta ani, yon  na  bata  mga  gimo  ta  kagayyaan. \\\smallskip
\gll Piro,  daw  kino  man  \textbf{ya}  na  ga-tunuga  en  ta  oras  ta ani, yon  na  bata  mga  ga-imo  ta  ka-gayya-an. \\
but  if/when  who  \textsc{emph}  \textsc{att}  \textsc{lk}  \textsc{i.r}-sleep  \textsc{cm}  \textsc{nab}\textsc{s}  time/hour  \textsc{nabs}  harvest
\textsc{d3adj}  \textsc{lk}  child  \textsc{pl}  \textsc{i.r}-do/make  \textsc{nabs}  \textsc{nr}-shame-\textsc{nr} \\
\glt `Whoever the \textbf{heck} sleeps at the time of harvest, those children\footnote{The expression \textit{yon na bata} ‘those children’ is a figure of speech used in didactic illustrations and proverbs. It does not refer literally to children.} are the ones doing the most shameful thing.’ [JCOB-L-02 7.5]
\z
\ea
Uļa  a  naļam  lain  pa  \textbf{ya}  agian  ko  ya en. \\\smallskip
 \gll Uļa  a  na-aļam  lain  pa  \textbf{ya}  agi-an  ko  ya en. \\
     \textsc{neg.r}  1\textsc{s.abs}  \textsc{a.hap.r}-know  different  \textsc{emph}  \textsc{att}  pass-\textsc{nr}  1\textsc{s.gen}  \textsc{def.f} \textsc{cm} \\
\glt ‘I did not know that my path now/then was \textbf{\textsc{still}} different (from the path taken earlier to get there).’ [DBON-C-08 2.6]
\z
\ea
\label{bkm:Ref441519766}\label{bkm:Ref52607980}
Man-o  tak  nabui  saging  no  ya  ta  uļa  \textbf{ya}  daon? \\\smallskip
 \gll Man-o  tak  na-bui  saging  no  ya  ta  uļa  \textbf{ya}  daon? \\
why  because  \textsc{a.hap.r}-live  banana  2\textsc{s.gen}  \textsc{def.f}  \textsc{nabs}  \textsc{neg.r}  \textsc{att}  leaf \\
\glt ‘Why the \textbf{heck} is your banana plant living when there are no leaves!?’ (The speaker is angry because his banana plant has withered. The speaker planted the top of a plant having leaves but the addressee planted the trunk without leaves and his is growing well. The mistaken expectation is that the top with leaves should grow rather than the trunk alone.) [CBWN-C-16 6.10]
\z

The second-position adverb \textit{paran} ‘perhaps’ occurs in questions to indicate that the speaker is not sure about what he is asking about and wants the addressee to give her opinion \REF{bkm:Ref441522128}. It is also used in rhetorical questions when trying to get the addressee to think about something \REF{bkm:Ref441522154}.  When it occurs with the irrealis negator \textit{dili} in the phrase \textit{dili paran}, it is asking ‘won’t this perhaps be right?’ \REF{bkm:Ref441522272}. This adverb is not used widely and may be borrowed from \isi{Tagalog} or \isi{Hiligaynon}.

\ea
\label{bkm:Ref441522128}
Ino  \textbf{paran}  ti  miad? \\\smallskip
 \gll Ino  \textbf{paran}  ti  miad? \\
what  perhaps  \textsc{d}1\textsc{nabs}  good \\
\glt ‘Which \textbf{perhaps} of these is good?' [BGON-L-01 12.4]
\z
\ea
\label{bkm:Ref441522154}
Ugaling,  man-o  \textbf{paran}  tak  mga  kaoy  i  impurtanti? \\\smallskip
 \gll Ugaling,  man-o  \textbf{paran}  tak  mga  kaoy  i  impurtanti? \\
however  why  perhaps  because  \textsc{pl}  tree  \textsc{def.n}  important \\
\glt ‘However, why \textbf{perhaps} are the trees important?’ (This is in a hortatory speech trying to influence people to plant trees.) [ROOB-T-01 6.1]
\z


\ea
\label{bkm:Ref441522272}
Dili  \textbf{paran}  danen  ya  mag-ambaļ  sanglit  na  magaiskwila  man ya, “Oy,  mga  kaapuan  ta  ya  pugya  uļa  gid  gapinsar?" \\\smallskip
 \gll Dili  \textbf{paran}  danen  ya  mag-ambaļ  sanglit  na  maga-iskwila  man ya, “Oy,  mga  kaapuan  ta  ya  pugya  uļa  gid  ga-pinsar?" \\
\textsc{neg.ir}  perhaps  3\textsc{p.abs}  \textsc{def.f}  \textsc{i.ir}-say  because  \textsc{lk}  \textsc{i.ir}-attending.school  \textsc{emph}
\textsc{def.f}  oh.no  \textsc{pl}  ancestor  1\textsc{p.incl.gen}  \textsc{def.f}  long.ago  \textsc{neg.r}  \textsc{int}  \textsc{i.r}-think \\
\glt `Won’t \textbf{perhaps} they (descendants) say because of going to school, “Oh man, our ancestors long ago really did not think”?’ (This comes from the same speech as in the previous example trying to motivate people to plant trees by saying that if they don’t, this is what their descendants will think.) [ROOB-T-01 9.8]
\z

The second-position adverb \textit{pļa} indicates something that is unexpected or surprising to the participants in the story or to the speaker herself \REF{bkm:Ref441348269}.

\ea
\label{bkm:Ref441348269}
Gatingaļa  mangngod  din  tak  ambaļ  manong  din,  ``Ittaw nyan  \textbf{pļa}  baboy  na  tļunon!" \\\smallskip
 \gll Ga-tingaļa  mangngod  din  tak  ambaļ  manong  din,  ``Ittaw nyan  \textbf{pļa}  baboy  na  tļunon!" \\
\textsc{i.r}-wonder  younger.sibling  3\textsc{s.gen}  because  say  older.brother  3\textsc{s.gen}  person
\textsc{d2adj}  unexpected  pig  \textsc{lk}  wild \\
\glt `His younger sibling was wondering because his older sibling said, “That person \textbf{surprisingly} is a wild pig.”{}' (This is a story about a brother who leads his younger brother to believe that he killed a person on the mountain just to see how he reacts. When they go to the mountain the younger brother sees that the older brother killed a wild pig, not a person.) [RBWN-T-02 4.2]
\z

More than one post-verbal adverb may co-occur. In general, the order is first the intensity adverbs \textit{nang} ‘only/just’, \textit{gid} ‘really/truly’, \textit{pa} ‘even/still’; second the clausal adverbs, such as \textit{kon} or \textit{man}; and third the aspect adverbs \textit{pa} ‘incompletive’ or \textit{en} ‘completive’ \REF{bkm:Ref329776500}{}-\REF{bkm:Ref329776503}. \tabref{tab:orderofsecondpositionadverbs} displays the order of second-position adverbs. Their glosses are given in \REF{bkm:Ref113436041} above, but are repeated in \tabref{tab:orderofsecondpositionadverbs} for convenience. Text examples of multiple second-position adverbs follow.

\begin{table}
\caption{Order of second-position adverbs}
\label{tab:orderofsecondpositionadverbs}
\begin{tabularx}{\textwidth}{lQp{2.3cm}}
\lsptoprule
First & Second & Third \\
\midrule
Intensity adverbs & Modal/epistemic\newline  adverbs & Aspectual\newline adverbs \\
\midrule
\textit{(pa) nang} ‘even/still just’ & \textit{imo} ‘emphasis/importance /attention getter’ & \textit{pa}\newline ‘incomplete’ \\
\tablevspace
\textit{(pa) gid} ‘even/still, really’ & \textit{kon} ‘hearsay’ & \textit{en}\newline ‘complete’ \\
\tablevspace
\textit{(pa) man} ‘even/still, emphasis’ & \textit{inta} ‘wish/intent’\footnotemark{} & \textit{anay}\newline \mbox{‘first{\slash}for a while’} \\
\tablevspace
\textit{pa} ‘even/still’ & \textit{tuo} ‘intention, please’ (polite word) \\
\tablevspace
\textit{anay} ‘please’ (polite word) & \textit{daan} ‘ahead of time, immediately' \\
\tablevspace
\textit{nang gid} ‘just really’ & \textit{isab} ‘again, repetition’ \\
\textit{eman} ‘again, resumption’ & \textit{dagli} ‘momentarily’ \\
& \textit{taan} `perhaps' (inference) & \\
& \textit{man} \textit{gyapon} ‘just the same’ & \\
& \textit{gani} ‘truly’ & \\
& \textit{abi/paryo abi} ‘for instance’ (softens requests{\slash}commands and states what another may be thinking) & \\
& \textit{ba} teasing/joking irony/sarcasm & \\
& \textit{i, an, ya} various attitudes and strong emotions & \\
& \textit{paran} ‘perhaps, isn’t that so?’ (in rhetorical questions) & \\
& \textit{pļa/pļang} ‘unexpected, surprise’ & \\
\lspbottomrule
\end{tabularx}
\end{table}
\footnotetext{ \textit{Inta} is a clause-level optative adverb, and as such may appear in the disjunct (clause-initial) position. It has roughly the same meaning in its clause-internal usage as a second-position adverb. We normally gloss it as \textsc{opt} for `optative mood'.}
\ea
\label{bkm:Ref329776500}
Ta  ame  na  pag-ambaļ  ki  kanen  gapati \textbf{nang}  \textbf{man}  \textbf{en}  Pedro  i. \\\smallskip
 \gll Ta  ame  na  pag-ambaļ  ki  kanen  ga-pati \textbf{nang}  \textbf{man}  \textbf{en}  Pedro  i. \\
\textsc{nabs}  1\textsc{p.excl.gen}  \textsc{lk}  \textsc{nr.act}-speak  \textsc{obl.p}  3s  \textsc{i.r}-believe
just  \textsc{emph}  \textsc{cm}  Pedro  \textsc{def.n} \\
\glt `During our talking to him, Pedro \textbf{just} began to believe/obey.’ [DBWN-L-22 56.6]
\z
\ea
\label{bkm:Ref329776503}
Uļa  a  \textbf{nang}  \textbf{gid}  \textbf{en}  gasagbak. \\\smallskip
 \gll Uļa  a  \textbf{nang}  \textbf{gid}  \textbf{en}  ga-sagbak. \\
\textsc{neg.r}  1\textsc{s.nabs}  just  \textsc{int}  \textsc{cm}  \textsc{i.r}-noisy \\
\glt ‘I really just did not make noise.’ [VAWN-T-15 5.7]
\z
\ea
Nakatinir  \textbf{pa}  \textbf{gid}  di  a  medical team  ta  darwa  pa  duminggo  daw  gapauli  \textbf{gid}  \textbf{man}  \textbf{en}. \\\smallskip
 \gll Naka-tinir  \textbf{pa}  \textbf{gid}  di  a  medical team  ta  darwa  pa  duminggo  daw  ga-pa-uli  \textbf{gid}  \textbf{man}  \textbf{en}. \\
\textsc{i.hap.r}-stay  \textsc{inc}  \textsc{int}  \textsc{d}1\textsc{loc}  \textsc{inj}  medical team  \textsc{nabs}  two  \textsc{inc}  week
and   \textsc{i.r}-\textsc{caus}-go.home  \textsc{int}  too  \textsc{cm} \\
\glt `The medical team \textbf{even} \textbf{really} stayed here two more weeks and they \textbf{really} \textbf{also} were allowed to go home.’ [JCWN-T-21 19.1]
\z
\ea
Sabat  \textbf{pa}  \textbf{gid}  \textbf{isab}  kano  i,  ambaļ  din,  “I said how much.'' \\\smallskip
 \gll Sabat  \textbf{pa}  \textbf{gid}  \textbf{isab}  kano  i,  ambaļ  din,  “I said how much." \\
reply  \textsc{inc}  \textsc{int}  again  American  \textsc{def.n}  say  3\textsc{s.erg} \\
\glt ‘the American \textbf{really} replied \textbf{again}, he said, “I said how much?”' [EFWN-T-09 3.5]
\z
\ea
… gakereg  ka  \textbf{gid}  \textbf{pa}  \textbf{man}. \\\smallskip
 \gll … ga-kereg  ka  \textbf{gid}  \textbf{pa}  \textbf{man}. \\
{} \textsc{i.r}-trembling  2\textsc{s.abs}  \textsc{int}  \textsc{inc}  \textsc{emph} \\
\glt ‘You are \textbf{really} \textbf{still} trembling also.’ [EMWN-T-05 4.13]
\z
\ea
… pugya  mga  baļay  ta  Iloilo  i  mga  nipa  \textbf{pa}  \textbf{nang}. \\\smallskip
 \gll … pugya  mga  baļay  ta  Iloilo  i  mga  nipa  \textbf{pa}  \textbf{nang}. \\
 {} long.ago  \textsc{pl}  house  \textsc{nabs}  Iloilo  \textsc{def.n}  \textsc{pl}  nipa  \textsc{inc}  just/only \\
\glt ‘…long ago the houses in Iloilo were only \textbf{really} \textbf{just} nipa.’ [LGON-L-01 10.1]
\z
\ea
Gaidad  \textbf{nang}  \textbf{pa}  kanen  ta  sampuļo  daw  annem  na  taon. \\\smallskip
 \gll Ga-idad  \textbf{nang}  \textbf{pa}  kanen  ta  sampuļo  daw  annem  na  taon. \\
\textsc{i.r}-age  just/only  \textsc{inc}  3\textsc{s.abs}  \textsc{nabs}  ten  and  six    \textsc{lk}  year \\
\glt ‘She \textbf{just} \textbf{still} was the age of sixteen years.’ [EMWN-T-06 4.5]
\z

The two adverbs \textit{pirmi} ‘always’ and \textit{dayon} ‘immediately’ may occur either preverbally as the first constituent in the clause, or in second position. Their unmarked position is second, but they occur in first position at heightened, intense, or climactic sections of a narrative.

When \textit{dayon} ‘immediately’ occurs clause-initially, it has the discourse function of marking new developments in the story, especially those that are exciting, lead to the climax, or are the climax itself (examples \ref{bkm:Ref425603569}{}-\ref{bkm:Ref425607917}). In these cases, it is glossed as ‘right away’. When in second position, \textit{dayon} expresses manner of action \REF{bkm:Ref425607938}{}-\REF{bkm:Ref425607965}, in which case it is glossed as ‘immediately’.

\ea
\label{bkm:Ref425603569}
Pre-verbal \textit{dayon} \\
Nakita  din  ake  na  mangngod  na  nalemmes. \textbf{Dayon}  din  kamangen  daw  duaļ  din  ta  baybay  daw ilingan  nanay  ko  daw  tatay  na  bata  nyo  nalemmes. \\\smallskip
 \gll Na-kita  din  ake  na  mangngod  na  na-lemmes. \textbf{Dayon}  din  kamang-en  daw  duaļ  din  ta  baybay  daw \emptyset{}-iling-an  nanay  ko  daw  tatay  na  bata  nyo  na-lemmes. \\
\textsc{a.hap.r}-see  3\textsc{s.erg}  1\textsc{s.gen}  \textsc{lk}  younger.sibling  \textsc{lk}  \textsc{a.hap.r}-drown\footnotemark{}
right.away  3\textsc{s.erg}  get-\textsc{t.ir}  and  take  3\textsc{s.erg}  \textsc{nabs}  beach  and \textsc{t.ir}-go-\textsc{apl}  mother  1\textsc{s.gen}  and  father  \textsc{lk}  child  2\textsc{p.gen}  \textsc{a.hap.r}-drown\\
\glt `He saw my younger sibling who (almost) drowned. \textbf{Right} \textbf{away} he got her/him, and took him/her to the beach, and he went to my mother and father (to say) that your child has (almost) drowned.’ [LCWN-T-01 2.8-9]
\footnotetext{The verb \textit{lemmes} does not assert that the person died, though they could have. In this case the sibling did not “drown to death” in the English sense, though ‘drown’ is the best short gloss we can come up with for this verb.}
\z
\largerpage[2]
\ea
Pagsil-ing  din,  \textbf{dayon}  gabot  Pedro  ta  sundang  din daw tigbasen  lima  din  na  gibit  ta  prasko. \\\smallskip
 \gll Pag-sil-ing  din,  \textbf{dayon}  gabot  Pedro  ta  sundang  din daw tigbas-en  lima  din  na  ga-ibit  ta  prasko. \\
\textsc{nr.act}-peek  3\textsc{s.gen}  right.away  pull.out  Pedro  \textsc{nabs}  machete  3\textsc{s.gen} and chop-\textsc{t.ir}  hand  3\textsc{s.gen}  \textsc{lk}  \textsc{i.r}-hold  \textsc{nabs}  bottle \\
\glt `When he was looking (in the bottle), \textbf{right} \textbf{away} Pedro pulled out his machete and chopped his hand that was holding the bottle.’ [JCWN-T-20 17.9]
\z


\ea
Pagkakita    Pedro  na  nuļog  Maria  i,  \textbf{dayon}  din balik daw  labyugan  ta  kaļat. \\\smallskip
 \gll Pagka-kita    Pedro  na  na-uļog  Maria  i,  \textbf{dayon}  din balik daw  \emptyset{}-labyug-an  ta  kaļat. \\
\textsc{nr.act}-see  Pedro  \textsc{lk}  \textsc{a.hap.r}-fall  Maria  \textsc{def.n}  right.away  3\textsc{s.gen} return and  \textsc{t.ir}-throw-\textsc{apl}  \textsc{nabs}  rope \\
\glt `When Pedro saw that Maria fell, \textbf{right} \textbf{away} he returned and threw a rope (to her).’ [EMWN-T-06 5.4]
\z
\ea
\label{bkm:Ref425607917}
Pagkita  danen  ki  yaken,  \textbf{dayon}  gaus-os daw mandļagan. \\\smallskip
 \gll Pag-kita  danen  ki  yaken,  \textbf{dayon}  ga-us-os daw ma-ng-dļagan. \\
\textsc{nr.act}-see  3\textsc{p.gen}  \textsc{obl.p}  1s  right.away  \textsc{i.r}-slide.down
and \textsc{a.hap.ir-pl}-run \\
\glt ‘Seeing me, \textbf{right} \textbf{away} they slid down (the tree) and ran off.’ (The children were climbing trees on a beach believed to have evil spirits. So when they saw a man coming up from the sea they assumed he was an evil spirit and were afraid of him. But his boat had turned over when he was fishing.) [EFWN-T-11 15.5]
\z
\ea
\label{bkm:Ref425607938}
second-position \textit{dayon}: \\
Pagtapos  nay  luag  ta  kaoy  ya,  gadiritso kay \textbf{dayon}  ta  pambot  ya  na  makay  miling  Duļļo. \\\smallskip
 \gll Pag-tapos  nay  luag  ta  kaoy  ya,  ga-diritso kay \textbf{dayon}  ta  pambot  ya  na  m-sakay  m-iling  Duļļo. \\
\textsc{nr.act}-finish  1\textsc{p.excl.gen}  watch  \textsc{nabs}  tree  \textsc{def.f}  \textsc{i.r}-straight 1\textsc{p.excl.abs}
immediately  \textsc{nabs}  motorboat  \textsc{def.f}  \textsc{lk}  \textsc{i.v.ir-}ride  \textsc{i.v.ir}-go  Duļļo \\
\glt ‘When we finished looking at the tree, we \textbf{immediately} went straight to the boat to ride to go to Dullo.’ [DBWN-T-24 3.8]
\z
\ea
A,  ginikanan  an  a  na  uļa  nakaintindi,  uļa  nakasat-em,    mambaļ  en  \textbf{dayon},  “Indya  kwarta  no  ya?” \\\smallskip
 \gll A,  ginikanan  an  a  na  uļa  naka-intindi,  uļa  naka-sat-em,    m-ambaļ  en  \textbf{dayon},  “Indya  kwarta  no  ya?” \\
Ah,  parents  \textsc{def.m}  \textsc{inj}  \textsc{lk}  \textsc{neg.r}  \textsc{i.hap.r}-understand  \textsc{neg.r}
\textsc{i.hap.r}-comprehend  \textsc{i.v.ir}-say  \textsc{cm}  immediately  where  money  2\textsc{s.gen}  \textsc{def.f} \\
\glt `Parents who do not understand, do not comprehend, will \textbf{immediately} say, “Where is your money?”' (Parents say this to a child who wants to go to college when the parents can’t afford it.) [JCOB-L-02 13.3]
\z
\ea
Tungtungan  din  sanga  ya.  Nuļog  \textbf{dayon}  kanen ya. \\\smallskip
 \gll ...-Tungtong-an  din  sanga  ya.  Na-uļog  \textbf{dayon}  kanen ya. \\
\textsc{t.r}-get.on.top-\textsc{apl}  3\textsc{s.erg}  branch  \textsc{def.f}  \textsc{a.hap.r}-fall  immediately  3\textsc{s.abs} \textsc{def.f} \\
\glt ‘He got on the branch. He fell \textbf{immediately}.’ [LSWN-T-01 4.2]
\z
\ea
Tapos  ame  na  recording,  gakaan  kay  \textbf{dayon} ta  ice cream. \\\smallskip
 \gll Tapos  ame  na  recording,  ga-kaan  kay  \textbf{dayon} ta  ice cream. \\
after  1\textsc{p.excl.gen}  \textsc{lk}  recording  \textsc{i.r}-eat  1\textsc{p.excl.abs}  immediately
\textsc{nabs}  ice cream \\
\glt ‘After our recording, we \textbf{immediately} ate ice cream…’ [EMWN-T-09 9.4]
\z
\ea
\label{bkm:Ref425607965}
Daw  nabao  isya  na  gabagnes,  kanen masakitan  ta  gettek.  Umawen  \textbf{dayon} manligan  ya  daw  surano  para  teyepan bai na  nabao. \\\smallskip
\gll Daw  na-bao  isya  na  ga-bagnes,  kanen ma-sakit-an  ta  gettek.  Umaw-en  \textbf{dayon} manligan  ya  daw  surano  para  \emptyset{}-teyep-an\footnotemark{} bai na  na-bao. \\
if/when  \textsc{a.hap.r}-continual.pregnancy  one  \textsc{lk}  \textsc{i.r}-pregnant  3\textsc{s.abs}
\textsc{a.hap.ir}-pain-\textsc{apl}  \textsc{nabs}  stomach  call-\textsc{t.ir}  immediately
midwife  \textsc{def.f}  and  shaman  for  \textsc{t.ir}-blow.with.ginger-\textsc{apl}  woman
\textsc{lk} \textsc{a.hap.r}-continual.pregnancy \\
\footnotetext{The word \textit{teyep} is a special word for ‘blow’ meaning what a shaman does with the ginger as part of diagnosing an illness. The usual word for ‘blow’ is \textit{eyep.}}
\newpage
\glt `When a pregnant woman has a continual pregnancy, her stomach hurts her. Call \textbf{immediately} the midwife and shaman in order to blow ginger on the woman with a continual pregnancy.’ [VAOE-J-05 1.4]
\z

Similarly, when \textit{pirmi} ‘always’ occurs initially in the clause it usually is at places that have heightened tension in the story (\ref{bkm:Ref441602096}). The unmarked position is second in the clause \REF{bkm:Ref441602112}.

\ea
\label{bkm:Ref441602096}Pre-verbal \textit{pirmi} \\
A,  gakwa  ki  yaken,  \textbf{pirmi}  a  nang  en  na  pangamuyo ta  Dios. \\\smallskip
 \gll A,  ga-kwa  ki  yaken,  \textbf{pirmi}  a  nang  en  na  pangamuyo ta  Dios. \\
Ah,  \textsc{i.r}-whatchamacallit  \textsc{obl}  1s  always  1\textsc{s.abs}  only/just  \textsc{cm}  \textsc{lk}  pray
\textsc{nabs}  God \\
\glt `Ah, when (the fish) watchamacallit on me (took the bait), I \textbf{continually} prayed to God.’ (This is a story about a fisherman who caught a big fish and did not want to let it  go even though his boat had overturned.) [EFWN-T-10 4.7]
\z
\ea
\label{bkm:Ref441602112}
second-position \textit{pirmi} \\
Mam,  taan  sadya  kaw  \textbf{pirmi}  tak  gakitaay  kaw ta  inyo  na  mga  utod  daw  arey. \\\smallskip
 \gll Mam,  taan  sadya  kaw  \textbf{pirmi}  tak  ga-kita-ay  kaw ta  inyo  na  mga  utod  daw  arey. \\
ma’am  perhaps  enjoy  2\textsc{p.abs}  always  because  \textsc{i.r}-see-\textsc{rec}  2\textsc{p.abs} \textsc{nabs}  2\textsc{p.gen}  \textsc{lk}  \textsc{pl}  sibling  and  friend \\
\glt `Ma’am, perhaps you are \textbf{always} enjoying (it) because you, your relatives and friends are seeing each other.’ [AFWL-L-01 4.2]
\z

When an adverb such as \textit{pirmi} occurs in clause-final position, the speaker draws special attention to the adverb \REF{bkm:Ref441602199}.

\ea
\label{bkm:Ref441602199}
Final \textit{pirmi} \\
Gapangamuyo  a  ki  kaon  \textbf{pirmi}. \\\smallskip
 \gll Ga-pangamuyo  a  ki  kaon  \textbf{pirmi}. \\
\textsc{i.r}-pray  1\textsc{s.abs}  \textsc{obl.p}  2s  always \\
\glt ‘I am \textbf{\textsc{always}} praying for you.’ (This is the last sentence in a personal letter.) [VAWL-C-12 2.8]
\z

The adverb \textit{inta} ‘\textsc{optative}/wish/intention’ can occur pre-verbally outside the clause (as a disjunct adverb, as described in \sectref{bkm:Ref441599506}), or in second position in the clause, in which case it describes an unfulfilled intention sometimes idiomatically translated as ‘should have’. Examples \REF{bkm:Ref443721187} and \REF{bkm:Ref443721212} illustrate \textit{inta} in second position:

\ea
\label{bkm:Ref443721187}
Mļagan  man  kon  \textbf{inta}  mga  ittaw  tak  nakita  danen  en na manigir  danen  napatay. \\\smallskip
 \gll M-dļagan  man  kon  \textbf{inta}  mga  ittaw  tak  na-kita  danen  en na manigir  danen  na-patay. \\
\textsc{i.v.ir}-run  \textsc{emph}  \textsc{hsy}  \textsc{opt}  \textsc{pl}  person  because  \textsc{a.hap.r}-see  3\textsc{p.erg}  \textsc{cm} \textsc{lk}  manager  3\textsc{p.gen}  \textsc{a.hap.r}-dead \\
\glt `The people \textbf{should} \textbf{have/intended} \textbf{to} \textbf{run} (intention not fulfilled) because they saw that their manager died.’ [MBON-T-04 12.10]
\z
\ea
\label{bkm:Ref443721212}
Yi  na  manakem  uļa  \textbf{inta}  kanen  napatay  tak kanen  gatago  naan  ta  kasilyas  ya. \\\smallskip
 \gll Yi  na  manakem  uļa  \textbf{inta}  kanen  na-patay  tak kanen  ga-tago  naan  ta  kasilyas  ya. \\
\textsc{d}1\textsc{adj}  \textsc{lk}  older  \textsc{neg.r}  \textsc{opt}    3\textsc{s.abs}  \textsc{a.hap.r}-dead  because
3\textsc{s.abs}  \textsc{i.r}-hide  \textsc{spat.def}  \textsc{nabs}  restroom  \textsc{def.f} \\
\glt `This older person \textbf{should} not have died (but he was shot when running back to the house to get some money) because he hid in the restroom.’ [BCWN-C-04 6.7]
\z

Some adverbs such as \textit{sigurado} ‘surely’, \textit{siguro} ‘perhaps’, \textit{basi} ‘maybe’, \textit{dapat} ‘must/ought’ and \textit{kinangļan} ‘necessary’ besides their modification of the verb or clause, can also function as uninflected complement-taking predicates. See the discussion of complement clause constructions in \chapref{chap:clausecombining}, \sectref{sec:na-complementclauses}.

\subsection{Prepositional adverbs}
\label{bkm:Ref480640397} \is{prepositional adverbs}\is{adverbs!prepositional}

Clause-level modifiers following the non-absolutive prepositional marker \textit{ta} are described in \sectref{bkm:Ref439924168}. In this section we briefly describe and exemplify adverbs that appear in this construction. Adverbs following \textit{ta} never occur before the main verb of the sentence, as allowed for locative oblique elements. Also, unlike other oblique elements these structures never occur with a complex preposition---the marker is always and only \textit{ta}. These \textit{prepositional adverbs}\is{prepositional adverbs} tend to express the extent, degree or value of the action, as listed in \REF{bkm:Ref441309255}.

\ea
\label{bkm:Ref441309255}
\begin{tabbing}
\hspace{2cm} \= \kill
Prepositional adverbs \\
\textit{ta miad} \> ‘well’ \\
\textit{ta usto} \> ‘completely/well/properly’ \\
\textit{ta tudo} \> ‘with all effort’ \\
\textit{ta matuod} \> ‘truly’ \\
\textit{ta dakmeļ} \> ‘thickly’ \\
\textit{ta bakod} \> ‘muchly’ \\
\textit{ta inay-inay} \> ‘slowly’ \\
\textit{ta tise/sise} \> ‘a little bit’
\end{tabbing}
\z

Examples \REF{bkm:Ref423465666}{}-\REF{bkm:Ref423465676} illustrate prepositional adverbs from the corpus. These post-verbal adverbs occur after any other post-verbal adverbs discussed above and any Referring Phrases.

\ea
\label{bkm:Ref423465666}
Ta pitto ya na mga priso gaamat patay duma ya tak danen pasilutan gid \textbf{ta miad}. \\\smallskip
 \gll Ta  pitto  ya  na  mga  priso  ga-amat  patay  duma  ya  tak danen  pa-silut-an  gid  \textbf{ta}  \textbf{miad}. \\
\textsc{nabs}  seven  \textsc{def.f}  \textsc{lk}  \textsc{pl}  prisoner  \textsc{i.r}-gradually  dead  some  \textsc{def.f}  because
3\textsc{s.abs}  \textsc{t.r}\textbf{-}punish\textbf{-}\textsc{apl}  \textsc{int}  \textbf{\textsc{nabs}}  \textbf{well/good} \\
\glt `Of the seven prisoners, some gradually died because they were punished \textbf{well}.’ [JCWN-T-20 26.2] \\\smallskip

* … \textbf{ta} \textbf{miad} na danen pasilutan.
\z
\ea
Gani kabataan i advice ko ake na mag-iskwila \textbf{ta usto}. \\\smallskip
 \gll Gani  ka-bata-an  i  advice  ko  ake  na  mag-iskwila  \textbf{ta}  \textbf{usto}. \\
so  \textsc{nr}-child-\textsc{nr}  \textsc{def.n}  advice  1\textsc{s.gen}  1\textsc{s.gen}  \textsc{lk}  \textsc{i.ir}-school  \textsc{nabs}  well \\
\glt ‘So children, my advice (is) to go to school (i.e., to study) \textbf{well}.’ [LTOE-C-01 2.2]
\z
\ea
\label{bkm:Ref423465676}
Dayon kon papakang buļag tamboļ ya \textbf{ta sikad tudo}. \\\smallskip
 \gll Dayon  kon  pa-pakang  buļag  tamboļ  ya  \textbf{ta}  \textbf{sikad}  \textbf{tudo}. \\
right.away  \textsc{hsy}  \textsc{t.r}-hit  blind  drum  \textsc{def.f}  \textsc{nabs}  very  intense \\
\glt ‘Right away the blind one hit the drum \textbf{very} \textbf{hard}.’ (7.12]
\z
\subsection{Locative adverbs}
\label{bkm:Ref52477189}

Locative adverbs\is{locative adverbs}\is{adverbs!locative} are words that refer specifically to locations, such as \textit{here} and \textit{there} in English. Some other grammars of Philippine languages call these kinds of words “deictics” or “deictic pronouns”. We prefer the term locative adverb because these words distribute in sentences like adverbs, and are not distinguished by case. However, there are certain similarities between locative adverbs and demonstrative pronouns. First, locative adverbs exhibit the same four spatial distinctions as demonstrative pronouns, and, except for D4, the forms are clearly parallel to the non-absolutive demonstrative pronouns (see \chapref{chap:referringexpressions}, \tabref{tab:demonstrativepronounsadjectives}). Second, locative adverbs have general and precise forms. Finally, the precise forms occur in long and short variants. We have yet to discern any meaningful difference between the long and the short forms of precise locative adverbs.

\begin{table}
\caption{Locative adverbs}
\label{tab:locativeadverbs}
\fittable{
\begin{tabular}{lll}
\lsptoprule
& General & Precise \\
\midrule
D1 near speaker & di & unti/ti/duti/enti \\
D2 near addressee & dyan & unsan/san/ensan \\
D3 somewhere in the area of speaker and addressee & don & untan/tan/entan \\
D4 far away, out of sight & dya & unso/so/enso \\
\lspbottomrule
\end{tabular}
}
\end{table}

The following are some examples of locative adverbs in context:

\ea
Lugay  munta  baybay,  Manang,  tak  \textbf{untan}  pugya   baļay  daen. \\\smallskip
 \gll Lugay  m-punta  baybay,  Manang,  tak  \textbf{untan}  pugya   baļay  daen. \\
long.time  \textsc{i.v.ir}-go  beach  older.sister  because  \textsc{d3loc.pr}  long.ago
house  3\textsc{p.gen} \\
\glt ‘Some time past (she) was going to the beach, Older Sister, because \textbf{right} \textbf{there} long ago was their house.’ [CBON-T-03]
\z
\ea
Yaan  ta  Iloilo  tama  \textbf{don}  buļag  na  gapangayo. \\\smallskip
 \gll Yaan  ta  Iloilo  tama  \textbf{don}  buļag  na  ga-pangayo. \\
\textsc{spat.def}  \textsc{nabs}  Iloilo  many  \textsc{d3loc}  blind  \textsc{lk}  \textsc{i.r}-request \\
\glt ‘In Iloilo there are many blind \textbf{there} who ask for things.’ [TTOB-L-03 10.10]
\z

The unmarked position for locative adverbs is post-verbal, either before or after the first noun phrase (\ref{bkm:Ref52348615}{}-\ref{bkm:Ref113439285}). Personal pronouns and shorter adverbs occur before the locative adverb. An oblique location phrase with the definite spatial marker \textit{naan} and a noun can occur clause-finally further specifying the location (\ref{bkm:Ref52439592}).

\ea
\label{bkm:Ref52348615}
Tukad  danen  \textbf{dya}.  Padaļa  din  utod  din  \textbf{dya}. \\\smallskip
 \gll Tukad  danen  \textbf{dya}.  Pa-daļa  din  utod  din  \textbf{dya}. \\
go.uphill  3\textsc{p.abs}  \textsc{d}4\textsc{loc}  \textsc{t.r}-carry  3\textsc{s.erg}  sibling  3\textsc{s.gen}  \textsc{d}4\textsc{loc} \\
\glt ‘They went up there. He took his sibling \textbf{there}.’ [RBON-T-01 6.8]
\z
\ea
Padaļa  nay  \textbf{dya}  a  kabaong  ta  ake  na  bayaw naan  ta  simbaan  ta  mga  sundaļo. \\\smallskip
 \gll Pa-daļa  nay  \textbf{dya}  a  kabaong  ta  ake  na  bayaw naan  ta  simba-an  ta  mga  sundaļo. \\
\textsc{t.r}-carry  1\textsc{p.excl.erg}  \textsc{d}4\textsc{loc}  \textsc{inj}  coffin  \textsc{nabs}  1\textsc{s.gen}  \textsc{lk}  sibling-in-law
\textsc{spat.def}  \textsc{nabs}  worship-\textsc{nr}  \textsc{nabs}  \textsc{pl}  soldier \\
\glt `We took the coffin of my brother-in-law \textbf{there} to the church of the soldiers.’ [VAWN-T-15 3.2]
\z
\ea
\label{bkm:Ref113439285}
Nabatyagan  ko  gid  \textbf{dya}  na  uļa  gid  man  en  gasakit. \\\smallskip
 \gll Na-batyag-an  ko  gid  \textbf{dya}  na  uļa  gid  man  en  ga-sakit. \\
\textsc{a.hap.r-}feel-\textsc{apl}  1\textsc{s.erg}  \textsc{int}  \textsc{d}4\textsc{loc}   \textsc{lk}  \textsc{neg.r}  \textsc{int}  also  \textsc{cm}  \textsc{i.r}-pain \\
\glt ‘I really felt \textbf{there} that there was no more pain.’ [JCWN-T-22 8.13]
\z

Locative adverbs and the definite spatial oblique marker \textit{naan} sometimes occur together, for example, \textit{naan dya}. However, when a general locative adverb is in the pragmatically marked pre-verbal position, then \textit{naan} must occur before it (exx. \ref{bkm:Ref52439592}{}-\ref{bkm:Ref52439595}). When the spatial marker and the locative adverb are pre-verbal, any clitic pronouns and clitic adverbs must also occur in pre-verbal position (\ref{bkm:Ref52439592}).
\ea
\label{bkm:Ref52439592}
Naan  kay  \textbf{dya}  gakitaay  kay,  kami Maria  daw  Neneng  naan  Gaisano. \\\smallskip
 \gll Naan  kay  \textbf{dya}  ga-kita-ay  kay,  kami Maria  daw  Neneng  naan  Gaisano. \\
\textsc{spat.def}  1\textsc{p.lexcl.abs}  \textsc{d4loc}  \textsc{i.r}-see-\textsc{rec} 1\textsc{p.excl.abs} 1\textsc{p.excl.abs}
Maria  and  Neneng  \textsc{spat.def}  Gaisano \\
\glt ‘(It was) there we met each other, I and Maria and Neneng, in Gaisano.’ [BMON-C-05 10.2]
\z

The following orders are also possible:
\ea
\label{bkm:Ref52439595}
a. \textit{Naan dya kami gakitaay}. \\
b. \textit{Naan kay/kami dya gakitaay}. \\ 
\z

But the following are impossible:
\ea
a. * \textit{Naan dya kay gakitaay}. \\
b. * \textit{Naan dya gakitaay kay}. \\
c. * \textit{Naan dya gakitaay kami}.
\z

Example \REF{bkm:Ref52439955} illustrates a precise locative pronoun.

\ea
\label{bkm:Ref52439955}
Nļaman  no  na  kaselled  aren ta  ubra \textbf{unti}  nang  man  ta  ame  i  na  munisipyo, naan  ta  upisina  i  ta  trisurer. \\\smallskip
 \gll Na-aļam-an  no  na  ka-selled  aren ta  ubra \textbf{unti}  nang  man  ta  ame  i  na  munisipyo, naan  ta  upisina  i  ta  trisurer. \\
\textsc{a.hap.r}-know-\textsc{apl}  \textsc{2serg}  \textsc{lk}  \textsc{i.exm}-go.inside  1\textsc{s.abs+cm} \textsc{nabs}  work \textsc{d1loc.pr}  only  also  \textsc{nabs} 1\textsc{p.incl.gen}  \textsc{def.n}  \textsc{lk}  town.hall \textsc{spat.def}  \textsc{nabs}  office  \textsc{def.n}  \textsc{nabs} treasurer \\
\glt `You know that I got to go to work just \textbf{right} \textbf{here} in our town hall, in the office of the treasurer.’ [DBWL-T-20 8.5] 
\z

One difference between the distribution of general and precise locative adverbs is that with the precise forms the definite spatial oblique marker \textit{naan} is optional when in preverbal position and usually does not occur. Compare the following to examples \REF{bkm:Ref52439592} and \REF{bkm:Ref52439595}:

\ea
\textbf{Unti}  ko  man  nakita  isya  na  waig  na  pasaļok  nang  ta kabo  daw  mandok  ka. \\\smallskip
 \gll \textbf{Unti}  ko  man  na-kita  isya  na  waig  na  pa-saļok  nang  ta kabo  daw  m-sandok  ka. \\
\textsc{d1loc.pr}  1\textsc{s.erg}  also  \textsc{a.hap.r}-see  one  \textsc{lk}  water  \textsc{lk}  \textsc{t.r}-dip  only  \textsc{nabs} dipper  if/when  \textsc{i.v.ir}-fetch.water  2\textsc{s.abs} \\
\glt `(It was) \textbf{right} \textbf{here} I even saw one (source of) water that is just dipped out with a dipper when you fetch water.' [DBWN-L-21 2.10]
\z

\ea
\textbf{Naan}  \textbf{unti}  gabetang  en  simbaan  ta  ebes  i.  Piro \textbf{naan}  \textbf{unso}  datas  na  simbaan. \\\smallskip
 \gll \textbf{Naan}  \textbf{unti}  ga-betang  en  simba-an  ta  ebes  i.  Piro \textbf{naan}  \textbf{unso}  datas  na  simba-an. \\
\textsc{spat.def}  \textsc{d}1\textsc{loc.pr}  \textsc{i.r}-placed  \textsc{cm}  worship-\textsc{nr}  \textsc{nabs}  down  \textsc{def.n}  but
\textsc{spat.def}  \textsc{d}4\textsc{loc.pr}  high  \textsc{lk}  worship-\textsc{nr} \\
\glt `(It is) \textbf{right} \textbf{here} the church was placed below (the hill). But (it is) \textbf{right} \textbf{there} (on the hill) is the high church.’
\is{adverbs|)}
\z

\section{Prepositional Phrases}
\label{bkm:Ref481473714} \is{Prepositional Phrases|(}

In this section we will treat prepositional phrases (non-absolutive constituents expressing oblique elements) as optional clause-level modifiers, similar to \isi{adjunct adverbs}\is{adverbs!adjunct}. Prepositional phrases consist minimally of an oblique case-marked RP, that is, an RP with one of the \isi{oblique} case particles, \textit{ki} for personal names or \textit{ta} for all other head nouns (see \chapref{chap:referringexpressions}, \sectref{sec:caseinreferringexpressions}). Spatial prepositional phrases\is{spatial Prepositional Phrases}\is{Prepositional Phrases!spatial} may also contain the \isi{spatial definite particle} \textit{naan} (variants \textit{yaan}, \textit{nyaan}, or \textit{an}), and an optional spatial preposition\is{prepositions!spatial|(} (PREP) before the RP. The template for prepositional phrases is given in \REF{bkm:Ref479920100}, and some examples are presented in \REF{bkm:Ref117667676}:
\ea
\label{bkm:Ref479920100}
\textit{naan} (PREP) RP\textsubscript{obl}
\z
\ea
\label{bkm:Ref117667676}
a.  \textit{naan dani ta apoy} ‘near to the fire’ [AGWN-L-01 3.18] \\
b.  \textit{naan dani (ki) Pedro} ‘near Pedro' (elicited) \\
c.  \textit{naan tengnged ta bai na gamasakit} ‘next to the woman who was sick’ [EDWN-T-03 2.17] \\
d.  \textit{naan apaw ta iya na mata} ‘above his/her eye’ [EDWN-T-03 2.21] \\
e.  \textit{madyo ta duma na banwa} ‘far from other towns/countries’ [JCWL-T-19 4.2] \\
f.  \textit{tengnged  ta baļay danen} ‘next to their house’ [BMON-C-05 3.22] \\
g.  \textit{naan ta Manila} ‘in/at Manila’ [BMON-C-05 3.3] \\
h  \textit{naan ta Cagayancillo} ‘on Cagayancillo’ [VAWN-T-15 5.5] \\
I  \textit{naan Cagayancillo} ‘on Cagayancillo’ [LTOE-C-01 1.3] \\
j  \textit{ta Cagayancillo} ‘on Cagayancillo’ [JCWN-L-33 2.1]
\z

Nearly all the spatial prepositions (listed in \ref{bkm:Ref329066261}) transparently derive from relational nouns, that is, nouns that specify a relational term, such as top, bottom, side, inside, outside, and so on. These spatial prepositions indicate the location or direction of the following RP in relation to the proposition expressed in the clause. They also may indicate certain logical relations.

\ea
\label{bkm:Ref329066261}
\begin{tabbing}
\hspace{2cm} \= \hspace{4.8cm} \= \kill
Spatial prepositions \\
Form \> Meaning as Preposition  \>  Meaning as relational noun \\
\textit{apaw} \> ‘on top/above/up/upward’ \>   ‘top’ \\
\textit{daļem} \> ‘on the bottom/deep/below’ \>   ‘bottom/deep/underneath’ \\
\textit{ebes} \> ‘below/down/downward’ \> ‘downstairs/underneath??’ \\
\textit{tengnged} \> ‘next to/beside’     \>   ‘place next to’ \\
\textit{(a)tubang} \> ‘in front of’  \>          ‘front’ \\
\textit{kilid} \>`at/on the side of'   \>     ‘side’ \\
\textit{tudtod} \> ‘behind’ \>           ‘back’ \\
\textit{selled} \> ‘inside of’  \>          ‘inside’ \\
\textit{gwa} \> ‘outside of’     \>       ‘outside’ \\
\textit{dapit} \> ‘in direction of’ \>       ‘direction’ \\
\textit{madyo/adyo} \> ‘far’ \>           ‘far place’ \\
\textit{dani} \> ‘near’      \>      ‘close place’
\end{tabbing}
\z

As prepositions, these forms usually follow the definite spatial marker \textit{naan}.
\ea
Tapos,  gapungko  kay  \textbf{naan}  \textbf{dani}  \textbf{ta}  \textbf{apoy}  agod may  pagdangga. \\\smallskip
 \gll Tapos,  ga-pungko  kay  \textbf{naan}  \textbf{dani}  \textbf{ta}  \textbf{apoy}  agod may  pag-dangga. \\
then  \textsc{i.r}-sit.down  1\textsc{p.excl.abs}  \textsc{spat.def}  near  \textsc{nabs}  fire  so.that
\textsc{ext.in}  \textsc{nr.act}-hot \\
\glt `Then we sat \textbf{close} \textbf{to} \textbf{the} \textbf{fire} so that there was (something) to warm (us) up.’ [AGWN-L-01 3.19]
\z
\ea
Ta  iya  na  pagpanaw  nakaagi  kanen  ta  mga  tindaan \textbf{naan}  \textbf{kilid}  \textbf{ta}  \textbf{daļan}. \\\smallskip
 \gll Ta  iya  na  pag-panaw  naka-agi  kanen  ta  mga  tinda-an \textbf{naan}  \textbf{kilid}  \textbf{ta}  \textbf{daļan}. \\
\textsc{nabs}  3\textsc{s.gen}  \textsc{lk}  \textsc{nr.act}-go/walk  \textsc{i.hap.r}-pass  3\textsc{s.abs}  \textsc{nabs}  \textsc{pl}  sell-\textsc{nr}
\textsc{spat.def}  side  \textsc{nabs}  road \\
\glt `During his walking, he happened to pass some stores \textbf{on} \textbf{the} \textbf{side} \textbf{of} \textbf{the} \textbf{road}.’ [VPWN-T-05 2.4]
\z
\ea
Eviok  isya  man  na  lugar  \textbf{naan}  \textbf{dapit}  \textbf{ta}  \textbf{Mamaan}. \\\smallskip
 \gll Eviok  isya  man  na  lugar  \textbf{naan}  \textbf{dapit}  \textbf{ta}  \textbf{Mamaan}. \\
Eviok  one  too  \textsc{lk}  place  \textsc{spat.def}  towards  \textsc{nabs}  Mamaan \\
\glt ‘Eviok (is) a place too that (is) \textbf{towards} \textbf{Mamaan}.’ [EFWN-T-11 9.6]
\z

The definite demonstrative \textit{naan} can also occur without a spatial preposition, in which case the meaning is general location---‘in’, ‘at’, or ‘on’. In such cases, the non-absolutive marker \textit{ta} or \textit{ki} often drops out, suggesting that \textit{naan} is almost grammaticalized as a third oblique case marker\is{oblique case markers} (exx. \ref{bkm:Ref480787107}{}-\ref{bkm:Ref52441569}). If a place is very familiar, then it seems more often the non-absolutive case marker drops out.
\ea
Gamasyar  a  \textbf{naan}  \textbf{ta}  \textbf{baļay}  danen. \\\smallskip
 \gll Ga-ng-pasyar  a  \textbf{naan}  \textbf{ta}  \textbf{baļay}  danen. \\
\textsc{i.r}-\textsc{pl}-visit  1\textsc{s.abs}  \textsc{spat.def}  \textsc{nabs}  house 3\textsc{p.gen} \\
\glt ‘I went visiting \textbf{at} \textbf{their} \textbf{house}.’ [BMON-C-04 1.10]
\z
\ea
Gapundo  en  bļangay  \textbf{naan}  \textbf{ta}  \textbf{Anini-y}. \\\smallskip
 \gll Ga-pundo  en  bļangay  \textbf{naan}  \textbf{ta}  \textbf{Anini-y}. \\
\textsc{i.r}-anchor  \textsc{cm}  2.masted.boat  \textsc{spat.def}  \textsc{nabs}  Anini-y \\
\glt ‘The boat anchored \textbf{at} \textbf{Anini-y}.’ [VAWN-T-18 6.1]
\z
\ea
\label{bkm:Ref480787107}
Gapit  kay  pa  \textbf{naan}  \textbf{Duļļo}. \\\smallskip
 \gll Ga-apit  kay  pa  \textbf{naan}  \textbf{Duļļo}. \\
\textsc{i.r}-stop.off  1\textsc{p.excl.abs}  \textsc{inc}  \textsc{spat.def}  Duļļo \\
\glt ‘We still stopped off \textbf{at} \textbf{Dullo}.’ [DBON-C-06 4.8] \\
(\textit{ta} has dropped out of \textit{naan ta Duļļo}).
\z
\ea
\label{bkm:Ref52441569}
Piro  uļa  ka  gambaļ  \textbf{naan}  \textbf{yaken} na  nadayaran ka  sa  asta  nang  en  gapudpod. \\\smallskip
 \gll Piro  uļa  ka  ga-ambaļ  \textbf{naan}  \textbf{yaken} na  na-dayad-an ka  sa  asta  nang  en  ga-pudpod. \\
but  \textsc{neg.r}  2\textsc{s.abs}  \textsc{i.r}-say  \textsc{spat.def}  1s
\textsc{lk}  \textsc{a.hap.r}-good-\textsc{apl}
2\textsc{s.abs}  \textsc{d}4\textsc{nabs}  until  just  \textsc{cm}  \textsc{i.r}-crumble.apart \\
\glt `But you did not say \textbf{to} \textbf{me} that you are pleased with that until it just crumbled apart.’ [MAWL-C-03 4.8] \\
(\textit{ki} dropped out of \textit{naan ki yaken}).
\z

The following examples illustrate these prepositions in their usages as \isi{relational nouns}\is{nouns!relational}.

\ea
\textbf{Naan}  \textbf{ta}  \textbf{apaw}  tallo  na  kanyon  gabatang  ta  atubangan  ta  Lipot. \\\smallskip
 \gll \textbf{Naan}  \textbf{ta}  \textbf{apaw}  tallo  na  kanyon  ga-batang  ta  atubangan  ta  Lipot. \\
\textsc{spat.def}  \textsc{nabs}  above  three  \textsc{lk}  canon  \textsc{i.r}-put  \textsc{nabs}  front  \textsc{nabs}  Lipot \\
\glt ‘\textbf{On} \textbf{the} \textbf{top} (of a hill) three canons were positioned in front of Lipot (community).’ [JCWN-T-24 5.3]
\z
\ea
Lugay  ambaļ  gid  mangngod  ko  mag-angad  kay  kon  \textbf{naan}  \textbf{ta}  \textbf{apaw}  tak  naan  dya  galin  kanta  ya. \\\smallskip
 \gll Lugay  ambaļ  gid  mangngod  ko  mag-angad  kay  kon  \textbf{naan}  \textbf{ta}  \textbf{apaw}  tak  naan  dya  galin  kanta  ya. \\
long.time  say  \textsc{int}  younger.sibling  1\textsc{s.gen}  \textsc{i.ir}-look.up 1\textsc{p.excl.abs}  \textsc{hsy} \textsc{spat.def}  \textsc{nabs}  above  because  \textsc{spat.def}  \textsc{d}4\textsc{loc}  \textsc{i.r}-from  sing  \textsc{def.f} \\
\glt `Then my younger brother said look up \textbf{at} \textbf{the} \textbf{top} (of the mast of the boat) because there (was where) the singing was coming from.’ [VAWN-T-19 4.6]
\z


\ea
Piro  \textbf{naan}  \textbf{ta}  \textbf{dani}  \textbf{ko}  may  namungko  en. \\\smallskip
 \gll Piro  \textbf{naan}  \textbf{ta}  \textbf{dani}  \textbf{ko}  may  na-ng-pungko  en. \\
but  \textsc{spat.def}  \textsc{nabs}  near  1\textsc{s.gen}  \textsc{ext.in}  \textsc{a.hap.r-pl}-sit  \textsc{cm} \\
\glt ‘But \textbf{at} \textbf{the} \textbf{place} \textbf{close} \textbf{to} \textbf{me} there was one (wild pig) sitting.’ [RCON-L-01 3.1]
\z
\ea
Naan  unti  gabetang  en  simbaan  \textbf{ta}  \textbf{ebes}  \textbf{i}. \\\smallskip
 \gll Naan  unti  ga-betang  en  simba-an  \textbf{ta}  \textbf{ebes}  \textbf{i}. \\
\textsc{spat.def}  \textsc{d}1\textsc{loc.pr}  \textsc{i.r}-put  \textsc{cm}  worship-\textsc{nr}  \textsc{nabs}  below  \textsc{def.n} \\
\glt ‘Here the church was positioned \textbf{at} \textbf{the} \textbf{place} \textbf{below}.’ (This is talking about a church built on the top of a hill that was a fort and a church built at the base of the hill.) [BBOE-C-01 1.6]
\z

As with all nouns, \isi{relational nouns}\is{nouns!relational} may themselves be modified by genitive case RPs:

\ea
Pagtapos  sa  en  gapanaog  kay  en  naan  ta  sikad bungyod  na  bukid  daw  gaagi  \textbf{naan}  \textbf{ta}  \textbf{kilid}  \textbf{ta}  \textbf{mga}  \textbf{karsada} na  sikad  daļem  panan-awen… \\\smallskip
 \gll Pag-tapos  sa  en  ga-panaog  kay  en  naan  ta  sikad bungyod  na  bukid  daw  ga-agi  \textbf{naan}  \textbf{ta}  \textbf{kilid}  \textbf{ta}  \textbf{mga}  \textbf{karsada} na  sikad  daļem  panan-aw-en… \\
\textsc{nr.act}-after  \textsc{d}4\textsc{nabs}  \textsc{cm}  \textsc{i.r}-go.down  1\textsc{p.excl.abs}  \textsc{cm}  \textsc{spat.def}  \textsc{nabs}  very slope  \textsc{lk}  mountain  and  \textsc{i.r}-pass  \textsc{spat.def}  \textsc{nabs}  side  \textsc{nabs}  \textsc{pl}  street
\textsc{lk}   very  deep  vision-\textsc{t.ir} \\
\glt `After that, we came down a very steep mountain and passed along \textbf{the} \textbf{side} \textbf{of} \textbf{the} \textbf{street} which had a view very far down....’ [RMWN-L-01 5.1]
\z

\ea
Pabay-an  nang  napatay  \textbf{naan}  \textbf{ta}  \textbf{tengnga}  \textbf{ta}  \textbf{daļan} tak  adlek  na  kani  danen  madapil. \\\smallskip
 \gll Pa-bay-an  nang  na-patay  \textbf{naan}  \textbf{ta}  \textbf{tengnga}  \textbf{ta}  \textbf{daļan} tak  adlek  na  kani  danen  ma-dapil. \\
\textsc{t.r}-leave.alone  just/only  \textsc{a.hap.r}-dead  \textsc{spat.def}  \textsc{nabs}  middle  \textsc{nabs}  road
because  afraid  \textsc{lk}  later  3\textsc{p.abs}  \textsc{a.hap.ir}-mixed.up.with/in \\
\glt `The dead one was just left \textbf{in} \textbf{the} \textbf{middle} \textbf{of} \textbf{the} \textbf{road} because (the people) were afraid that later they will get mixed up (in the murder).’ [JCWN-T-20 22.2]
\z


\ea
Gaubra  kay  ame  ta  salad  na  tampayas  \textbf{naan} \textbf{ta}  \textbf{gwa}  \textbf{ta}  \textbf{ame}  \textbf{na}  \textbf{baļay}. \\\smallskip
 \gll Ga-ubra  kay  ame  ta  salad  na  tampayas  \textbf{naan} \textbf{ta}  \textbf{gwa}  \textbf{ta}  \textbf{ame}  \textbf{na}  \textbf{baļay}. \\
\textsc{i.r}-make  1\textsc{p.excl.abs}  1\textsc{p.excl.gen}  \textsc{nabs}  salad  \textsc{lk}  papaya  \textsc{spat.def}
\textsc{nabs}  out  \textsc{nabs}  1\textsc{p.excl.gen}  \textsc{lk}  house \\
\glt `We were making papaya salad \textbf{in} \textbf{the} \textbf{outside} \textbf{part} \textbf{of} \textbf{our} \textbf{house}.’ (Porches and sometimes livingrooms are called \textit{gwa} on Cagayancillo.) [VAWN-T-16 2.6]
\z
\ea
Pag-abot  danen  naan  ta  suba,  palubbas  danen  iran  na bayo  daw  pabatang  danen  \textbf{naan}  \textbf{ta}  \textbf{kilid}  \textbf{ta}  \textbf{suba}  daw  maglangoy. \\\smallskip
 \gll Pag-abot  danen  naan  ta  suba,  pa-lubbas  danen  iran  na bayo  daw  pa-batang  danen  \textbf{naan}  \textbf{ta}  \textbf{kilid}  \textbf{ta}  \textbf{suba}  daw  mag-langoy. \\
\textsc{nr.act}-arrive  3\textsc{p.gen}  \textsc{spat.def}  \textsc{nabs}  river  \textsc{t.r}-undress  3\textsc{s.erg}  3\textsc{p.gen}  \textsc{lk}
clothes  and  \textsc{t.r}-put  3\textsc{s.erg}  \textsc{spat.def}  \textsc{nabs}  side  \textsc{nabs}  river  and  \textsc{i.ir}-bathe \\
\glt `When they arrived at the river, they took off their clothes and put (them) \textbf{on} \textbf{the} \textbf{bank} \textbf{of} \textbf{the} \textbf{river} and then bathed.’ [CBWN-C-25 5.4]
\z

The following is an example of a relational RP, \textit{apaw an ta apador} ‘the top of the cabinet’, expressed as absolutive via the applicative derivation in the main verb:

\ea
Pasangatan  din  ta  bayo  \textbf{apaw}  \textbf{an}  \textbf{ta}  \textbf{aparador.} \\\smallskip
 \gll Pa-sangat-an  din  ta  bayo  \textbf{apaw}  \textbf{an}  \textbf{ta}  \textbf{aparador.} \\
\textsc{t.r}-put.away-\textsc{apl}  3\textsc{s.erg}  \textsc{nabs}  clothes  top  \textsc{def.m}  \textsc{nabs}  cabinet \\
\glt ‘S/he put some clothes away in \textbf{the} \textbf{top} \textbf{part} \textbf{of} \textbf{the} \textbf{cabinet}.’
\is{prepositions!spatial|)}
\z


As shown above, the \isi{spatial prepositions}\is{prepositions!spatial} occur with the definite spatial marker \textit{naan} (including variants \textit{nyaan} and \textit{yaan}). The directional prepositions\is{directional prepositions|(}\is{prepositions!directional|(} usually do not. These are given in \REF{bkm:Ref480806122}.

 %longdistance
\ea
\label{bkm:Ref480806122}
\begin{tabbing}
\hspace{2.5cm} \= \kill
Directional prepositions \\
\textit{alin} \> ‘from’ \\
\textit{punta/munta} \> ‘(going) towards/to’\footnotemark \\
\textit{asta} \> ‘until/including’  (From Spanish \textit{hasta} ‘until’/’up to’) \\
\textit{keteb} \> ‘until’ \\
\textit{tuman} \> ‘to the point of’
\end{tabbing}
\footnotetext{These are lexicalized forms of the verb ‘go’. As prepositions, there seems to be no difference in meaning between \textit{punta} (bare form of the root) and \textit{munta} (irrealis form).}
\z

The directional prepositions \textit{munta}, \textit{asta}, and \textit{keteb} never occur with the spatial marker, though sometimes \textit{alin} does. As with positional prepositions, a non-absolutive case marker and a Referring Phrase follow directional prepositions, and the non-absolutive marker may drop out, especially before proper names \REF{bkm:Ref480806358}.

\ea 
Pag-abot  nay  ta  liyo,  gapanaw  kay eman  \textbf{munta}  \textbf{ta}  \textbf{Teresa}. \\\smallskip
 \gll Pag-abot  nay  ta  liyo,  ga-panaw  kay eman  \textbf{munta}  \textbf{ta}  \textbf{Teresa}. \\
\textsc{nr.act}-arrive  1\textsc{p.excl.gen}  \textsc{nabs}  otherside  \textsc{i.r}-go/walk  1\textsc{p.excl.abs}
again.as.before  going  \textsc{nabs}  Teresa \\
\glt `When we arrived at the other side (of the river), we started walking again as before \textbf{going} \textbf{to} \textbf{Teresa} (a community).’ [BGON-L-01 1.10]
\z
\ea
\label{bkm:Ref480806358}
Tapos,  gaplano  isab  daen  Mam  daw  Sir  na  makay kay  nang  en  ta  bus  na  \textbf{punta}  \textbf{Manila}. \\\smallskip
\gll Tapos,  gaplano  isab  daen  Mam  daw  Sir  na  makay kay  nang  en  ta  bus  na  \textbf{punta}  \textbf{Manila}. \\
then  \textsc{i.r}-plan  again  3\textsc{p.abs}  Ma’am  and  Sir  \textsc{lk}  \textsc{i.v.ir}-ride
1\textsc{p.excl.abs} just/only  \textsc{cm}  \textsc{nabs}  bus  \textsc{lk}  going  Manila \\
\glt `Then Ma’am and Sir were planning again that we ride a bus \textbf{going} \textbf{to} \textbf{Manila}.’ [AGWN-L-01 6.5]
\z
\ea
May  nakita  a  man  na  mga  waig  na  gailig \textbf{alin}  \textbf{ta}  \textbf{bukid}. \\\smallskip
 \gll May  na-kita  a  man  na  mga  waig  na  ga-ilig \textbf{alin}  \textbf{ta}  \textbf{bukid}. \\
\textsc{ext.in}  \textsc{a.hap.r}-see  1\textsc{s.abs}  too  \textsc{lk}  \textsc{pl}  water  \textsc{lk}  \textsc{i.r}-flow from  \textsc{nabs}  mountain \\
\glt `I saw some water too flowing \textbf{from} \textbf{the} \textbf{mountain}.’ [AGWN-L-01 2.3]
\z


Two directional prepositional phrases may occur in one sentence as long as they are different directions.
\ea
\textbf{Alin}  \textbf{ta}  \textbf{baga}  \textbf{din}  daw  \textbf{munta}  \textbf{ta}  \textbf{batiis}  pagapos  kanen  i. \\\smallskip

Prepositional Phrase 1 \hspace{1.6cm}   Prepositional Phrase 2 \\
\gll \textbf{Alin}  \textbf{ta}  \textbf{baga}  \textbf{din}  daw  \textbf{munta}  \textbf{ta}  \textbf{batiis}  pa-gapos  kanen  i. \\
from  \textsc{nabs}  shoulders  3\textsc{s.gen}  and  going  \textsc{nabs}  feet  \textsc{t.r}-tie.up  3\textsc{s.abs}  \textsc{def.n} \\
\glt ‘\textbf{From} \textbf{his} \textbf{shoulders} and \textbf{going} \textbf{to} \textbf{his} \textbf{feet} he was tied up.’ [MBON-T-07a 9.7]
\z
\ea
Sigi  legged  danen  ta  uling  \textbf{alin}  \textbf{ta}  \textbf{uļo} \textbf{ya} \textbf{asta}  \textbf{ta}  \textbf{ikog}  \textbf{din}  \textbf{ya}. \\\smallskip

\hspace{5.8cm} Prepositional Phrase 1 \\
\gll Sigi  legged  danen  ta  uling  \textbf{alin}  \textbf{ta}  \textbf{uļo} \textbf{ya} \\
continue  rub  3\textsc{p.erg}  \textsc{nabs}  charcoal  from  \textsc{nabs}  head  \textsc{def.f} \\\smallskip

 Prepositional Phrase 2 \\
\gll \textbf{asta}  \textbf{ta}  \textbf{ikog}  \textbf{din}  \textbf{ya}. \\
until  \textsc{nabs}  tail  3\textsc{s.gen}  \textsc{def.f} \\
\glt `They kept rubbing the charcoal (on the fish) \textbf{from} \textbf{the} \textbf{head} \textbf{to} \textbf{his} \textbf{tail}.’ [JCON-L-07 16.2]
\z

The directional prepositions can also express \isi{temporal relations}:

\ea
Kalabanan  ta  mga  taon  \textbf{alin}  \textbf{ta}  \textbf{buļan}  \textbf{ta}  \textbf{Nubimbri} \textbf{asta}  \textbf{ta}  \textbf{buļan}  \textbf{ta}  \textbf{Abril}  sigi  en  adlaw. \\\smallskip

\hspace{4.1cm} Prepositional Phrase 1 \\
\gll Kalabanan  ta  mga  taon  \textbf{alin}  \textbf{ta}  \textbf{buļan}  \textbf{ta}  \textbf{Nubimbri} \\
most  \textsc{nabs}  \textsc{pl}  year  from  \textsc{nabs}  month/moon  \textsc{nabs}  November \\\smallskip

 Prepositional Phrase 2 \\
\gll \textbf{asta}  \textbf{ta}  \textbf{buļan}  \textbf{ta}  \textbf{Abril}  sigi  en  adlaw. \\
until  \textsc{nabs}  month/moon  \textsc{nabs}  April  continue  \textsc{cm}  sun \\
\glt `Most years \textbf{from} \textbf{the} \textbf{month} \textbf{of} \textbf{November} \textbf{till} \textbf{the} \textbf{month} \textbf{of} \textbf{April} it is continually sunny.’ [JCWE-T-14 3.1]
\z

\newpage
\ea
Yi  ake  na  inagian  na  \textbf{asta}  \textbf{anduni}  dili  ko gid  malipatan. \\\smallskip
 \gll Yi  ake  na  <in>agi-an  na  \textbf{asta}  \textbf{anduni}  dili  ko gid  ma-lipat-an. \\
\textsc{d}1\textsc{abs}  1\textsc{s.gen}  \textsc{lk}  <\textsc{nr.res}>pass-\textsc{apl}  \textsc{lk}  until  now/today  \textsc{neg.ir}  1\textsc{s.erg}
\textsc{int}  \textsc{a.hap.ir}-forget-\textsc{apl} \\
\glt `This is my experience \textbf{that} \textbf{until} \textbf{now/today} I can’t forget.’ [VAWN-T-18 7.1] 
\is{prepositions!directional|)}\is{directional prepositions|)}
\z

The logical prepositions\is{logical prepositions|(}\is{prepositions!logical|(} never occur with the spatial location marker \textit{naan}. They can precede either a Referring Phrase or a clause. When they precede a clause, they function as subordinating conjunctions (see \chapref{chap:clausecombining}, \sectref{sec:finiteadverbialclauses} on finite adverbial clauses).

\ea
\begin{tabbing}
\hspace{2cm} \= \kill
Logical prepositions \\
\textit{para} \> ‘for/beneficiary/for the sake of’  (From Spanish \textit{para} ‘for’) \\
\textit{tenged} \> ‘about/because of’ \\
\textit{parti} \> ‘about’ \\
\textit{paagi} \> ‘by means of’ \\
\textit{duma} \> ‘with (accompaniment)’
\end{tabbing}
\z

The preposition \textit{para} can mean ‘for the purpose of something’ or ‘for the benefit of someone’. It can precede a referring phrase or a clause nominalized with \textit{pag}{}- (see \chapref{chap:clausecombining}, \sectref{sec:nominalizationsasadverbialclauses}).

\ea
Daw  ugtu  gani  en  nanay  an  gaeļeb-eļeb  \textbf{para}  \textbf{ta}  \textbf{yapon} eman  kani  lub-ong. \\\smallskip
 \gll Daw  ugtu  gani  en  nanay  an  ga-eļeb-eļeb  \textbf{para}  \textbf{ta}  \textbf{yapon} eman  kani  lub-ong. \\
if/when  noon  truly  \textsc{cm}  mother  \textsc{def.m}  \textsc{i.r}-\textsc{red}-slice.thin  for  \textsc{nabs}  supper
again.as.before  later  cooked.cassava \\
\glt `When (it is) noon truly the mother cuts thinly (cassava) \textbf{for} \textbf{our} \textbf{supper} again as before later, cooked cassava.’ [ICOE-C-01 3.2]
\z
\ea
Paimbargo  din  mga  manok,  baboy  daw  ano  pa  man makita  din  \textbf{para}  \textbf{ta}  \textbf{pamilya}  \textbf{din} … \\\smallskip
 \gll Pa-imbargo  din  mga  manok,  baboy  daw  ano  pa  man ma-kita  din  \textbf{para}  \textbf{ta}  \textbf{pamilya}  \textbf{din}… \\
\textsc{t.r}-confiscate  3\textsc{s.erg}  \textsc{pl}  chicken  pig  if/when  what  \textsc{inc}  also
\textsc{a.hap.ir}-see  3\textsc{s.erg}  for  \textsc{nabs}  family  3\textsc{s.gen} \\
\glt `He confiscated chickens, pigs and whatever  else he saw \textbf{for} \textbf{his} \textbf{family}…’ [JCWN-T-20 10.5]
\z
\ea
… yon  gamit  \textbf{para}  \textbf{ta}  \textbf{pag-ubra}   \textbf{ta}  ikam. \\\smallskip
 \gll … yon  gamit  \textbf{para}  \textbf{ta}  \textbf{pag-ubra}   \textbf{ta}  ikam. \\
{} \textsc{d}3\textsc{abs}  use  for  \textsc{nabs}  \textsc{nr.act}-make/work  \textsc{nabs}   mat \\
\glt ‘… those (previously mentioned pandan and buri) are used \textbf{for} \textbf{making} \textbf{mats}.’
\z

The logical preposition \textit{parti} ‘about’ is nearly synonymous with the preposition \textit{tenged}  ‘about’, as in examples \REF{bkm:Ref423444497}{}-\REF{bkm:Ref423444699}, except that \textit{tenged} can also precede a reason -‘because of something’, as in examples \REF{bkm:Ref423445020}{}-\REF{bkm:Ref443727092}. Both can precede a referring phrase or a \textit{pag}{}- clause \REF{bkm:Ref423444714}{}-\REF{bkm:Ref423444699}.

\ea
\label{bkm:Ref423444497}
Daw  gusto  no  maļaman  a  \textbf{parti}  \textbf{ta}  \textbf{ame}  \textbf{na}  \textbf{lugar}, ame na lugar  sikad  lineng… \\\smallskip
 \gll Daw  gusto  no  ma-aļam-an  a  \textbf{parti}  \textbf{ta}  \textbf{ame}  \textbf{na}  \textbf{lugar}, ame na lugar  sikad  lineng… \\
if/when  want  2\textsc{s.erg}  \textsc{a.hap.ir}-know-\textsc{apl}  \textsc{inj}  about  \textsc{nabs}  1\textsc{p.excl.gen}  \textsc{lk}
place 1\textsc{p.excl.gen}  \textsc{lk}  place  very  peaceful \\
\glt `If you want  to know \textbf{about} \textbf{our} \textbf{place}, our place is very peaceful…’ [EFWL-T-07
11.1]
\z
\ea
Yi  na  isturya  ni  \textbf{parti}  \textbf{ta}  \textbf{kapri}. \\\smallskip
 \gll Yi  na  isturya  ni  \textbf{parti}  \textbf{ta}  \textbf{kapri}. \\
\textsc{d}1\textsc{abs}  \textsc{lk}  story  \textsc{d}1\textsc{pr}  about  \textsc{nabs}  giant \\
\glt ‘This very story is \textbf{about} \textbf{a} \textbf{giant}.’ [CBON-T-03 1.1]
\z
\ea
Yaken  i  may  isturya  a  man  \textbf{tenged}  \textbf{ta}  \textbf{darwa}  \textbf{i} \textbf{na}  \textbf{mag-utod}  \textbf{daw}  \textbf{pari}. \\\smallskip
 \gll Yaken  i  may  isturya  a  man  \textbf{tenged}  \textbf{ta}  \textbf{darwa}  \textbf{i} \textbf{na}  \textbf{mag-utod}  \textbf{daw}  \textbf{pari}. \\
1\textsc{s.abs}  \textsc{def.n}  \textsc{ext.in}  story  1\textsc{s.abs}  too  about  \textsc{nabs}  two  \textsc{def.n}
\textsc{lk}  \textsc{rel}-sibling  and  friend \\
\glt `As for me, I have a story too \textbf{about} \textbf{two} \textbf{brothers} \textbf{and} \textbf{a} \textbf{friend}.’ [RBWN-T-02 1.1]
\z
\ea
Tudluan  ko  kanen  \textbf{tenged}  \textbf{ta}  \textbf{ame}  \textbf{na}  \textbf{ambaļ}. \\\smallskip
 \gll \emptyset{}-Tudlu-an  ko  kanen  \textbf{tenged}  \textbf{ta}  \textbf{ame}  \textbf{na}  \textbf{ambaļ}. \\
\textsc{t.ir}-teach-\textsc{apl}  1\textsc{s.erg}  3\textsc{s.abs}  about  \textsc{nabs}  1\textsc{p.excl.gen}  \textsc{lk}  say \\
\glt ‘I will teach her \textbf{about} \textbf{our} \textbf{language}.’ [VAWL-T-13 7.5]’\label{bkm:Ref423444714}
\z
\ea
\label{bkm:Ref423444699}
Saysay  \textbf{parti}  \textbf{ta}  \textbf{paglebbeng}. \\\smallskip
 \gll Saysay  \textbf{parti}  \textbf{ta}  \textbf{pag-lebbeng}. \\
describe  about  \textsc{nabs}  \textsc{nr.act}-bury \\
\glt ‘A description \textbf{about} \textbf{burying}.’ [CBWE-C-06 1.1]
\z


\ea
Yi  na  isturya  \textbf{tenged}  \textbf{ta}  \textbf{ame}  \textbf{na}  \textbf{pagleddang}. \\\smallskip
 \gll Yi  na  isturya  \textbf{tenged}  \textbf{ta}  \textbf{ame}  \textbf{na}  \textbf{pag-leddang}. \\
\textsc{d}1\textsc{adj}  \textsc{lk}  story  about  \textsc{nabs}  1\textsc{p.excl.gen}  \textsc{lk}  \textsc{nr.act}-sink \\
\glt ‘This story is \textbf{about} \textbf{our} \textbf{sinking}.’ [CBWN-C-11 1.1]\label{bkm:Ref423445020}
\z
\ea
Gani  ta  iran  na  pagnubig  kaysan  daen magapon dili  pa  kuli  \textbf{tenged}  \textbf{ta}  \textbf{biskeg}  \textbf{na}  \textbf{angin}  \textbf{daw}  \textbf{darko} \textbf{na}  \textbf{mga}  \textbf{baļed}. \\\smallskip
 \gll Gani  ta  iran  na  pag-nubig  kaysan  daen magapon dili  pa  ka-uli  \textbf{tenged}  \textbf{ta}  \textbf{biskeg}  \textbf{na}  \textbf{angin}  \textbf{daw}  \textbf{darko} \textbf{na}  \textbf{mga}  \textbf{baļed}. \\
so  \textsc{nabs}  3\textsc{p.gen}  \textsc{lk}  \textsc{nr.act}-haul.water  sometimes  3\textsc{p.abs} \\
all.day.long \textsc{neg.ir}  \textsc{inc}  \textsc{i.exm}-go.home  because  \textsc{nabs}  strong  \textsc{lk}
wind  and  big.\textsc{pl} \textsc{lk}  \textsc{pl}  wave \\
\glt `So during their hauling water, sometimes they stayed until late afternoon, not able to go home \textbf{because} \textbf{of} \textbf{strong} \textbf{winds} \textbf{and} \textbf{big} \textbf{waves}.’ [VPWE-T-01 2.8]
\z
\ea
\label{bkm:Ref443727092}
Ta  barangay  Sta. Cruz  isya  nang  gid  a  napatay  ya.  Dili man  en  \textbf{tenged}  \textbf{ta}  \textbf{masakit}  daw  dili  ta  subļa  en  na  dikstros … \\\smallskip
 \gll Ta  barangay  Sta. Cruz  isya  nang  gid  a  na-patay  ya.  Dili man  en  \textbf{tenged}  \textbf{ta}  \textbf{masakit}  daw  dili  ta  subļa  en  na  dikstros… \\
\textsc{nabs}  community  Sta. Cruz  one  just/only  \textsc{int}  \textsc{inj}  \textsc{a.hap.r}-dead  \textsc{def.f}  \textsc{neg.ir}
\textsc{emph}  \textsc{cm}  because  \textsc{nabs}  sickness  if/when  \textsc{neg.ir}  \textsc{nabs}  too.much  \textsc{cm}  \textsc{lk}  dextrose \\
\glt `In the community of Sta. Cruz only one really died. (It was) not \textbf{because} \textbf{of} \textbf{the} \textbf{sicknesss}, but rather (because) of too much dextrose…’ [JCWN-T-21 17.3]
\z

The preposition \textit{paagi} ‘by means of’ can occur with a referring phrase or a subordinate \textit{pag}{}- nominalized action clause.

\newpage
\ea
Uyi  kitaay  danen  ta  iya  na  sawa \textbf{paagi}  \textbf{man}  \textbf{ta}  \textbf{kaļat}  \textbf{nang} … \\\smallskip
 \gll U-yi  kita-ay  danen  ta  iya  na  sawa \textbf{paagi}  \textbf{man}  \textbf{ta}  \textbf{kaļat}  \textbf{nang}… \\
\textsc{emph}-\textsc{d}1\textsc{abs}  see-\textsc{rec}  3\textsc{p.abs}  \textsc{nabs}  3\textsc{s.gen}  \textsc{lk}  spouse
by.means.of  too  \textsc{nabs}  rope  only \\
\glt `\textsc{These}, he and his wife saw/met each other \textbf{by} \textbf{means} \textbf{of} \textbf{only} \textbf{a} \textbf{rope}…’ [MBON-T-07a 13.1]
\z
\ea
Kanen  i  giling  ta  yi  na  puļo  \textbf{paagi}  \textbf{ta} \textbf{pagsakay}  \textbf{ta}  \textbf{lunday}  para  manglaya. \\\smallskip
 \gll Kanen  i  ga-iling  ta  yi  na  puļo  \textbf{paagi}  \textbf{ta} \textbf{pag-sakay}  \textbf{ta}  \textbf{lunday}  para  ma-ng-laya. \\
3\textsc{s.abs}  \textsc{def.n}  \textsc{i.r}-go  \textsc{nabs}  \textsc{d}1\textsc{adj}  \textsc{lk}  island  by.means.of  \textsc{nabs}
\textsc{nr.act}-ride  \textsc{nabs}  outrigger.canoe  \textsc{purp}  \textsc{a.hap.ir}-\textsc{pl}-cast.net \\
\glt \textsc{‘He} came to this island \textbf{by} \textbf{means} \textbf{of} \textbf{riding} \textbf{an} \textbf{outrigger} \textbf{canoe} in order to go fishing with a cast net.’ [VAWN-T-17 2.2]
\z

The last examples in this section illustrate the preposition \textit{duma} ‘with (accompaniment)’. Recall that this word is also used as a noun to mean “companion”, and as a verb to mean “to accompany”:

\ea
May  darwa  buok  na  tallog  na  linaga  na  kupad-kuparen man  \textbf{duma}  \textbf{ta}  \textbf{duma}  \textbf{na}  \textbf{kinupadkupad}. \\\smallskip
 \gll May  darwa  buok  na  tallog  na  l<in>aga  na  kupad-{}kupar-en man  \textbf{duma}  \textbf{ta}  \textbf{duma}  \textbf{na}  \textbf{k<in>upad-kupad}. \\
\textsc{ext.in}  two  piece  \textsc{lk}  egg  \textsc{lk}  <\textsc{nr.res}>boil  \textsc{lk}  \textsc{red}\sim{}slice\textsc{-t.ir}
also  with  \textsc{nabs}  other  \textsc{lk}  <\textsc{nr.res}>\textsc{red}\sim{}slice \\
\glt `There are two boiled eggs to be sliced also \textbf{with} \textbf{the} \textbf{other} \textbf{sliced} \textbf{things}.’ [JCWE-T-16 4.6]
\z
\ea
Yo  waig  din  ya  en  na  palaga  ya  \textbf{duma}  \textbf{ta}  \textbf{tang}ļ\textbf{ad}  \textbf{ya,} yo  inemen. \\\smallskip
 \gll Yo  waig  din  ya  en  na  pa-laga  ya  \textbf{duma}  \textbf{ta}  \textbf{tang}ļ\textbf{ad}  \textbf{ya,} yo  inem-en. \\
\textsc{d}4\textsc{abs}  water  3\textsc{s.gen}  \textsc{def.f}  \textsc{cm}  \textsc{lk}  \textsc{t.r}-boil  \textsc{def.f}  with  \textsc{nabs}  lemon.grass  \textsc{def.f}
\textsc{d}4\textsc{abs}  drink-\textsc{t.ir} \\
\glt `That one, its water that was boiled \textbf{with} \textbf{the} \textbf{lemon} \textbf{gras}s, that one drink (it).’ (medicine for high blood pressure) [DBOE-C-04 9.3]
\is{prepositions!logical|)}\is{logical prepositions|)}
\is{Prepositional Phrases|)}
\z
