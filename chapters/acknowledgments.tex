\addchap{\lsAcknowledgementTitle} 

We are very grateful to many Kagayanen speakers who have befriended Carol, taught her to speak Kagayanen, encouraged her, and helped her in many ways including contributing stories and many long conversations. It is impossible to list everyone here and so we will just mention those who worked with Carol for a significant time period. These include \name{Henry}{Bungalso}, \name{Mermor}{Ballena}, \name{Levi}{Bonales}, \name{Darlie}{Bundal}, \name{Jocelyn}{Bundal}, \name{Ruby}{Bundal}, \name{Lorebeth}{Obra}, and \name{Edith}{Tapalla}. Without their patient and invaluable help this grammar would not exist. 

During the time Carol lived in Kagayanen communities, several SIL linguistic consultants worked with her in analyzing and writing on different aspects of the Kagayanen language. These include \name{Sherri}{Brainard}, \name{Mary Ruth}{Wise}, \name{Lou}{Hohulin}, \name{Doris}{Payne} and Thomas Payne. The things learned through these consultations have been incorporated throughout this grammar.
In the early stages of writing this grammar \name{Mike}{Cahill}, another SIL linguistic consultant, worked with Carol on the Phonology during a 6-week academic writing workshop in Bagabag, Philippines. Others have commented on various chapters and have given us good advice and ideas for improving the grammar. These include \name{Janet}{Allen}, \name{Michael}{Boutin}, Pastor \name{Jehu Pedigan}{Cayaon}, \name{Sharon Joy}{Estioca}, \name{Melanie Bundal}{Fresnillo}, \name{Jacqueline}{Huggins}, \name{Paul}{Kroeger}, \name{Louise}{MacGregor}, \name{Ken}{Manson}, \name{Ricardo}{Nolasco}, \name{Steve}{Quakenbush}, and \name{Doug}{Trick}. Our sincere apologies for any omissions. 

Carol is wholeheartedly thankful for the many who have supported her in the Kagayanen project for many years, including churches, family and friends. A special mention is due Professor \name{Walter}{Zorn} who sparked Carol’s love of languages and inspired her in continuing to study languages during her first years at Great Lakes Bible College in Lansing Michigan. Previous to this she had no interest or experience in languages or linguistics. Carol is also grateful to her teachers at Fort Wayne Bible college, SIL Dallas, and the University of Texas at Arlington. Though all her teachers were important in her training, the ones who were the most encouraging during her linguistic M.A. course at SIL and UTA were \name{Ilah}{Fleming}, \name{Robert}{Longacre}, \name{Terry}{Malone}, \name{Kenneth}{Pike}, and \name{Eunice}{Pike}. Also, many friends during her training days and since have given her the confidence and hope to persevere in working in the Philippines, learning Kagayanen, describing it, and writing a complete grammar. These include friends and roommates during the Dallas training, friends and supporters in the USA, as well as SIL and TAP (Translation Association of the Philippines) friends and colleagues in the Philippines. Carol is especially indebted to her teammates on the Kagayanen project: \name{Scott}{MacGregor} and \name{Louise}{MacGregor}, \name{Jacqueline}{Huggins}, \name{Josephine}{Wan} and \name{Michael}{Wan}, and \name{Carla}{Morgan}. Without their support, advice, discussions and insightful knowledge of the Kagayanen culture and language this grammar would not be what it is now. 

Thomas Payne is a latecomer to this project, having consulted during a grammar workshop in 1990, and having served as a regular grammar consultant since 2010. We would like to thank the Canada Institute of Linguistics, SIL International, SIL Philippines, the Fulbright Foundation, and the Pike Center for Integrative Scholarship for encouraging and supporting Thomas’ participation.